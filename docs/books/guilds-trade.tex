 % Use  to include (non HTML-escape) variable foo instead of {{{foo}}}
\documentclass{book}

%% Pacake inclusion
% Unicode support if xelatex is used
\usepackage{fontspec}
\usepackage{xunicode}

\usepackage[english]{babel} % Language support
\usepackage{fancyhdr} % Headers

% Allows hyphenatations in \texttt
\usepackage[htt]{hyphenat}





% Included if the stdpage option if set to false
\usepackage[a5paper, top=2cm, bottom=1.5cm,
  left=2.5cm,right=1.5cm]{geometry} % Set dimensions/margins of the parge


\makeatletter
\date{}

% Redefine the \maketitle command, only for book class (not used if stdpage option is set to true)
\renewcommand{\maketitle}{
  % First page with only the title
  \thispagestyle{empty}
  \vspace*{\stretch{1}}
  
  \begin{center}
    {\Huge \@title   \\[5mm]}
  \end{center}
  \vspace*{\stretch{2}}
  
  \newpage
  % Empty left page
  \thispagestyle{empty}
  \cleardoublepage

  % Main title page, with author, title, subtitle, date
  \begin{center}  
    \thispagestyle{empty}
    \vspace*{\baselineskip}
    \rule{\textwidth}{1.6pt}\vspace*{-\baselineskip}\vspace*{2pt}
    \rule{\textwidth}{0.4pt}\\[\baselineskip]
    
    {\Huge\scshape \@title   \\[5mm]}
    {\Large }
    
    \rule{\textwidth}{0.4pt}\vspace*{-\baselineskip}\vspace{3.2pt}
    \rule{\textwidth}{1.6pt}\\[\baselineskip]

    \vspace*{4\baselineskip}

    {\Large \@author}
    \vfill
    
  \end{center}
  
  \pagebreak
  \newpage
  % Copyright page with author, version, and license
  \thispagestyle{empty}
  \null\vfill
  \noindent
  \begin{center}
    {\emph{\@title}, © \@author.\\[5mm]}
    {This work is free of known copyright restrictions.\\[5mm]}
  \end{center}
  \pagebreak
  \newpage
}


% Redefine headers
\pagestyle{fancy}
\fancyhead{}
\fancyhead[CO,CE]{\thepage}
\fancyfoot{}



%%%%%%%%%%%%%%%%%%%%%%%%%%%%%%%%%%%%%%%%%%%%%%%%%%%%%%%%%%%%%%%%%
% Command and environment definitions
%
% Here, commands are defined for all Markdown element (even if some
% of them do nothing in this template).
%
% If you want to change the rendering of some elements, this is probably
% what you should modify.
%
% Note that elements that already have a LaTeX semantic equivalent aren't redefined
% : if you want to redefine headers, you'll have to renew \chapter, \section, \subsection,
% ..., commands. If you want to change how emphasis is displayed, you'll have to renew
% the \emph command, for list the itemize one, for ordered list the enumerate one,
% for super/subscript the \textsuper/subscript ones.
%
%%%%%%%%%%%%%%%%%%%%%%%%%%%%%%%%%%%%%%%%%%%%%%%%%%%%%%%%%%%%%%%%%%%

% Strong
\newcommand\mdstrong[1]{\textbf{#1}}

% Code
\newcommand\mdcode[1]{\texttt{#1}}

% Rule
% Default impl : (displays centered asterisks)
\newcommand\mdrule{
  \nopagebreak
  {\vskip 1em}
  \nopagebreak
  \begin{center}
    ***
  \end{center}
  \nopagebreak
 {\vskip 1em}
 \nopagebreak
}

% Hardbreak
\newcommand\mdhardbreak{\\}

% Block quote$
\newenvironment{mdblockquote}{%
  \begin{quotation}
    \itshape
}{%
  \end{quotation}
}


% Code block
%
% Only used if syntect is used for syntax highlighting is used, else
% the spverbatim environment is preferred.





\makeatother

\title{Guilds, Trade, and Agriculture}
\author{Arthur Penty}

\begin{document}

% Redefine chapter and part names if they needs to be
% Needs to be after \begin{document} because babel



\maketitle

\setcounter{tocdepth}{0}
\setcounter{secnumdepth}{0}
\tableofcontents
\chapter*{Preface}
\label{chapter-0}
In a series of articles recently contributed to the \emph{Daily News} under the title “Europe in Chaos,” the writer deduced the doom of modern civilization from the general tendency of the ratio of exchange to fall since the Armistice. In his last article he suggested that “Perhaps the Guild Socialists have seen a vision of the ultimate solution,” and then went on to say, “but if so they must descend from the clouds and begin to construct their system here and now.” For “if things are allowed to drift for another two or three years it will be too late.”

This little book accepts the general point of view of European affairs as enunciated in those articles and seeks to carry the discussion one stage nearer to practical politics. If Guild Socialists are not to be seen everywhere at work constructing their system, it is not due to the absence of any will or desire in the matter but to the fact that except in respect of Building Guilds they have no clear notion of how exactly to get to work. We believe we know the ultimate solution; but hitherto it has not been quite clear to us what is the next step. The recent divisions among Guild Socialists witness only too clearly to the perplexity that has overtaken the movement. It occurred to me whilst reading the articles already mentioned, that perhaps this perplexity was due to the fact that the Guild theory and policy was inadequate to the extent that it had been built up around the problem of production to the neglect of the problem of exchange. It was inevitable perhaps that this should be so, since we were led in the first instance to believe in the essential Tightness of Guild organization from a study of the problems of production rather than of exchange. Moreover, the particular form that Guild theory has taken is in no small measure due to the fact that it arose to combat the bureaucratic tendencies of Collectivism. In this light the defect of the Guild theory is not that what it affirms is not true, but that other aspects of truth have escaped its attention.

The present volume aims at remedying this defect by stating Guild theory and policy from the point of view of exchange. In so far as it differs from the previous Guild theory it is a difference of emphasis. Instead of making the establishment of Guilds the central issue, it treats Guilds as a means to an end the end being the maintenance of the Just Price–in the belief that the establishment of the Just Price is the solution of the problem of exchange in so far as this problem is a question of money, and values. It moreover shows that as far as England is concerned, the revival of agriculture is the necessary corollary of any stabilization of the exchanges. By thus widening the issues it becomes possible to carry the Guild idea into spheres where hitherto it has not entered.

Mention has been made of the articles entitled “Europe in Chaos.” By the kind permission of their author, Mr. J. S. M. Ward, and the Editor of the \emph{Daily News}, I am able to include them in this volume as an Appendix. The articles are the summary of Mr. Ward’s book, since published, entitled \emph{Can our Industrial System Survive?} (W. Rider \& Sons, Ltd., 2s. 6d.). It is a book I cannot speak too highly of, for if facts and figures could awaken us to the realities of the situation that confronts us it should do so, while it is entirely indispensable to any one who is anxious to understand the problem.

It remains for me to thank Dr. P. B. Ballard for his assistance in preparing the MS. for press.

A. J. P. 66 Strand-on-Green, W. 4. \emph{February} 1921.

\chapter{The Need of a Social Theory}
\label{chapter-1}
Whatever differences of opinion may exist as to the best way of facing the problem confronting society, a general consensus is growing up that the present order is doomed. It is agreed that things are going from bad to worse, and that it is only a matter of time–a few years at the most–before the great crisis will arrive that will determine whether England is to go the way of Russia and Central Europe–to anarchy and barbarism–or to be reconstructed on some co-operative or communal basis.

Which of these two ways things will go depends upon our action in the immediate future. If we allow ourselves to drift, then in a few years’ time we shall arrive at the state of affairs we know by the name of Bolshevism. For “Bolshevism is the last resort of desperate starving men”;\footnotemark[1] and starvation is at the end of our story, as we shall begin to understand more clearly when the reasons for the present impasse are understood. From this fate there is no possible means of escape, except by boldly facing the problem that confronts us and resolutely taking in hand the reconstruction of society from its very foundations upwards. Nothing less than that is any use at all. For it is the foundations that are giving way. And so, unless we act while yet there is time, there can be no saving of our civilization.

Meanwhile the difficulty that confronts reformers and statesmen alike is to know how to act. All their lives they have lived on certain phrases and shibboleths, and in a very literal sense taken no thought of the morrow. They have talked about progress and emancipation and our glorious civilization, which, in spite of defects, they have never failed to remind us is superior to any civilization of the past. And now Nemesis is overtaking us. A few years of war and our glorious civilization is seen to be crumbling and our statesmen and reformers are entirely at a loss to explain how such a thing could possibly happen, for they lack any comprehension of the problem of our society as a whole. They have for so long been concerned with the secondary things in society and have so persistently neglected the discussion of primary and fundamental principles, that they are without the mental equipment which a great crisis demands.

Evidence of their lack of grip on reality is forthcoming on every hand. Men who know what they want go straight ahead. They act with promptitude and decision. But in these days, if one were to judge only by appearance, one would say that the great idea in politics is to wait until you are pushed, and then to yield with a becoming dignity. But of course that is only appearance. The real explanation is that our statesmen and politicians have lost their way, and they are without a compass to guide them. In other words they have become opportunists because they have lost their faith, and they have lost their faith because the social theories upon which they relied have become untenable. Before the war the gospel of economic individualism that had been the faith of the nineteenth century was already discredited, while collectivism, which sought to take its place, was proving unworkable in practice. But the war has completed the destruction of these beliefs, and in consequence their adherents flounder about, attempting first this and then that in the hope that by some unexpected turn of events a path will be open to them. But it all avails nothing. For without a belief they lack conviction; and this prevents them from acting with unity of purpose or continuity of effort in any direction. Among the thousand and one things that claim their immediate attention they are unable to distinguish those which are of primary and fundamental importance from those that are secondary. So when by chance they stumble upon something which if persisted in would give results, they lack the determination to go forward, and the moment they come up against some obstacle they turn round and run. So it will be until we can establish a social theory that will give such an explanation of the facts as will guide them. For there is no such thing as a purely practical problem, inasmuch as behind every practical question is to be found a theoretical one.

Now the underlying cause of the collapse since the war of all social and economic theories that had secured any widespread organized support is that one and all of them took our industrial system for granted as a thing of permanence and stability. This is just as true of Socialist as of capitalist economics, inasmuch as all Socialist theories presupposed that a time would come when the workers would be able to take over capitalist industry as a going concern. The consequence is that Socialist and Labour leaders are as much perplexed as capitalists themselves at the sight of the system crumbling to pieces. The possibility of this dissolution had never occurred to them, and they have no idea how to stop it. And this is no wonder. For their belief in the permanence of industrial organization was so absolute that it led them to reject all ideas that were incompatible with the industrial system; and as all ideas of a fundamental nature inevitably came into collision with the industrial system it meant in practice that they refused to recognize any fundamental ideas whatsoever, so they are consequently left stranded without an idea that has any relevance to the present situation. The Bolsheviks alone are not disillusionized; and they are not disillusionized because in spite of their economic formulae their faith is in the class war. So firm are they in their belief that things will naturally right themselves once the workers attain to power, that they actually discourage speculation regarding the future as something that diverts energy from their central object of attaining power.

Recognizing, then, that the collapse of existing economic theories is due to the fact that they accepted industrialism as a thing of permanence and stability, it follows that any new social theory adequate to the situation must be based upon principles that are antipathetic to industrialism. Such principles are, I believe, to be deduced from the informal philosophy of the Socialist movement which is to be distinguished from its formal and official theories. The formal theories of Socialism based upon the permanence of industrialism are now happily discredited for ever. But the informal philosophy of the movement stands unimpaired, for it is based upon something far more fundamental than any economic theory–the permanent needs of human nature. On its negative side it is a moral revolt against capitalism ; on the positive side it rests upon the affirmation of the principles of brotherhood, mutual aid, fellowship, the common life. These are the things that the Socialist movement finally stands for; and they grow by reaction. In proportion as existing society becomes more hopeless, more corrupt, more unstable, men will tend to take refuge in idealism ; and this idealism the informal philosophy of the Socialist philosophy supplies. Such people have hitherto accepted the economic theories of Socialism as convenient formulae to give shape to their moral protests. But intellectual comprehension among them was rare, inasmuch as most of them swallowed the theories without tasting them. When they do taste them, they spew them out.

The deduction to be made from all this is that Socialism is finally a moral rather than an economic movement. It is because of this that it has gathered strength in spite of the discrediting of its successive theories. It is this that we must build upon. Our aim should be to bring economic theory into a direct relationship with this informal moral philosophy, to dig as it were a channel in which its whole strength may flow instead of being wasted in the sands of contradictory beliefs and impossible doctrines.

\footnotetext[1]{In this country Bolshevism is the last resort of disillusionized social theorists.

}\chapter{On Wages and Foreign Trade}
\label{chapter-2}
In the preceding chapter I urged the necessity of a social theory that would bring economics into a direct relationship with the informal Socialist philosophy with its ideas of brotherhood, mutual aid, fellowship and the common life. Recent events have brought into a new prominence the antagonism that exists between the head and the heart of Socialism.

During the war wages were raised to keep pace with the increasing cost of living. Nowadays, when prices are falling, the demand is made by employers that a corresponding reduction shall be made in wages. Behind this demand is the contention of employers that foreign trade cannot be restored and unemployment lessened while costs of production in this country remain as high as at present. The more reasonable trade unionists are disposed to accept this view on the assumption that the employers are willing to accept a corresponding reduction in profits. But the extremists refuse to accept any lowering of existing standards of wages without a struggle.

Now, from the point of view of formal Socialist theory, the extremists who refuse to consider a reduction in wages are in the right. If the relations of Capital and Labour are the mechanical ones postulated of Socialist theory the workers are justified in demanding that they shall enjoy a permanent increase in wages. Nor can there be any doubt whatsoever that they are ultimately in the right. If it was possible in the fifteenth century for the town worker to be paid a wage that worked out six or seven times the cost of his board and the agricultural worker two-thirds of this amount,\footnotemark[1] it is on the face of things extra- ordinary that with our enormously increased productivity it should yet be impossible to pay the workers a wage which covers little more than bare necessities. Yet a close examination reveals the fact that the present system of industry is so wasteful and built up on a basis so false that it cannot be made to pay the wages that the workers are theoretically justified in demanding. It is apparent that the increases cannot come in the particular way Labour expects or by their particular methods. It is not in the nature of things. Industry as it exists to-day in our great industrial centres is dependent upon foreign trade, and so long as it is so dependent it will be necessary to compete. Except, therefore, where we enjoy some monopoly or other artificial advantage, we shall only be able to compete successfully by producing as cheaply as possible, and that involves lower wages than were paid during the war. There is no getting away from this. If we are to remain an industrial competing nation, the workers must be prepared to accept such wages as will enable our manufacturers to compete successfully.\footnotemark[2] If they are not satisfied with so little–and there is no reason why they should be–the present system must be changed.

It is here we come to the popular Socialist fallacy. The present system is not changed merely by changing its ownership, since if the workers succeeded in getting possession of industry to-morrow they would be subject to the same economic laws to which employers are subject to-day, and they would be compelled to act much in the same way because they would be required to run the same machine. But if we wish to change the system we must recognize the necessity for industry to become as far as possible independent of foreign markets. This involves the revival of agriculture, for only by such means can the home markets be restored. In so far as industry could depend upon the home markets, we should be able to exercise control over the conditions of industry, and real fundamental changes in the position of the workers could be introduced. But it is vain to suppose that any such change can be introduced so long as industry rests on the economic quicksand of foreign markets. It becomes apparent therefore that if the position of the workers is to be improved they must take longer views. There is no such thing as “Socialism now.” But there is such a thing as Socialism in ten years’ time if the workers could be persuaded to follow a consistent policy over such a period of time. The trouble is that the workers, as indeed most people in every class, think of the social problem in the terms of their own jobs. The engineer wants a solution in the terms of engineering; the bootmaker in the terms of boots; the clerk in the terms of clerking. It is natural, perhaps, but none the less impossible, for it disregards the action of those world-wide economic forces which dominate all nations in proportion as they become dependent upon foreign trade.

I said that if the position of the workers is to be improved they must take longer views. It is clear that modern industrial activities are essentially transitory in their nature. Quite apart from the war, it is manifest that sooner or later the situation that exists to-day must have arisen, for the existing arrangement whereby goods are produced at one end of the earth and food at the other does not possess within itself the elements of permanence. It owes its existence to many things, but by far the most important to the fact that we were the first to employ machinery in production. This virtual monopoly that we had for so long encouraged the growth of cross- distribution. But it is uneconomic and therefore cannot last, for it is apparent that other things being equal, it must be cheaper to produce goods near the markets than at a distance from them. An arrangement may be uneconomic, but custom and inertia will combine to perpetuate it long after the circumstances which brought it into existence have disappeared. The war woke up many of our former customers to this fact. Before the war they were content to produce food and raw materials, and relied upon us in the main for their manufactured goods. During the war we could not supply their wants, and they took to manufacturing all kinds of things for themselves. As these manufactures are carried on near to where the raw materials are found or produced, it is manifest that we cannot hope to recover these markets. They must gradually slip from our hands. We cannot expect to export in the future such large quantities of manufactured goods to Australia, Canada, South America and elsewhere as hitherto. Meanwhile, in order to finance the war, we disposed of most of our foreign investments. The result of it all is that our industries will be unable to provide work for such numbers as hitherto. Not being able to sell goods to the food-producing nations, we shall soon be without the money to pay for the food we must import to keep our population alive–a fact that is brought home to us by the constant falling of the rate of exchange.

It appears therefore that though the reversal of our Russian policy, the complete removal of blockades and the provision of credits for the restoration of European trade would relieve the unemployed problem, it cannot hope to solve it, since a wider view of the situation leads to the conclusion that such relief can only be temporary. The renewal of trade facilities with Russia and Central Europe might relieve the congested state of the home market, but it will not provide us with the wherewithal to buy food, because Europe has no food to give us in exchange for our goods. If food is to be obtained, we must give something in exchange to the countries which produce it or we must produce it for ourselves. And as those countries upon which we have been accustomed to rely for a supply of food are beginning to produce their industrial wares for themselves, it follows that the only way to meet the situation is to take measures to produce as much food as possible for ourselves by the revival of agriculture. By no other means can the balance of exchange be restored. Agriculture is fundamental, since the price of food determines the cost of everything else. If therefore we neglect to revive agriculture, we shall be exploited by the countries who do produce food, and this, by raising the price of our manufactures, will in turn increase our difficulties in competing in other markets. It is insufficiently recognized that during the war the agricultural populations all over the world have been becoming rich while the industrial ones have become poor. It is not improbable therefore that capitalism, declining in the towns, may rehabilitate itself through agriculture. It certainly will do so unless Socialists are very much more wide awake than they have hitherto been.

Though at the moment the change which we are required to make will be difficult and inconvenient to the people affected, it will, if taken in hand with resolution, prove undoubtedly to be a blessing, for our society is top heavy, and the revival of agriculture is a movement in the direction of a return to the normal. But even with agri- culture revived it is questionable whether we shall in the long run be able to support our present population. In so far as this is true, there is only one remedy, and that is emigration. And here the real trouble begins. Emigration has so often been advocated as an excuse for postponing reforms at home that a natural and justifiable suspicion attaches to any one who advocates it, as Mr. Lloyd George found out recently when he suggested it as a remedy for unemployment. But it was not only with critics at home that he had to contend. The Dominions themselves lost no time in announcing that they had unemployed problems of their own and therefore could not assume responsibility for ours. And there the matter was allowed to drop. Nevertheless I am persuaded that emigration is a necessary part of the solution of our problem, and by one means or another it must be rendered practicable. That England, having sold her foreign investments and lost her oversea markets, cannot hope even with agriculture revived to support her present population is demonstrable beyond doubt. But that our Dominions, with their vast empty spaces of fertile land that can produce the food and supply the raw materials of industry, cannot find room for our surplus population is a paradox a paradox moreover that needs to be explained, since it is impossible to deny that such is the situation in our colonies to-day. It suggests the question: Why does our economic system produce such contradictory results? What is it that has got such a strangle-hold upon all modern industrial nations?

\footnotetext[1]{The wages of the artisan during the period to which I refer (the fifteenth century) were generally, and through the year, about 6d. per day. Those of the agricultural labourers were about 4d. I am referring to ordinary artisans and ordinary workers... It is plain the day was one of eight hours... Sometimes the labourer is paid for every day in the year, though it is certain he did not work on Sundays and principal holidays. Very often the labourer is fed. In this case, the cost of maintenance is put down at from 6d. to 8d. a week. Food was so abundant and cheap that it was sometimes thrown in with the wages (\emph{Six Centuries of Work and Wages}, by J. E. Thorold Rogers, pp. 327-8).

}\footnotetext[2]{What I say here must not be interpreted as giving any countenance to the indiscriminate reduction of wages that has begun to take place since these words were written. Where high wages are demonstrably the cause of stagnation in an industry, as in many cases there is every reason to believe they are, they must be reduced to get the machine going again. But it seems that the original idea of taking something off the highest wages corresponding to the lowering of the cost of living is being used as an excuse for reducing the wages of the lowest paid workers, because such workers, being unorganized, are defenceless. Not only is this inhuman, but it is uneconomic. The fallacy involved in such reductions is exposed in the concluding paragraph of the next chapter.

}\chapter{The Tyranny of Big Business}
\label{chapter-3}
I concluded the last chapter by asking what it was in the economic system of industrial nations that had got such a strangle-hold upon them.

The usual answer is of course to ascribe the general paralysis to the economic reactions that followed the war. In the immediate sense this is partially true. But of itself it is an insufficient explanation, for it is evident that the disease existed and was rapidly developing before the war. Let us therefore begin our inquiry by focusing our attention upon a most evident symptom and consider the widespread tyranny of big business. The success of these large organizations has been so dazzling that they have almost succeeded in silencing critics as to the ultimate validity of their activities. They have claimed to be the last word in efficiency, and to be justified as evidence of the survival of the fittest. For most people this has been a sufficient apology, and they have inquired no further. But we are unwilling to accept them at their own valuation, since we are persuaded that in them and their methods the immediate cause of the paralysis that is overtaking industry is to be discovered.

It will not be denied that expansion is to our industrial system the breath of life. So long as the system could continue expanding, it worked in spite of its shortcomings and injustices. But our economic system is so fearfully and wonderfully made that it cannot remain stationary. Once the limit of expansion is reached, contraction sets in, and with it all manner of internal complications begin to make their appearance. The honour of placing a limit to this process of expansion belongs to large financial and industrial organizations which have overreached themselves. So keen have they been on making money that they have ignored all other considerations, and for a generation they have been at work undermining the very foundations on which their prosperity ultimately rested. The changed position of the pioneer since big business got under way will bring this point home. The pioneer is the advance guard of civilization. He goes out into uninhabited places, he clears the land, and it is by means of his conquests that the area of civilization is extended. That he should continue his work is necessary for the continuance of our civilization; for, as I have already said, expansion is to it the breath of life. And how has big business treated the pioneer? The answer is, it has simply strangled him. The pioneer is isolated. He is dependent upon dealers for his supplies and for the marketing of his produce. In the old days of colonial expansion there were many such competing dealers, and this fact ensured him favourable terms; but a time came when big business got the upper hand. And then tilings changed. The pioneer found himself at the mercy of some trust or syndicate that was in a position to bleed him white and did not hesitate to do so. When news was noised abroad of the treatment to which those who went on the land in the colonies were subjected, no new men ventured. They no longer went forth with the proverbial half a crown in their pockets to embark upon some new enterprise with a feeling of assurance and confidence. For they began to realize that they had not a dog’s chance of success. It was thus that the initiative and enterprise that made our colonies was strangled. The age of expansion came to an end and our colonies began to develop their own unemployed problems. That is why nowadays they have no place for the emigrant. Contraction has set in.

A generation ago it was the custom to belaud these large organizations; to assume that because they were successful they represented a higher form of industrial organization; and, on the grounds of the necessities of social evolution, to condone the immorality of their methods as inevitable in a time of transition. It was supposed that by suppressing competition they were laying the foundations of the communal civilization of the future, and that when their great work of amalgamation and centralization was completed they would pass into the hands of the people. To-day we realize that this was a vain delusion. We no longer justify them as the fittest to survive. We have begun to ask the question as to whether they can survive at all. For they have been too short-sighted to make any provision for the future. Systems of organizations that have endured in the past were always careful to see that a ladder existed whereby the coming generation could rise step by step until they reached the summit of their callings. By such means these organizations renewed themselves. With such an eye to the future the Mediaeval Guilds jealously guarded the position of the apprentice. Apprenticeship was “an integral element in the constitution of the craft guilds, because in no other way was it possible to ensure the permanency of practice and continuity of tradition whereby alone the regulations of the guilds for honourable dealing and sound workmanship could be carried on from generation to generation.”\footnotemark[1] And this principle was not only understood by the craftsmen, but it was understood by the merchants in the past who, we read, were accustomed to sneer at the East India Company because it could not “breed up” merchants of initiative and independence. And this feeling against joint-stock enterprises continued until the middle of last century, when it yielded at last to the force of circumstances consequent upon the industrial revolution, and the principle of limited liability became admitted in law.\footnotemark[2]

The evil inherent in joint-stock companies was not fatal to them at first, since before the acknowledgment of the principle of limited liability in law they were few and far between, and so it became possible for them to renew their organization, wherever it was defective, by recruiting from outside their ranks. But once they become general, the evil inherent in them rapidly developed; for it soon became apparent that the divorce of ownership from management upon which they were based brought into existence horizontal and class divisions between those in their employ; and this spread disaffection everywhere. For men began to find that their future depended less on themselves than on the attitude of their immediate superiors towards them. In the higher ranks, these circumstances led to those jealousies and feuds by which all large organizations are distracted. In the lower ones it led to apathy and indifference; for when large organizations took away liberty from the individual they took away from him all living interest in his work. The effect of it all has been the destruction of the sense of responsibility. This results in a loss of efficiency. Expenses go up and up, and there seems to be no stopping them. Recourse is had to amalgamations. But it is all of no avail; for the soul has gone out of the body and there is no restoring it.

There is no restoring the \emph{morale} of these large organizations, because they have succeeded in destroying confidence and goodwill everywhere by the short-sightedness of their policy. For not only are limited companies responsible for the flood of commercial dishonesty and legalized fraud that have simply overwhelmed modern society, but under their aegis Labour has become more and more embittered. It is widely recognized nowadays that the mass of men have no disposition to do any more work than they can help. This is in the main due to these large organizations which lead men to feel that not they but others are going to profit by their labour. So long as competence was rewarded and honour appreciated there was an incentive for men to work. If they became efficient they might get on to their own feet, and the presence of a number of men with such ambitions in industry gave a certain moral tone to it that reacted upon others. But when, owing to the spread of limited companies, all such hopes were definitely removed; when technical ability, however great, went unrecognized and unrewarded, and proficiency in any department of industry incurred the jealousy of “duds” in high places, demoralization set in. All the old incentives were gone, and no one was left to set a standard. The suppressed impulses of men whose ambitions were thwarted turned into destructive channels. The rising generation, feeling themselves the defenceless victims of exploitation, are in open rebellion. They refuse any longer to make profits for others, and this refusal is accompanied by a spirit that is anything but conciliatory. There is, I am persuaded, a close connection between the spread of Bolshevism and the exploitation of the young. The hopeless position in which they find themselves, without prospects of any kind, is largely responsible for their uncompromising temper and a certain impatience and ruthlessness that disregards circumstances. It is insufficiently recognized that Bolshevism here is in no small degree a rising of the younger generation against the old. Can we wonder?

While on the one hand big business finds itself threatened by the disaffection of the workers, on the other it is perplexed by the contradictions of its own finance. The faith of financiers has hitherto been placed in reducing the costs of production. It was assumed that any reduction of costs would be automatically followed by an increase of demand. But is this so? Such a policy is no doubt a sound business proposition from the point of view of the individual capitalist who is anxious to find ways and means of increasing his market. But it has obvious limitations when applied generally. To the individual capitalist bent on increasing his market it matters nothing how the costs of production are reduced. But when generally applied it makes all the difference in the world whether such reductions are effected by improved methods of production or by lowering wages. For the latter method, by reducing pur- chasing power, undermines demand. We see therefore that demand does not depend ultimately upon a reduction of costs, but on the distribution of wealth. In so far therefore as big business sets about to centralize wealth it undermines demand. But again, in so far as increased production is necessary to maintain its financial stability, there is necessitated an increased demand. We see then that to seek to centralize wealth and to increase production is to travel in opposite directions at the same time; for while centralization of wealth tends to undermine demand, increased production presupposes increased demand. Can we wonder that a deadlock has overtaken industry? The immediate cause may be the war, but it is clear that the problem is far more fundamental, and that the deadlock would have arrived quite apart from the war. Like political despots our commercial magnates are beginning to find that the successes of despotism exhaust its resources and mortgage its future.

\footnotetext[1]{An introduction to the \emph{Economic History of England}, by E. Lipson, pp. 282-3.

}\footnotetext[2]{See chapter on Limited Liability Companies in my \emph{A Guildman’s Interpretation of History}.

}\chapter{On Investing and Spending}
\label{chapter-4}
Considering the anti-climax in which big business is seen to be ending, the question arises: What is it that has impelled it on such a fatal course?

With the individuals immediately concerned, love of money, power and personal ambition has doubtless been the mainspring of their activity. But it is a mistake to attribute too much to purely personal influences, inasmuch as such men are the instruments rather than the cause of developments. Their freedom of choice can be exercised only within certain well-defined limits. Those limits are determined by the current ideas and practice of finance, to which all their activities must have reference. To understand therefore the cause of the deadlock that is overtaking industry, we must inquire into those principles of finance which are accepted by all who engage in commercial activities.

In this connection it may be held that there is a sense in which it is true to say that the City has been the victim of a false economic philosophy. For though the principles of that philosophy have on the whole followed and justified economic practice rather than directed it, yet there can be little doubt that commercial men would not have embarked upon their latter-day enterprises with such abandon and self-confidence had they not believed that they were supported by the thought of the age. And indeed, apart from Ruskin and his followers, who were comparatively few in number, the thought of the age did on the whole, until a few years before the war, support the City in its doings. Collectivists objected to the proceeds of industry going into the pockets of a few, but they accepted the principles governing City finance. They did not perceive that apart from the way the earnings of industry were distributed there was anything fundamentally wrong in these principles of finance. Yet that there must be something very fundamentally wrong needs no demonstration to-day. Big business is too manifestly breaking down to be able to justify itself any longer.

Ultimately of course what is wrong is the modern philosophy of life, with its worship of wealth–its belief that the acquisition of money precedes the attainment of all other good things in this universe. But to change these values (and they must be changed) is the work of time, and we are unfortunately faced with immediate issues, the legacy of generations of false philosophy. To deal with them it is necessary to know the proximate cause of things, and the proximate cause of the activities of big business is undoubtedly the theory of investments as popularly understood. That theory teaches that money is never so usefully employed as when it is invested in some productive enterprise, and it recognize no limit to the possibility of such investments. Nearly all people with money accept this theory as a truth that is axiomatic, and consider them- selves as doing a positive service to the community when they reinvest any spare money they may have for further increase instead of spending it in some way or other. For, as they are accustomed to say, money so invested provides employment.

This is the philosophy of the rich to-day. If we went back a couple of generations to the old Tory school we should find that they believed it was not the investing but the spending of money that gave employment. Though neither of these conflicting philosophies is ultimately true, the old Tory idea is infinitely nearer the truth than the current one; while as a practical working philosophy for the rich there is simply no comparison between the two. For whereas money spent does return into general circulation, the effect of investing and reinvesting surplus money is in the long run to withdraw it from circulation, much in the same way as if it were hoarded. Nay, it is actually worse than if it were hoarded. Hoarded money may undermine demand, but it does not increase supply, whereas when reinvestment proceeds beyond a certain point it increases supply and undermines demand at the same time.

It is apparent that in a society in which economic conditions were stable a balance between demand and supply would be maintained. The money made by trade would be spent, and in this way it would return into general circulation. Thus a reciprocal relationship would be maintained between demand and supply. In former times money was spent upon such things as architecture, the patronage of arts and letters, the endowment of religion, education, charitable institutions, and such-like ways. Expenditure upon such things stimulated demand and created employment, while it tended to bridge the gulf between rich and poor. The only defence that is ever made for the existence of a wealthy class in society is that but for their expenditure in such ways our great monuments of architecture, educational and other endowments would never have come into existence. I am not quite sure how far this is true. But it is manifest that when the rich did dispose of their wealth in such public ways they were in a position that could be defended on the grounds of expediency if not of equity. But what defence can be put up for the rich to-day who have so completely lost all idea of function as to be unwilling to spend at all except upon themselves; who fail to support charitable institutions; who are so inaccessible to culture as to neglect the patronage of arts and letters; who so lack confidence in their own judgment as to be unable to patronize the crafts of to-day and take refuge in antiques; who are unwilling to spend money upon architecture, nay, who can only be persuaded to buy pictures when assured they are good investments–in a word, who have no idea what to do with their surplus wealth except to reinvest it for further increase, that is to use it for the purpose of undermining the economic system that permits them to live such useless existences. But perhaps they know best!

This is no exaggeration. The misuse of surplus wealth by the rich upsets the balance between demand and supply. And this is productive of waste. For when more money is invested in any industry than is required for its proper conduct, the pressure of competition is increased; for any increase in the pressure of competition means that money that should be spent is invested to increase supply; and this increases the selling costs by encouraging the growth of the number of middlemen who levy toll upon industry, while it increases the expenditure on advertisements and other overhead charges. Thus we see it transfers labour from useful to useless work. Further, it encourages the over-capitalization of industry by burdening industry with a dead load of watered capital. These things react to raise the price of commodities on the one hand and to demoralize production on the other. For in the effort to produce dividends on this watered capital all moral scruples are thrown overboard. Thus we see there is a direct connection between the perpetual reinvestment of surplus money for further increase and the unscrupulous methods of big business. Once an industry has experienced a boom on the Stock Exchange, its doom is sealed. It becomes grossly over-capitalized, and every kind of dirty trick and smart practice is resorted to in the attempt to produce dividends on the watered capital. Attempts are invariably made to squeeze more out of labour. The disaffection of labour to-day is in no small measure the reaction against this kind of thing.

In former times the rich appeared to have some notion that there was such a thing as a limit to the possibilities of compound interest. But after the introduction of machinery the possibilities of making money increased so enormously as to remove from their minds any sense of limitations. In demanding that all money shall bear compound interest, finance is committed to an absolutely impossible principle; as must be apparent to any one who reflects on the famous arithmetical calculation that a halfpenny put out to five per cent, compound interest on the first day of the Christian era would by now amount to more money than the earth could contain. This calculation clearly demonstrates that there is such a thing as a limit to the possibilities of compound interest; yet what we call “sound finance” to-day proceeds upon the assumption that there is no limit. In consequence, it invests and reinvests surplus wealth and loads industry with a burden it cannot bear. For if wages were reduced to the lowest figure capable of keeping body and soul together and prices raised to the highest limit, productive industry could not be made to yield the returns which the conventional system of invested funds now requires. Can we wonder that capitalism is breaking down?

\chapter{On Producing More and Consuming Less}
\label{chapter-5}
The development of foreign trade was a primary cause in leading the rich to abandon their habit of spending their surplus wealth in public ways and to invest and reinvest it for the purpose of further increase. The discovery of America and the sea route to India transferred prosperity from the Hanseatic and other inland towns to seaports and countries with a good seaboard. The change was very profitable to English merchants, who began to secure a larger share of the commerce of the world. Moreover, it stimulated British industries, and the rich began to find increasing opportunities for profitable investment.

These changes were accompanied by certain changes in economic thought. In the seventeenth century there arose the Mercantile school of economists whose central idea was to increase the wealth of the nation by foreign trade; and as a means towards this end they taught that the rule to follow was “to sell more to strangers yearly than we consume of theirs in value.” Translated into the terms of industry this doctrine becomes that of “producing more and consuming less.” By following this advice, money was made, and in the terms of cash men became wealthy. But unfortunately this was not the only consequence, for this policy brought into existence the problem of surplus production. This surplus was in the first instance deliberately created in order to take advantage of the opportunities of making money that the exploitation of distant markets afforded. But after machinery was invented, it became the plague of our society, for surplus goods increased so enormously in volume that it became a matter of life and death with us to find markets in which to dispose of our goods. For, as a consequence of this money-making policy our society has become economically so constituted that we cannot live merely by producing what we need, but must produce all manner of unnecessary things in order that we may have the money to buy the necessary things, of which we produce too little.

But the evil does not end with ourselves. In the long run, this policy defeats its own ends. It is obviously impossible as a world policy because all the nations cannot be increasing their production and decreasing their consumption at the same time. The thing is simply impossible. Hence it came about that once the employment of machinery began to give us an unfair advantage in exchange, one nation after another was drawn into the whirlpool of industrial production. And in proportion as this came about we were driven further and further afield in the search for markets, until a day came at last when there were no new markets left to exploit. When that point was reached competition became fiercer and fiercer, until the breaking-point arrived and war was precipitated.

The crisis came in Germany. Immediately it is to be traced to the Balkan War, which closed the Balkan markets to her, and to the fact that after the Agadir crisis in 1911 the French capitalists withdrew their loans from Germany, and these things combined to bring the German financial system into a state of bankruptcy; for this system, built upon an inverted pyramid of credit, could not for long bear the strain of adverse conditions. But the ultimate reason why the crisis made its appearance first in Germany was undoubtedly due to the fact that more than any other nation she had forced the pace of competition. In the fifteen years before the war Germany had quadrupled her output. The rate of productivity, due to never-slackening energy, technique and scientific development, was far out-stripping the rate of demand, and there was no stopping, for production was no longer controlled by de/nand but by plant. In consequence, a day came when all the world that would take German-made goods was choked to the lips. Economic difficulties appeared, and then the Prussian doctrine of force spread with alarming rapidity. War was decided upon for the purpose of relieving the pressure of competition by forcing goods upon other markets, and to cheapen production by getting control of additional sources of raw material. Hence the demand for colonial expansion, the destruction of the towns and industries of Belgium and Northern France, and the wholesale destruction of shipping by the submarine campaign. They all had one object in view: to relieve the pressure of competition and to get more elbow-room for German industries. The idea of relieving the pressure of competition by such commercial sabotage was not a new one. It had been employed by Rome when she destroyed Carthage and Corinth and the vineyards and olive groves in Gaul out of commercial rivalry. The Germans, who copied the methods of the Romans in so many ways, followed them also here.

Had Germany succeeded in bringing the war to an early conclusion, it is possible that this policy would have been successful to the extent of giving her industries a temporary relief from the pressure of competition. But instead of being terminated in a few months as she had intended, the war dragged on for over four years, and this exhausted the economic resources of Europe. When the Armistice was signed there was a world shortage of the necessaries of life and it became necessary, if Europe was not to disintegrate economically, that efforts should be made to resume at once normal trade relationships. But unfortunately the Big Four into whose hands arrangements for Peace had fallen, were not, as Mr. Keynes has told us, interested in economics. What they were interested in was military guarantees against a renewal of hostilities, territorial questions and indemnities. And so it came about that the realities of the economic situation, the urgency of which permitted no delay, were entirely disregarded. For while on the one hand the Peace terms ignored the fact that the war had left < many in a state of economic exhaustion, and that therefore she could only pay indemnities on the assumption that she experienced a trade revival; on the other hand the huge figures at which the indemnities were fixed, and the continuance of the blockade, by interfering with the operations of normal economic forces, precluded the possibility of any such revival.

Meanwhile, unmindful that the war had been precipitated by the fact that the industrial system had reached its maximum of expansion, the doctrine was preached in this country that salvation was to be found in a policy of maximum production. That the world shortage of the necessaries of life demanded that efforts should be made to make good the deficiency, no one will be found to deny, for in many directions making good the deficiency was a race against time. But the advocates of a policy of maximum production were as little concerned as the Peace Conference with the realities of the economic situation. They were not interested in the increased production of food, or finally in any other form of necessary production, but in finding ways and means of repaying the War Loan without resort to a capital levy. And this is where they went astray. For not only was Labour alienated inasmuch as it saw in this policy an attempt to shift the burden of war taxation on to the shoulders of the producers, but it led its advocates to demand the increased production of everything and anything regardless of the fact that the policy of the Peace Conference both in regard to Russia and to Central Europe was to close their markets to us, and that during the war other nations deprived of their accustomed supplies from us had taken to manufacturing for themselves. The result has been what various writers on economics foresaw–that indiscriminate production was followed by a glut, and the unemployed are on our streets.

To add to the public bewilderment, the cry has gone up of late that the needs of national economy demand that we consume less; and the average man is a little concerned to know what is meant when he is urged to produce more and to consume less. The answer is that, absurd and contradictory as it sounds, it is nevertheless the principle upon which our glorious civilization has been built. It is a principle with four hundred years of history to support it, but at last the limits of industrial expansion necessary to its continuance have been reached. For, as I have said before, our economic system must either be expanding or contracting. And as it so happens that as the age of expansion has come to an end, the age of con- traction naturally follows. The Government, impervious to arguments, has at length had to yield to the force of facts. It has ceased to admonish all and sundry to increase their production, and the word has gone round to reduce production and ration employment, and for each man to work fewer hour; for the opinion in the commercial world to-day is that less production rather than more is the remedy for our present difficulties.

Though there may probably be temporary revivals of trade, the process of contraction now definitely inaugurated will, I am persuaded, continue. For just as hitherto the normal trend of affairs was, in spite of recurring depressions, from expansion to expansion, so now when the tide has turned the normal trend will be from contraction to contraction a tendency that can only be checked by a complete change in the spirit and conduct of industry such as is involved in return to fundamentals. This truth is vaguely apprehended to-day, though at the moment men are at a loss to know how to translate it into the terms of actuality. But now when disillusionment has overtaken society there is a prospect that right reasoning may prevail and a path be found. Let us try to discover it.

\chapter{Fixed Prices versus Speculation}
\label{chapter-6}
We have seen that the existing system of industry and finance is rapidly reaching a deadlock. What is to be done in the circumstances?

The first thing to do is to effect such repairs of the old machine as will enable it to run a little longer in order to gain time to build the new one, which we must have in running order before the existing machine breaks down completely. For such a purpose such measures as the reversal of our Russian policy, the removal of all blockades, and the provision of credits for the renewal of trade with Central Europe are indispensable. It will be unnecessary for me to do more than mention them, as steps towards their fulfilment have already been taken. But it is necessary to insist that though these measures may bring relief by enabling our merchants to dispose of their surplus stocks, yet the relief would only be temporary, inasmuch as if the Continental nations get on their feet again they will begin to compete with us in other markets. If on the contrary they do not recover their industrial position but relapse into more primitive conditions, they will not have sufficient surplus to enable them to buy our manufactures. If these facts were clearly recognized and the necessary measures taken in hand, then we should have nothing to fear. But the danger is that the moment any improvement in trade is felt we shall stop thinking and pursue the silly old game, comforting ourselves with the illusion that the dislocation of trade was due entirely to the war, and that there is nothing organically wrong with the industrial system.

The truth, however, must be faced. We are in an economic cul-de-sac, and there is but one path of escape; and that is to get back somehow to the primary realities of life. It must be recognized that we are up against the consequences of centuries of injustice, usury, and Machiavellianism in politics and business, and that there is finally no escape except to return to the principles af justice, honesty and fair dealing, upon which all civilizations rest.

Reduced to its simplest terms, the change necessary to enable society to escape from the deadlock that is threatening industry is conveniently expressed in the well-known formula:–“the substitution of production for profit by production for use”; and the first step in that direction will be taken when we begin to establish a system of fixed prices throughout industry. For though the Just Price rather than the fixed price is the ideal to be attained, yet it can only be realized by stages. The fixed price therefore is to be regarded as a step towards the Just Price, because the people will never be satisfied with a fixed price that is not a Just Price.

The difference between a fixed price and the Just Price almost explains itself. Fixed prices are those that are uniform and are not determined by competition; but such prices may be anything but just, as many fixed prices during the war were anything but just. A Just Price would bear a certain definite relationship to the cost of production, measured in labour units. It would mean that some things would be sold for more and other things for less than at present. In the case of a few useful and necessary things it might mean that they would be sold for more than at present, because useful labour is invariably underpaid; while it so happens that they are often sold by retail dealers with little or no profit in order to attract customers, and provide opportunities for selling other goods, generally useless and unnecessary things, that carry a handsome profit. It will be necessary therefore, if production for profit is to give way to production for use, to readjust all such selling prices so that the price in each case may correspond to the actual cost of production, since until prices are so adjusted no change in the motive of industry is possible. For with prices determined by competition the producer must think primarily in the terms of profits if he is to remain solvent.

In all kinds of ways the present system of prices is demoralizing. Some years ago when I had some experience of the furniture trade, I made the interesting discovery that it stereotyped the forms of design. It came about this way. Profits were put on certain things and not on others. Certain things in general demand, such as chests of drawers, bureaus, chairs and small tables were sold without profit, while other things such as dining tables, bookcases, sideboards, heavy curtains and carpets carried good profits. Simpler kinds of furniture were sold at cost price and sham ornamental pieces at exorbitant ones. A designer therefore, in the employ of the furnishing houses, could only exercise his fancy within certain narrow limits. The furniture had to be elaborate, and the curtains had to be heavy or there would be no profits. He might know that some other arrangement would be infinitely more effective, but he was not allowed to carry it out, for in that case the public would not be prepared to pay a price that would give a working profit, though to provide such a profit it might only cost half of what the sham elaborate design cost. The public would not think they were getting value for money, and therefore would refuse to buy. This illustration may serve to show how unjust prices strangle creative work. They have strangled the effort to revive design and handicraft; for when conditions obtain which will not allow men to do things in the way they know they should be done, they lose interest in their work and begin to think only of profits.

No doubt many who have had experience of other trades could add their testimony of the peculiar effect unjust prices have had. But in general it may be said that the effect of unjust prices is to transfer labour from useful to useless work, with its corollary that useful work is nearly always badly paid. The result is that men insensibly learn that it is easier and more honourable to live by exploiting labour for profit or by trafficking or by money-lending, or by speculating, or by some parasitic art–by any means in fact except by doing work which is useful and desirable for the purposes of human life. It is thus that occupations have come to be esteemed in proportion as they win money, afford comfort and leisure, and confer individual power and distinction. The effect of it all is to produce social demoralization. It exalts false social values; this in turns corrupts everything else, for it encourages lying and fraud of every kind, and ends by creating an atmosphere of social lies so dense that few people know where they are. Divorced from all useful work they have no final test of truth. In consequence they become dissatisfied, they are at the mercy of every fashion of opinion, and finally like the builders of Babel they end in a confusion of tongues, no man being able to make himself intelligible to any one else.

Thus we see that unfixed and unjust prices divert energy from production to speculation. It will remain impossible for people to be interested in the ultimate social utility of anything they do so long as the price they are to get for their labour is settled by competition. The reason why the commercial motive is for the most part absent among professional men is precisely because the price of their services is fixed; and it will tend to disappear from industry once prices are fixed. The professional man is able to put his best into his work because he has not to worry about how much he has to receive for his services, and it will be the same in industry when the same conditions obtain.

Uncertainty as to price dislocates industry in every direction, and has handed production over to the speculator with consequences that are grossly demoralizing. It is only possible for a man to plan ahead if he knows where he stands. The farmer, for instance, must plan four years ahead. He must arrange for a rotation of crops. If he knows he can dispose of his produce at a certain definite fixed price he can concentrate all his attention upon getting the best out of his land. He can go ahead. But if uncertainty as to price surrounds him on every side he will not produce on such a large scale, for he will need to keep a greater reserve of capital in case of need. Moreover he will have to be for ever thinking about prices, of when and where to sell, and this will prevent him from making the best use of his land. There is no greater illusion than to suppose that the motive of profit stimulates efficiency. Only love of work can do that, and nothing detracts from love of work so much as economic uncertainty. I am convinced that the decline of quality in production is due far less to avarice than to the demoralization that accompanies such uncertainty.

Further, the determination of prices by competition leads inevitably to injustice. In the event of a shortage the producer exploits the consumer; in the event of a surplus the consumer exploits the producer. The producer may be ruined as English farmers were in the years 1879-90. This ruin has reacted to make living increasingly expensive for everybody. It is thus that unfixed prices leads to unrest among the workers by introducing an element of uncertainty into the real value of wages. It leads moreover to disaffection all round. The consumer is indignant when he is exploited by the producer, and the producer when ruined becomes a centre of discontent. On the other hand the producer who has profited by the system hardens his heart towards the rest of the community because he believes that they would have done the same as he has done if they had had the chance. He therefore resents criticism as a personal injustice. It is thus that competition in prices ends in the promotion of class hatred–of enmity between the haves and the have-nots.

Then again unfixed prices lead to economic instability. Before the war, because we were living upon the moral capital of centuries of tradition and stability, the danger inherent in allowing prices to be determined by competition was apparent only to a few, though as a matter of fact economic conditions every year became more unstable. But during the war, when restraining influences were removed, profiteering became rampant and what hitherto had only been apparent to a few was seen by the many. It was seen that no society could endure that allowed prices to be fixed in this way, inasmuch as it could only end by shaking to pieces the economic system itself. Hence it was that, faced with this peril, the Government sought to limit by means of fixed prices the profits that could be made by any manufacturer or middleman. After the war, the Government brought in the Profiteering Act to enable it to continue to exercise the power of fixing maximum prices of articles in general use, which fixing had during the war been done under D.O.R.A. Its operation, however, was limited to six months, and since its expiration there has been a return to the system of competitive prices for such articles, and prices have begun to fall. Some of the controls, however, that deal with the price of food and raw material still remain. They exist independently of this Act.

It can occasion no surprise that measures that interfered so much with the ways of business should be unpopular in the commercial world. The business man has the conviction that what is in his personal interest is necessarily in the interests of the community, since as society lives by commerce he assumes that anything that interferes with the liberty of commerce can be in the interests of nobody. Moreover, to him speculation is the soul of business, and as any extension of fixed prices over industry would put an end to speculation he can see nothing but demoralization overtaking the world when business loses its soul. No doubt he is perfectly honest in believing this. To men who accept business operations at their face value there is no other conclusion. The question to us, however, is whether the world does not want another soul quite different from the one that business affords; whether, if the moral tone of industry is ever to be raised, business as we understand it must not go, nay, whether business itself can carry on much longer at all on its present basis.

But it was not only business men who objected. Socialists and Labour men also objected. But their objection was of a different order, and was due to the fact that, having \emph{a priori} ideas in their minds as to the way the millennium is to be ushered in, they look with suspicion upon any idea that has not hitherto found a place in their programme. To them the fixation of prices was nothing more than a means of satisfying popular clamour and postponing the day of substantial reform, and for this reason they never seriously considered it. Perhaps they may, when they awaken to the fact that it is an idea with potentialities in it little suspected by its promoters.

But there are other and more valid objections to the Government’s policy of fixed prices. Their enforcement was accompanied by vexatious and irritating interferences of all kinds. With this objection I can entirely sympathize. But I would point out that it does not invalidate the principle of fixed prices. What it does invalidate is the instrument that was used for enforcing them. That instrument was the bureaucratic machine–the system of control from without. It is clumsy and irritating, but the Government had no option but to use it, for the Guild the system of control from within–was non-existent. And the Guild, as we shall see later, is the only instrument that can fix prices properly.

Then there is the objection that certain things went off the market as soon as prices were fixed. This again does not invalidate the principle of fixed prices. What it does do is to demonstrate the impossibility of enforcing fixed prices against the will of a trade. Here again the solution is to be found in the institution of Guilds. For a Guild would contain everybody who worked in a trade, not the few people who were in a position to exploit it, and if everybody in a trade had a voice in the matter we may be assured that they would act democratically for the good of all, and not merely in the interests of a few.

Finally there is the objection of the man-in-the-street, to whom fixed prices were popular during the war when they prevented prices going higher, but who turned against them when the control prevented them falling to a lower level. This objection again is valid. But it does not invalidate the principle of fixed prices. What it does invalidate is the fixed price that is not a Just Price; and as the fixed prices during the war were not Just Prices, it was desirable that control should be removed to enable prices to return to the normal.

\chapter{Guilds and the Just Price}
\label{chapter-7}
I concluded the last chapter by pointing out that the man-in-the-street does not object to the principle of a fixed price but to the fixed price that is not a Just Price. As it happens that the Just Price was the central economic idea of the Middle Ages, let us pause to consider what it meant in those days.

The Just Price in the Middle Ages was primarily a moral idea. By that I mean that it owed its establishment to moral rather than to economic considerations. It was the idea that between two persons bent on honest and straightforward dealing it is possible to arrive at something that may be regarded as a Just Price. Indeed, as a matter of fact, when this idea pervades the whole community, as it did at one time in the Middle Ages, conditions are created that make it a comparatively easy matter to translate such a principle into practice; for under such circumstances prices remain more or less stationary, and every article acquires a traditional price. As a moral precept, the idea of the Just Price was maintained by the Church and supported by the words of the Gospel, “Whatsoever that men should do unto you, do ye also unto them.” To buy a thing for less or sell a thing for more than its real value was considered in itself unallowable and unjust, and therefore sinful, though exceptional circumstances might sometimes make it permissible. The institution of buying and selling wares, it was held, was introduced for the common advantage, and this common advantage could only be maintained if there was equal advantage to both parties. Such equality was defeated if the price which one of the parties received was more or less than the article sold was worth.

Under the auspices of the Guilds, the Just Price became a fixed price. Indeed it is true to say that the Guilds were organized to maintain the Just Price. For it is only by relating the Guild regulations to this central idea that they become intelligible. To maintain the Just and Fixed Price, the Guilds had to be privileged bodies having an entire monopoly of their respective trades over the area of a particular town or city; for it was only by the possession of a monopoly that a fixed price could be maintained, as society found to its cost when the Guilds lost their monopolies. Only through the exercise of authority over its individual members could the Guild prevent profiteering in its forms of forestalling, regrating, engrossing and adulteration. Trade abuses of this kind were ruthlessly suppressed in the Middle Ages. For the first offence a member was fined; the most severe penalty was expulsion from the Guild, which meant that a man lost the privilege of following his trade in his native city.

But a Just and Fixed Price cannot be maintained by moral action alone. If prices are to be fixed throughout industry, it can only be done on the assumption that a standard of quality can be upheld. As a standard of quality cannot finally be defined in the terms of law, it is necessary for the maintenance of a standard, to place authority in the hands of craftmasters a consensus of whose opinion constitutes the final court of appeal. In order to ensure a supply of masters it is necessary to train apprentices, to regulate the size of the workshop, the hours of labour, the volume of production, and so forth; for only when attention is given to such matters are workshop conditions created that are favourable to the production of masters. Thus we see that all the regulations–as indeed the whole hierarchy of the Guild–arise out of the primary object of maintaining the Just Price.

The Just and Fixed Price when maintained by the Guilds left no room for the growth of capitalism by the manipulation of exchange currency, for it demanded that money should be restricted to its legitimate use as a medium of exchange. Unconsciously, the Mediaeval Guilds stumbled upon the solution of the problem of currency which had perplexed the lawgivers of Greece and Rome and broke up their civilizations, as in these days it is breaking up ours. The idea is a simple one–so simple in fact that one wonders how ever it came to be overlooked. Currency, or in other words money, is a medium of exchange. The problem is how to restrict it to its legitimate use. So long as it is fairly and honourably used to give value for value; so long in fact as money is used merely as a token for the exchange of goods, then a society will remain economically stable and healthy. But unfortunately such a desideratum does not follow naturally from the unrestricted freedom of exchange, that is by allowing prices to be determined by the higgling of the market; because under such circumstances there is no equality of bargaining power. The merchants and middlemen, because they specialize in market conditions, find themselves in a position to exploit the community by speculating in values. Standing between producers and consumers, they are in a position to levy tribute from each of them. By refusing to buy they can compel producers to sell things to them at less than their real value; while by refusing to sell they can compel consumers to buy things from them at more than their real value; and by pocketing the difference they become rich. The principle remains the same when the merchant becomes a manufacturer, the only difference being that the exploitation becomes then more direct. For whereas as merchant he exploits the producer indirectly by buying the product of his labour at too low a cost, in his capacity as manufacturer he exploits labour direct. All commercial operations partake of this nature. Their aim is always to defeat the ends of fair exchange by manipulating values. By so doing, money is \emph{made} as we say, and the problem of riches and poverty is created. It is a bye-product of this abuse of exchange. For this evil there is only one solution–the solution provided by the Guilds–to fix the price of everything; for when all prices are fixed there is no room left for the speculator. There is nothing to speculate in.

The Guilds in the Middle Ages existed in the towns. But they never came into existence in the rural areas. There were no agricultural Guilds. This was the weak place in the Mediaeval economic armour; for it is obvious that if the Just Price was finally to be maintained at all, it would have to be maintained everywhere, both in town and country. That Guilds were never organized in the rural areas is to be explained immediately by the fact that in the eleventh and twelfth centuries, when Guilds were organized in the towns, currency had not spread into rural areas, for Feudalism existed there and exchange was still carried on by barter. But the ultimate reason is to be found in the fact that the function that the Guilds performed in the regulation of exchange and currency was not understood at the time, while by the time currency had spread into rural areas the validity of the Just Price had come to be challenged by the lawyers, who maintained the right of every man to make the best bargain he could for himself. They found authority for their attitude in the Justinian Code, belief in the infallibility of which had accompanied the revival of Roman law. This challenge undermined the moral sanction on which the Just Price ultimately rested. It had the success of an appeal to a lower motive,, and it was thus that the revival of Roman law introduced into Mediaeval society those very elements of corruption with which it had been associated at Rome. Its ultimate effect has been to reproduce in the modern world those very same economic difficulties and social disorders that paved the way for the break up of the Roman Empire. For Roman law was not, like Mediaeval law, designed to enable good men to live among bad, but to enable rich men to live among poor; and as such it had been designed to bolster up, in the interests of public order rather than for the maintenance of justice, a society that had been rendered corrupt by an unregulated currency.

While the lawyers were blind to the significance of the Just Price, the Church was equally blind to the need of Guild organization for its maintenance. It thought, as many religious people think to-day, that the world can be regenerated by individual moral action alone. It never realized that a high standard of commercial morality can only be maintained if organizations exist to suppress a lower standard. Hence it came about that while the Church inculcated the doctrine of the Just Price in the pulpit, the confessional and the ecclesiastical courts, it never stressed the need of organization; and so the peasants who accepted such teaching found themselves eventually at the mercy of those who followed the teaching of the lawyers. It was thus that the moral sanction of the Just Price lost its hold on the country population and the way was opened for the growth of capitalism and speculation. The moral sanction of the Just Price being undermined, the Guilds found it increasingly difficult to maintain fixed prices. In the sixteenth century the whole system broke down entirely, as a consequence of the suppression of the monasteries, which upset the economic equilibrium of society, producing widespread unemployment, and the wholesale importation of gold from South America, which doubled prices all over Europe.

So ended the Guilds; and it is only recently that we have begun to realize what the world lost with them. For they fulfilled a function of fundamental importance to society, since in maintaining the Just Price they prevented people from speculating in exchange. With their disappearance society lost entire control over its economic arrangements, and the world has been at the mercy of economic forces ever since. It is no exaggeration to say that society will continue to be at their mercy until the Guilds are restored, for by no other means can speculating in exchange be suppressed. The restoration of the Guilds therefore provides the key to the economic problem. The control of prices is a precedent condition of success in any effort to secure economic reform, inasmuch as until prices are fixed it will be impossible to plan or arrange anything that may not be subsequently upset by the fluctuations of the market. It is a necessary preliminary to any securing of the unearned increment for the community, since until prices are fixed it will always be possible for the rich to evade attempts to reduce their wealth by transferring any taxation imposed upon them on to the shoulders of other members of the community.

\chapter{How the Great Change may Come}
\label{chapter-8}
Granted then that it is essential to the solution of the problems confronting us that prices be fixed and the Guilds restored, we must try to answer the question: How is it to be done?

Here, to some extent, we enter the realm of uncertainty. The translation of any idea into the terms of actuality depends upon circumstances, and as it is impossible to foresee with precision what will happen in the future, it is impossible to define exactly every step that must be taken. On the other hand the popularization and acceptance of any idea will, if there is any truth in it, tend to create the circumstances necessary to its practical attainment. To some extent, therefore, salvation is by faith and propaganda.

But it is not entirely a matter of faith. We have certain definite things to go upon. We have unmistakable evidence that “the new social order is developing its embryo within the womb of existing society.” In the Trade Union movement, for instance, we have, to use Mr. Chesterton’s words, “a return to the past by men ignorant of the past, like the subconscious action of some man who has lost his memory.” In which light the proposal to transform the Unions into Guilds is seen to be an effort to give conscious direction to a movement which hitherto has been entirely instinctive. There is, moreover, historical continuity in this idea, inasmuch as the Trade Unions are the legitimate successors of the Mediaeval Guilds; not only because the issues which concerned them could not have arisen but for the defeat of the Guilds, but because they acknowledge in their organizations a corresponding principle of growth. The Unions to-day with their elaborate organizations exercise many of the functions that were formerly performed by the Guilds–such as the regulation of wages and hours of labour, in addition to the more social duty of giving timely help to the sick and unfortunate. Like the Guilds, the Unions have grown from small beginnings until they now control whole trades. Like the Guilds also, they are not political creations, but voluntary organizations that have arisen spontaneously to protect the weaker members of society against the oppression of the powerful. They differ from the Guilds only to the extent that, not being in possession of industry and corresponding privileges, they are unable to accept responsibility for the quality of work done and to regulate prices. But their performance of this latter function cannot be withheld much longer, for the growth of economic instability and uncertainty is exercising such a paralysing influence upon the conduct of industry that the instinct of self-preservation must before long compel a return to the idea of a fixed and Just Price; and, as we saw that it is impossible to maintain fixed prices for more than a few staple articles, by means of bureaucratic control from without, it follows that any general fixation of prices is impossible apart from the co-operation of each trade as a whole. When membership of a trade organization is confined to employers it exhibits the vices of a trust. But when it includes every worker by hand or brain, it will display the virtues of a Guild. For honesty and fair dealing will always find the support of the majority.

But how may this necessity be translated into the terms of practical politics? The success that has followed the organization of Building Guilds in different parts of the country might appear to suggest that the future is to be found by working upon such lines. But it is manifest that there is a limit to such a policy of encroaching control. The organization of Building Guilds was possible because of circumstances peculiar to the building trade, i.e. the housing shortage which provided the immediate opportunity; Labour controlled municipal councils that were in a position to give them work; and the fact that in the building trades the element of fixed capital, so important in other large industries, is unimportant in comparison with the charges connected with each particular job,–materials and labour entailing almost the whole costs of the building industry. These circumstances made the application of the principle of industrial self-government a fairly simple proposition. But it obviously could not be applied to other large industries where immense fixed capital is required, and where the market is not so easily localized.

Considerations of this kind lead me to the conclusion that the Guilds will arrive some other way. Recent developments lead me to suppose that if the change will not be catastrophic it will at any rate be dramatic, inasmuch as it is possible that their organization may be encouraged to stabilize the exchanges. The demand of Labour that the Government should step in and organize trade by barter with other nations, in order to break down the barriers set up by the fluctuations of exchange, is evidence that thought is moving in some such direction, for such a departure would necessitate the organization of Guilds, as the only way of avoiding the evils consequent upon the creation of enormous Government departments to carry through the work. Moreover, except on a Guild basis it will be impossible to guard against abuses. For if trade were organized on a basis of national barter it would be necessary in order to adjust shares of the industries engaged to put some price on the outgoing and incoming goods. If this were left in the hands of Government officials it would become as scandalous as the munition contracts were during the war, for the Government would have to deal with a crowd of profiteers who would think of nothing except how to secure advantages for themselves, and the public outcry and dissatisfaction would be as great as against the profiteers during the war. For this problem there is but one solution, and that is for each trade to be organized on a basis of self-government in order that prices may be fixed and standards of conduct enforced or in other words by the organization of Guilds. Precedent for such development is to be found in the later days of the Roman Empire when the Government, having assumed responsibility for the provision of an adequate food supply, began to delegate functions to organized groups of workers. Our knowledge of what happened in this way is very scanty, but there is sufficient evidence to believe that a time came when efforts were made to balance the centralizing bureaucratic tendency by decentralization as much as possible, and that group organization on something resembling a Guild basis came into existence.

Development upon such lines appears to me to be the probable thing. But. meanwhile our manufacturers, who have taken fright at the prospect of being undersold in the home markets, are pressing the Government to bring in an Anti-Dumping Bill as it is called by those who are opposed to it, or “The Safeguarding of our Industries Bill,” as it is called in Government circles. The demand is evidently the result of panic, since as far as I can ascertain the fear of dumping is largely unsupported by facts. The Continent is not in a position to export goods in vast quantities. On the contrary the depreciation of the exchange that has scared our manufacturers is itself evidence that the Continent is importing more than it is exporting,\footnotemark[1] and the only remedy is for the Continent to produce more. But if we refuse to take their goods it is evident that their exports will cease and therefore our exports to them.

It is to be observed that when things go wrong and people are at a loss to understand the cause, they invariably seek salvation in the adoption of a reversal of policy, whether it has anything to do with the facts or not. Thus because it has so happened that since the war things have not automatically adjusted themselves by Free Trade, they imagine a remedy is to be found in Protection. But what reason is there to think that this proposed remedy would not be worse than the disease? The application of the principle of Free Trade in the past regardless of circumstances may be regrettable. But is a general tariff imposed still more regardless of circumstances likely to produce better results? Free Trade may not contain all the truth some of its advocates claim. But it does contain some truth that is valuable in a period of transition like the present when exchanges have to be built up again.

The true alternative to Free Trade is not Protection, but a system of fixed prices under Guild regulation. To the capitalist demand for Protection the workers should reply that as it is undesirable for the State to grant privileges except to those who are willing to accept corresponding responsibilities, the only terms on which they may ask for Protection is that they are willing to submit to Guild regulation; which means that they must agree to sell their goods, in the home market at any rate, at Fixed and Just Prices, and that they give the workers a status in their respective industries. If such conditions were accepted, the issue between Free Trade and Protection disappears, and here I would observe that the function of a Guild is not to organize industry but to regulate it in the same way that professional societies to-day enforce a discipline among their members. All other issues, such as whether the members of the Guild should be organized in self-governing workshops, or whether they should have small workshops of their own as happened in the Middle Ages, are secondary. They are matters of opinion, preference or experience. But they are not germane to the idea of a Guild as an organization enforcing a certain standard of conduct and efficiency over a whole trade. My own opinion is that under the control of the Guilds, different forms of workshop organization would exist. Men of gregarious instincts would prefer the self-governing workshop, while men of a masterful or solitary disposition would prefer to work alone. But they would all have to abide by the Guild regulations, or suffer expulsion. However, these things’ are largely a matter of opinion. If the Guilds are to arrive dramatically it is manifest that they will have to adapt themselves to the circumstances that exist. If we can secure a return to the principles of honesty and fair dealing, that is all we can expect at the beginning. The rest will follow in due course.

Once the guildization of industry takes place in one country, the example will speedily be followed by others. For the problem is international. All the nations of the earth are having to face the same problem and learn the same lesson at the same time. All are engaged together in the bitter but salutary process of discovering their souls–some as victors, the others as vanquished. They are all getting heartily sick of the economic struggle; while the rich who hitherto have obstructed the path of reform are nowadays on the defensive. We may be assured therefore that whatever vision is coming to ourselves as a result of the breakdown of our civilization, similar visions are coming to others; and may it not be that beneath the class hatreds, beneath the oppositions of the hour, a profound principle of unity is at work, and that our late enemies may, when at last some ray of light breaks, rise simultaneously with ourselves to substitute international co-operation for international strife and competition? With fixed prices throughout industry, economic competition would automatically come to an end, and with it cross distribution would tend to disappear; for no one would be tempted to buy goods at a distance for some temporary advantage of gain. The qualitative ideal of production would tend to replace the quantitative one, for as the reinvestment of surplus wealth would no longer be possible it would no longer go to provide more machinery for more cut-throat competition, but would tend to be spent on the arts, upon building, education, and the other amenities of civilization. Pleasure would be restored to work.

It will have been noticed that I have discussed the coming of the new social order in the terms of Guilds and currency rather than in the terms of property, as customary in Socialist circles. The reason for this is that I am persuaded that to begin with property is to tackle the problem at the wrong end, and the difficulty experienced in translating the labour programme into action is due finally to that fact. At every step in the reconstruction of society it will be necessary to interfere with property, yet all the same the centre of gravity of the economic problem is to be found in currency rather than property; for currency is the vital thing, the thing of movement, it is the active principle in economic development, while property is the passive. It is true that profits that are made by the manipulation of currency sooner or later assume the form of property, but the root mischief is not to be found in property but in unregulated currency. To solve the problem of currency by the institution of the Just Price under a system of Guilds, is to bring order into the economic problem at its active centre. Having solved the problem at its centre, it will be a comparatively easy matter to deal with property, which lies at the circumference. Property owners would be able to offer no more effective resistance to change than hitherto landlordism has been able to offer to the growth of capitalism. By such means the reconstruction of society would proceed upon orderly lines. All it would be necessary to do would be to exert a steady and constant pressure over a decade or so, and society would be transformed completely. But to begin with property is to get things out of their natural order, for it is to proceed from the circumference to the centre, which is contrary to the law of growth. It is to precipitate economic confusion by dragging up society by its roots; and this defeats the ends of revolution by strengthening the hands of the profiteer; for the profiteer thrives on economic confusion. Of what use is it to seek to effect a redistribution of wealth before the profiteer has been got under control? since so long as men are at liberty to manipulate exchange, they will manage somehow to get the wealth of the community into their hands. Thus, we see that the solution of the social problem, as indeed of every other problem in this universe, resolves itself finally into one of order. Take issues in their natural order and everything will straighten itself out beautifully. All the minor details or secondary parts will fall into their proper places. But approach these same issues in a wrong order and confusion results. No subsequent adjustments can remedy the initial error. This principle is universally true. It is as true of writing a book or of designing a building, as of conducting a revolution. The secret of success in each case will be found finally to rest upon the perception of the order in which the various issues should be taken.

\footnotetext[1]{From January 1919 to September 1920 we sold to the war-stricken countries of Europe £658,750,000 and only imported £239,500,000 worth of goods (\emph{Can Our Industrial System Survive?} by J. S. M. Ward, p. 74).

}\chapter{Agriculture and Emigration}
\label{chapter-9}
The corollary of the substitution for international competition of international co-operation is the revival of agriculture, for it implies a return to the idea of communities that are as self-contained as circumstances will allow; and such communities inevitably rest upon agriculture.

In an earlier chapter I showed that the revival of agriculture was necessary alike to the solution of our unemployed problem and to provide us with food now that the days of our industrial supremacy are numbered. But it is necessary also for another reason: to ensure a healthy population. It came as a surprise to most people in this country that recruiting statistics revealed the fact that we had a larger percentage of physical inefficients than any other country at war. But it is not surprising, remembering that no other country in the world has such a large proportion of her population living in crowded towns nor been industrialized for anything like the same length of time. These statistics prove that a town population gradually loses its vitality. In the past this vitality was every generation renewed by a stream of population from the country. In this light a peasantry on the soil is to be regarded as a reservoir from which the towns replenish their stock, and therefore agriculture stands on a different basis to that of any other industry, and its welfare should be protected at all costs. From a mercantile point of view it matters little whether the population be engaged in the production of food or motor-cars. But from a national point of view there is all the difference in the world, since the production of food guarantees a nation’s future while the production of motor-cars does not. Yet when we remember how big business dominates national policy we cannot be surprised that, being, as we saw, heedless of its own future it should be equally heedless of that of the nation. If, therefore, one aspect of the return to fundamentals is a return to the principles of justice, honesty and fair dealing, the other aspect is a return to the land; to a life lived in closer contact with the elemental forces of nature.

When one thinks of the revival of agriculture, or the colonization of England as some would prefer to call it, the first obstacle one feels to be in its path is the great discrepancy between the earnings of the town and of the country workers. The first step towards reform therefore demands that the wages of agricultural workers be raised. The recently formed unions of agricultural workers are doing invaluable work in this direction. But they are meeting and will continue to meet with the resistance of the farmers to their demands, and it is doubtful whether the farm labourers will ever be in a sufficiently strong position to force the hands of the farmers very much. For there is only one time in the year when they could strike with advantage to themselves, and that is at harvest time. Even then their success would be doubtful. The resistance of the farmers to such demands we are apt to ascribe to a grasping nature. But I doubt whether it can entirely be attributed to that. The uncertainties of farming due to natural causes are increased tenfold by the effects of speculation, and the returns of one harvest may be swept away by the manipulation of prices in distant financial centres. So the farmer’s one idea is to build up a reserve against a possible change of fortune, and this constant preoccupation tends to develop in time a mean and grasping disposition.\footnotemark[1] But the difficulty could be got over by a changed attitude towards questions relating to agriculture. Prices must continue to be guaranteed by the Government, and there must be no question of a return to the old system of competitive prices, as there would be no question if the implications were understood. In consideration for such guaranteed prices, the farmers should agree to raise the wages of their labourers.

At first sight this suggestion seems to be rather Utopian ; and no doubt it will so remain as long as the Coalition remains in power. But would it be different if a Labour Government held the reins? I am not sure. For while the Coalition would doubtless refuse to act in this way out of fear of Labour becoming too powerful, a Labour Government might find difficulties owing to jealousies in the ranks of Labour. The Labour Party is responsible to its own supporters, and as it consists mainly of town workers it is possible that it would object to such action on the grounds that it was creating a privileged class of workers. This is not entirely a matter of imagination. The Building Guilds have found themselves up against this kind of thing. Propagandists on their behalf have found opposition to them from other classes of workers who fear that the workers in the building trade may become a privileged class. This is one of the weaknesses of the Socialist appeal. For whereas the doctrine of Socialism is espoused by many (I hope the majority) from altruistic motives, it has nevertheless secured the support of large bodies of workers by its appeal to their immediate self-interest; and self-interest is apt to be short-sighted. In this case it has led the workers to demand an impossibility–that any advantages that accrue to Labour shall accrue to all at one and the same time.

Perhaps the most practical way of meeting this difficulty is to get into the minds of the workers some idea of the structure of society, and the need of drastic reconstruction. Very few of them to-day have any idea that society needs to be reconstructed. They are of course familiar with the word, but not, I fear, with the idea. The evils of society, they have been told, are incident to capitalism, and they imagine that once a Labour Government is returned to power capitalism will be abolished and we shall live happily ever afterwards. Beyond that they do not go. They have no idea that society is crumbling to pieces, and needs to be rebuilt in the same way that a house that is falling down will need to be rebuilt, and that in rebuilding society it is no more possible for the benefits that are to be conferred upon Labour to be conferred simultaneously upon all, than it is possible in building a house for all the bricks to be laid at one and the same time. Nor can they have any idea until differentiation is made between primary and secondary production, and they come to realize that in any rebuilding of society it is necessary to deal with primary activities before it is possible to deal with secondary ones.

The idea needs to be popularized that agri- culture is fundamental; that it forms the base of the pyramid of production; and that as it has been allowed to decline in this country the reconstruction of agriculture must take precedence over all other industries. The revival of agriculture is immediately important, because it would absorb so many of our unemployed; a thing that is so obvious that one wonders how it is that in the present crisis the two are never connected by our leaders. But the case is even stronger still in the long run, since, if we neglect to revive agriculture, it is a certainty that in a few years’ time we shall be left without food; for, as I have already pointed out, the countries that supplied us with food are taking to manufactures, so they will not require our goods. Therefore, as we shall have nothing to give them in exchange for food, we must take to growing our own.

But there are deeper reasons than these of mere expediency why agriculture should be revived. If the economic problem is to be handled successfully we must be as self-supporting as possible. It is simply impossible to initiate drastic reform so long as our industries are dependent upon foreign markets, for in this case the factors governing the problem are outside of our control. The modification of a tariff or a war, the discovery of some new raw material or some other such event in some remote corner of the globe, dislocates the labour of those at home, while all the time our fortunes remain in the hands of capitalist adventurers. Under a system of international markets the workers become parasitic upon the capitalist, because he alone can find outlets for goods. Indeed, so long as industry is dependent upon foreign markets, production will be very much of a gamble. It will depend upon speculation, and this is incompatible with the reconstruction of society. But if agriculture were revived, a large home market would become available. If the agricultural worker were paid as he should be paid, it would react to the benefit of the town workers by relieving the pressure of competition in the towns. We should soon find that a prosperous peasantry was our greatest economic asset. The raising of the wages of the agricultural workers would, moreover by putting the labourers on their feet again, pave the way for the organization of Agricultural Guilds. Such Guilds would regulate prices, be centres of mutual aid, buy and sell and do the other work undertaken by agri- cultural organization societies. They should, moreover, administer the land ; and in this connection I would suggest that the land should be owned as well as administered by the local Guilds. This suggestion is offered as an alternative to nationalization, in order to avoid the evils of bureaucracy.

The Agricultural Guilds would be mixed or undifferentiated organizations. They might be likened to the Guilds Merchants of the Middle Ages to the extent that the village carpenters, smiths and other isolated workers would be included in them, and also that the functions of the parish councils would be merged in them in the same way that the Guilds merchant and municipalities were identical ; or they might be likened to the village communes of pre-feudal days, differing from them to the extent that whereas the village communes exchanged by barter, these Agriculture Guilds would regulate currency by means of fixed prices. It may also be assumed that the strip system would not be reverted to. As to whether land is best cultivated on large or small holdings I am not prepared to dogmatize, as opinion of those with practical experience is so divided. But, after all, it is a secondary matter, and one that may well be left for the Guilds themselves to decide. It would be an important issue if our ideal were one of peasant proprietors, but not if the country is to be colonized by groups of workers organized into Guilds.

Colonization by groups is also the key to the problem of emigration. What is so distasteful to most people in connection with emigration to-day is the isolated feeling of the man who emigrates alone; for not only is he separated from his friends, but he is left entirely to his own initiative; and town-bred people naturally hesitate from venturing upon a career so full of hazards. Such men when they do emigrate rarely settle down in the land of their adoption. They cherish the hope of making a pile and returning home. It is this spirit that has corrupted colonial life, and has brought into existence in our colonies in less than a century social problems as bad if not worse than ours which have taken centuries to develop. This was not the case with the early emigrants who settled in America and elsewhere. They founded societies which were comparatively stable; and not the least of the things that enabled them to found such societies was that some of the old Mediaeval communal spirit survived among them; and so they emigrated in groups, a custom that has survived among Italians and Eastern Europeans to this day. And this fact makes all the difference. For when men and women emigrate in groups they are held together by personal and human ties, and can render each other mutual aid and support. In consequence they settle down in a way that emigrants who go individually never do. We were successful in the past as a colonizing power because this communal spirit obtained ; colonization has become impossible with us now because of the individualism that is rampant. For this individualism has built up trusts and syndicates and other monopolies that suck the life-blood out of the emigrant

\footnotetext[1]{Since this was written a fall of prices of wheat is reported.

}\chapter{Machinery and Unemployment}
\label{chapter-10}
The revival of agriculture raises the question of the employment of machinery, and this in turn raises so many other questions that it is necessary to pause and consider them.

At the moment, I am not concerned to discuss whether, considered in the abstract, machinery is or is not a desirable thing, since making no claims to be anything more than a means to an end, it can be demonstrated to be either good or bad, according to the philosophy we hold. It is hopeless, therefore, to attempt to secure acceptance of any conclusion regarding its ultimate use until some unanimity of opinion is first established in the realms of philosophy and belief. But meanwhile we are confronted with a very practical question about which we must make up our minds: What is to be our primary aim in reviving agriculture? Is it to provide us with food, or to find work for the unemployed, or what? The question is a pertinent one, because the defence of unregulated machinery has hitherto rested on the belief that in the long run it would emancipate mankind from the curse of Adam by reducing labour to a minimum and thus set us free to pursue the higher ends of life. As to whether most of those who may claim to have been so liberated show any disposition to follow higher pursuits, I do not for the moment inquire, though the “Gentleman with a Duster” has a different tale to tell. But it may be observed that nowadays when this prophecy that machinery is destined to liberate mankind from the necessity of toil shows some signs of being fulfilled, when for the moment we have produced enough and to spare, we are panic-stricken at the army of unemployed in our streets and worry ourselves to death to find them some work to do. The situation reminds us of an ancient Hindu story of a man who went to a great yogi for a formula to raise the devil. The yogi was quite willing to oblige him, but warned him before doing so that once the devil was raised up he must be kept in employment or he would turn and devour him. The man, however, was not to be intimidated, so he took the formula and raised the devil by his incantations; he had plenty of work, and managed for a long time to keep the devil fully occupied. But a time came when work began to run out, and he lived in terror of his destruction at the hands of the unemployed monster. In desperation he went back to the yogi to seek advice. “Well,” said the yogi, “I told you what to expect. But do not despair. Take this dog to your devil and ask him to straighten its tail. That will keep him busy for ever.” Even so is it with our industrial system, not leisure but terror is at the end of its story. We must find it work to do, and the only work we can find is about as utilitarian as straightening the dog’s tail.

Now the reason why we act in such illogical, contradictory ways towards machinery is because in proportion as it tends to become automatic it raises questions which nobody can answer. If machinery is to reduce labour to a minimum, then it follows that some other method of payment must be instituted from the one customary to-day. For payment to-day is for work done, and that is no use if work is to be abolished. The necessity of making some such fundamental change is at the back of the minds of some of those who devise credit schemes and advocate consumer’s credits which are to be distributed independent of work done. None of these schemes will bear looking into. But they do face up to a difficulty that the modern world prefers to ignore: how people are to be paid in the machine society. Perhaps an effort to find a solution of this problem is the clue to Marx, and is what he was really after when he advocated the abolition of the wage system. Anyway, he makes his whole theory of social evolution dependent upon the development of machinery. He saw clearly that the machine era would end in the dilemma that faces us to-day, inasmuch as the end of machine production was to be the creation of an unemployed problem that could not be solved by the time-honoured methods. The solution he proposed was of the nature of a leap in the dark. The unemployed were to rise, take possession of the means of production and exchange, and at the end of it was his structureless Communist state. That was as far as he saw. His thinking came to an end at the point where the real difficulties begin, and viewing the problem to-day at closer quarters it is not quite as simple as it appeared to Marx. We can see the unemployed problem but not the Communist state arising out of it, though we do see the possibility of revolution.

For my own part, I do not believe that there is a solution of this problem on modernist lines. It seems to me that the tradition of payment for work done is so deep-rooted that men will continue to think and act in such terms in spite of the fact that this tradition is challenged by the use of machinery. The difficulty is to think in any other terms, and I feel it to be a part of wisdom to accept the present method of distributing purchasing power by means of payment for work done as irrevocable, whatever the implications may be. For of the choices before us, either the abolition of such a means of distributing purchasing power or limiting the use of machinery, the latter seems to me simplicity itself compared with the former, for I can think in the terms of the latter but not of the former, and neither, apparently, can any one else. Those who try go mad.

Assuming, then, that the present method of distributing purchasing power is to persist, it follows that our aim should be to regulate machinery in such a way as not to dislocate our method of payment for work done, and in this connection it is to be observed that the opportunity for reducing such a principle to practice presents itself in connection with the revival of agriculture. For agriculture is fundamental, and we could build on its base a new society that would gradually replace our present one. In this sense we get a new start, and it is up to us to make up our minds now. The question arises in connection with the use of agricultural machinery. To what extent is it to be used? During the war it was everywhere encouraged because it was a matter of urgency to produce food. But if agriculture is to be revived now it will have the further object of providing work for the unemployed. Let us therefore face the fact that the more machinery we employ the less work there will be for the unemployed. Hence, if our primary aim be to provide work for the unemployed, the less machinery we use the better. On the contrary, if we decide that it is foolishness not to use machinery, let us be clear what we are going to do with the unemployed. To my way of thinking, there is only one sane thing to do if machinery is to be used, and that is to employ the same number of men as would be required if no machinery were used, and to reduce the number of hours worked as more machinery is used. For I insist that in all such questions consideration of the human factor should come first. I believe the ultimate cause of our confusion is to be found in the fact that it is our custom to put it last, and to assume that the right thing to do is to put other considerations–financial or mechanical–first, leaving the human factor to take care of itself as best it can. Let us hope that the magnitude of the present unemployed problem which refuses to solve itself will be the precursor of better things by forcing upon us the necessity of giving human considerations the first place.

One of the advantages of the solution I suggest is that while machinery would be used, we should not be at its mercy, for the economic system would be independent of it. Supposing, for instance, a day came when owing to the shortage of petrol the use of tractors had to be abandoned, the economic system would not break down, for it would be constructed with a margin of safety. All it would mean would be that those employed upon the land would have to work longer hours. If machinery were employed in this way it would do the things it professes to do. It would reduce drudgery, and it would give more leisure, and it is possible that craft developments might follow, for there is no reason why home crafts should not be joined up to the pursuit of agriculture in the future as in the past, when the long winter evenings were employed in this way. Any additional leisure that the use of machinery would give might be so employed, though as a man’s living would be secured by his agricultural earnings, it would be optional. The objection to the use of machinery would fall to the ground if its actual use corresponded to its theoretical justification. But what hitherto has made all discussion on the subject so hopeless, is that while in practice machinery is used for one purpose, it is theoretically justified for another, while belief in its benevolence was so confident and absolute that it seemed to matter little to people what motive prompted its use so long as it was used. We must break with this sloppy-minded attitude towards machinery, and learn to reason about it as we reason about other things, for it is a certainty we shall never be able to control it until we think intelligently about it.

But there are other and more fundamental questions connected with the use of machinery that must not be lost sight of. One of these is the exhaustion of natural resources which follows its unregulated use. Mention has been made of the petrol shortage. It is estimated that the supply in America will only last another twenty-five years. We are engaged in a war in Mesopotamia to secure another source of supply. America has designs upon Mexico for the same object. Borings are being made to discover a source of supply in this country ; no doubt there are others. But some day or other there will be no more, and it is sheer folly, to say the least, to commit ourselves to methods of production and transport that depend upon supplies that are limited. For under such circumstances our position will become desperate as natural resources tend to become exhausted. To reduce the position to its lowest denomination in the terms of cash, the cost of the wars in which we shall be involved in order to get possession of new sources of supply ought to be counted against any savings that are made in other directions, and what is true of petrol is true of other materials. Industrial production uses up all materials at such an alarming rate that some day we shall be left hard and dry if the matter is not taken in hand. It would be well for us to be forewarned in time, and now that the opportunity of a fresh start presents itself, some reason should be brought to bear on the question. If we lay it down as a maxim that the first principle of a normal civilization is that it should be as self-contained as possible, the second is that it should in no sense be living upon capital, but should arrange its production in such a way that it should largely reproduce itself. This is not entirely possible, for we must make use of mineral wealth to some extent. But wisdom suggests that our resources should be conserved and not wasted in the reckless, spend-thrift way we are accustomed to do. Our newspapers are full of indignation against the waste by Government Departments, but scarcely a word is ever said of the thousand times more serious waste of natural resources, though one would have thought that the paper shortage should have made them think.

The modern problem is so elusive that it is generally difficult to prove a certain tendency to be evil or dangerous. We may, however, test the truth of many tendencies by their bearing upon agriculture, and here it is to be observed that the tendency of all modern developments is to rob agriculture of its manures. Human and animal manures are natural fertilizers. In this respect hitherto there existed a reciprocal relationship between man and nature. Food consumed was returned to the earth as manures. But when the water-carriage system of sewage came along with its superior convenience, which is undeniable, this chain of reciprocity was broken, and resource was had to chemical manures, and the guano deposits of South America. Attention was called to this problem twenty-five years ago by Dr. Vivien Poore, who wrote a book on the subject, \emph{Rural Hygiene}, the object of which was to prevent the spread of the water-carriage system of sewage into rural areas. He showed how the water-carriage of sewage produced the typhoid fever germ ; that our sanitary measures were designed to protect ourselves against this germ; how introduced into rural areas it poisoned water supplies and necessitated enormous expenditure on schemes to get water from distant and unpolluted sources ; that the manurial value of the human excrement was destroyed by water-carriage, and that chemical manures were no substitute for the natural organic manures. But the warning was ignored, the water-carriage system was convenient, and the manufacture of sanitary goods was a vested interest, and so nothing was done. One more problem was added for posterity to solve. Since the development of motor transport the problem is aggravated, for it robs agriculture of horse manures. When mention is made of these things, the reply we get is, that in the future it will be possible to extract the nitrogen from the air. Whether or not this is a really practical proposition, or whether the nitrogen will remain eternally in the air I do not know, but it illustrates the thoughtless way in which we are content to go on. We study only our immediate convenience, create enormous problems, and trust to sheer chance to getting out of them. The many and wonderful discoveries of science have apparently reacted to create this spirit. It prevents us from exercising any forethought in respect to things of a fundamental nature by confirming us in the belief that something is bound to turn up. Our national life is lived after the manner of a spendthrift who is prepared to squander one fortune on the chance of another being left to him.

Again, this thoughtlessness is encouraged by the complexity and pace of modern life–a consequence of the misapplication of machinery–which militates against reflection and clear thinking of all kinds. Nowadays there is no time for anything, the complexity of our society bewilders people. No one can deny these things, yet if it is true that the development of speed and complexity beyond a certain point is evil, then we have a clear case for the regulation of machinery. But here we run up against the prejudices of the thoughtful just as much as the thoughtless. Let us examine them.

Take first the economists. They will deny \emph{in toto} the existence of a machine problem, affirming that the evils that have accompanied the use of machinery are due entirely to the peculiar economic conditions which existed at the time of its introduction, or in other words, that machinery has been misapplied because it arrived at a time when the accepted social gospel was that of economic individualism, from which it follows that what we have to do is to substitute some form of economic co-operation for economic individualism when the machine problem would automatically solve itself. But is such reasoning valid? If it be true, as I am willing to admit, that the economic problem preceded the machine problem, and is therefore more fundamental, it is equally true that morals are more fundamental than economics. If, therefore, our practical activity is to be related only to those things that are fundamental, then it must be based upon morals and not upon economics. Economists can’t have it both ways. Either we base our activities upon ultimate truth, in which case we abandon economics and pursue morals, or we base them upon a series of proximate truths, in which case the problem of machinery takes its place alongside that of economics. We may agree that the substitution of economic co-operation for economic individualism must precede the control of machinery, but such co-operation would not ensure its control in the face of the popular prejudice in favour of its unrestricted use.

But economists are not the only people with prejudices on this question. There are the moralists who affirm that there is no such thing as a machine problem, inasmuch as machinery is non-moral, and its application will, therefore, be good or bad according to the motive that inspires its use. The weakness of this argument is that it assumes that the intelligence of the user corresponds always with his moral intention. We know that in other departments of activity this does not by any means follow, and that a man’s motive may be good and his actions bad, or \emph{vice versa}. The case, therefore, for regulating machinery rests finally on precisely the same grounds as any other kind of regulation. Firstly, to restrain those whose motives are bad from injuring society by their actions, and secondly, to prevent those who with the best of motives do through ignorance things which in their ultimate effects are harmful.

But how may machinery be regulated? What kind of regulation is needed? The answer is that our attack must not be directed primarily at machinery, but at the system of the division of labour that lies behind it. For that system was the great factor in the destruction of the creative impulse in industry, and we may be sure it will not reappear until it is destroyed. Moreover, the abolition of the division of labour cuts at the very base of the quantitative ideal of production, which is immediately responsible for the misapplication of machinery. If we keep in mind the central idea–the general principle that machinery needs to be subordinated to man–I think we shall find that, generally speaking, the issue is one between large and small machines. We should forbid large machines in production on the principle that large machinery tends to enslave man because he must sacrifice himself mentally and morally to keep it in commission, whereas the use of small machines has not this effect, because they can be turned on and off at will, as, for instance, is the case with a sewing machine. Exceptions would have to be made to this rule, as in the case of pumping and lifting machinery where no question of keeping it in commission necessarily enters. The difficulty of deciding whether a machine was or was not harmful would not be difficult to determine once the general principle were admitted that machinery needs to be subordinated to man.

\chapter{On Morals and Economics}
\label{chapter-11}
I concluded the last chapter by answering objections of doctrinaire economists and moralists to the regulation of machinery. As it is not improbable that they will object to my general position, it is necessary for me to anticipate their attacks. The economists will object to the conception of the Just Price because it involves moral considerations, and they demand an economic solution of the problem of society that is independent of morals; the moralists, on the contrary, will maintain that my policy is impracticable inasmuch as it presupposes a moral revolution to make it effective.

Respecting the economic objection, I deny \emph{in toto} that there is any such thing as a purely economic solution of our problems, because I do not believe that there is such a thing as a fool-proof society. The search for it is as hopeless as the search for perpetual motion, and it has been at the bottom of all the confusion in Socialist economics and policy in the past. It is responsible, too, for the gulf that separates formal Socialist theory from its informal philosophy. Of course it is to be admitted that there are certain things in economics in which morals play no part. Such factors in the economic situation as the inequalities of nature, the fluctuations due to a good or bad harvest, have nothing to do with morals. But such things do not impugn the fact that economics in the larger sense presupposes certain moral assumptions any more than because a man’s life is determined partly by accidental circumstances it is to be explained apart from morals. This heresy goes back to Ricardo. Before he wrote economists always based their reasoning upon certain moral assumptions. They were either like the Mediaeval economists concerned to understand how economic managements could be brought into relation with the highest morality, or like Adam Smith they postulated human selfishness as the motive force of economic activity. But what they never thought of doing was to affirm the existence of economics apart from morals. In the hands of Marx this heresy received a new development. He turned the tables completely, inasmuch as he made morals dependent upon economics, and this way of reasoning has persisted among the more doctrinaire elements in the Socialist movement ever since, in spite of the fact that Ruskin nearly fifty years ago exposed the fallacy in the first few pages of \emph{Unto this Last}. The reason for the survival of this fallacy is perhaps to be found in the fact that at the present time it seems impossible to interfere with the course of economic development by action of a purely moral order. But this is to be misled by appearance, for moral action only influences economic development over a long period of time, the moral action of one generation determining the economic environment of the next. It is this that leads so many people to suppose that economics and morals exist apart.

I have more sympathy with the objection that will be raised by the moralists, because they happen to be right in theory, but mistaken as to the actual facts. It is not true to say that the maintenance of the Just Price presupposes a higher moral development than that which exists to-day. For this again is to be misled by appearances. The popular outcry during the war against profiteering demonstrates that moral standards still exist among the people, while the success of the Socialist movement as I explained in the first chapter is due to the fact that it has some qualities of a moral revolt. It is true that Socialists have been primarily concerned with the popularization of certain economic doctrines, but in order to obtain a hearing for them they have been obliged to attack the ideal of wealth. It has thus come about as a consequence of these attacks, repeated from one end of the country to the other during this last thirty years, that a changed moral attitude towards wealth has come into existence, and has influenced large bodies of people entirely unaffected by Socialist theories. I think it is no exaggeration to say that in this direction the Socialist movement has accomplished a moral revolution comparable only to that effected by the Early Christians, who attacked wealth as vigorously as any Socialist, though perhaps with a different object. Recognizing this, I feel it ill becomes moralists who have rarely attacked wealth at all to talk about the need of a moral revolution before the Just Price and other measures could be established. They should be told that the moral revolution in this direction is an accomplished fact.

The awakening of the public conscience in regard to collective morality should not be overlooked because simultaneously with it personal morality has suffered a decline. To some extent that decline is doubtless due to certain Socialist teachers, though not to the influence of the movement as a whole. To a far greater extent it is to be attributed to the economic pressure under which most people live in these days. Sexual morality is not improved by the fact that such a large proportion of the rising generation find it difficult to marry, nor does overcrowding and the housing shortage improve matters. While again the commercial morality imposed by large concerns upon those they employ gets steadily lower and lower. But what men do under duress scarcely affects their character at all. Father Dolling, after years of intimate experience of the Portsmouth underworld, an environment of almost inevitable vice and crime, came to the conclusion that its inhabitants were comparatively spiritually innocent. “Our falls in Portsmouth,” he says, “entailed no complete destruction of character, hardly any disfigurement at all. Boys stole, because stealing seemed to them the only method of living–girls sinned–unconscious of any shame in it, regarding it as a necessary circumstance of life if they were to live at all. The soul unquickened, the body alone is depraved, and, therefore, the highest part is still capable of the most beautiful development.”\footnotemark[1] It is the same in the commercial world. Men pursue immoral methods in business, because it seems to them the only way of living, and remain unconscious of any sin in the matter. When they do become conscious they revolt. The Socialist movement draws its recruits from among those who are in moral revolt, and that is why I am persuaded it is only finally to be understood as a moral revival. Socialists talk about changing the system, not because they are indifferent to morality, but because they realize the impossibility of acting on moral precepts amid such adverse circumstances. Such being the motive force behind the demand for a change of the system, there is no reason to doubt that the moral effort necessary to the enforcement of the Just Price will be forthcoming. Nay, it can be said that nothing less than a desire to enforce such principles of honesty and fair dealing could suffice to bring the Guilds into existence.

What may be doubted, it seems to me, is not whether the moral effort necessary to the maintenance of the Just Price under a system of Guilds would, in the event of their establishment, be forthcoming, but whether our attachment to city life may not stand in the way of a revival of agriculture until it is too late. The town worker has become accustomed to a life of bustle, crowded streets, trams, railways, cinemas, etc., that makes him restless under simpler conditions. He has, moreover, become dependent upon a complicated machine. That machine allots to him a single specialized task and supplies his other needs. In consequence he has lost the habit of doing things for himself, and depends more and more upon buying what he needs, and this has undermined those qualities of resourcefulness, forethought and patience that are the necessary accompaniment of an agricultural life. To break with this tradition before the system comes to grief is the real obstacle in our path, for everything combines against us. We have not only to contend with the inertia common to all reform, but with the rooted habits of city populations. And it may be that just as it was impossible to make people realize the possibility of a European War before it was upon us, so it will be impossible to induce people to realize our industrial position until starvation threatens us if more food is not produced. If this is the case we shall drift and drift until the only way of meeting the situation will be some form of agricultural conscription, and our countryside will be dotted with tents and huts to house workers engaged in a desperate effort to cope with the problem of food. That is what things must come to if we do not wake up before long. Meanwhile it is the plain duty of publicists of all kinds to bring home to the people the realities of the situation–to make them face the facts, and in the short space of time allotted to us to assist in the cultivation by every means at our disposal of such habits of industry and self-reliance as will enable the people to change their ways of life with the minimum of dislocation.

\footnotetext[1]{\emph{Ten Years in a Portsmouth Slum}, by Father Dolling.

}\chapter{Industrialism and Credit}
\label{chapter-12}
IT has been my aim in this little volume to concentrate attention on certain issues that I feel to be primary and fundamental. There are other things that must be done, such as the liquidation of the war loan, but on this issue I have nothing to add to what has already been said by others. The expenditure of surplus wealth by the rich in the ways it used to be spent would also do much to mitigate the evils of unemployment. But no one at the present time can act in these matters except the rich, and as it seems impossible to persuade them to do anything that ultimately matters, it is just as well to base our politics upon the assumption that they will continue as at present–hoping against hope and doing nothing.

The reason why we should concentrate our attention upon the things that are fundamental is precisely because they have been for so long neglected, while their importance to us is proportionate to their neglect. They were neglected because, as I have said before, they are antipathetic to industrialism, and industrialism appeared before the war to be built upon a rock, and few believed it carried within itself the seeds of its own destruction. On the contrary, the world was cursed with an easy-going belief that though evils undoubtedly existed they were merely incidental to the system, and could be remedied by half- baked measures of reform. But that serene self-confidence nowadays is gone, and I can write in a different way. It is no longer necessary for me to plead for recognition of the fact that internal evidence pointed to the break-up of our industrial civilization, since nowadays we can see it dissolving before our very eyes–the only evidence which the modern world is prepared to believe. In these circumstances self-preservation suggests that delay is dangerous, and that it is necessary to get to work at once to build the new society before the existing one breaks down completely.

Meanwhile, those who still retain any belief in the possibility of saving existing society from disruption are concentrating on the problems of credit. With the effort of those who are attempting to overcome the barriers to a revival of foreign trade set up by the depreciation of the exchanges, and with those who would break the monopoly of the banks, I have every sympathy. But with those who imagine that the problems of credit can be cured by some Morrison’s Pill it is different. In my opinion they are living in a world of illusions. One of these pill schemes, that formulated by Major C. H. Douglas, and for which the \emph{New Age} has stood sponsor, demands special consideration, since, as the editor of that journal acted as sponsor for National Guilds, it has come to be discussed as if it were an approach to the Guilds.

It will not be necessary for us to consider this scheme in all its details. It will be sufficient for us to discuss its central idea. And here I would observe that though I cannot accept Major Douglas’s scheme, I recognize that he has attacked a real problem, though I may add that to some of us it is not a new one. Briefly, then, Major Douglas faces the fact that the policy of Maximum Production inevitably results in a deadlock, upsetting the balance between demand and supply, but instead of tracing it to the causes I have enumerated in Chapters 3, 4 and 5–that is ultimately to certain moral causes–he ignores morals entirely, and traces the phenomenon entirely to its immediate cause in our system of credit. This leads him to seek a solution of the problem in the terms of accountancy. He proposes to correct the discrepancy between demand and supply by selling goods below cost. There is, of course, nothing new in selling below cost. Manufacturers have resorted to it in every financial crisis when they have overproduced, to get rid of their surplus stocks. What is new is this: Appreciating the fact that the present financial crisis is no ordinary one that will pass by the normal operations of supply and demand, he exalts a practice that has hitherto been resorted to as a measure of temporary expediency into a permanent principle of finance. For according to Major Douglas, the selling price of articles must always be below cost, while he proposes that the difference between the actual costs of production and the selling price shall be made up to the producers by the Government in treasury notes. That is the gist of the scheme. It is the only idea we need discuss, as the others are merely accessory to it.

Now, the first and most obvious objection to this scheme is that such a wholesale issue of paper money would depreciate the currency. Major Douglas proposes to guard against this by the fixation of prices. To which I answer that if this measure were to be effective, prices would have to be fixed simultaneously for all commodities in all industries, since if the scheme were applied gradually and prices fixed below cost in one industry and not in the others the prices of commodities that were unfixed would rise to restore the balance. But to fix prices simultaneously in all industries is impossible, for in these days of international markets the unit to be considered is not this country, but the world. Under such circumstances the proposition is unthinkable. It is the \emph{reductio ad absurdum} of our economic system. I have advocated fixed prices (but not selling below cost), but I recognize clearly that a system of fixed prices could only be introduced gradually, and it seems to me that any scheme to be practical must be based upon that assumption.

The truth is Major Douglas has confused cause and effect. He sees that the operations of industry to-day are governed by the credit facilities in the control of the banks, and so he concludes that the whole problem is that of credit–or if that is not entirely true, it is true to say that he thinks the problem of credit is capable of a separate and detached solution. It will clear the air to say that the problem of credit is not the central but the last phase of the disease It is the dilemma in which a civilization based upon usury finds itself at the finish. It makes its appearance because the limit of usury has been reached. And it is because of this that the problem is not to be resolved finally in the terms of accountancy, but of morals. For centuries the desire for profits has been the driving force in industry. It has been behind our indus- trial developments and brought into existence our vast complicated civilization. Nowadays the limits of this development have been reached because the limits of compound interest have been reached, and the centralizing process is complete. Recognizing the fundamental nature of this problem, it is vain to suppose that a solution can be found for this misdirection of activities merely by a re-shuffling of the cards, which is what Major Douglas’s scheme amounts to. On the contrary, the only thing that can lift us out of the economic morass into which we have fallen is finally the discovery of a new principle, the emergence of a new motive, a new driving force. The experience of history teaches us that there is finally only one power in this universe capable of supplying this need and successfully challenging this commercial spirit, and that is religion. To be more precise–Christianity, and Christianity as it was understood by the Early Christians who attacked the ideal of wealth and property as vigorously as any Socialist. It was Christianity that re-created civilization after it had been disintegrated by the capitalism of Greece and Rome, and if our civilization is to survive, it will be due to the re-emergence of this same spirit.

But it will be said that if we are to wait until a revival of Christianity is accomplished we are lost, for it is impossible to expect wholesale conversions while the problem confronting us develops with such rapidity. To which I answer that I am speaking of the ultimate solution ; not of immediate measures. But it would clarify our thinking enormously about immediate practical measures if we considered them in the light of the teachings of Christianity instead of the materialist philosophy. No one who thought clearly in the terms of Christianity could ever fall into the credit or Bolshevik heresies because he would not think in the terms of Industrialism. And he would not think in the terms of Industrialism because he would realize its central principle was a denial of everything that Christianity stands for: “Take no thought saying, What shall we eat? or, What shall we drink? or, Wherewithal shall we be clothed? (for after all these things do the Gentiles seek:) for your heavenly Father knoweth that ye have need of all these things. But seek ye first the kingdom of God, and his righteousness; and all these things shall be added unto you.” This is the method of Christianity. But Industrialism is the organization of society on the opposite assumption. Seek ye first material prosperity, and all other things shall be added unto you it says. But experience proves not only that they are not added, but in the long run the material things themselves which have been so anxiously sought are taken away.

Meanwhile, let us accept the fact that the day of our industrial supremacy is over, and that we cannot hope any longer to export such vast quantities of goods to distant markets as hitherto. As our industries will not be able to give employment to such vast numbers of workers, agriculture must be revived to provide at the same time work for the unemployed, and the food we shall in the future be unable to obtain unless we produce it for ourselves. This will necessitate a drastic land policy. It is a matter of life and death to us, and no vested interests must be allowed to stand in the way, any more than they were allowed to stand in the way of the conduct of the war. Men must be trained in agriculture and planted on the land with their families. And they must be organized in groups under Agricultural Guilds.

As it is improbable, even with agriculture revived and England colonized, for work to be provided for more than a part of our unemployed, we must be prepared for emigration on a vast scale. Here again organization must be in groups. We must in fact plant new societies as the Greeks did when they colonized. There must be agriculturalists, craftsmen, doctors and others necessary to fulfil the various needs of these communities.

In order that our colonies may absorb our surplus population the individualistic commercial philosophy which has dominated life must be abandoned and a return made to those old principles of organization and fair dealing which are crystallized for us in the idea of the Guilds and the Just Price. The popularization of these ideas should accompany all efforts of reform, for they are the two poles, as it were, of sanity in social arrangements.

Then the handicrafts must be revived and machinery controlled, otherwise the problems which perplex us will speedily reappear in these new centres. It is possible that in the future machinery may turn out to be a blessing instead of the curse which it is to-day. But if its course is to be turned from destruction to construction we shall need to think about it intelligently, and the first sign of grace in this direction will be a determination to control it. Once the principle were admitted its practical application would not be difficult. It could be gradually brought under control by taxing its use where it was socially undesirable. In other directions its use might be prohibited entirely. Where questions of foreign trade were involved agreement would have to be reached with other countries.

The measures I have enumerated are the things most fundamental. They would become the first practical steps towards the creation of the new world order. Though the unemployed problem is at the moment a great perplexity to us, its appearance is a necessary circumstance in the transition to a better order. Henceforth politics will orientate themselves around the problem of the unemployed, and the association of the unemployed problem with social reconstruction should convert idealism into the terms of practical politics. For just consider what a fundamental change of attitude this unemployed problem may bring about. Hitherto it has been the custom in all questions of policy to put the material factor first and to let the human factor shift for itself as best it could–to put the interests of capital before the interests of life. Henceforth this order will be reversed. The urgency of the unemployed problem will compel us to give human considerations the first place, and it must continue to do so. This of itself will effect an intellectual revolution. Political science, which in modern times has been literally upside down, inasmuch as it put the last things first, should develop into a real science of human affairs.

Whether, however, these plans are to be realized or not, all depends on the attitude of the next two or three years. Afterwards it will be too late. Unless the present extraordinary spirit of apathy can be shaken and drastic action taken to deal with the situation, it is to be feared we shall drift into a state of anarchy, lawlessness and wild revolt from which there can be no appeal except to force. The danger is that instead of getting to work to lay the foundation of the new social order, of building the new system while the old one is falling to pieces, we may, encouraged by some brief revival of trade, deceive ourselves into believing that there is no need of fundamental change, or waste our time in discussing all kinds of secondary issues–things against which we are for the most part powerless, inasmuch as they are symptomatic of the break-up of the old order–all kinds of temporary measures, excessive Government expenditure, high prices, high wages, diminished output, etc., anything in fact except the real central issues upon which our whole future depends. Then nothing will get done until it is too late, and starvation becomes chronic among us, and Bolshevism as the scourge of God comes upon us. If Bolshevism does come here we shall have deserved it. For we are in an infinitely better position to face the problems that the war has left than the Continental nations, for not only is our rate of exchange better, but we are an Empire with vast empty spaces ready to take our surplus population. One thing alone can defeat us, and that is \mdstrong{Apathy}.

\chapter*{Appendix. Europe in Chaos\footnotemark[1]}
\label{chapter-13}
\section*{I}
To-day the rates of exchange on London are:

\begin{mdblockquote}
New York, 3.45⅛-3.45¼ (15s. 7d to the £).

Berlin, 256-257 (mark about of ⅞ of 1d.).

Paris, 58.75-58.80 (franc about 4d.).


\end{mdblockquote}
The evidence, on which rests the arguments of these articles, is found in the London rates of exchange current since the Armistice. Thus it will be advisable to give a few of the outstanding rates.

In the case of the United States the par of exchange is \$4.8665 to the £; on December 5, 1918, the rate was \$4.770; while on November 20, 1920, the rate was \$3.470. The rates for the Argentine were 5.040 pesos to the £ (parity), 4.665 (December 5, 1918), and 4.582 (November 20, 1920). For Japan the corresponding figures are 9.800, 8.972 and 6.857 yen to the £.

Thus it will be seen that the rates between London and the two great industrial nations of the East and of the West have moved considerably in a direction adverse to London. In other words, the dollar is now 40.2 per cent., and the yen 42.9 per cen., above their par value.

In the case of the Argentine, the movement is not so pronounced, and the peso is only 10 per cent, above its par value. Still, this is sufficiently disquieting when one remembers the foodstuffs that we are in the habit of buying from that country.

Returning to Europe, there are only two countries whose London rates of exchange are above par. These are Holland and Switzerland, both small nations that remained neutral while the tide of war swept round them. The Dutch florin on November 11, 1920, was 6.3 per cent, and the Swiss franc 13.7 per cent, above their par values, the actual rates being 11.39 florins and 22.19 francs respectively.

The remaining neutral nations of Europe are now at or below par–Sweden exactly at par, Norway and Denmark 29.7 per cent., Spain 4.5 per cent. But when we consider our late allies, and enemies, we find as we progress eastwards the position getting worse and worse. On November 11th, two years after the Armistice, the franc in Paris was 56.2 per cent, below its par value, the Italian lira 72.6 per cent., the Portuguese escudo 85.8 per cent., the German mark 91.9 per cent., the Bohemian kroner 91.4 per cent., the Austrian kroner 97.9 per cent., and the Polish mark 98.6 per cent.

Now let us understand these percentages. Remember that a depreciation of 100 per cent, means that a currency is worth literally nothing for exchange purposes. Then we can see how near the currencies of Europe are approaching to this absolute zero.

Now what is the meaning of this? And how does it affect you and me? And what is the future of Europe? Before we can answer these grave questions we must understand the economic structure of Europe as it existed before the war.

The present industrial system is of recent growth. It was only at the close of the eighteenth century that the ingenuity of man devised means by which the processes of manufacture could be carried out by power instead of hand labour. It is only during the last century that the physicist and the chemist entered into our industrial life.

The results of the “industrial revolution” were far-reaching. It was directly responsible for the modern factory system, which gathered the peoples into large towns and enabled the industrial countries of Europe to maintain a vastly increased population. In fact, one can say that if the present industrial system were destroyed half the population of Europe must either emigrate or starve.

The system by which Europe lived in pre-war days may be described very shortly. Europe imported her food and raw materials from overseas and exported, in exchange, the products manufactured from the raw materials she had imported the year before. These products were of greater value than the raw materials owing to the skill and labour Europe put into them, and on this difference Europe lived.

This is a rough outline of the system, but it needs several qualifications. In the first place, the difference in value mentioned above was greater than Europe’s actual needs. This difference was invested by Europe either in home or overseas industries. This capital, in turn, helped to create more wealth. Thus the European shareholders received additional payments from abroad in the shape of interest.

Secondly, Europe rendered many services to the rest of the world; her vessels carried American and Japanese goods, her merchants dealt in them, her banks financed the movements of these goods, and her insurance companies protected them. All these services were paid for and the payment, like the interest, came in the form of goods–chiefly food and raw materials.

The essence of this system was exchange. We gave our finished products and our services in exchange for our food and raw materials. And the lever that operated this system was called the Bill of Exchange.

A Bill of Exchange is, in simple words, a statement of claim by a creditor on his debtor. Now an American creditor needs payment in dollars, and a British creditor in sterling, a Frenchman in francs. The weight of gold in a sovereign, a 10 dollar piece, etc., is fixed by each country’s laws.

Thus the gold exchange value between sovereigns and dollars can be easily calculated, and this is called the par of exchange. So it is normally open to the debtor to send gold in payment to his creditor. But the actual shipment of gold involves expense, so he usually adopted the second method.

This method is for him to find a creditor in his own country who will sell him a Bill of Exchange or claim upon a debtor in the country to which he himself owes money, or in other words, an Englishman importing cotton from America has to buy dollars with which to pay for his cotton : for the American exporter, as a rule, needs payment in his own currency.

Similarly an American buying goods from England has to buy sterling, so he gets in touch with the Englishman who wishes to buy dollars, and the transaction is arranged to their mutual advantage. But if England has imported more than she has exported, there will be several Englishmen trying to buy dollars for each American who wishes to sell dollars for sterling. So the price of dollars rises–for the demand exceeds the supply.

In other words, the rate of exchange will move against England. Normally this movement will have a limit, for English debtors will find it cheaper to ship gold. But if we have prohibited the export of gold, or if we have a paper currency which can be expanded at will, then, as we see to-day, there is no automatic check to the amount an exchange may depreciate.

It must be remembered that an expansion of currency means a rise in prices, and the price of foreign bills or foreign currency is not exempt from this law. Again, a paper currency brings Gresham’s law into operation, and drives the gold out of circulation. Thus there is no gold available which can be used for payment of foreign debts, and the importers in that country are forced to pay inflated prices for their foreign bills.

We have now described the system by which Europe lived before the war. We see that it depended on a cycle of exchange, and that the cycle was operated by the bill of exchange, and the value of the bill of exchange was maintained, if necessary, by gold shipments. We see that an unsound currency destroys that safeguard, and thus strikes a heavy blow at the system on which Europe lived.

The next point to consider is the effect of the war. During the four years of war every effort of the various belligerents was directed towards their mutual destruction. Thus in England the Government took control of the industries of the country and directed their energies to the manufacture of munitions, which were used, not only to destroy themselves, but also the products and factories of pre-war industry. The effect of this was to reduce the country’s exports to a minimum, while the imports of raw materials for munitions increased enormously. The same applied to all the Allies, and the effect was to pile up an enormous debt owed by the Allies to America.

To liquidate this debt, the Allies were forced in the first place to sell their overseas investments. This entailed the loss of the interest Europe had been receiving from overseas. Later on the Allies in turn were forced to borrow money from overseas.

Thus, in addition to the loss of her former interest, Europe henceforward had to pay interest abroad. This meant that to preserve the trade balance Europe ought to increase her exports at a time when all her energies were absorbed in the war.

Again, before the war, Europe paid for many of her imports by rendering services to the world. But in time of war she was unable to render these services, and so lost another source of payment. Since the war this has been partially recovered. But the Mercantile Marine of Europe has not yet recovered from its war losses. Moreover, there has been a noticeable increase in the shipping owned by the rest of the world. Similarly, the banking and merchant system had been thoroughly disorganized.

Finally, every belligerent had to find the money necessary for the prosecution of the war. This was done in the first place by means of taxes and long-term loans. These absorbed the surplus income and the savings of the various countries, and so diverted them from their normal purpose of developing the economic life of Europe, and turned them to the purposes of war and destruction. But no country wholly paid for the war by these means, and so the deficit had to be met by an increase in their “floating debt.”

The effect of a large floating debt, in its best form, is to absorb all the ready money of the country, which normally is put to a productive use; in its worst form a large floating debt leads to inflation of the credit and the currency of a nation. By inflation is meant the artificial creation of fresh purchasing power without a corresponding supply of commodities which can absorb this purchasing power. Thus the prices of commodities rise, and with them the price of foreign currencies. Or in other words, the foreign exchanges of the belligerents were depreciated as a result of inflation.

Thus the effect of the war was to shatter in every possible way the cycle of trade which upheld the pre-war economic structure of Europe. It dammed the pre-war flow of exports from Europe. It reversed the pre-war flow of interest which formerly paid for some of Europe’s imports. It disorganized the means Europe had of rendering services to the world. It used up and destroyed the stocks of raw materials which Europe possessed. And finally it artificially inflated the currency and credit of Europe, and by depreciating her exchanges rendered it even more difficult for her to obtain those raw materials she needed to restart her flow of exports. The cycle has been broken, and it remains an open question whether it can be repaired.

\section*{II}
WE have seen that the effect of the war was to make Europe break every law upon which her economic structure rested. The position at the Armistice was tragically simple, but most of us were too blind to see it.

Briefly, Europe was swept bare of raw materials, and had no finished products with which to buy them. Her overseas investments had been sold and, in addition, money had been borrowed from abroad with which to pay for the war. Her commercial and financial system was shattered, and her mercantile marine crippled by the submarine campaign. All her savings, all her energy, had been directed to the purposes of destruction.

There was very little left with which to re-start the industries on which her very life depended. All she possessed was large quantities of paper money, which were more or less useless for the purpose of replenishing her stocks of raw material.

Even so, the full story has not yet been told. The war’s toll in life and suffering must still be reckoned in the account. Even those who returned unharmed found that they had lost their habits of regular work. Again, no account has been taken of the actual destruction that the war was the cause of–the farm-lands of the Somme, and the coal-mines of Lens.

Above all, we must add in the loss caused by the collapse of Russia, which was the granary of Europe. If we total up all these items, we see how great was the danger facing us at the conclusion of the war.

It may be urged that Europe recovered from the Napoleonic wars a century ago. This is true, but it should be remembered that our industrial system was still in its infancy. Practically every country was still self-supporting, and had a far smaller population to maintain. At that date Europe was still mainly agricultural, and the different countries were not then bound together into the component parts of one huge machine.

But if the facts in 1918-19 were as we have stated them, what steps were taken at the Peace Conference to save Europe from the effects of the war? To speak quite frankly, the Peace Conference hardly recognized their existence.

Thus they discussed the possibilities of obtaining indemnities from Germany. They did not realize that the only possible way was to take over all German industries, supply them with raw materials, and the people with food, and run them as a “going concern” for what they could get by way of profit. This might have been brutal, but the German people would have been properly fed.

Again, under the name of self-determination, the old Austrian Empire was dismembered. The new States that arose out of it promptly erected customs barriers one against the other, thus failing to realize that they could only exist if they looked upon themselves as one economic unit. The result is now too obvious namely, that a state of financial and economic chaos has given rise to a state of destitution and starvation.

Finally, as a result of the Peace Treaties the Allies have been left with huge military commitments all over the world at a time when every penny was needed to re-start the industrial machine. It is clear now what should have been done. It should have been seen that the restarting of European industries was a race against time, and that compared to this nothing else was of the slightest importance. Food, materials and labour should have been sent at once to where they were needed, and no effort should have been spared to ensure that this was done.

Instead of this petty quarrels have broken out in Fiume, in Poland, in Asia Minor, in Mesopotamia, and, in fact, all over the world. These have all involved expense, and forced the Governments of Europe to resort to further inflation. No Government is guiltless, and least of all the Peace Conference.

The result of this is seen in the exchange movements that have taken place since the Armistice. They show us the result of national extravagance, especially of our military adventures. At this season of the year Europe has to purchase the world’s crops of wheat, cotton, etc., without which she cannot exist. She has nothing with which to pay for them, except a few manufactured products and paper money.

The value that the world attaches to this paper money is shown by the present rates of exchange. Compare, too, the present rates with those current a year ago, when Europe was purchasing last year’s crops, and it ill be clear that Europe is slowly sinking under the burden that the war placed on her shoulders.

Thus, in December 1919 our pound was worth 3 dollars 8 1 cents in New York; in November 1920 only 3 dollars 44 cents. In December 1919 our pound would buy 41.03 francs, 49.63 lire, and 181.53 marks (contrast even these rates with the par of exchange). In November 1920 these rates were 57.17 francs, 95.13 lire, and 262.89 marks. This shows the extent to which the dry rot has spread during the year.

The cause of this rot is plain–Europe must buy in order to live, but she has nothing to sell. And unless the cycle of trade is restarted, she will still have nothing to sell.

It may be asked: “What steps have the Governments taken in order to rectify this position?” The answer is that most steps taken by the Governments have resulted in aggravating the position.

There is no need to call attention to the extravagance of the various Governments. The word is on every one’s lips, and the pity is that people do not realize the direction in which extravagance is leading us. For every fresh load of debt, every fresh issue of paper money brings the final tragedy nearer–when the machine on which we depend, and which is already tottering under the blows dealt it by the war, will be unable to serve us any longer.

It is easy to give examples of the results of this extravagance–Continental inflation and British E.P.D. The first renders it more and more impossible for the Continent to buy, the second renders it difficult for our industries to produce the goods the Continent needs at a price they can pay. But whatever it results in, one thing is clear : This extravagance must cease, if Europe is to be saved.

Among other steps taken, various Governments have attempted to regulate their exchanges. This was done during the war with fair success; but it meant the loss of our overseas securities, and also the raising of foreign loans. But after the war the exchanges had to be left to find their own level, with the result we now see.

Sporadic attempts, however, have been made to arrest their fall. The fall in sterling was arrested temporarily by the shipment of gold last March to New York, and by reduction in purchases. A month later France arrested the fall of the franc by a very drastic series of import restrictions. That they were only partly successful was probably due to further Government extravagance.

The Portuguese Government tried to fix their exchanges by an arbitrary decree, but found that the only result was to cut off their imports. Lastly, our own attempts to regulate the currency in India and East Africa have met with a large amount of justifiable criticism.

Any attempt of this kind is bound to fail, for the rates of exchange are but a symptom of the economic illness from which a country is suffering. It is useless to remove the symptom without removing the cause of the illness, and so it is useless to regulate a rate of exchange while leaving the cause of its depreciation untouched.

If any further evidence is needed of the breakdown of the economic machine, it is found in the wild price fluctuations that have been rampant during the past two years. It is comparatively easy to trace these price movements and also their causes. The Armistice found Europe swept bare of all her stocks of goods. Her industries were all mobilized for the production of munitions, while her peoples had an ever-increasing supply of paper money in their pockets.

As industry resumed a peace footing, orders flowed in from all over the world, and every factory was filled up with orders for months ahead. It is no wonder that prices began to soar, while speculation was rampant, and huge profits became the rule. Nor is it surprising that the workers claimed a share in these profits, and a better wage with which to meet the rise in prices.

Disastrous strikes followed until these higher wages were granted, and the mere granting of them entailed a rise in production costs, which forced the manufacturer to maintain his swollen prices. Nor can the Governments be absolved from profiteering. The British Government enforced a Profiteering Act at home, while they were selling coal at £10 per ton on the Continent.

The effect of these high prices quickly showed itself in the rates of exchange, and by the middle of this year the Continental exchanges had depreciated so badly that Europe was unable to buy our goods. Then the break came, and prices began to fall. This fall brought with it dwindling profits, in some cases enforced liquidation, and in most trades unemployment for the workers. The boom in British trade was broken, and the slump is only now beginning.

This is the position to-day. Europe is dying for lack of our goods, but Europe cannot produce the goods she needs in order to pay for ours. For we cannot take payment in paper money and depreciated currencies. So our export trade is going, our industries are being slowly strangled, and our men are being thrown out of work. That is what the collapse of Europe means to us, and now we can only see the beginning. Remember that as we go eastward from the Bay of Biscay the exchanges become worse and worse until we reach Russia, where the rouble is absolutely worthless. Watch the rates of exchange, and then ask yourself, “What will be the end?”

\section*{III}
We have seen that before the war Europe supported a larger population than she could have fed from her own produce by exporting finished goods, by the interest on her overseas loans and the payment for her services.

The war has decreased or destroyed the last two sources of income and replaced them by claims for interest on war loans, which means that henceforth she must export more than she imports instead of being able to do the reverse as she did in 1913.

Moreover, the supply of finished goods with which she bought next year’s food and raw materials no longer exists. Unless, however, she can get these essentials she cannot restart her industrial system, and having no goods to give she can only offer paper money. This being only of use if foreigners can exchange it for goods, has continued to depreciate steadily since peace was made, as there are not the goods.

In short, the position, after two years’ peace, is, as shown by the rates of exchange, far worse than in 1918. Therefore, Europe is slowly drifting into a state of bankruptcy, which means that ultimately she will no longer be able to buy the bare necessities of life. When that happens the whole system must collapse. What that means is shown by the condition of affairs in Russia, a country, which, being mainly agricultural, should have been able to feed itself if any European country could.

The possibility of such a catastrophe is so terrible that so far no one has dared to suggest it, but the writers feel that unless people realize where they are drifting no efforts to avert it will be made till it is too late. They do not say that even now it is impossible to save Europe, though it will be no easy task, but they do say, if things are allowed to drift for another two or three years it will be too late then. It is certainly possible to save Great Britain to-day; by then it may be too late.

Unfortunately there are not wanting other indications that our civilization is in danger. We can only tabulate these briefly, but whenever in history a civilization has been approaching its end similar indications have appeared.

They include a marked laxity in the morals and an open challenge to the established moral codes. For example, “The Right to Motherhood” shows what is meant. The failing influence of the orthodox faiths, love of luxury and extravagance at a time when tens of thousands are suffering from want ; a spirit of lawless violence, coupled with a strange apathy on the part of a large section of the community, are characteristic indications of a decaying civilization.

Though these vices are noticeable in Great Britain to-day, they are not nearly so marked as in many Continental countries, and only emphasize the more the fact that Great Britain is still healthier than the Continent.

As the situation on the Continent goes from bad to worse, we find it increasingly difficult to sell our goods. We, above all countries, are dependent on our export trade, and it is poor consolation for us to know that America is suffering in proportion, even more severely in her export trade, from the same cause. America can feed herself still, whereas we cannot. To her, external trade is almost a luxury, to us it is an absolute necessity. Without it, half our population will starve.

Already we are witnessing the gradual closing of our Continental markets, and almost a panic among our manufacturers at the possibility of being undersold in the home markets by the Continent, but this aspect of the case was dealt with in the \emph{Daily News} on November 26th.

Unless the decline on the Continent is stopped, this strangling of our industries will continue, and it behoves us now to consider seriously what we shall do in that event. There is no need for panic, but that is far less likely than apathy and contemptuous unbelief till the crisis is on us. By then it will be too late. Rather let us take such a possibility into our reckoning, and begin to prepare alternative plans.

If Europe can be saved, then gradually things will right themselves, and the first thing to be done is for every Government at home or abroad to reduce its expenditure to the very lowest that is possible, even if this entails the abandonment of desirable social schemes or valuable military positions. We simply cannot afford them.

Every country must not merely increase production, but see that the goods made are exchanged for the things they must have. It is no use filling warehouses with goods which our neighbours cannot buy because their exchanges are so badly depreciated. We in Great Britain must open up new markets, if necessary, by means of barter, particularly with countries other than the United States of America, from which we can get food or raw materials–for example, Poland and Russia.

But supposing Europe cannot be saved, what will happen? Briefly it will be impossible to transport the excessive millions in Europe overseas. What will happen to them is what has happened in Russia, and to-day is happening in Poland and Austria they will die.

Those who survive will revert to an agricultural race, with but simple industries and no elaborate industrial system. Do not think that this picture is too highly coloured. Five years ago, would you have thought it possible that Russia would have reached the condition she is in to-day? Russia, remember, represents one-quarter of the earth’s surface. As we move eastward from the Bay of Biscay to the frontiers of Russia, we find that the exchanges fall consistently. On November 30th France was 57.80 francs to the, Italy 95.50, Germany 250, Austria 1,175, Poland 1,750, Hungary not quoted, Russia —! And on the frontiers of Poland gather a pack of starving men looking hungrily westward.

What is the alternative to the present system if it does not recover? It is not Bolshevism. That is the last resort of desperate, starving men. It may come when the last agony of dissolution is upon Europe, but it cannot reorganize and feed the present large population. It has already appeared sporadically in Hungary, Germany and Italy. It has been driven underground–perhaps–but only for a time. If you want to prevent Bolshevism see that the people are well fed. That, however, is just what we are unable to do in many parts of Europe. The machine that did is broken by the war, it is still freely rotating, but each month it moves slower and with more difficulty.

If we cannot save Europe, can we at least save ourselves?

Yes! if we prepare in time, Great Britain can be saved, but it will not be the Great Britain we knew and loved before the war. With our Continental markets gone, and our export trade crippled, we shall not be able to support our present population.

A drastic land policy would settle on the countryside millions who are now congregated into industrial areas. Millions would have to emigrate to our Overseas Dominions and Colonies. Herein lies our strength. We are an Empire with vast empty spaces, with lands which can produce the food and raw materials we shall still need, and supply us with the simple things of life, which we can barter with the more primitive peoples of Europe.

Our industries will dwindle, but our geographical position will enable us to remain a great seafaring and merchant race. As the last outpost of the industrial west (by then the United States of America) we can still carry the merchandise of those countries which cluster round the Pacific to those who dwell in Russia and barter them for the minerals and raw materials they are prepared to offer.

But it will be a smaller England with probably less than half its present population and perhaps a humbler member of the British Empire than it is to-day. Do not let us suppose that we can continue indefinitely to export huge quantities of manufactured goods to Australia, Canada, or even South Africa. During the war these countries have been developing their own manufactures near the spot where they produce raw materials. This process is bound to continue.

Has it ever struck you how the centre of industry, commerce and civilization has shifted ever westward? In classical days the Mediterranean was the centre, in the Middle Ages it was the Baltic, where the Hanseatic League ruled. In the sixteenth century it shifted to the Atlantic. What if it is again moving to the Pacific, where America and Australia face China and Japan? Look at the rates of exchange of Japan and the United States of America if this possibility seems fantastic.

But it takes time to move millions of men, and if the industrial system is breaking down, what will take its place? State Socialism cannot, for it presupposes a huge industrial machine. Perhaps the Guild Socialists have seen a vision of the ultimate solution, but, if so, they must descend from the clouds and begin to construct their system here and now.

It is useless to imagine our Guildsman will straightway become a saint. He will be exactly the same man who at present forms part of the industrial system. In time a better system may produce more perfect men, but they must evolve by degrees.

Meanwhile, the wise man uses the material he has to hand, and in truth the average Briton, despite his faults, is still the cream of the earth. In short, why not begin to build the new system to-day, so that it is a running machine by the time the old one breaks down completely? But still, perhaps, this appears a nightmare dream. What if, after all, it is but the darkness before the dawn of better things? Nightmare or vision of the dawn, take your choice, but look at the writing on the wall and ask which country follows Russia, and the answer is given to you by \emph{the rates of Exchange}.

\footnotetext[1]{\emph{Daily News}, December 11th, 14th and 16th. See Preface.

}

\end{document}
