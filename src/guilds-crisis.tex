\documentclass{book}
\usepackage{fontspec}
\usepackage{xunicode}
\usepackage[english]{babel}
\usepackage{fancyhdr}
\usepackage[htt]{hyphenat}
\usepackage[a5paper, top=2cm, bottom=1.5cm, left=2.5cm,right=1.5cm]{geometry}
\makeatletter
\date{}
\pagestyle{fancy}
\fancyhead{}
\fancyhead[CO,CE]{\thepage}
\fancyfoot{}
\makeatother
\title{Guilds and the Social Crisis}
\author{Arthur Penty}
\begin{document}
\thispagestyle{empty}
\vspace*{\stretch{1}}
\begin{center}
	{\Huge \@title   \\[5mm]}
\end{center}
\vspace*{\stretch{2}}
\newpage
\thispagestyle{empty}
\cleardoublepage
\begin{center}
	\thispagestyle{empty}
	\vspace*{\baselineskip}
	\rule{\textwidth}{1.6pt}\vspace*{-\baselineskip}\vspace*{2pt}
	\rule{\textwidth}{0.4pt}\\[\baselineskip]
	{\Huge\scshape \@title   \\[5mm]}
	{\Large }
	\rule{\textwidth}{0.4pt}\vspace*{-\baselineskip}\vspace{3.2pt}
	\rule{\textwidth}{1.6pt}\\[\baselineskip]
	\vspace*{4\baselineskip}
	{\Large \@author}
	\vfill
\end{center}
\pagebreak
\newpage
\thispagestyle{empty}
\null\vfill
\noindent
\begin{center}
	{\emph{\@title}, © \@author.\\[5mm]}
	{This work is free of known copyright restrictions.\\[5mm]}
\end{center}
\pagebreak
\newpage
\setcounter{tocdepth}{0}
\setcounter{secnumdepth}{0}

\chapter*{Preface}
\label{chapter-0}
This book is, among other things, an attempt to formulate a policy for Guildsmen in the event of a revolution. Prophecy is always dangerous, and it is, of course, conceivable that a sudden enlightenment might descend upon the governing class of this country such as would enable them to steer safely through the social rapids which lie ahead. It must be confessed, however, that such a miracle is highly improbable, considering that they do riot apparently possess sufficient understanding to retain the loyalty of such a naturally conservative body of men as the police. Prudence, therefore, suggests the wisdom of accepting revolution as inevitable, and of shaping Guild policy in the light of it, in order that we may not be taken by surprise. For revolution is at the same time a great opportunity and a great danger. If it should come upon us while we are unprepared, it is almost a certainty we should drift into anarchy. On the other hand, if anticipated, it might be used for the purposes of reconstruction.

The circumstance that, owing to the excesses of the Bolshevik regime in Russia, the idea oi revolution is no longer popular in this country does not affect the position one iota. For revolutions are not definite political acts which owe their origin to a more or less temporary mood of the people, but are forced upon people by the fact that a particular political and economic system has reached a deadlock. For when normal activities can no longer find an outlet there is bound to come a bursting of barriers. Such an impasse, I hope to show, is bound to follow the economic policy of the Government, which may be summarized in the term “Maximum production.” It is a policy which must either issue in revolution or other wars, which if the public allow could be used to relieve the pressure of the markets by the creation of a demand for armaments. It has been said that Governments are never overthrown, but that they commit suicide, and, frankly confessed, our Government seems impelled by a kind of fate towards such an ending.

As the assumption underlying my arguments is that Germany will not repay our War Loan, it is necessary to point out that even if she were made to pay, the crisis would not be averted. In this event we should have to provide her with work, and this would react to increase unemployment in this country. I wish it were otherwise, for justice demands that Germany should be made to suffer; but I cannot overlook the fact that its economic reaction upon ourselves would be as unfavourable as the introduction of slaves was to the freemen of the Roman Empire.

It remains for me to thank the Editor of the \emph{New Age} for permission to reprint the two concluding chapters.

A. J. P.

66 Strand-on-Green, W. 4.

September 1918

\chapter{The Economic Cul-de-sac}
\label{chapter-1}
In spite of the repeated assurances of Cabinet Ministers and others that things after the war are going to be very different from what they were before, there is little either in their words or actions to suggest that they have any idea of what the forthcoming changes are likely to be. Though they talk a great deal about reconstruction, and have set up a Ministry of Reconstruction to elaborate plans for our guidance in the future, it becomes more evident every day that it is readjustment rather than reconstruction that engages their attention There is nothing either in the general principles laid down for the guidance of the Committees set up by the Ministry,\footnotemark[1] or in such of their reports as have already come to hand, to suggest that the governing class are in any way conscious of the need of reconstruction. On the contrary, all the reports agree in taking existing society in its main essentials for granted as a thing of permanence and stability, little suspecting the real peril that confronts us and seeking only to effect such detailed adjustments as they suppose are necessary to enable society to recover from the shock and dislocations occasioned by the war. One of the Committees only that concerned with the Labour Unrest shows any sign of alarm, while even here there is little or no suspicion that the trouble is irremovable so long as industrialism exists. On the contrary, the trouble is regarded merely as a form of distemper to be remedied by the balm and plaster of the Whitley Report.

While making this general comment on the work of the Ministry, I must not be interpreted as deprecating entirely the work of the Committees. The problems of demobilization and the supply and distribution of raw materials are problems of fundamental importance, though they partake of the nature of readjustment rather than of reconstruction. However much our eyes are fixed on the future, however much we may be persuaded that the only way to avoid catastrophe is finally to take such measures to strengthen the base of society as are involved in a return to first principles, the fact remains that we must live from day to day during the period of transition. And in order that we may so live, in order that the economic reaction of the war may not precipitate anarchy, society as it exists to-day must be propped up. To such an extent the work of the Committees is valuable, and to such an extent the various systems of control which are being introduced into so many departments of production and distribution are to be approved, even though they do involve bureaucratic methods of administration. If the temporary nature of these arrangements be admitted, then no harm can come of them. The danger is that these props, instead of being regarded as scaffolding necessary to the rebuilding of society, should be mistaken for permanent structural arrangements, for they touch no vital social issue. Though at the moment they put a boundary to the growth of anarchy, they do not seek to remove its cause, and no scheme which does not seek first and foremost to remove the cause of social anarchy is worthy of the name of reconstruction.

That readjustment rather than reconstruction was the aim of the Ministry is apparent not only from the terms of reference to the Committees, but from their manner of setting to work. Had reconstruction been their aim, they would not immediately have set up a number of Committees to deal with the various aspects of the problem presented, but would have sought first to establish some general unanimity of opinion as to its cause. It was, I suppose, because they regarded the war as a colossal accident rather than as the goal towards which industrialism inevitably tended that they made no such effort. And this is where they went astray. If the war were entirely due to the personal ambition of the Kaiser and the lust for conquest of the Pan-Germans, then there would be no more to be said. But if on inquiry we find there to be causes much more fundamental and intimately connected with the economic expansion to which industrialism had committed all the nations of the West, the situation wears a very different complexion. For it will then be seen that readjustment is not only insufficient to meet the perfectly legitimate demands of labour, but cannot even save the governing class itself from complete annihilation in the near future.

In such circumstances it becomes apparent that if a scheme of reconstruction is to be formulated which shall be in relation to the facts of the case, we must make our starting-point an inquiry into the causes of the war, and in this connection it will be convenient to begin with the Kaiser and his personal responsibility. Evidence seems to point to the fact that though the Kaiser’s arrogant and bombastic spirit was a great factor in the development of the war spirit in Germany, yet at the last moment he was reluctant to sign the declaration of war. The Kaiser is evidently a weak man, and had doubtless to screw his courage up to take the final step, as is evidenced by the testimony of Dr. Mühlon, formerly a director of Krupps, who has told the world the story of how r early in July 1914 the Kaiser informed Herr Krupp von Bohlen that he would declare war as soon as Russia mobilized, adding “and this time the people would see that he would not turn back.”

That is conclusive. But there is a question arising out of this. Why did the Kaiser say “this time”? It had reference to the Agadir crisis of 1911, when the Kaiser came to an agreement with France over matters in dispute in Morocco without having occasion to resort to war. This pacific act of the Kaiser did not please the Pan-Germans, who denounced him in the Berlin Press as a coward and a traitor, and so, being a weak man, he yielded to their clamour. But why did the Pan-Germans desire war?

Prince Lichnowsky has told us that when the British Government showed the utmost readiness to meet the wishes of Germany in its desire for colonial expansion and a treaty defining the respective spheres of influence of the two Powers had been arranged, the German Government refused to sign it upon the only terms on which Sir Edward Grey would become a party to it—namely, that it should be given publicity. The answer is, of course, that as the Pan-Germans desired war under all circumstances and determined that nothing should stand in its way, they deprecated the publication 6f a treaty which would have knocked the bottom out of their propaganda. Had the treaty been published, it would have been impossible publicly to maintain the theory that Germany was surrounded by a world of enemies, cut off from any peaceful expansion by the envious jealousy and the encirclement policy of British statesmen.

But why did the Pan-Germans desire war apparently under any circumstances? The usual answer is, of course, to say that Germany was ambitious, desired world dominion, that she had become so saturated with the spirit of war and conquest that she had become incapable of thinking politically except in the terms of war. While this undoubtedly was the case, it does not explain why war broke out in 1914 instead of before, for such a spirit had been present in Germany since 1871. The reason is, I think, to be found in the economic condition at which Germany had then arrived. The financial strain in Germany in the three or four years preceding the war had become so terrible that it is conceivable that the war was as much caused by the desire, for relief from such trying circumstances as by the warlike proclivities of the German ruling class. German trade had been built up upon a highly organized system of credit; and as that system showed signs of breaking down, German statesmen and financiers apparently had come to the conclusion that the only way to save the country from financial disaster was to secure the huge indemnities which would follow upon a successful war. That is the reason, I believe, why Germany refused to sign the treaty with Great Britain. It was because its statesmen felt that while it would make war impossible, it would not solve the economic problem with which Germany was confronted in 1914.

Before the outbreak of the war the joint-stock system of banking in Germany was in a very rotten condition. Germany was trading upon a broadly extended system of credit, controlled through the Reichsbank by the Government. Under the Reichsbank flourished a system of four hundred and twenty-one joint-stock banks. In February 1914 the ninety-one principal joints-tock banks had owing to them from various debtors 6,068,000,000 marks, while their indebtedness was 8,600,000,000 marks, or in other words, they were insolvent—a fact which is not surprising when we learn the highly speculative nature of the enterprises which they were accustomed to finance. The stability hitherto of the English banks rests on the fact that they can only invest in gilt-edged securities. But the German banks would apparently finance anything, no matter how speculative. Many of them had been heavily engaged in promoting doubtful ventures at home and abroad, such as the building of railways in Russia, Asia Minor and South America, while in order to encourage German export trade they were accustomed to grant long credits to foreign customers without near prospects of payment. It w r as by such means that Germany had hoped to secure the commercial hegemony of the world. But she had overreached herself. The system was clearly breaking down.

It will be unnecessary for me to go deeply into this matter, but a moment or two spent over the greatest of the joint-stock banks—the Deutschebank—will be worth while. “On paper this limited company, which must not be mistaken for the Imperial State Bank, is an imposing institution. Its securities and reserves amount to 425,000,000 marks, or 21,000,000, of which 250,000,000 marks are capital and 175,000,000 reserve, figures which will compare reasonably well with one or other of the smaller joint-stock banks of this country or of France. But where the English joint-stock banks or the Credit-Lyonnais, let us say, are largely institutions of deposit, doing only very conservative financial business, the Deutschebank, which has lately absorbed the Bergisch-Marckischebank, employs the greater part of its capital and its resources in speculations of a very doubtful type, or definitely and absolutely employs the deposits entrusted to it for political ends or the extension of German interests. In Turkey, for instance, the Deutschebank has employed itself in the building of railways, in the farming of the octrois; in Berlin it has attempted to found a petroleum monopoly under the control of the Government, and it has advanced more than 100,000,000 marks for the purpose of saving the Fuersten-Conzern.

“This Princes-Concern was an immense syndicate of princes and courtiers who were determined to obtain their share of the industrial development of Germany. They built hotels, factories, immense shops, where they traded in every possible article of commerce; they speculated in building land; and last year (1914) the whole concern came to the ground with an immense crash, threatening with absolute ruin several of the princely houses of Germany. That the Deutschebank should have tried to come to the rescue of this concern was nothing more or less than dishonesty to its depositors, or, if that is too strong a statement, it is exact to say that at the date of the outbreak of the war the Deutschebank, in spite of its advance of 100,000,000 marks, was very far from having established the Fuersten-Conzern on anything like a satisfactory basis.”\footnotemark[2]

Corroborative testimony to the economic depression which had overtaken Germany prior to the war is to be found in the Reports of H.B.M.’s Consular Agents in Germany. Reading them makes it fairly apparent that by the end of 1912 the German industrial system had reached its limit of expansion, and that the competition of French, Japanese, English, and Scotch manufacturers was either closing markets to the Germans, or was actually making inroads in the German home trade, and it becomes evident that the German financial system, built on an inverted pyramid of credit, could not for long bear the strain of adverse conditions. Germany was committed to a policy of indefinite industrial expansion, and signs were not wanting that that expansion had reached its limit. Professor Hauser\footnotemark[3] tells us in this connection that the ratio of productivity, due to never-slackening energy, technique and scientific development, was before the war far outstripping the ratio of demand. Production was no longer controlled by demand, but by plant. What the Americans call overhead expenses had increased to such an enormous extent that no furnace could be damped down and no machine stopped, or the overhead expenses would eat up the profits, and the whole industrial organization come crashing down, bringing with it national bankruptcy. In other words, the commercial history of the German Empire was one of enormous artificial expansion obtained not infrequently by cutting prices to such an extent that there were no available profits when the expansions were secured. Since the opening years of the present century the whole financial position of Germany has, in fact, been one of long anxieties, qualified by short periods of hectic confidence.

But, it will be said, if the German economic system was really breaking down before the war, how is it that she has been able to finance the war for four years? For if such had been the case, would not the strain of the war have broken it down long ago?

The answer is that though in the long run the continuance of war tends to wreck every economic system, its immediate effects may be otherwise, inasmuch as the exigencies of war can be used by an all-powerful Government to perpetuate a financial system which is moving towards bankruptcy by changing temporarily the basis on which it rests. Let me explain how this works.

It will not be disputed that under normal peace conditions every financial system rests upon confidence. The maintenance of this confidence In these days rests upon an ability to make profits, for it is only by making profits that interest on loans and other financial obligations can be met. Should the pressure of competition become so severe that the margin of profit is reduced beyond the point at which obligations can be met, confidence goes, and if the great majority of people in a community are in these difficult straits economic stagnation results. Such a state of things might exist in a society in which a small minority of the community was very wealthy. Economic stagnation in such circumstances would not mean that there was not wealth in a country, but that the possessors of wealth withhold their money from circulation because they cannot see a return for their capital. The evidence I have given appears to show that some such state of financial stagnation had overtaken Germany in the two years preceding the war. Competition had become so keen and profits so reduced that confidence had been largely destroyed, and money withdrawn from circulation. But once war was declared the system began to work again, because finance rested no longer on confidence but on force. In other words, war introduced a change in financial operations to the extent that confidence came to rest no longer upon personal solvency, but upon Government solvency, which in turn rested upon faith in German arms to secure huge war indemnities at the conclusion of peace. Realizing the root trouble in German finance, namely, small profits which disposed the possessors of wealth to withhold money from circulation, I do not see how war indemnities could provide a remedy. But that is by the way. The important thing was the German people thought so, and that set the financial machine in motion again.

Enjoying this illusion, the possessors of wealth, who in peace times withheld their money from circulation because they could see no return for it, lent it to the Government when it declared war, partly out of fear, lest if they did not support the Government their country might be invaded or they might be compelled to acquiesce in an unfavourable peace, but primarily because the Government promised them interest on their loans. Further, under the plea of urgency to which the war gave justification, the credit of the German subject was propped up by the Government, which, acting through the Reichsbank, put a value by fiat on securities which are now unsaleable, and can only have value on the hypothesis of a German victory: values on German business concerns abroad sequestrated and possibly to be confiscated, values of concessions that may never be returned to Germany, values of export houses that may never be able to regain their markets, values of ships seized or sunk, etc. By placing a credit value on such securities, the Government could borrow money. The expenditure of these loans by the Government put money into circulation, which the German Government borrowed again from the people into whose hands it had passed, paying the interest out of further borrowing. This process could be continued so long as the belief persisted that the country would be ultimately victorious, and it took a long time to destroy this belief, for by the exercise of arbitrary power the German Government had managed to merge with its shaky structure of public credit the whole structure of private credit as well. The limit is only reached when the accumulations of interest to be paid cannot be met out of further borrowings.

It will be with the return of peace that the real troubles will begin. This will be not merely because of the political and financial complications which will arise in every belligerent country over the repayment of their war loans, and the problems of demobilization and unemployment, but because the economic lesson which the war should have taught has not been heeded. The war is still regarded by most people as a colossal accident, a stupendous misfortune which has overtaken the world. Individual thinkers here and there have seen its connection with industrialism, have seen that the war was precipitated by the fact that industrialism, at least in Germany, had reached its limit of expansion. But nowhere is there any public recognition of the fact, and this is where the danger lies. For it is certain that the whole of Western civilization was travelling in the same direction, and, apart from the war, would soon have found itself in this same economic cul-de-sac, from which the only escape is backwards. It is a paradox, but it is nevertheless true, that what we term expansion ends finally in congestion. The congestion which for so long followed every attempt to break the line in France symbolizes the congestion which has entered into every department of modern activity. Everything in modern life is congested—our politics, our trade, our professions and cities have one thing in common: they are all congested. There is no elbow-room anywhere, and, as I have said, there can be but one path of escape, and that is backwards.

Modern thinkers, although they will sometimes admit that many things in life have their limits, nevertheless find it difficult to believe that there is such a thing as a limit to economic development. Somehow or other they imagine that economic expansion can go on for ever, and deny absolutely the possibility of such a thing as an economic deadlock overtaking industry. I believe that a terrible disillusionment awaits them, for events very soon after the war will prove my contention in a way more forcible than logic unless, of course, in the meantime .the danger is clearly recognized and measures are taken for meeting it. Judging by the trend of opinion, such a course seems extremely unlikely.

Let me try to show why industrial expansion must eventuate in an economic deadlock. I will begin by defining an economic deadlock as a state of affairs in which the balance between demand and supply is so completely upset that only changes so drastic and fundamental as to amount to a revolution can by any possibility restore it again. What is there improbable about such a situation arising? The balance has been upset many times during the last hundred years, and after a time it is true things have adjusted themselves again. But what reason is there to suppose that the balance will always be restored, any more than to suppose that because a man has recovered several times from some serious illness he will always be able to offer effective resistance? We know that such is not the case, and that the constant recurrence of illness will so weaken a man’s constitution that in the end he succumbs. The same holds good with respect to economic evils which attack the body politic. They undermine this and undermine that until finally they bring disaster. That this is not popularly recognized is due to the long period of time which elapses between the first symptoms and the final catastrophe. When the evil first appears it gives rise to alarm. People predict dreadful consequences, and they are right, but the delay seems to disprove them. Familiarity breeds indifference. Then apologists appear, and the various stages of the disease are heralded as signs of progress, until finally all ideas of right and wrong become so confused that when the final crisis arrives the foundations of right thinking have become so completely undermined that nothing can prevent collapse.

Let me argue the point another way. If there is no limit to the possibilities of production there must be no limit to consumption, because the volume of production can only increase on the assumption that there is a corresponding increase in consumption. But is it not apparent that there must be a limit to the possibilities of consumption? If by automatic machinery we could increase production a thousandfold the balance between demand and supply would be upset and an economic deadlock created, for it is a certainty we could not increase our consumption to a corresponding degree, except by recourse to a war a thousandfold more destructive than the present one.

That is the answer, I think, to those people who agree in theory that there is a limit to consumption, but deny that we are in any way reaching this limit. The proof that there is such a thing as a limit to consumption lies in the fact that we are at war. We are at war to decide, among other things, which nation or group of nations shall have the right to dominate the markets of the world. If the limit of consumption has not been reached, why should there be this struggle, why this intensification of competition? Surely it can only mean that, having reached this limit, we are in an economic cul-de-sac, that we are unable to go forward and too proud to go back.

\footnotetext[1]{In introducing the Bill for the establishment of a Ministry of Reconstruction (July 27, 1917) the Home Secretary (Sir George Cave) explained that it would be concerned with 1. The restoration of normal conditions in connection with commerce and industry and the development of trade in the light of the experience gained by the war; 2. The restoration of the normal rights of persons affected by war conditions and improvement in conditions also suggested by the circumstances of the war.

}\footnotetext[2]{The quotation is from \emph{When Blood is their Argument}, by Ford Maddox Hueffer (Hodder \& Stoughton), which in spite of its gory title is one of the most interesting books I have read on pre-war conditions in Germany. Corroborative testimony as to the rotten state of the joint-stock banks is to be found in Professor J. Laurence Laughlin’s \emph{Credit of the Nation}s (Scribners, New York), to which I am also indebted.

}\footnotetext[3]{\emph{Germany’s Commercial Grip of the World}, by Professor Hauser of Dijon (Eveleigh Nash).According to Messrs. Farrow and Crotch in the space of fifteen years Germany quadrupled her output, and in consequence a day came when all the world that would take German-made goods was choked to the lips. Economic difficulties began to make themselves felt in Germany, and then the Prussian doctrine of force spread with alarming rapidity. War was decided upon for the purpose of relieving the pressure of competition by forcing goods upon other markets (\emph{The Coming Trade War}, by Thomas Farrow and Walter Crotch: Chapman \& Hall, 2S. 6d.).

}\chapter{Maximum Production and Scientific Management}
\label{chapter-2}
THE underlying cause of the destruction of the balance between demand and supply, which in turn has been the economic cause of the war and will lead us afterwards into an economic cul-de-sac, is the sin of avarice, which leads people to be for ever reinvesting their surplus wealth for further increase instead of spending it upon crafts and arts.

This mania—for it is nothing less—is of quite modern origin. In the Middle Ages, as in the East to-day, it was the custom of people to spend or invest their wealth in beautiful things. They would spend their all upon fine buildings, furniture, metal-work, rugs, or jewellery. Incidentally, this is why people who were poor according to modern standards invariably lived in a beautiful environment. It was natural for these people to spend their wealth in this way because when the laws against usury were strict there was no other way to spend it. But with the relaxation of the Mediaeval laws against usury, and the rise of Protestantism, which sought to accommodate morals to the practice of the rich, a change gradually took place. Still, in spite of gross inequalities in the division of wealth, the balance between demand and supply was fairly maintained, since, so long as hand production obtained, a natural boundary prevented the growing tendency of people to reinvest surplus wealth for further increase from developing beyond a certain point. But with the coming of machinery and the limited liability company this boundary was removed, and opportunities for investment presented themselves at every turn. It was thus that the old idea that surplus wealth should be spent upon the arts first fell into disuse, and then was forgotten. When people build nowadays they no longer regard it as a means of consuming a surplus, but as a speculation by which they hope to increase their riches. This applies not only to building, but to pictures, which are bought to-day as investments.\footnotemark[1]

Had the governing class any grip of the economic situation, they would have seized upon this issue as being the central one for themselves, and by diverting surplus wealth into its proper channel have sought to readjust the balance between demand and supply. But in spite of all that has happened, and is happening, they seem to be entirely blind to the situation. They never for one moment reflect on the general economic situation, which in their minds appears to be entirely obscured by two issues considered by them of more immediate importance, namely, how to secure our commercial supremacy after the war against the competition of Germany, and how to repay the war loan. Being practical men—that is, men who can never see the wood for the trees—they concentrate on these two issues, disregarding entirely the wider considerations involved. Faced, apparently, by a dilemma and seeing no sure path of escape, they close their eyes to half of the facts of the situation and plunge wildly forward in a desperate bid for safety. How else can the advocacy of maximum production and scientific management be explained? If it is not a policy of desperation, what is it? For no one could advocate it who has made any attempt to see the problem as a whole. It is a gambler’s last throw with the dice loaded against him.

I feel well advised in making this assertion, for all the facts of the situation point to this conclusion. The advocates of maximum production and scientific management make no attempt to think as statesmen who take all sides of a problem into consideration; they do not even think of the class interests of capitalists, for maximum production can be shown to be contrary to their interests as a class; they think as individual capitalists who interpret national problems in the terms of their own businesses. There can be no doubt about this, for it is only by thinking in such terms that it is possible to make out a case for these proposed innovations. Their reasoning is arcadian in its simplicity. To repay the war loan and to maintain our commercial supremacy after the war, it is necessary to make more money and to produce more cheaply. These ends are to be attained by maximum production on a basis of scientific management. What could be simpler? Scientific management will reduce the cost of production, and will therefore allow us to compete more successfully with Germany, while maximum production increases opportunities for the making of profits. Such a policy is without doubt a sound business proposition from the point of view of the individual capitalist who has to consider ways and means of holding his own in the market and meeting his financial obligations after the war. But it is not possible for many of them to adopt it without imperilling the stability of the whole social and economic system. For it has this defect, when considered from a national point of view, that it increases immeasurably the discrepancy between demand and supply. It trespasses further on the margin of economic safety. In a word, it is a proposal to take a short cut by sailing too near the wind, and as after the war the political and economic atmosphere will be charged with storms and tempests, the chances are that the ship of state will capsize.

To realize the danger of this proposal it is only necessary to enlarge the area of the problem. Granted that maximum production and scientific management would enable our manufacturers to produce more cheaply and to make more money, would it enable them to give more employment? For unemployment is going to be the problem of problems after the war, and a policy which does not make this issue its starting-point is no policy at all. It is an evasion of the whole difficulty. In comparison, how to repay the war loan, and how to maintain our position in the markets of the world are matters of quite secondary importance, since the whole future of our civilization depends upon our capacity to deal successfully with unemployment. Failure means not only revolution, but a relapse into anarchy and barbarism.

“But,” it will be said by the advocates of this insane policy, “making good the shortage which has been occasioned by the war, the revival of agriculture, protection for home markets, and bounties for key industries will provide work for some time to come, and so there is no immediate danger. Unemployment there probably will be, but it will not be of such dimensions as to imperil the stability of society.” To which I answer that though by such means we may put off the evil day, they leave the central problem essentially unaltered. The reason for this is to be found by again enlarging its area. For the problem is really an international one. All the other belligerent nations will have to face the same problems as ourselves. If we adopt maximum production, they in turn will be compelled to adopt it in self-defence, while in so far as by means of Protection and bounties we encourage home industries at the expense of foreign ones, the result will be a decreased purchasing power in other nations, which in turn will deprive us of markets for our surplus goods. On this issue the Free Trade argument is perfectly sound. I am in favour of Protection for other reasons—for military and political reasons, and because apart from it the regulation of our internal economic arrangements will remain impossible. But the idea that by means of Protection our volume of trade can be increased appears to me to be altogether illusory.

I said that if we adopt maximum production other nations will be compelled to do the same in self-defence. Where shall we be then? The competition will be more severe than ever. Profits will decline, and how is that going to help us to repay the war loan? So that finally we see that maximum production defeats its own ends, even from the point of view of its promoters. Sooner or later the truth will have to be faced (and the sooner the better) that the only way to repay the war loan is to effect such a radical revolution in our methods of taxation as w r ill enable the wealthy class to liquidate the debt among themselves. All efforts of the wealthy to evade their responsibilities by attempts to shift the burden on to the shoulders of other classes must in the nature of things not only fail in the end, but will be accompanied by a measure of retribution that they will not easily forget. The new world, it is true, is going to be different from the old, but it rests with the wealthy class whether the transition is going to be one of orderly progression or revolution. For if it be true, as I have already shown, that industrialism before the war had reached its limit of expansion, then it follows that the reorganization of industry on a basis of scientific management must be accompanied by the growth of a permanently unemployed class—a class which tends gradually to increase. For, as the whole underlying basis of modern industry is one of expansion, it follows that once the limit is reached, contraction must take its place. Here again there will be no stopping the tendency, once it gets fairly in motion, apart from a return to those first principles of social organization which we abandoned four hundred years ago.

While maximum production is calculated to make trouble for us in the markets, scientific management will make trouble for us in the workshop. It is not a policy calculated to pour oil on troubled waters, but rather to add fuel to the flames of discontent. For the moment appearances are to the contrary. Labour Ministers have been brought into line, and are doing their best to induce the workers to scrap their old prejudices in favour of limitation of output, while holding out promises of increased earnings if they will join hands with the employers in an effort to increase the volume of production by accepting scientific management. But promises are one thing and fulfilment is another. The workers’ instinct in favour of limiting output is not altogether a prejudice, though it may appear as such to capitalists and others. On the contrary, it is born of experience, and an experience not to be gainsaid. The workers know that such a policy keeps them employed, whereas when more than the average is produced the markets are glutted and unemployment results. This has been the experience of the maximum production policy in America, where a factory will work at full pressure for several months and then close down until its surplus stock can be disposed of. It is experiences of this kind which have led the American Labour Unions to adopt an attitude of uncompromising hostility towards scientific management. It may be possible for our Labour Ministers to persuade the workers to give it a trial. But they will not acquiesce for long, for the old difficulties will soon reappear, and then the old troubles will begin again.

But there are other and deeper reasons for the hostility of labour. Scientific management irritates the workers. They dislike the kind of supervision which it entails. Labour is essentially human and does not care about being scientifically managed. Its idea is to manage industry some day itself, and so it naturally looks with suspicion upon a system which proposes to deprive the worker of what remains of his skill and to transfer all labour knowledge to the management. For scientific management is a good scavenger. It is out for every scrap of trade knowledge it can get. Following the machine, it proposes to clean up the last vestiges of craftsmanship, and to put the ship-shape touches to modern industry, “Each one of these ‘scientific’ propositions is perfectly familiar to the workman in spite of the rather naive assurance of the efficiency engineers that they are new. He has known them in slightly different guise for a century past. The new thing is the proposition to develop what has been in the past the tricks of the trade into a principle of production. Scientific management logically follows, and completes the factory process.”\footnotemark[2]

It is important to note that it \emph{completes} the factory process. As such it is a cul-de-sac. Mr. J. A. Hobson, in an article on scientific management, brings home the truth of that assertion. “Indeed,” he says, “were the full rigour of scientific management to be applied throughout the staple industries, not only would the human costs of labour appear to be enhanced, but progress in the industrial arts itself would probably be damaged. For the whole strain of progress would be thrown upon the scientific manager and the consulting psychologist. The large assistance given to technical intervention by the observation and experiments of intelligent workmen, the constant flow of suggestion for detailed improvements would cease. The elements of creative work still surviving in most creative labour would disappear. On the one hand there would be small bodies of efficient taskmasters carefully administering the orders of expert managers; on the other, large masses of physically efficient but mentally inert executive machines. Though the productivity of existing industrial processes might be greatly increased by this economy, the future of industrial progress might be imperilled. For not only would the arts of invention and improvement be confined to the few, but the mechanization of the great mass of workmen would render them less capable of adapting their labour to any other method than that to which they had been drilled. Again, such automatism in the workers would react injuriously upon their character as consumers, damaging their capacity to get full human gain out of any higher remuneration that they might obtain. It would also injure them as citizens, disabling them from taking an intelligent part in the arts of political self-government. For industrial servitude is inimical to political liberty. It would become more difficult than now for a majority of men, accustomed in their workday to mechanical obedience, to stand up in their capacity as citizens against their industrial rulers when, as often happens, upon critical occasions, political interests correspond with economic cleavages.”\footnotemark[3]

There is one comment to make on this quotation. Mr. Hobson’s reference to “large masses of physically efficient executive machines” does not receive medical support. \emph{American Medicine} comments editorially on the result to labour of efficiency schemes designed to relieve it of “wasted” effort.

“Working along with his partner the efficiency engineer, the speeder-up has managed to obtain from the factory worker a larger output in the same period of time. This is done by eliminating the so-called superfluous motions of the arms and lingers i.e. those which do not contribute directly to the fashioning of the article under process of manufacture... The movements thought to be superfluous simply represent Nature’s attempt to rest the strained and tired muscles. Whenever the muscles of the arms and fingers, or of any part of the body for that matter, undertake to do a definite piece of work, it is physiologically imperative that they do not accomplish it by the shortest mathematical route. A rigid to-and-fro movement is possible only to machinery; muscles necessarily move in curves, and that is why grace is characteristic of muscular movement and is absent from a machine. The more finished the technique of a workman and the greater his strength, the more graceful are his movements, and, what is more important in this connection, vice versa. A certain flourish, superfluous only to the untrained eye, is absolutely characteristic to the efficient workman’s motions.

“Speeding-up eliminates grace and the curved movements of physiological repose, and thus induces an irresistible fatigue, first in small muscles, second in the trunk, ultimately in the brain and nervous system. The early result is a fagged and spiritless worker of the very sort that the speeder-up’s partner—the efficiency engineer—will be anxious to replace by a younger and fresher candidate, who, in his turn, will soon follow his predecessor if the same relentless process is enforced.

“It will always be necessary to consider workers as human beings, and charity and moderation in the exaction of results will usually be found the part of wisdom, as representing a wise economy of resources. This scientific charity, however, is something quite apart from the moral effect on the personnel of due recognition of their long service, and of loyalty which is likely to accompany it.”\footnotemark[4]

So after all it appears that the workers’ prejudice is not altogether without some foundation, and as it so happens that the workers are masters oi the position to the extent that they must be willing to co-operate with the efficiency engineer if a scheme is to be evolved suitable to a particular trade, the pill has to be gilded if they are to swallow it. This is the secret of the bonus system and promises of high wages, as it is doubtless the secret of the Whitley scheme. For, according to Mr. F. W. Taylor, its pioneer, scientific management requires of industry a new ethical standard, and involves a complete revolution both on the part of the management and the men. But if I am not mistaken, the anxiety of our new industrialists to introduce this new ethical standard is a case of crying peace, peace, when there is no peace. For industrialism has exhibited disruptive tendencies since the day of its birth—disruptive tendencies which have hitherto only been held in check by the military organization. But for the military, industrialism could never have been introduced. The Luddite anti-machinery riots bear witness to the opposition that had to be overcome, while every stage of its development has been punctuated by the military on whose assistance capitalists have been able to rely in their warfare with the workers for the suppression of riots which developed out of strikes. So there is a sense in which it may be affirmed that industrialism and militarism rest to-day on a common foundation. The war, as I have already shown, was precipitated by the economic crisis which had overtaken industrialism in Germany. The idea that militarism could be abolished and industrialism retained is quite illusory. For if militarism went, a check would be removed which so far has prevented industrialism from bearing its bitterest fruit. The workers would rise against its tyranny if they felt that they no longer need submit, and it looks as if scientific management would bring the trouble to an issue. Under the new dispensation it is to play the part of \emph{agent provocateur} until the workers rise and rebel.\footnotemark[5]

Meanwhile, there is some consolation in the fact that as every industrialized nation after the war will be confronted by the same problems, all the nations involved in the struggle are learning the same lesson at the same time. All of them will discover that industrialism is a cul-de-sac from which the only escape is backwards. There is reason therefore to hope that beneath the fierce and cruel oppositions of the hour a profound principle of unity is at work, and that when after the war the dream of a glorified industrialism is dispelled, common action may be taken to put an end not only to militarism, but also to the industrial warfare of which it is the bitter fruit.

\footnotetext[1]{After the Franco-German War the French saved themselves by putting in hand extensive building operations, or, in other words, by spending money. The defect of our Housing Scheme from this point of view is that it is not being undertaken in order to spend money, but as an investment. This different spirit betrays the lack of insight into economic questions by the governing class.

}\footnotetext[2]{\emph{American Labor Unions}, by Helen Marot (Henry Holt \& Co., New York),

}\footnotetext[3]{J. A. Hobson, \emph{Sociological Review}, July 1913.

}\footnotetext[4]{\emph{American Medicine}, April 1913, quotation from \emph{American Labor Unions}, by Helen Marot.

}\footnotetext[5]{The relations of industrialism and militarism are discussed in other terms in Mr. L. P. Jack’s book \emph{From the Human End.}

}\chapter{The Return to Mediaevalism}
\label{chapter-3}
A consideration of the issues raised in the foregoing chapters points to the conclusion that capitalism is about to commit suicide. Having reared the industrial system upon a basis of social and economic injustice, capitalists are driven from one desperate expedient to another in a vain effort to attain economic, stability. But these efforts will avail nothing, for the crisis ahead cannot be met by men whose primary interest is in maintaining the capitalist system. Hence their dilemma.

It is because industrialism is finally based upon social injustice that the balance between demand and supply has been upset For this phenomenon is but the reflection in the economic sphere of the destruction of the balance of power in the body politic which followed the destruction of the Guilds at the time of the Reformation, when the people lost control of those things which immediately affected their lives. Uncontrolled by Guilds, industry could no longer be related to human needs. It became subject to mass movements entirely incapable of control by any human agency whatsoever, whether collective or individual, and it has gone on floundering ever since, while Parliament, which came to usurp all power in the State, has in turn been drawn into the sweep of these invisible world-currents.

In one sense it is true to say that the present state of things marks a condition into which civilization has drifted, and is the result of no policy, no forethought, no design. And yet in another sense this is not true. The modern State has become what it is because for the last four hundred years the governing class have sought to perpetuate the injustices established by the Reformation. It was because the governing class was living on the plunder of the monasteries and the Guilds that they were in the past led to blacken Catholicism, to condone usury, to misrepresent the Guilds and to give support to .false political and economic theories. They did this because in no other way could they justify themselves. While they denied the people the right to manage their own affairs through the agency of Guilds—the only institution through which the people are capable of exercising control—they found that they themselves were unable to control the economic situation. When they found that their meddling only made matters worse, they came to drift, to adopt the policy of \emph{laissez-faire}, which the force of circumstances has brought to an end, but which leaves them in a sad dilemma. For whereas things have reached such a pass that something must be done, they find that not only are they without any rational social theory to guide them in the task of reconstruction, but that the prejudice against Mediaeval society which has been created by lying historians in the past stands in their way, because it has led men to look with suspicion upon all normal social arrangements. In rejecting the Guild, political philosophers denied the chief corner stone of any sane political theory, and have in consequence been driven into error after error and into compromise after compromise in a vain endeavour to find solutions to problems which for minds with their perverted outlook are insoluble.

To Mediaeval social arrangements we shall return, not only because we shall never be able to regain complete control over the economic forces in society except through the agency of restored Guilds, but because it is imperative to return to a simpler state of society. Further development along present lines can only lead to anarchy. For anarchy is the product of complexity. It comes about in this way: the growth of complexity leads to confusion, because when any society develops beyond a certain point the human mind is unable to get a grip of all the details necessary to its proper ordering. Confusion leads to misunderstandings and suspicions, and these things engender a spirit of anarchy. No one will deny that such a spirit is rife to-day, and it is difficult to avoid the conclusion that it is a sign that modern society is beginning to break up. We are certainly beginning to turn the corner, and once it is turned there will be no stopping until we get back to the Mediaeval basis. We shall travel of course by stages. But we shall get there eventually because we shall find no rest, no stability, until we reach our destination. There will be no stopping at any half-way house; so much is certain.

Meanwhile it is interesting to note how Mediaeval economic principles are insinuating themselves into latter-day practice as a consequence of the force of circumstances. We have not yet attained to the Mediaeval conception of a Just Price, but the necessity of putting a boundary to the depredations of the profiteer has revived its Mediaeval corollary–the Fixed Price. Being a practical people with machinery as our god, we indignantly repudiate the idea that it is in the interests of society that machinery be controlled. Yet all the same machinery is being controlled in Lancashire and Yorkshire to-day\footnotemark[1]—it is true as measures of war emergency consequent upon the shortage of cotton and wool, but it is none the less significant on that account; for if the war is not to be regarded as a colossal accident but as something towards which the whole modern polity inevitably tended then we may be sure that the forces at work which make control necessary to-day will make it necessary in the future. The cotton shortage may come to an end; but Lancashire is losing its Indian market because of an adverse tariff, as indeed it is losing other markets through the growth of competition—circumstances which bring home to us the fact that industrialism has reached its limit of expansion. Wisdom might have suggested years ago the desirability of regulating the output of cotton. For it would surely have been better to have introduced such regulations than to be for ever lowering the standard of quality in order to adjust the balance between demand and supply which the use of an ever-increasing number of spindles necessitated. Is it not strange that nothing short of a war of universal dimensions could induce Lancashire to face up to the situation? I should like to believe that wars would be impossible in the future, but the unwillingness or inability of mankind to face the simple facts of society apart from them does not leave much room for hope.

The examples I have given of the tendency of latter-day economic practice to follow Mediaeval lines are interesting, but the strongest evidence of all in support of the hypothesis that a return to Mediaevalism is essential to the preservation of society is to be found in the success of the National Guild movement which proposes to transform the Trade Unions into Guilds. For there is historical continuity in the idea, inasmuch as the Trade Unions are the legitimate successors of the Mediaeval Guilds, not only because the issues with which they have concerned themselves have arisen as a result of the suppression of the Guilds, but because they acknowledge in their organization a corresponding principle of growth. The Unions to-day with their elaborate organizations exercise many of the functions which were formerly performed by the Guilds—such as the regulation of wages and hours of labour, in addition to the more social duty of giving timely help to the sick and unfortunate. Like the Guilds, the Unions have grown from small beginnings until they now control whole trades. Like the Guilds also, they are not political creations, but voluntary organizations which have arisen spontaneously to protect the weaker members of society against the oppression of the more powerful. They differ from the Guilds only to the extent that, not being in possession of industry and of corresponding privileges, they are unable to accept responsibility for the quality of work done and to regulate the prices, The National Guild proposal therefore to transform the Trade Unions into Guilds by giving them a monopoly of industry is thus seen to be an effort to give conscious direction to a movement which hitherto has been entirely instinctive—which is, to use Mr. Chesterton’s words, “a return to the past by men ignorant of the past, like the subconscious action of some man who has lost his memory.”\footnotemark[2] And the propaganda has met with a phenomenal success—a success which I have some right to say has been out of all proportion to the amount of work put into it or the means at the disposal of its advocates, and which therefore can only be finally explained on the assumption that it voices a felt need; that the balance of power in society has become so upset that men instinctively support the Guild idea as a means of restoring the equilibrium.

It is safe to say that the Guild propaganda would not have been followed with the success it has had but for the co-operation of certain external happenings. In the first place there is the growing distrust of Parliament and centralized government. In the next there is the increasing sense of personal insecurity and loss of personal independence which has followed the growth of large organizations. Then there is the war and the Munitions Act, which gave the workers a taste of Collectivism and the enormous growth of bureaucracy, which has brought home to many people the utter inadequacy of such a method for meeting really vital problems. In consequence almost everybody has come to feel that some fundamental change must be made, and as the road forward is impassable, there is no alternative but to go back. I am aware of course that many National Guildsmen would not go to such lengths. Their concern is with the problem of transforming the Unions into Guilds, which they can justify as going forward. All the same it is a step backwards of a very fundamental order, for it is nothing less than a proposal to reverse the practice and judgment of the last four hundred years. I say “practice and judgment,” but I place practice first because I do not seriously think that the present state of things owes its existence to any reasoned judgment whatsoever. It was established first by force and attempted justifications were made afterwards. That is the history of all modern ideas.

We may agree with the National Guildsmen that the first step is for the workers to take over the control of industry, and that in order to do this they must for the present accept industry as it actually exists. {[}\textasciicircum{}3{]} But if they are not to be involved in the catastrophe which threatens the modern world, they should be sufficiently frank with themselves to know in what direction we are travelling; for there will be no time to discuss properly the issues involved when the transfer actually takes place. One fundamental issue—the incompatibility of democratic control with highly centralized organization—is being realized, so there is nothing to fear in that direction. No difficulties are likely to be put in the way of the growth of local autonomy. The trouble is likely to come over the unemployed problem which will certainly follow the demobilization of the forces and the closing down of the munition factories in spite of the shortage which must be made good. National Guildsmen will be as powerless as capitalists to face this problem unless in the meantime they make up their minds in what direction society is travelling.\footnotemark[3] Socialists generally have not emancipated themselves entirely from Capitalist ways of thinking. Almost without exception they still think about finance in commercial terms, while Guildsmen have not always learned to think primarily in the terms of things. Yet Guild finance must differ as fundamentally from commercial finance as Guild organization differs from commercial organization. To make a long story short, Guild finance means the abolition of finance as we understand it. For finance to-day means nothing more than finding ways and means of using money for the purposes of increase, and obviously Guilds can have nothing to do with such a motive. It follows that in proportion as the Guild principle of fixed prices can be applied, opportunities for making money by the manipulation of exchange will tend to disappear, while in proportion as the workers come into the possession of industry, opportunities for investment will likewise come to an end. Bookkeeping there will be, but bookkeeping is not what we understand by finance. From this point of view the primary aim of the Guild is to guard society against the evils of an unregulated currency by restricting currency to its legitimate use as a medium of exchange.

{[}\textasciicircum{}3{]} Something approximating to National Guilds was organized under the Menshevik Regime in the Russian Revolution. But the good work which was then done was rendered nugatory by the action of the Bolsheviks, who, raising the cry that the capitalists were creeping back to the control of industry, urged the workers to elect to their Workshop and Factory Committees not those best qualified to administer the work, but those who were the exponents of Bolshevik views. It was thus the reign of the demagogue was inaugurated in Russia and industrial chaos made its appearance. It is to be hoped that we shall have the sense not to fall into this pitfall.

The introduction of a change so fundamental in the conduct of industry will create a host of problems with which it will be necessary to deal. For with the change many occupations will automatically come to an end, and if society is not to relapse speedily into anarchy it is important that the situation should be intelligently anticipated. All who find themselves unemployed should be put upon free rations until such time arrives as they can become absorbed in the new social system. There is no other way of preventing bloodshed. Meanwhile the surplus workers should be put upon the land, for not only would this measure have the merit of immediately relieving the situation, but the revival of agriculture would confer the permanent benefit of strengthening society at its base, while it would react to restore normal conditions in industry. Of course some discrimination would need to be shown, as in the case of old people who would be unable to adjust themselves to the new conditions and should be pensioned off.

While the revival of agriculture would relieve the unemployed problem, it would by no means solve it. Such a \emph{desideratum} can only be reached by such a complete change in the purpose and scope of industry as is involved in the substitution of a qualitative for the present quantitative ideal of industry. This is a big question, and presupposes a revolution not only in our methods of production but in our ways of thinking, habits of life and personal expenditure. As I have discussed this question and its implications at some length in my \emph{Old Worlds for New},{[}\textasciicircum{}5{]} it will not be necessary for me to repeat the argument I there used. Suffice it here only to say that such a change in concrete terms means the revival of handicraft together with a definite limitation of the use of machinery. That the revival of handicraft would assist us in our efforts to cope with the unemployed problem becomes apparent when we realize that with a reversion to handicraft we should no longer be haunted by the problem of surplus goods which has followed in the wake of unregulated machine production. Anyway it is apparent that if men are unemployed they must either be provided for or left to starve. Would it not be wiser to employ them as handicraftsmen than to compel them to live on doles while being employed on some useless and unnecessary work? This issue must be faced. It cannot be evaded any longer, because nowadays, when there are no new markets left to exploit, it will be impossible to put off the evil day by dumping our surplus products in foreign markets.

To ordinary sane men such reasoning is conclusive. Unfortunately, however, the decision in such matters does not rest with them to-day, but with the “politically educated” members of society—that is with men whose natural instincts have been perverted by the training of their minds on false issues in the supposed interests of capitalists and the \emph{status quo}. That Socialists and Labour men generally are just as much victims of our false academic tradition as members of the governing class does not lessen but increases the danger, for by depriving the working class of their natural leaders, it is surely bringing about the rule of the mob. It is tragic, but still it is nevertheless true to say that, generally speaking, the more highly educated a man is to-day the more likely he is to be wrong. This is the secret of the power of the Northcliffe Press, of the Billing verdict, as of the impotence of our governing class. The feeling against leaders, rightly interpreted, is really a demand for leaders whose instincts are sound. The good men believe the wrong things. That is our root trouble to-day.

\footnotetext[1]{The Cotton Control Board administering the cotton trade in Lancashire states the number of spindles each factory may use. The operatives work a fortnight and then take a week’s holiday for which they are paid, men receiving 25s. and women 15s. The Wool Control in Yorkshire proceeds along similar lines.

}\footnotetext[2]{\emph{A Short History of England,} by G. K. Chesterton.

}\footnotetext[3]{George Allen \& Unwin, Ltd. 3s. 6d. net.

}\chapter{The Spiritual Change}
\label{chapter-4}
The danger inherent in the growing disrespect for all forms of authority is that from being a perfectly legitimate protest against spurious forms of authority and culture it may develop into a revolt against authority and culture in general. To the Neo-Marxist whose faith is absolute in the materialist interpretation of history this may seem a matter of no consequence. But to those who realize the dependence of a healthy social system on living traditions of culture it is a matter of some concern. For whereas a false culture like the academic one of to-day tends to separate people by dividing them in classes and groups and finally isolating them as individuals, a true culture like the great cultures of the past unites them by the creation of a common bond of sympathy and understanding between the various members of the community.

The recovery of such a culture is one of our most urgent needs, for some such unifying principle is needed if society is to be reconstituted. If the overthrow of capitalism is not to be followed by anarchy, this dual nature of the social problem must be acknowledged. For it is apparent that if a change in the economic system is to be made permanent it will need to be accompanied and fortified by a change in the spirit of man. Most Socialist activity to-day is based upon the assumption that one will necessarily follow more or less automatically as a consequence of the other, and that all it is necessary to do is to seek to abolish economic insecurity under a restored Guild system and the materialist spirit would disappear as a matter of course. But such reasoning, I submit, is fallacious. Even granting that it could be proved that the social problem had its origin in a purely economic cause, it does not follow that to effect economic change in the right direction would automatically produce the change we desire on the spiritual side of life, because, as we are all creatures of habit, the materialist habit of mind would tend to persist when the cause which originally created it had been removed. What most Socialists fail to realize is that the material and spiritual sides of the problem must be attacked simultaneously if reaction is not to result. Otherwise it is a certainty that the one which at the moment is left standing would wreck the other. We know that a religious revival to-day would not effect permanent results unless it were accompanied by a change in the economic system. For precisely the same reason a change in the economic system cannot be permanent unless accompanied by a corresponding change in the spirit of man. Apart from a change in the spirit of man, it is conceivable that a restored Guild system, instead of laying the basis of a happy and prosperous society, would, under materialist direction, degenerate into a number of warring groups, in which the groups in an economically weak position would be ground down by those in a stronger one. All the circumstances which now so rightly shock the Socialist conscience would be reproduced. The tree would still only bear thistles, for self-interested human nature must ever inflict suffering on those that are weak. Economic change is therefore impotent to redeem society unless it is accompanied by such a change in the spirit of man as is tantamount to a religious awakening, “For,” to quote de Maeztu, “men cannot unite immediately among one another; they unite in things, in common values, in common ends.”\footnotemark[1] The materialist philosophy of organized Socialism supplies no common aim capable of uniting men for the purposes of reconstruction; on the contrary, it can only unite them for the purposes of destruction, for the overthrow of the existing system. Once that is done, Socialists must split up among themselves, for their lives are governed by no common denominator. Like the builders of Babel, they will be overtaken by a confusion of tongues—for such is the inevitable end of all materialist systems.

The more one thinks about the social problem the more one comes to see that economic health in a community is dependent upon morals; and the more one thinks about morals the more one comes to realize that their roots are finally to be found in religious conviction. Brotherhood is only possible on the assumption that evil motives can be kept in subjection, and the experience of history seems to prove that only a religion which appeals to the heart and conscience of men is capable of this. If evil motives can be kept in subjection, then the kingdom of God upon earth can be realized, but on no other terms. This, I take it, was the central truth and purpose of Christianity throughout its great historic period. By strengthening man it sought to establish and fortify the normal in life and society. That Christians at times have been drawn to other ideals is true, but that the central aim of Christianity was the establishment of the kingdom of God upon earth the wonderful architecture and social organization of the Middle Ages bears witness.

We have moved so far away from the Middle Ages that it is difficult for us to conceive of life as it was then lived or religion as it was then understood. Religion then was not a thing to be indulged in by people who had a bias in that direction and ignored by others—something apart from life with little or no influence on the main current of affairs—but was the creative force at the centre of society; the mainspring and guiding principle that shaped art, politics, business and all other activities to a common end. It was moreover a culture which united king and peasant, craftsman and priest in a common bond of sympathy and understanding; for, unlike modern culture, it did not depend upon books and so did not raise an intellectual barrier between the literate and the illiterate, but united all, however varying the extent of their knowledge and understanding. The mason who carved the ornaments of a chapel or cathedral drew his inspiration from the same source of religious tradition as the ploughman who sang as he worked in the field or the minstrel who chanted a story in the evening. Modern education at the best is a poor substitute for the old culture which came to a man at his work. The utmost it can do is to give us an opportunity of reading in books descriptions of a beautiful life which once existed in reality. And let us never forget that the central mystery around which this life moved was religion. This fact is the last one the modernists are willing to admit. They may be fascinated by the glamour and romance of the Middle Ages, by its wonderful architecture and its social organization. But it may be said of them what Mr. Chesterton said of Ruskin, “that he wanted all parts of the cathedral except the altar.”

In accounting for the changes which destroyed Mediaeval Society and inaugurated the modern world, it is customary in economic circles to ascribe them to the Reformation and the Great Pillage which accompanied it. But the Reformation itself was the consequence of that many-sided movement which we know as the Renaissance which in turn was the direct consequence of that awakened interest in Greek and Roman literature, science and art in the fourteenth century in Italy which followed the Revival of Learning. So that when we search for the impulse which first set in motion the forces which have created the modern world we find it in the labour of scholars who ransacked libraries in their enthusiasm for the culture of the pagan world.

The immediate results of their labour were full of promise. The rediscovery of the literature and art of the ancient world had a wonderfully stimulating effect on the imagination of Europe, inclining as it did at the beginning to give a certain added grace and refinement to the vigorous traditions of Mediaevalism. It seemed, indeed, for a time as if the Renaissance was really what its name implies—a rebirth—and that life itself, casting off the fetters which bound it, was to come to its own at last. But it was not to be. Early in the sixteenth century its morning splendour in Italy received a check, and as time wore on it became more and more evident that the glories of the Renaissance were over and that its tyrannies had begun. For what happened in Italy happened wherever it succeeded in establishing itself. Its immediate effect was always that of a stimulant which for a time quickened things into a vigorous life. After this reaction set in. A kind of staleness overcame everything. Mankind suffered spiritual atrophy. Religion and art withered as a consequence of the forces set in motion, and in spite of attempted revivals, I've never succeeded in becoming properly rooted again, nor will they ever do so until the false values which the Renaissance imposed upon the world are banished. For, briefly, it may be said that the fundamental error of the Renaissance was that it everywhere concentrated attention upon secondary things to the neglect of the primary ones. In its enthusiasm for learning it came to exalt knowledge above wisdom, science above religion, mechanism above art. The misdirection of energy which has followed these false valuations has literally turned the world upside down, so that, like a pyramid balanced upon its apex, it remains in a state of unstable equilibrium. For there can be no peace so long as the major powers which alone are capable of giving direction to society are subjected to the caprice and domination of the minor ones.

It is a fact not without significance that science alone has profited by the changes associated with the Renaissance. I say it is not without significance because science is not a creative but a destructive force. Let there be no mistake about this. Science always destroys. There are of course some things—disease, for instance—which need to be destroyed, and in destroying these science does useful work. But the usefulness of science is strictly limited. As the handmaid of religion and art its services may be invaluable. For it is their function to know the \emph{why} of things, whereas science only concerns itself with the \emph{how}. And in a healthy society the \emph{why} would take precedence to the \emph{how}. When this natural order is reversed and science assumes the leadership, society lives in peril of its existence. For the liberation of natural forces which science aims at effecting is to liberate forces which man is powerless to control. It is no accident that science has become the servant of militarism. Too proud to accept spiritual direction, it was left no choice in the matter.

The materialist spirit which science has helped to engender shows itself irreconcilably hostile to all the higher interests of mankind. All men who care for spiritual things are conscious of this antagonism. But hitherto opinion has been divided as to the best means of combating it. Feeling themselves more or less powerless in the face of the vast mechanism of industrialism, many such men are inclined to take the view that industrialism must be accepted to-day as an established fact, and urge upon all who are conscious of its limitations to seek to supplant its materialist direction by a spiritual one. This view, which has the advantage of appearing broad and magnanimous, has the further one of reconciling men temporarily to the servitude to which they must submit. Nevertheless, it is both impracticable and fallacious. The very magnitude of the industrial system forbids it, thus making of our would be industrial reformers utterly impracticable dreamers. Small machines may be used by man, but large machinery acquires a will of its own. The men who direct it soon find out that they can only remain solvent on the assumption that they are willing to sacrifice everything to the all-absorbing interest of keeping the vast machinery in commission. Hence it comes about that it is the tendency of industrialism to throw out all men who are unwilling to bend their will to the will of the machine. It is in the nature of things that this should be so. For there are only two possible lines of development. Either industry must be brought into relation with what we regard as the permanent needs of human nature, or human nature is not to be regarded as a fixed quantity and must adapt itself to the needs of industry. There is no third position such as the proposed spiritual control of industrialism would suggest.

Though there exists to-day an undoubted antagonism between the material and spiritual sides of life, it has not always been so. Whether such antagonism exists or not is all a matter of proportion. Up to a certain point in the development of civilization no antagonism is felt. The material and spiritual aspects of life go hand in hand. But beyond a certain point this is no longer the case. Separation begins. Henceforth further development of one side can only be at the expense of the other. It is not a case of any one definitely willing this separation. It simply happens as a loss of balance consequent upon an undue concentration upon the problems appertaining to one side of life. In this sense things are to be regarded not as necessarily good or bad in themselves, but may be either according to the proportion they bear to each other. As in chemistry we know that the elements composing any compound substance will combine with others in a certain definite and fixed proportion, and in no other, so it appears that in society the material and spiritual elements can only combine organically when they co-exist in a certain definite proportion.

Exactly what that proportion is it is impossible in words to say. What, however, we do know is that the material side of life is to-day abnormally over-developed while the spiritual side is to an equal extent under-developed, and this is sufficient for practical purposes. For our business being to restore the balance now destroyed, we are right in supporting whatsoever tends to increase spiritual activities on the one hand and to limit material ones on the other. In reality, however, this is not two forms of activity but one, inasmuch as both reforms must proceed simultaneously. The material development is to-day so overwhelming and its force is so irresistible that there can be no such thing as a widespread spiritual reawakening so long as the material crust in which our life is embedded remains unimpaired. That crust will need to be broken before the spirit of man can move freely again, and there is every reason to believe it will be broken before long. For the determination of the Government, capitalists and others to carry the industrial system after the war to its logical conclusion is the surest way of ending it, for all the contradictions which now underlie our civilization will then come into the light of day. Once that happens, the system will not be able to go on. The lie upon which it is built will be out, and there will be no hiding the truth any longer. We shall have to face the facts because the facts will be facing us. Unable so much as to entertain the idea of a limit to material expansion or to conceive of a social order fundamentally different from our own, the governing class are nevertheless unconsciously preparing the way for the new social order by seeking political suicide, which of course is the only thing they can do considering they cannot go forward and are too proud to go back. For “pride goeth before a fall.”

Far be it that any words of mine should deter our governing class from the pursuit of a policy which is so full of beneficent promise for the future of mankind. My concern is not with them, but with the Socialist and Labour movements, which I fear may fall into the same pit. For the situation after the war will be full of dangers for men who have hitherto based their policy upon the assumption that industrialism has come to stay. They have assured themselves so often that “we cannot go back” that they will be entirely helpless when confronted with a situation through which they cannot go forward. If therefore they are not to be taken by surprise, if after the war we are not to go to pieces as Russia did after her revolution, it is urgent that the leaders of the Socialist and Labour movements should pause and think. If they do not, then the col{]} apse of the present order will leave society entirely without leaders, at the mercy of our Jacobins and Bolsheviks, who, like their predecessors in the French and Russian Revolutions, will make the anarchy complete by facing every issue as it arises, not with the understanding which comes from broad and humane sympathies, but in the narrow and mechanical way which is only possible to minds drilled in the materialist misinterpretation of history.

That is where I will leave the matter. I have drawn attention to the danger which threatens us, and I have suggested within certain limits the direction in which a solution may be found. If you ask for a more detailed plan I reply that such is undesirable, for a purpose wedded to details may easily suffer shipwreck. Our need, on the contrary, is an aim sufficiently noble to unite men coupled with an understanding and determination to mould circumstances as they arise. A precedent condition of success upon such lines is a clear and widespread recognition of the problem confronting us as it actually exists. If this could be secured half of the battle would be won, and we need have no fear as to our ability to improvise measures when the crisis comes. Meanwhile two prejudices stand in the way of such a desideratum. One is our utterly irrational faith in the stability of industrialism; the other is an ignorance where it is not a wilful misrepresentation of the past. Let us not forget that in history, as Mr. Chesterton has reminded us, there has never been a Revolution which did not in some measure aim at being a Restoration.

\footnotetext[1]{\emph{Authority, Liberty and Function in the Light of the War}, by Ramiro de Maeztu (Geo, Allen \& Unwin, qs. 6d.).}

\chapter{The Function of the State}
\label{chapter-5}
\begin{quotation}\
	This is the reason why the law was made, that the wickedness of men should be restrained through fear of it, and that good men could safely live amongst bad men; and that bad men should be punished by the law and should cease to do evil for fear of the punishment.

	(From the Feuro Juzzo, a collection of laws Gothic and Roman in origin, made by the Hispano-Gothic King Chindasvinto, A.D. 640. In the National Library of Spain, Madrid.)
\end{quotation}

It is typical of the confusion in which a generation of Collectivist thinking has involved social theory that when to-day men speculate on the attributes of the State in the society of the future they invariably proceed upon the assumption that its primary function is that of organization. The syndicalist, with his firmer grip on reality, realizing that the State is an extremely bad and incompetent organizer, rightly comes to the conclusion that if the State can find no better apology for its existence it is an encumbrance—a conclusion from which I can see no escape for such as conceive organization to be the primary function of the State.

National Guildsmen, though accepting the State as essential to a well-ordered society, have not always been able to escape from this dilemma. Mr. Hobson\footnotemark[1] dismisses the idea of organization being the primary function of the State, but conceives of it as spiritual, though the examples he gives in support of his contention, with the exception of education, namely, foreign policy, public health and local government, appear to me to be more mundane than spiritual. This contention, however, is begging the question. It is not a satisfactory answer to the Syndicalist. It suggests the existence of activities with which a Guild Congress may not be qualified to deal, but it offers us no clear principle for guidance. Mr. Hobson’s understanding of “spiritual” is different from mine; and I would say that if the State cannot justify itself as an organizer, it certainly cannot do so as a spiritual influence. Not only does it not exercise any spiritual influence to-day, but it is questionable if the State has ever done so in the past. On the contrary, the State appears to exercise a baneful influence on whatever spiritual activities it has taken under its protection. Most people would agree that the influence of the State upon the Anglican Church has been a most depressing one; while it is significant that in the one section of this Church which is to-day alive—the High Church—advocates of disestablishment are to be found. Nobody will be found to defend our national educational system or to maintain that the participation of the State in the task of education has in any way fulfilled the expectations of its promoters. Nor, again, can any one maintain that the patronage of the arts by the State exhibits any degree of insight or understanding. It is, I believe, in the nature of things that this should be so, for the State is of the earth earthy. The problem of temporal power which engages its attention does not tend to create an atmosphere favourable to the growth and development of things spiritual.

If, then, the State is not to be justified as an organizer nor can it exercise spiritual functions, on what grounds is it to be justified? The experience of history provides the answer. The function of the State is to give protection to the community—military protection in the first place, civil protection in the next, and economic protection in the last. Let me deal with economic protection first; for if I am to be understood at all it is necessary to make it clear that I refer to something very different from the Protection of current politics. Protection is a double-edged sword and may just as easily be a curse as a blessing. Protection against the economic enemy beyond the seas is the necessary corollary of any stable economic system. But protection against the economic enemy at home is the primary necessity, for it means the protection of the workers against exploitation. It involves a restoration of the Guilds. By chartering these the State gives economic protection to the community.

The connection between an economic protection of this order and military and civil protection may not at first sight be obvious. But a little thought will perhaps show that they are mutually dependent. All these forms of protection have this one thing in common—they seek to guard society against the depredations of the man of prey. Economic protection or privilege is demanded for the Guild in order to prevent the man of prey from securing his ends by means of trickery. Civil protection is demanded in order to prevent the same type of man from securing his ends by means of personal violence. Military protection is demanded in order to secure the community against attacks from without, which is the inevitable consequence of the domination of an adjacent people by men of this type. From this point of view the differing psychology of nations is to be explained. The internationalist may be right in affirming that, taken in the mass, men are very much alike all over the world. But in practical affairs what makes the difference is the type of man that dominates a civilization, for the dominating type gives the tone to a community, and it is that which in politics must be reckoned with.

The manifest truth of this view of the function of the State has been obscured by two things: firstly, by the undoubted fact that in our day the State is very much at the mercy of the man of prey; and secondly, by the acceptance of reformers of Rousseau’s doctrine of the “natural perfection of mankind.” The first may or may not be a reason for giving the existing State an unqualified support, since law is no longer enacted to enable \emph{good men to live among bad}, but to enable \emph{rich men to live among poor}. The second is a more serious matter, because it tends to confirm the man of prey in the possession of the State by standing in the way of the only thing that can finally dislodge him—the growth of a true social philosophy. It has always been a mystery to me why Rousseau’s doctrine should have found acceptance among Socialists. How they reconcile their belief in the natural perfection of mankind with their violent hatred of capitalists I am entirely at a loss to understand. If the domination of the modern world by capitalists is not to be explained on the hypothesis that when the State withdrew economic protection from its citizens by suppressing the Guilds the capitalists, by a process of natural selection, came to dominate the lives of the more scrupulous members of society, then how is it to be explained? To exonerate capitalists from personal responsibility by blaming the “system” is pure nonsense, because it presupposes the existence of a social system independent of the wills of its individual members, and especially of capitalists who are its dominating type. Moreover to speak of capitalism as the capitalist system is itself a misnomer, for it is not in any sense a system. On the contrary, capitalism is a chaotic and disorderly growth, while every effort to bring order into it reacts to increase the prevailing confusion. Socialists are right in hating capitalists; they are wrong in denying the only rational justification for that hatred—original sin. I insist upon a frank recognition of this fact because I do not see how the Guilds are to be restored apart from it. Just in the same way as the modern Parliamentary system is the political expression of the doctrine of the natural perfection of mankind, so the Guild system in the Middle Ages was the political expression of the doctrine of original sin. About this no two opinions are possible. The Mediaevalists realized that rogues are born as well as made, and that the only way to prevent the growth of a cult of roguery such as oppresses the modern world is to recognize frankly the existence of evil tendencies in men and to legislate accordingly. It was for this reason that they sought to suppress profiteering in its various forms of forestalling, regrating and adulteration; for they realized that rogues are dangerous men, and that the only way to control them is to suppress them at the start by insisting that all men who set up in business should conform to a strict code of morality in their business dealings and daily life. Liberalism, with its faith in the natural perfection of mankind, was based upon the opposite assumption—that the best will come to the top if men are left free to follow their own desires. They sought to inaugurate an industrial millennium by denying economic protection to the workers, while they dreamed of a day when military protection would no longer be necessary. Both of these illusions have been shattered by the war, but the doctrine upon which they were built—the natural perfection of mankind—remains to perpetuate our confusion. When it, too, is shattered we may recover the theory of the State.

\footnotetext[1]{\emph{Guild Principles in Peace and War}, by S. G. Hobson (S, Bell \& Son).

}\chapter{The Class War}
\label{chapter-6}
\section{I}
There can be little doubt that the struggle which will decide the form which Socialist thought and action must finally take will be fought between the Neo-Marxists and Guild Socialists. For though the immediate practical proposals of the two movements have sufficient in common for the differences to appear to a Collectivist as the differences between the moderate and extreme parties into which all movements tend to divide, yet they are finally separated by principles which are as the poles asunder, and Socialists must before long choose between them. As the situation develops they must cleave either to a purely materialist or to a spiritual conception of the nature of the problem which confronts us. They cannot remain in their present indeterminate state.

Though a collision between the two movements is inevitable, so far nothing more than skirmishes between outposts have taken place. Yet they are sufficient to indicate upon what lines the attack of the Neo-Marxists is likely to develop Guild Socialism, it appears, is not acceptable to men whose central article of faith is the class war Though Guild Socialism has arisen in opposition to Collectivism, and though, I believe, when has reached its final form, it will be found to be farther removed from Collectivism than Neo-Marxism itself, nevertheless, Mr. Walton Newbold\footnotemark[1] tells us that the Neo-Marxists firmly and honestly believe it to be a bureaucratic variation of Collectivism intended to perpetuate the authority of the middle class.

That the Neo-Marxists should have chosen this line of attack is significant. It testifies to what is uppermost in their minds. For though in their propaganda they demand social justice for the workers, it is manifest that class-hatred rather than the desire for justice is the mainspring of their actions. I hold no brief for the middle class. It has many and grievous faults, and it pays for them dearly in defeat, in isolation, in lack of hold upon the modern world. So far from seeking to save itself in the manner which the Neo-Marxists suspect, it has not to-day sufficient faith to believe it might be successful if it made the attempt, and it is increasingly reconciling itself to an idea of Marx which the Neo-Marxists appear to have forgotten—that the middle class will become merged in the proletariat. Anyway, on no other hypothesis except pure idealism can I explain the action of those middle-class Socialists who have sought to advocate the Guilds. For if they imagine they are going to save the middle class by the promotion of a system of democratic organization in every unit of which they would be in a hopeless minority, then all I can say is that they must be fools of the first order and are entitled to the contempt with which Mr. Newbold regards them. Further, if the Neo-Marxist contention is correct they must explain why the National Guilds League opposed the Whitley Report, for the middle class has certainly nothing to lose by its adoption.

Facts of this kind are not to be gainsaid. The reason why Guild Socialists propose to include the salariat in the Guild is a purely practical one. The simplest way to bring the capitalist system to an end is for the workers to take over the industries of the country as they actually exist. This is common sense and nothing more. Modern industry is a very complex affair, and our daily needs require that the various people concerned in industry can be persuaded to co-operate together. But if any radical change is to be brought about, and the spirit of co-operation maintained, it can only be on the assumption that the workers are magnanimous when they are victorious. This is the way all the world’s great conquerors have consolidated their power; and the workers will never be able to carry through a successful revolution until they understand it. For magnanimity disarms opposition. But to preach the class war is to court failure in advance, for it is to seek the establishment of power, not on a basis of magnanimity, but of suspicion; and this robs victory of its fruits by rendering politically impracticable those very measures which, if enacted, would make victory permanent. In such circumstances, the defeated become desperate, are afraid to give in, and, seeing no hope for themselves in the new order, they band themselves together to restore the old. It is thus that revolution is followed by counter-revolution and the workers are defeated.

The right method, it seems to me, is not to preach revolution, but to preach ideas. It is necessary to form in the mind of the people some conception of what the new social order will be like. When the mind of the people is saturated with such ideas one of two things must happen Either the Government must acquiesce in the popular demand, or revolution will ensue. The former is preferable because, as the change can then be inaugurated with cool heads, it is more likely to be permanent. It is no argument against this method to say that the Labour Party has failed. Firstly, because the Labour Party is an insignificant minority and therefore cannot exercise power; and, secondly, because the Labour Party never made up its mind what it really wanted. This latter reason makes it fairly safe to say that if the Labour Party should get into power at the next election it would not be able to effect radical change. In these circumstances our immediate work should not be to bully the Labour Party, which, in the nature of things, can only reflect opinion, but so to clarify our ideas that unanimity of opinion will make its appearance in the Labour movement. The danger is that the people may succeed to power before ideas are ripe. We might then expect a succession of violent conflicts proceeding from the attempt to realize an unrealizable thing. This is what happened in the French Revolution, when the Jacobins, obsessed with the idea of a democratic centralized government, refused to tolerate any other organizations within the State, thus opposing the formation of those very organizations which render a real democracy possible. The Neo-Marxists by repudiating State-action altogether seem to Guild Socialists to be falling into an error the exact opposite to that of the French Revolutionists. Their society would fall to pieces for lack of a co-ordinating power; if the present order were thrown over in its entirety, it would be impossible to improvise arrangements to meet the situation which would be created. We should be starved at the end of a fortnight.

If starvation has been the fate of Russia, which is an agricultural country, and where the class war in the main has meant only the abolition of landlords, how much more will it be the case in a highly industrialized State like our own which can be maintained only by a very high degree of co-operation, and where the middle class forms such a large proportion of the community. If the working class of Russia could not abolish 2\% of the population without precipitating social chaos, what chance have the working class in this country after abolishing 30\%? On the other hand, if the advice of Guild Socialists is followed and industries are taken over in the first place as they exist, the complete democratization of industry could at the most only be a matter of a few years, for the working class would be in a majority in every Guild.

That a scheme calculated to have such an effect should have originated among middle-class Socialists only appears incredible to Mr. Newbold and his friends because they will persist in approaching every question from the point of view of class. But it is not incredible when we realize that middle-class Socialists are often as much “fed up” with the existing system as members of the proletariat, though perhaps for different reasons. The misunderstanding and Consequent suspicion which Neo-Marxists have for middle-class Socialists is largely due to the fact that different motives bring them into the movement. Viewing everything from a purely economic point of view, the Neo-Marxists are unable to understand that men may be very dissatisfied with the existing state of society though they are in fairly comfortable circumstances. They may dislike the work they are compelled to do, or they may be interested in the arts, or some other subject, and finding commercialism opposed to all they want to do, come to hate the system. The more educated and the more imaginative a man is the more restless he will become under the present system, because the more he may find himself balked and thwarted in life. Most men love to do good work, and they learn to despise a system which compels them to do bad. With the typical Fabian the motive is apt to be purely philanthropic. It is this that has led them astray. They came to support bureaucracy because they wanted an instrument with which to abolish poverty; and in regard to anti-sweating legislation they have proved to be right. Their mistake was to advocate as a general principle a form of organization which is only to be justified under very exceptional circumstances for dealing with exceptional problems.

The idea that bureaucracy is a method of organization peculiarly acceptable to the middle class is a romantic illusion which exists entirely in the Marxist imagination. Some years ago (ten or more) I attended a meeting of the Fabian Society and heard Mr. Webb, while protesting against the attitude of certain Fabians who objected to officials, affirm that under Socialism all men would be officials. The announcement was received in dead silence as something altogether incredible. It was clear even then that Fabians did not altogether relish the idea of society being organized on a bureaucratic basis. Mr. Webb got his own way, not because the feeling of the meeting was with him, but because his critics could not at the time offer any alternative. The triumph of Mr. Webb in the Socialist movement was due entirely to the fact that he was definite and knew exactly what he wanted; whereas those who were opposed to him did not, and those who supported him were entirely unconscious of where his policy was leading. Many evil things come about this way; there are more fools in the world than rogues, and, generally speaking, we are much more likely to get at the truth of things by assuming that most men are fools than by assuming they are rogues. Let us not forget that the road to hell is often paved with good intentions. If Marxists would think more of psychology they would not be so full of suspicions. They would begin to understand that man is a many-sided and complex creature and is not to be explained entirely in terms of economics.

Such an understanding would revolutionize their policy. From being exclusive they would seek to become inclusive. Instead of espousing a doctrine which sets every man’s hand against his neighbour, they would seek the creation of a synthesis sufficiently wide to be capable of welding together different types of men in the effort to establish a new social order. Their present policy leads nowhere. Neo-Marxists may begin by repudiating middle-class Socialists as men whose interests are opposed to those of the working class. But if I am not mistaken, it will not end there. Before long they will be required to repudiate the parasitic proletariat as dependents of the rich; after which they will have to repudiate skilled workers as members of a privileged class. Where will working-class solidarity be then? Nowhere, I imagine; for the working class will be a house divided against itself. I say it will be. Truth to tell, it already is.

\section{II}
While the Guild movement acknowledges a different starting-point from that of the Neo-Marxists, it moves towards a different goal. That goal is symbolized in the word “Guild.” I wonder how many Neo-Marxists have ever pondered over the significance of that word. For it is a symbol of the past a past to which many Guildsmen hope to return. It was not idly chosen. The right to use it had to be fought for. It could not have been used by the National Guild movement had not the formulation of its policy been preceded by a movement or agitation which for a generation sought to remove prejudices against an institution in the past which an ever-increasing number of men to-day are coming to recognize as the normal form of social organization. This battle was fought out among our much-despised intellectuals—by historians, craftsmen, architects and others, who realized that the prejudice which had been created by interested persons in the past against Mediaeval institutions had become a peril to society. Leading men to look with suspicion upon all normal social arrangements, it tended to thwart all efforts to reconstruct society on a democratic basis by diverting the energies of the people into false channels. How much of the discord and ill-feeling which prevails between the different sections of the reform movement had its origin in prejudice against the past it is impossible to say; but it is a certainty that Collectivism as a theory of social salvation could only have been formulated by men whose minds had been formed on a false reading of history. And as the gospel of the class war owes its present popularity to the disappointment which followed attempts to reduce Collectivism to practice, the popular misconceptions of history are to be held responsible for much.

That the Neo-Marxists should consider the Guild movement to be merely a variation of Collectivism shows how completely they misunderstand not only the underlying purpose of the movement, but its history too. For not only are the principles of Collectivism and Guilds fundamentally opposed, inasmuch as the method of the former is control from without by the consumer, while the method of the latter is control from within by the producer, but Guildsmen were accustomed to attack Collectivism long before Marxists came to suspect it. But it was not until Socialists were disillusioned over Collectivism that Guildsmen could get a popular hearing. When in February 1906 my \emph{Restoration of the Guild System}, which contained a destructive analysis of Collectivism, appeared, it was held up to ridicule by the Socialist and Labour Press.\footnotemark[1] And now at last, when the current of opinion has turned in our favour, Mr. Newbold tells us that the Neo-Marxists regard the Guild movement as a variation of bureaucratic Collectivism. This opinion they arrive at, not from any careful economic analysis such as we have a right to expect from men who profess economic infallibility, but because, knowing something about psychology, which they do not, we refuse to join them in the class war; just as if the only differences which could possibly divide vSocialists \textbackslash{}vere differences of policy and that differences of principle were matters of no importance. Twelve years ago they wanted to rend us because we were not Collectivists; to-day, because they imagine we are.

The fundamental differences of principle which separate Guildsmen from Collectivists and Neo-Marxists alike will become more pronounced as the Guild scheme unfolds. The \emph{New Age} has said that National Guilds “is rather the first than the last word in national industrial organization.” It is in this light that the present proposals of the movement should be regarded. If a fuller programme has not hitherto been put forward it is not because Guildsmen will be satisfied with the present minimum, but because a general agreement has not yet been reached with respect to the more ultimate issues. Guildsmen have been forewarned by the fate of Collectivists from advancing a wide and comprehensive programme which has not been properly thought out, since only disaster can follow such a course. All the same, some unanimity of opinion is coming into existence in regard to wider issues, and as, generally speaking, it is in the direction I should like to see things go, I will venture my opinion for what it is worth as to our ultimate destination.

As I interpret the Guild movement, it is the first sign of a change in thought which will seek to I solve the social problem, not by a further development along present lines, which can only lead us to fresh disasters, but by effecting a return to the civilization of the Middle Ages. I do not mean by this that we shall in the future recover every feature of that era or that many things which exist to-day will not be retained in the future. I mean that in the first place we shall resume in general terms the Mediaeval point of view and that this will involve a return to Mediaeval ideas of organization. My reasons for believing this are that I think we are moving into an economic cul-de-sac from which the only escape is backwards; and that if the interests of life are to take precedence of the interests of capital we are inevitably driven into a position which approximates to that of the Mediaeval economists. The whole trend of economic development from Renaissance times onward, which has led to the enthronement of capitalism, has been to reverse the Mediaeval order.

In believing thus that capitalism will reach a climax in its development beyond which it can proceed no farther, I am at one with Marx in his interpretation of the evolution of capitalism. It seems to me that Marx predicted very accurately the trend of capitalist development. He foresaw that industry would tend to get into fewer and fewer hands, but it cannot be claimed that the deductions he made from this forecast are proving to be correct, for he did not foresee this war.\footnotemark[2] Not having foreseen this war, Marx did not foresee the anti-climax in which the present system seems destined to end. And this is fatal to his whole social theory, because it brings into the light of day a weakness which runs through all that he says—his inability to understand the psychological factor, and hence to make allowances for it in his calculations. Marx saw the material forces at work in society up to a certain point very clearly and from this point of view he is worthy of study. But he never understood that this was only one half of the problem and finally the less important half. Although Marx clearly foresaw the trend of economic development, he did not see that it had been accompanied by a loss of spirituality, and that simultaneously with the concentration of attention upon material things, religion and art had lost their hold over men. From this historical consideration it may be affirmed that the spirit of avarice grows in inverse ratio to the interest anc activity in religion and art. And as both of these activities were undermined by the changed outlook towards life and the forces set in motion by the Renaissance, the spirit of avarice became triumphant. In the same way that an epidemic to which healthy people are immune tends to spread rapidly among people of a low physical vitality, so avarice claims its victims among people to-day because, owing to the separation of religion and art from life, the mass of the people live in a state of low spiritual vitality.

An understanding of what I may call “the spiritual interpretation of history” will bring us nearer to an understanding of the Guild movement. It has been well described as a religion, an art and a philosophy, with economic feet. That is really what it is. For its aim is nothing less than to restore that unity to life which the Renaissance destroyed. Recognizing that every social system is but the reflection of certain ways of thinking certain ideas of life it seeks to change society by changing the substance of thought and life. But, unlike other movements which have aimed at spiritual regeneration, it deems it advisable to begin at the economic end of the problem in the belief that it is only by and through attacking material and concrete evils that a spiritual awakening is possible. For to quote the words of Mr. de Maeztu\footnotemark[3] “men cannot unite immediately among one another; they unite in things, in common values, in the pursuit of common ends.”

We can agree with the Neo-Marxists in recognizing that under the existing economic system the interests of capital and labour are irreconcilably opposed, and that no compromise is possible. Where we differ from them is in respect of issues about which we are not prepared to compromise. They envisage the problem primarily in the terms of persons and as a warfare between the classes We, on the contrary, see this conflict of interests as the inevitable accompaniment of a materialist ideal of life which rejects religion and art with their sweetening and humanizing influence. Tracing the existence of the problem to a different origin, we naturally seek for it a different solution. We meet the Marxist affirmation that the problem is material by affirming that it is both spiritual and material. And we part company by reminding them that “man does not live by bread alone.”

Finally, I would plead for a more generous attitude of mind among the various sections of the Socialist movement. If the existing economic system based upon competition is to be replaced by one based upon co-operation, the communal spirit must be substituted for the present individualist one. But the no-compromise policy of the Neo-Marxists tends to postpone the arrival of that spirit indefinitely by sowing the seeds of discord and suspicion everywhere. All movements rest upon trust and confidence, and these are impossible apart from a certain charity of spirit which will make some allowance for human weakness and mistaken judgments. For all men at times are apt to err. Would it not be wiser, therefore, instead of always accusing others of interested motives, to try first to understand them—to see whether difficulties are not to be explained on other grounds? If Neo-Marxists refuse such counsel and still maintain that their suspicions are justified and that only self-interests prevail, then in the name of logic I do not see how even they can claim to be an exception to this rule. What guarantee have we that they, like others, are not on the make? How are we to know that they are not seeking the support of the working classes for their own selfish ends? I do not say that this is so. What I do say is that it is the logical deduction from their position. And it is a deduction from the consequences of which they may not be able finally to escape. For if, by some chance, power should pass into their hands, they will be expected to live up to their promises. When they are in difficult circumstances, as all men in power find themselves at times, and have to choose between two evils, they must not be surprised if those whom they have had no option but to disappoint apply the same standards to themselves. It will be no use for them to plead extenuating circumstances, for extenuating circumstances are no part of the Neo-Marxist philosophy. And they must not expect more generosity from their supporters than they have extended to others. Out of fear of them they will be driven from one act of desperation to another, until finally they bring into existence a circle of enemies sufficiently strong to encompass their downfall. And their enemies will show them no mercy. Such was the fate of the uncompromising Jacobins of the French Revolution, and if I am not mistaken it will be the fate of Lenin and Trotsky to-morrow It is the fate of all political extremists who seek to establish power on a basis of suspicion.

\section{III}
Though the criticisms which Mr. Newbold has made against middle-class Socialists can be easily refuted, it is possible they have not been finally disposed of, inasmuch as the differences are much more fundamental than a mere misunderstanding. As always happens in respect of issues of a fundamental nature, people find it extremely difficult to say exactly what they mean, and it may be that the Neo-Marxists in their relations with the middle-class Socialists feel an instinctive antipathy which so far they have been unable to define.

Whatever may be the explanation of the antipathy shown by Mr. Newbold, I can scarcely think he really means what he says when he questions the right of middle-class Socialists to take part in Labour activities; for on that basis not only would he, as a middle-class person, be excluded, but it may be said that nearly all Socialist literature has been written and all the pioneer work has been done by middle-class persons, so that but for their assistance the Socialist movement would never have come into existence. I conclude, therefore, that he must mean something else.

It has been suggested that the secret of the trouble may be that Labour has “come of age,” and in consequence the advice of middle-class Socialists is resented much in the same way that a son is apt to resent the advice of a father who fails to realize that his son has grown up. The father’s advice may be right, but it is necessary for the son to act on his own initiative in order that he may feel his feet in the world.

Though this is an explanation of the estrangement, it does not satisfy me. I can scarcely think that the Labour movement is so short-sighted as to resent advice given by those outside of its class if it found such advice really helpful. The trouble is, I think, that until quite recently, when the Guild propaganda began to make headway, the intellectual leadership of the Socialist movement was entirely in the hands of the Fabians, and I fear they have queered the pitch for us. For their sympathies were not really democratic. It was poverty rather than wage-slavery they were anxious to abolish, and so, instead of seeking to interpret the subconscious instincts of the workers and to direct them into their proper channels, they sought to impose an economic system upon them which left human nature entirely out of account. As might have been expected, human nature has rebelled. The workers, having thrown over Collectivism, are trying to grope their way towards a solution of their problems. Left to their own resources, the workers have undoubtedly seized upon an important truth—that any solution of the economic problem must come as the result of a struggle—a truth that Guildsmen alone among intellectuals have recognized. Meanwhile, the repudiation by Labour of its leaders is not to be interpreted as a denial of the necessity for leadership, but rather as a protest against leaders who cannot lead, because their eyes are turned in the wrong direction.

Looking at the situation from this point of view, our immediate need is to define our position in regard to industrialism in terms that admit of no ambiguity. As a means towards this end it is imperative that we should in the first place not only look round and take stock of the situation which is developing, but anticipate within certain limits the situation which will have to be faced after the war. In this connection everything points to the coming of a great struggle between Capital and Labour. At the moment Labour has Capital at a disadvantage. But after the war Capital intends to get even again. According to all reports capitalists are everywhere sharpening their knives, determined, if they must die, that they will die fighting. Though I doubt not that in the long run Labour will be triumphant, I am by no means sure that victory will follow the first encounter—unless the Army makes common cause with Labour when it returns from France, which is not at all unlikely when we consider the bitter resentment which has been caused by the utterly inadequate pay and separation allowances. But in any case the outlook is not immediately very promising whichever side wins. If Capital is victorious we shall be committed to an industrial policy \textasciicircum{}which can only eventuate in further wars; for a state of things in which war is an ever-present contingency must be the inevitable consequence of the insane policy of for ever seeking to effect an increase in the volume of production, remembering that markets were already filled to overflowing before the war. On the other hand, if Labour wins, the immediate prospects are no more reassuring. There is a danger that in such an event we may pass through all the phases common to social revolutions ere sanity will prevail.

I say there is this danger. I do not, however, think it is inevitable. Whether or no we pass through all these phases depends upon the extent to which we can intelligently anticipate possible happenings in the future and can guard ourselves against pitfalls. This task should not be impossible, considering that we have the experiences of the Russian Revolution to draw upon. In our anticipated revolution, as in the Russian, the moderate party will come first. For we may be assured that whenever the Labour Party arrives with a majority in the House of Commons it will be composed of moderate men. It is the very moderation of the Labour Party that will be its undoing, for it will be unable to act decisively in any direction. This is easily understood when we remember that its members are held together by no common bond of principle. It is only necessary to read the reports of the Labour conferences to realize that the Labour Party does not know where it stands. Though Collectivism as a social theory is entirely discredited, the Labour Party is still vaguely Collectivist in one direction, while in the other its members are simple trade unionists with no general social theory—vaguely Liberal if they are anything at all.

Naturally it will be impossible for such a heterogeneous body to act with any unanimity and decision. It will be the old story over again. Just as after 1906, when the workers were dis appointed with the doings of the Labour Party they turned against it in violent disgust and inaugurated an internecine warfare which continued almost until the outbreak of war, so it may be expected that a similar disgust will follow the establishment of a Labour Government. For it will dilly-dally with things, and all its actions will be feeble. Then the great crisis will arrive, and our future history will depend entirely on the way it is met. Once confidence is destroyed in moderate men, there is a danger of things rushing to the opposite extreme. The Neo-Marxists (our Bolsheviks) will get their chance. They will point to the impotence of the Labour Party, accuse its leaders of lack of courage and a desire to make terms with the enemy and conspire to seize power and inaugurate the class war. If they succeed we shall go the way Russia has gone—to anarchy. But there is no reason why they should succeed. It will be our fault if they do. The situation could be steadied by a vigorous propaganda which would change the basis of the struggle from a warfare about persons to a warfare about ideas or things. Let me explain.

It is apparent, when we think about it, that the anticipated failure of a Labour Government could be accounted for in one of two ways. It could be ascribed to the corruption and moral cowardice of its members, or it could be attributed to lack of ideas—the absence of a social theory adequate to the situation which confronted them. The Neo-Marxists, envisaging the problem primarily in the terms of persons as a warfare between classes, would doubtless seize upon the personal aspect of the failure. Guildsmen, I hope, would be more generous in their criticisms. They should not accuse the Labour men of being knaves when they are transparently as innocent as fools. For who but fools would imagine it possible to find a solution to a political and economic problem the like of which has never been seen in history merely by means of a parliamentary majority united not by the possession of common principles but only in common aspirations? Who but fools could imagine that a majority so constituted could stand for one moment the shock of actuality? Realizing that the failure of a Labour Government may safely be predicted from its entire absence of social principles, Guildsmen should take every opportunity of driving this point home, insisting that goodwill is no substitute for ideas. They should, moreover, be careful to point out that Neo-Marxists differ from the Labour Party only to the extent of substituting ill will for good will inasmuch as the Labour Party and the Neo-Marxists have alike occupied their minds entirely with the problem of how power may be won to the utter neglect of the problem how it may be retained and used.

Not only are the Neo-Marxists without any social theory in the sense that they have never applied themselves to the task of elaborating the principles upon which a democratic and communal society must rest, but they appear to be unaware that one is necessary. All they see is that power to-day is in the hands of capitalists, and they want to see it transferred into those of the workers. That is very good so far as it goes. But it is insufficient for the purpose of reconstructing society, which they would be called upon to do if ever they succeeded to power; because if industry suddenly changed hands and the salariat were banished, as they propose, everything would not go on sweetly as before. The centre of gravity of industry would have completely changed. This change would introduce a host of problems that would demand immediate solution. It is vain to suppose that without clearly defined principles to guide them men unaccustomed to power would prove equal to the task. They would be like amateurs in possession of a powerful and unfamiliar weapon which, mishandled, would be much more likely to destroy them than the enemy.

As herculean a task as the solution of the economic problem is for any Government, its difficulties will be increased a thousandfold for the Neo-Marxists if ever they get into power; for their class-war policy carried into execution will complicate the economic problem by a psychological one of equal magnitude which, like the Bolsheviks, they will have no idea how to meet except by force. Now force in the hands of materialists always produces the very opposite effect to that which is intended, for materialists never understand psychology. But I fear it is useless to reason with Neo-Marxists about such things. They will never know anything about these problems until they are up against them, when they will be the most surprised people in the world.

Recognizing, then, the danger which would follow the success of the Neo-Marxists in such a crisis, Guildsmen should, by an intelligent anticipation of events, take measures to protect their flank. They should inaugurate a vigorous propaganda against the impossibilism of the Neo-Marxists. If in such an effort they are to succeed, it is essential before all things that the good faith of the Neo-Marxists be taken for granted, and that Guildsmen should seek to discredit them by carrying NeoMarxist ideas to their logical conclusion, showing how their excess of zeal must defeat their own ends by provoking reaction, since the mass of the people will become so weary of the anarchy which must follow the inauguration of the class war, that they will come to welcome a return of the old regime merely for the sake of peace and quietness. It should not be difficult to drive these truths home considering that both the Russian and the French Revolutions provide abundant illustrations of how class warfare fails to achieve its ends.

Further, Guildsmen must show the Neo-Marxists that their ideas are not only subversive of others but of themselves. Neo-Marxists are very fond of insisting “that the method prevailing in any society of producing the material livelihood determines the social, political and intellectual life of men in general,” but it never apparently occurs to them to make the deduction that in that case they and their gospel also become a part of the disease of society—a deduction which is not only evidenced by the fact that the Neo-Marxist gospel finds its warmest supports in those districts where industrialism is most highly developed, but that Neo-Marxists are so much a part of the system as to be incapable of imagining any other. They do not propose to change the system, but only its ownership.

From this point of view, it could easily be shown that in comparison with Guildsmen the Neo-Marxists are merely Conservatives; for Guildsmen have not only questioned industrialism, they have some idea of what to put in its place. They realize that as its retention must involve society in successive wars they must destroy it, or it will destroy them. It is the clear recognition of this fact that inclines an ever increasing number of Guildsmen to look back to the Middle Ages for inspiration and guidance. They do this not as romanticists but in soberness and truth.

\footnotetext[1]{Here is an extract from a review in the \emph{Labour Leader}, July 20, 1906: “Mr. Penty’s criticism of Socialism might have been written by a dweller in Cloud Cuckoo-Town. As the German evolved from the depths of his inner consciousness a camel which bore as much resemblance to the real thing as a kangaroo does to a cow, so Mr. Penty has evoked from the vast deeps a chimera equally grotesque.”

}\footnotetext[2]{The circumstance that Marx gave it as his opinion that the annexation of Alsace and Lorraine by Germany would lead at a later date to a European war does not acquit him, for the war he had in mind was a war of revenge, not an economic war, which this one certainly is.

}\footnotetext[3]{\emph{Authority, Liberty, and Function}, by Ramiro de Maeztu (George Allen \& Unwin, 4s. 6d.).

}

\end{document}
