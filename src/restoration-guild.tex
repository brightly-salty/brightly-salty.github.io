\documentclass{book}
\usepackage{fontspec}
\usepackage{xunicode}
\usepackage[english]{babel}
\usepackage{fancyhdr} 

\usepackage[htt]{hyphenat}





% Included if the stdpage option if set to false
\usepackage[a5paper, top=2cm, bottom=1.5cm,
  left=2.5cm,right=1.5cm]{geometry} % Set dimensions/margins of the parge


\makeatletter
\date{}

% Redefine the \maketitle command, only for book class (not used if stdpage option is set to true)
\renewcommand{\maketitle}{
  % First page with only the title
  \thispagestyle{empty}
  \vspace*{\stretch{1}}
  
  \begin{center}
    {\Huge \@title   \\[5mm]}
  \end{center}
  \vspace*{\stretch{2}}
  
  \newpage
  % Empty left page
  \thispagestyle{empty}
  \cleardoublepage

  % Main title page, with author, title, subtitle, date
  \begin{center}  
    \thispagestyle{empty}
    \vspace*{\baselineskip}
    \rule{\textwidth}{1.6pt}\vspace*{-\baselineskip}\vspace*{2pt}
    \rule{\textwidth}{0.4pt}\\[\baselineskip]
    
    {\Huge\scshape \@title   \\[5mm]}
    {\Large }
    
    \rule{\textwidth}{0.4pt}\vspace*{-\baselineskip}\vspace{3.2pt}
    \rule{\textwidth}{1.6pt}\\[\baselineskip]

    \vspace*{4\baselineskip}

    {\Large \@author}
    \vfill
    
  \end{center}
  
  \pagebreak
  \newpage
  % Copyright page with author, version, and license
  \thispagestyle{empty}
  \null\vfill
  \noindent
  \begin{center}
    {\emph{\@title}, © \@author.\\[5mm]}
    {This work is free of known copyright restrictions.\\[5mm]}
  \end{center}
  \pagebreak
  \newpage
}


% Redefine headers
\pagestyle{fancy}
\fancyhead{}
\fancyhead[CO,CE]{\thepage}
\fancyfoot{}



%%%%%%%%%%%%%%%%%%%%%%%%%%%%%%%%%%%%%%%%%%%%%%%%%%%%%%%%%%%%%%%%%
% Command and environment definitions
%
% Here, commands are defined for all Markdown element (even if some
% of them do nothing in this template).
%
% If you want to change the rendering of some elements, this is probably
% what you should modify.
%
% Note that elements that already have a LaTeX semantic equivalent aren't redefined
% : if you want to redefine headers, you'll have to renew \chapter, \section, \subsection,
% ..., commands. If you want to change how emphasis is displayed, you'll have to renew
% the \emph command, for list the itemize one, for ordered list the enumerate one,
% for super/subscript the \textsuper/subscript ones.
%
%%%%%%%%%%%%%%%%%%%%%%%%%%%%%%%%%%%%%%%%%%%%%%%%%%%%%%%%%%%%%%%%%%%

% Strong
\newcommand\mdstrong[1]{\textbf{#1}}

% Code
\newcommand\mdcode[1]{\texttt{#1}}

% Rule
% Default impl : (displays centered asterisks)
\newcommand\mdrule{
  \nopagebreak
  {\vskip 1em}
  \nopagebreak
  \begin{center}
    ***
  \end{center}
  \nopagebreak
 {\vskip 1em}
 \nopagebreak
}

% Hardbreak
\newcommand\mdhardbreak{\\}

% Block quote$
\newenvironment{mdblockquote}{%
  \begin{quotation}
    \
}{%
  \end{quotation}
}


% Code block
%
% Only used if syntect is used for syntax highlighting is used, else
% the spverbatim environment is preferred.





\makeatother

\title{The Restoration of the Guild System}
\author{Arthur Penty}

\begin{document}

% Redefine chapter and part names if they needs to be
% Needs to be after \begin{document} because babel



\maketitle

\setcounter{tocdepth}{0}
\setcounter{secnumdepth}{0}
\chapter*{Preface}
\label{chapter-0}
Readers of the following pages will probably be aware that the idea of restoring the Guild system as a solution of the problems presented by modern industrialism is to be found in the writings of John Ruskin, who put forward the proposition many years ago.

Unfortunately, however, as Ruskin failed to formulate any practical scheme showing how the Guilds could be re-established in society, the proposal has never been seriously considered by social reformers. Collectivism may be said to have stepped into the breach by offering a plausible theory for the reconstruction of society on a cooperative basis, and Ruskin’s suggestion was incontinently relegated to the region of impractical dreams.

My reason for reviving the idea is that while I am persuaded that Collectivism is incapable of solving the social problem, the conviction is forced upon me that our only hope lies in some such direction as that foreshadowed by Ruskin, and in the following chapters I hope to show that it is not impossible to discover practical ways and means of re-establishing the Guilds in our midst.

In order to understand the full significance of the present proposals they should be considered in conjunction with the theory put forward by Mr. Edward Carpenter in “Civilization, Its Cause and Cure.” Indeed the present volume aims at forging the links required to connect the theory there enunciated with practical politics.

Two other books which have an important bearing on the subject, and might be read with advantage, are Carlyle’s “Past and Present” and Matthew Arnold’s “Culture and Anarchy.”

\emph{Hammersmith}, \emph{August}, 1905.

\chapter{The Collectivist Formula}
\label{chapter-1}
Among the schemes which have been put forward as solutions of the social problem, Collectivism,\footnotemark[1], by reason of its close relationship to current problems, has alone secured any measure of popularity. Having discovered that “unfettered individual competition is not a principle to which the regulation of industry may be entrusted,” political philosophers rush to the opposite extreme, and propose to remedy the defect, first by the regulation, and finally by the nationalization of land, capital, and the means of production and exchange; which measures, we are told, by changing the basis of society from a competitive to a cooperative one, will, by providing the necessary conditions, eradicate every disease from the body politic.

Such a remedy would be perfectly reasonable if the evil to be combated were that of competition. But it is not. It is true that competition, as it manifests itself in modern society, is a force of social disintegration. But this is not because it is necessarily an evil thing; but because the conditions under which it is to-day pursued are intrinsically bad. That the competition of to-day differs from that of the past, we unconsciously recognize when we speak of commercial competition. Competition as it existed under the Guild System, when hours and conditions of labour, prices, etc., were fixed, was necessarily a matter of quality; for when no producer was allowed to compete on the lower plane of cheapness, competition took the form of a rivalry in respect to the greater usefulness or beauty of the thing produced. With the passing of the control of industry from the hands of the craft-masters into those of the financier and the abolition of the regulations of the Guilds, the era of commercial competition was inaugurated, and what was formerly a healthy and stimulating factor became a dangerous and disintegrating reactionary force; for competition between financiers means a competition for cheapness, to which all other considerations must be sacrificed.

F*r the sake of clearness, therefore, we will define the terms competition and commercialism, as follows: Commercialism means the control of industry by the financier (as opposed to the master craftsman) while competition means the rivalry of producers.

Viewing Collectivism in this light, we find that it seeks to eliminate, not commercialism, but competition. In so doing it establishes more securely than ever the worst features of the present system. The mere transference of the control of industry from the hands of the capitalist into those of the state, can make no essential difference to the nature of the industry affected. In Belgium, for example, where the bread-making and shirt-making industries have been nationalized, it has been found impossible to abolish sweating, or to introduce a shorter working day.\footnotemark[2] This abandonment of the principle of social justice at the outset doubtless foreshadows its abandonment for ever; since, as Collectivists become more and more concerned with practical politics, the difficulties of asserting their ideals, together with the establishment of their system of organization, will likewise be more and more increased. Moreover, after its establishment, the difficulties of reforming any State Department are well known; and how shall society be prevented from acquiescing in the present commercial abuses, under a collectivist \emph{régime}, and treating them as inevitable evils?

This question is a very pertinent one, since, according to the economics of Collectivism, every industry nationalized must be made to pay, and a Government charged with the administration of any industry would become interested in its continuance as a business, quite apart from its usefulness or otherwise, or whether or no it had been called into existence by some temporary and artificial need of modern civilization. Thus the Government at the present time, having nationalized the telegraphs, becomes interested in the continuance of gambling, the use of the system in connexion with the turf and the markets being its real basis of support, and not the comparatively insignificant percentage of work undertaken in respect to the more human agencies which require it. Again, were existing railways nationalized, Government would become interested in the continuance of wasteful cross distribution, called into existence by the competition of traders. Similarly, the gradual development of municipal trading and manufacture would tend to militate against the depopulation of towns. And this conservative tendency is inevitable, since Collectivism can only maintain its ground as a national system so long as it justifies the claims of its advocates to financial soundness.

Co-operation in its inception aimed at the establishment of an ideal commonwealth, but the cooperative ideal has long since departed from the movement, and little but a scramble for dividends remains. In like manner it is not unreasonable to suppose that Collectivism, having made its appeal for popular support on the grounds of its capacity to earn profits for the public, would suffer a similar degeneration. The electorate, in their profit-making zeal, would certainly not remedy abuses if their dividends were to be lowered, for they would still retain the superstition that only by producing dividends could their finances be kept in a healthy condition. Inasmuch as the ultimate control of industry would rest in the hands of the financier, production for profit and not for use would continue. For what other test can there be of a financier’s skill except his ability to produce profits?

In a word Collectivism means State Commercialism.

So long as the people are attached to their present habits of life and thought, and possess the same ill-regulated tastes, a State Department, charged with the administration of industry, would be just as much at the mercy of supply and demand as at present, while the fluctuations of taste would be just as disturbing to them as the fluctuations of public opinion are to the politician.

This brings us to the great political fallacy of the Collectivist doctrine—namely, the assumption that Government should be conducted solely in the interests of man in his capacity as consumer—a superstition which has survived the Manchester School; for a little consideration will convince us that in a true social system the aim of Government would not be to exalt either producer or consumer at the expense of the other, but to maintain a balance of power between the two.

The policy of exalting the consumer at the expense of the producer would be perfectly sound, if the evils of the present day were caused by the tyranny of the producer. But is it so? It is true that trade is in such a hopeless condition that the consumer is very much at the mercy of unscrupulous producers. The cause of this, however, is not to be found in the preference of producers generally for crooked ways, but in the tyranny of consumers which forces the majority of producers to adopt malpractices in self-defence. Doubtless all trade abuses have in their origin been the work of individual producers. They grow, in the first place, because without privileges the more honourable producers are powerless to suppress them, and secondly, because consumers generally are so wanting in loyalty to honest producers, and are so ready to believe that they can get sixpenny-worth of stuff" for threepence, that they deliberately place themselves in the hands of the worst type of producer. In every department of life the successful man is he who can lead the public to believe they are getting something for nothing, and generally speaking, if consumers are defrauded by producers, it is because they deserve to be. The truth of this statement will be attested by all who, in any department of industry, have made efforts to raise its tone. Everywhere it is the tyranny of the consumer that blocks the way. For example, anybody who has followed the history of the Arts and Crafts movement, and noted the efforts which have been made to raise the quality of English production must be convinced that this is the root of the difficulty. If the public were capable of a tenth part of the sacrifice which others have undertaken on their behalf, we might see our way out of the industrial quagmire.

This evil would not be remedied by bringing industry under State Control. Rather would it be intensified. Art in the past had its private patrons, and while these continue some good work may still be done. But the artist is powerless when face to face with a public body whose taste recognizes no ultimate standard, but taken collectively is always the reflection of the vulgarity its members see around them. As we may assume that private patrons would cease under Collectivism, so Art’s last support would disappear also. Whatever good work has been done for the public during the past century has been in the main the result of accident. Collective control foreshadows, not the abolition of poverty in our midst, by the direction of industry passing into the hands of wise administrators, but the final abandonment of all standards of quality in production, owing to the complete subjection of all producers to the demoralizing tyranny of an uninstructed majority.\footnotemark[3]

This commercial notion of Government solely in the interests of consumers leads the Collectivist into strange company. It leads him to acquiesce in such a pernicious system as the division of labor. Ruskin claimed that the subjective standard of human happiness, not the objective monetary standard assumed! by previous political economists, was the final test of the social utility of production. If we accept Ruskin’s position, surely we must consider man primarily in his capacity as producer. From this standpoint a man’s health, mental and moral, must depend upon the amount of pleasure he can take in his work. But we deprive the worker of this means of happiness and strive to replace it by such institutions as free libraries and popular lectures, which all lie outside the sphere of his real life. This policy would appear to be based on the idea that man should live a conscious double life. In the first place he must submit to any indignity he may be called upon to suffer by the prevailing system of industry, and secondly, he should aim in his leisure time at self-improvement. He thus destroys in the morning what he has built over-night. Like a mad sculler who pulls both ways at once he describes a rapid circle, and giddily imagines he is making immense progress forward. To unite these warring forces in man and to make him once more simple, harmonious and whole, he must again be regarded first and foremost in his capacity as producer.

Another reason for the primary consideration of the producer, which should be interesting to the democrat is that to legislate on the basis that all are consumers, while only some are producers, is obviously to put a premium upon idleness, for only the idle consume without producing. This fundamental defect of reasoning has thus rendered possible the paradox that while the Manchester School expended its moral indignation in protesting against idleness and luxury, by the very measures it advocated have idleness and luxury been mainly increased.

Let us pass on to a consideration of the principles of Collectivism in their application to particular problems. With regard to the question of Trusts, Collectivists assert that industries will become more and more subject to their domination, and that the State is then to step in and nationalize them.

Now, if we look at the matter carefully, we shall find that the development of industry into Trusts is by no means universal. It holds good in those branches of industry which deal with the supply of raw materials, in distribution, in railways and other monopolies, in the branches of production where mechanism plays an all important part and which command universal markets. On the other hand, there are branches of industry where no such development can be traced. It does not apply to those industries which, in the nature of things, rely upon local markets, such as the building trades; nor to those in which the element of taste enters, as the furnishing and clothing trades. It is true that the large capitalist exists in these trades, but this does not mean that the small builder, furnisher and clothier will eventually be thrust out of the market. The big contractor exists in the building trades, not because he can produce more cheaply than the smaller one, as a careful comparison of prices would show; nor is it because the work is better done, his \emph{raison d’être} is rather to be found in the circumstance that large building contrails can only be undertaken by builders possessed of large capital. Again, the existence of large firms in the furnishing and clothing trades cannot be taken as an indication of the growth of efficiency in those trades, such reduction of cost as has taken place having been obtained in the main at the expense of true efficiency; while again, the growth of large retail houses is in no sense due to a reduction of prices, rather has it been due in some measure to the same causes which brought the large building firms into existence, and to the system of advertising which leads an ignorant public to suppose they are getting a superior article for their money. Nay, if we go further into the matter, we shall find that so far from these huge organizations securing a higher degree of efficiency in production than smaller firms, they owe their very existence to the general degradation of industry—to the fact that the craftsman has so declined in skill, that he has become the cat’s-paw of capitalism. It is only where craftsmanship has declined, and the skilled craftsman has been replaced by the mechanical drudge, that capitalist control secures a firm foothold. It cannot be insisted upon too strongly that capitalist organization, whether private or public, is built upon and presupposes the degradation of the craftsman. Being organized for the production of indifferent work, they are normally working incapable of anything else; for in the production of good work, the craftsman must have liberty to follow the line of a consecutive tradition—a condition which capitalist organization denies, its function being not to develop a tradition of design in handicraft but to adjust the efforts of the craftsman to the whims of a capricious public.

This view, which was originally formed from personal observation and experience of the conditions now obtaining in industry, is amply corroborated by the testimony of Prince Kropotkin. In “Fields, Factories and Workshops,” he says: “The petty trades at Paris so much prevail over the factories that the average number of workman employed in the 98,000 factories and workshops of Paris is less than \emph{six}, while the number of persons employed in workshops which have less than five operatives is almost twice as large as the number of persons employed in the large establishments. In fad:, Paris is a great bee-hive where hundreds and thousands of men and women fabricate in small workshops all possible varieties of goods which require taste, skill and invention. These small workshops, in which artistic finish and rapidity of work are so much praised, necessarily stimulate the mental powers of the producer; and we may safely admit that if the Paris workmen are generally considered, and really are, more developed intellectually than the workers of any other European capital, this is due to a great extent to the work they are engaged in... and the question naturally arises: Must all this skill, all this intelligence, be swept away by the factory, instead of becoming a new fertile source of progress under a better organization of production? must all this inventiveness of the worker disappear before the factory levelling? and if it must, would such a transformation be a progress as so many economists, who have only studied figures and not human beings, are ready to maintain?"

Kropotkin here lays his finger on the weak point of modern sociological theories. They are based upon estimates of figures rather than estimates of men. The correct statement of this issue is perhaps to be found in the dictum that organization on a large scale secures efficiency up to a certain point, which varies in each industry, and when that point is reached, degeneration sets in. On the one hand the quality of the work declines, while on the other, administrative expenses show a tendency to increase out of their proper proportion, owing to the fact that personal control gradually disappears; and this is probably one of the causes which oblige many large firms gradually to adopt sweating practices. Expenses must be cut down somewhere, and the workers have to suffer.

And now that we have found Collectivist prognostications respecting the future of the Factory system to be based upon insufficient data, let us turn to Collectivist opinions respecting the future of machinery; in this connexion we observe that Collectivism teaches that machinery will be used more in the future than at present. The circumstance that many who identify themselves with Collectivism hold to the idea of William Morris, and quote him on sundry occasions, in no wise affects the Collectivist position, which is antagonistic to that held by Morris. Morris’s opposition to machinery was based in the first place upon the perception that there is no temperament in work produced by machinery, and in the next upon a recognition of the principle that its use tended to separate the artist and craftsman more widely than ever, whereas the restoration of industry to health demands their reunion.

But how is a reunion possible under a Collectivist régime? Surely if social evolution has separated the artist and craftsman, further progress along present lines must tend to separate them still further, and not to draw them together. Hence it is we feel justified in identifying Collectivism with the mechanical ideal of industry.

It may be said that the solution of our problems is to be found in a further development towards mechanical perfection, and this contention would be perfectly reasonable if the object of man’s existence were to make cotton and buttons as cheaply as possible; but considering that man has a soul which craves some satisfaction, and that the progress of mechanical invention degrades and stultifies it by making man more and more the slave of the machine, we feel justified in asserting that real progress lies along other lines. Up to a certain point it is true that mechanical invention is for the benefit of the community, but such inventions must be distinguished from the mass of mechanical contrivances which are the humble slaves of commercialism, and witnesses to the diseased state of society. To invent a machine to reduce the amount of drudgery in the world may reasonably be claimed as an achievement of Science; but to reduce all labour to the level of drudgery, to exploit Science for commercial purposes, is an entirely different matter. Machinery being a means to an end, we may test its social utility by considering the desirability or otherwise of the ends it is to serve. And what are the ends which have determined the application of machinery to modern industry? Not the satisfaction of human needs, or the production of beautiful things, but primarily the satisfaction of the money-making instinct, which, it goes without saying, is undesirable. There are very few things which machinery can do as well as hand labour, and so far as my personal knowledge extends, there is nothing it can do better. Hand riveted boilers are preferred to machine riveted ones; while the most delicate scientific instruments have to be made by hand. In fact, wherever careful fitting is valued the superiority of handwork is acknowledged. In the crafts, on the other hand, machinery is valueless, except for heavy work, such as sawing timber; though even here, where timber is exposed to view, it suffers in comparison with hand sawing and hewing, which has more temperament about it. In production, therefore, the only ultimate use of machinery to the community is that in certain heavy work it saves labour, which, considered from the point of view of the development of the physique of the race, is of very questionable advantage; or that it reduces the cost of production. This again, however, is a doubtful advantage, since the increase of material possessions beyond a certain point is extremely undesirable. Without machinery there would be plenty for all and to spare, if it were not for the greed of individuals; and machinery, by facilitating the production of goods in immense quantities, so far from eliminating the spirit of avarice by satisfying it, appears only to give it a cumulative force. Machinery has creeled the most effective class barrier yet devised. Again, considered in relation to locomotion the benefits of mechanism are very doubtful. If railways and steamboats have brought Chicago nearer to London, the world is more commonplace in consequence, and it is very much open to question whether the romance, the beauty, and the mystery of the world which mechanism seems so happy in destroying, may not in the long run prove to be the things most worth possessing, and the hurry and dispatch which are everywhere welcomed as the heralds of progress, admitted to be illusory.

Then Collectivists are in a quandary over the Fiscal Question. Finding themselves unable to accept either the position of the Protectionists or that of the Free Traders, the Fabian Society has formulated a scheme which is supposed to harmonize with the principles of Collectivism. In the tract entitled “Fabianism and the Fiscal Question,” the Society suggests, as a solution for the present crisis, that the trading fleet between ourselves and the colonies be impersonalized, when the conveyance of goods might be made free to all. Surely this would not lead towards Collectivism; rather would it intensify one of the worst evils of the present system which Collectivism proposes to cure—namely, the evil of cross distribution.

This brings me to the question of universal markets, which Collectivists generally assume to be a permanent factor in industry.

To some extent, of course, this will be so, and we must at the outset differentiate between a certain legitimate trade which in the nature of things must always exist, and its present abnormal development, which can only be regarded as symptomatic of disease.

That India should export tea to us appears quite reasonable, but why we should export cotton goods to India is not so clear. The former is a natural trade, because climatic conditions will not permit us to grow our own tea. The latter, however, is not ultimately rooted in actuality, but owes its existence to the creation of artificial conditions, to the circumstance that machinery for the purpose was first invented in Lancashire, and to the fact that we exploited foreign markets for our benefit in consequence. But this may not last. In the long run India must be able to manufacture cotton goods for herself, if the test to be applied is merely that of comparative cost, but when we remember that there are other factors in production which ought to be considered, and which will be taken into account when man re-awakens to the fact that profit is not the Alpha and Omega of production, the change is certain. The re-establishments of just standards of quality in production by the revival of art and the restoration of a sense of morality in trade demand the substitution of local for universal markets.

Of this there can be no question. For it is evident that one at least of the conditions of the restoration of the moral sense. in trade is that the cash nexus be supplanted by the personal nexus in trade relation, and this can only be possible under social conditions in which producer and consumer are known to each other. While again it may be argued that so long as universal markets are regarded as essential to trade, industry must continue to be of a speculative character, owing to the circumstance that supply precedes demand. To reverse this unnatural order of things is essential to production for use, and this involves, among other things, the restoration of local markets.

In like manner the necessities of Art demand the restoration of local markets. If beauty is ever to be restored, and the ordinary things of life are to be once more beautiful, it is certain that local markets will have to be restored. If Art were healthy the wholesale importation of articles of foreign manufacture would not obtain. An artistic public would, for the most part, demand goods of local manufacture, the beauty of which reflected those experiences common to their own life. Thus the English would not import Japanese Art, to any extent, recognizing that, though Japanese Art is admirable in Japan, it is yet so entirely out of sympathy with Western Art as to introduce an element of discord when placed in an English room; while again, for the same reason, the Japanese would not import English Art.

A possible objection to this assumption is that in the most vigorous periods of art a considerable trade was carried on in exchanging the artistic works of different countries, that, in fa6l, many of the finest examples of craftsmanship which were distributed over Europe in the Middle Ages and earlier often emanated from one centre. For instance, carved ivories were mostly made in Alexandria, and so far from the trade which existed in them acting in a way derogatory to the interests of Art, they, as a matter of fact, exercised a very stimulating influence upon the art of the age. To this I answer, that such a trade, which exists for the exchange of treasure, is a fundamentally different thing from a trade which exchanges the ordinary commodities of life, since while the former may operate to widen the outlook in the artistic sense, the effect of the latter is to precipitate all traditions of design into hopeless confusion and anarchy, because, when carried on on a large scale, production for foreign markets does not take the form of sending to other countries specimens of the best craftsmanship which a nation can produce, but of supplying cheap imitations of the genuine and native craftsmanship to other lands a most ruinous commerce; for while abroad the underselling of native craftsmanship tends to destroy the living traditions of those countries, its operations are no less harmful at home, by their tendency to confuse rather than consolidate a national tradition of design.

In the long run a universal trade in everyday commodities could only be favourable to Art on the assumption that internationalism were the condition of healthy artistic activity. And this is not so. An international art would involve the gradual elimination of all that is of local and provincial interest; and when this elimination is complete there is very little left. It would not be untrue to say that the Renaissance failed because its ideas were international, it strove to eliminate all that was of merely local interest in art, and the result was a final and complete imbecility such as never before existed.\footnotemark[4]

Similarly, when we turn to consider the financial side of Collectivism we discover similar fallacies. The nationalization of capital does not recommend itself to us as a solution of present day financial difficulties, since, according to one point of view the economic difficulty arises not so much from an unequal distribution of wealth as from the fact that so much of the labour of the community produces not wealth but “illth,” to use Ruskin’s word. The capital we account for in the columns of the ledger is, indeed, only of a very theoretical character. For, in spite of statistical calculations (which to all appearances may be used to prove anything it is desired to prove), we are not becoming richer, but poorer every year, and this we believe is to be accounted for by our system of finance, which, not studying things, but only the profit and loss account of them, fails to distinguish between what are assets and what are liabilities.

As an illustration of what I mean let us take a concrete instance Tramways. Now it is evident that from the point of view of the private capitalist, whose aim is the making of profit, that the possession of a tramway is to be reckoned as an asset. From the point of view of the community, however, it is altogether different. A municipal tramway is not an asset, but a liability in the national ledger. It is true that the possession of a tramway by a municipality enables the community to intercept profits which otherwise would swell the pockets of the private capitalist, but this does not constitute such a tramway a public asset; it merely decreases the liability. A tramway is a liability because it is not one of the ultimate needs of human society, but an artificial one, arising through the abnormal growth of big towns and cross distribution. If a man has to travel from New Cross to the City every day for employment he helps the tramway to pay its dividends, but he is the poorer for having to take the journey. He is perhaps richer by the time he saves as compared with the time he would lose in having to walk. But the fact that a man lives in one part of the town and works in another is itself an evil—reduced to the terms of national finance it is a liability, and no juggling of figures can make it into anything else. Hence it is that while convenience may suggest the expediency of municipalities owning their own tramways, we are not justified in reckoning them as national assets, or in supposing that the change from private to public ownership is a step in the solution of the social problem.

The same test may be applied to all the activities of Society—though the application of the principle will be very difficult. For exactly what in civilization will constitute an asset, and what a liability, will often be most difficult to determine. Perhaps on due consideration it may appear that civilization itself is entirely of the nature of a liability which man pays for by the sweat of his brow; that the secret of the present financial crisis is that civilization has become so artificial that he cannot pay the price demanded. At any rate, the more we reduce the number of our wants the richer intrinsically we become as a nation. Hence it appears to me that granting, for the sake of argument, that the nationalization of industry is possible, the proper course of action to adopt would be not to commence with the nationalization of the means of distribution, but with production, beginning at the bottom of the industrial scale with agriculture, and building up step by step from this bedrock of actuality, taking care always to avoid the multiplication of works of a temporary character, and building for posterity. It is precisely because ever since the commencement of the era of commercialism, we have individually and collectively proceeded upon the principle of letting posterity take care of itself, that society has become burdened with the maintenance of an ever increasing number of institutions to satisfy the temporary needs of society, that we are becoming poorer.\footnotemark[5]

Closely allied to the foregoing financial fallacy, and in some measure the cause of it, is the more or less unconscious acceptance by Collectivists of the opinion held by the Utilitarian Philosophers that the expenditure of surplus wealth upon art does not operate in the interests of the community. This is an error since from the point of view of national finance such expenditure provides a safety valve which prevents internal complications. The cutting down of expenditure upon Art does not, as Political Economists appear to argue, benefit the people, owing to the direction of surplus wealth into new productive enterprises, rather in the long run has it proved to have the opposite effect of aggravating the problem. Let us take an illustration.

A hundred men are engaged in production; let us make an artificial distinction, and say that seventy-five are engaged in the production of physical necessities, and twenty-five in the production of art (using the word art to indicate those things which do not directly contribute to the maintenance of the body). A machine is invented which enables fifty men to do the work which hitherto had given employment to seventy-five. The balance of production is now destroyed, for there will be a hundred men competing for seventy-five places. It is evident, therefore, if the balance in production is to be restored, one of two things must be done; either the hours of labour must be reduced all round, or the surplus profit created (be it in the hands of Consumer or Producer), must be used in employing the twenty-five displaced men upon the production of Art. Other factors may come in and modify the problem, such as the increased demand for utilities owing to their reduced price, but they are relatively insignificant owing to the fact that as it is not customary under such circumstances to raise the wages of the workers, the limit of the consumption of utilities is practically fixed. Neglecting this arrangement to provide employment for the displaced twenty-five men, disease is spread throughout industry by the destruction of the balance between demand and supply. They must find employment somehow, and so it happens under our commercial society they are used for fighting purposes, becoming travellers or touts in the competitive warfare for the trade which is now insufficient to give employment to all would-be workers. The benefit which the invention of the machine should bring to society is thus lost. The ultimate effect is not to cheapen but to increase the cost of commodities, since it tends to swallow up even the normal profits in fighting machinery, and prices have to be raised, or the quality lowered to make up the difference.

But the evil does not end here. For now, when the markets are filled to overflowing, there can be no mistaking the evil resulting from the practice to which an almost religious sanction has been given by our Political Economists, of systematically re-investing surplus wealth in new productive enterprises, since it tends to reduce wages by the over-capitalization of industry in addition to raising the cost of commodities. The congested state of our markets makes it exceedingly difficult for new industrial enterprises to be successfully floated. Investment is consequently taking the form of converting private businesses into limited liability Companies. Thus a private business with a real capital of say £50,000 is floated as a Company with a nominal capital of £75,000; the extra £25,000 going in goodwill and promotion expenses. And now that the business has more Capital it will be apparent that to maintain the same dividends as hitherto (necessary to maintain credit, if for nothing else), expenses must be reduced in every direction. Hence it generally happens that when a private firm is converted into a Company, unless a strong Trade Union exists, wages are cut down; if a Union prevents this, the old men are discharged to make room for younger and more energetic ones, while no opportunity is lost of increasing the price of commodities to the public or of adulterating the article to reduce its cost.

This, it is safe to say, is substantially what is taking place to-day. Yet, on the whole, Collectivists, while incidentally regretting the reduction of wages, welcome the change as a step towards the nationalization of capital. To me, however, this change wears a different aspect, for it is obvious that so long as we continue to accept the present principle of finance—that all capital should produce interest—and to harbour the utilitarian fallacy that expenditure upon Art is a dead loss to the community, the over-capitalization of industry must tend to increase. The fundamental fact is that so long as the present principles of finance remain unchallenged, the mere transference of capital from private to public ownership can have no appreciable effect on the problem, since a public body accepting these theories must, like a private manufacturer, put the interests of capital before the interests of life—and between these two there is eternal conflict.

The current commercial practice of reinvesting dividends is directly responsible for the development of class separation by withdrawing money from circulation where it is needed and causing it to circulate in orbits where it is not required. Whilst on the human side it is responsible for the great contrasts of extravagance and poverty in modern society, on the industrial side it has struck at the roots of all healthy production. It impoverishes works of real utility to create surpluses to be spent upon works of luxury—making Art an exotic and artificial thing, since its true basis is in utility and not in luxury. For it is one of the paradoxes of our so-called utilitarian age that it is always impossible to get sufficient money spent on real utilities to make them substantial. We first impoverish works of real utility, and having thus succeeded in rendering all useful labour utterly unendurable, we expend the surpluses in providing such diversions as free libraries, art galleries, and such like.

Yet why should utilities be expected to pay dividends? Why should it always be assumed that what is intended for use should yield a profit, and what is intended for luxury not? Why, for instance, in municipal expenditure should it be assumed that houses for the working classes should be self-supporting, while art galleries and free libraries are a charge upon the rates? Why should not some of the money which is spent upon these things be spent in making municipal houses more beautiful? And if it be right for one thing not to pay dividends, why not another? Frankly, I can see no reason, except the superstition of financiers and the fatal tendency of all things to crystallize into formulas. In the case cited, a more generous expenditure upon such things as municipal houses would do more to encourage Art than expenditure upon art galleries, if at the same time means could be devised whereby genuine and not commercial architects could be employed to build them. It is certain that the substantially-built houses and cottages ot the past were never built to earn dividends, and we shall never be able to house the poorest classes so long as we do expect these returns. The fact is that \emph{in really healthy finance, as in life, there is no formula}, and it is precisely because modern reformers have never seriously questioned the truth of modern principles of finance that they are powerless to introduce really effective measures of social reform.

Another instance of the failure of Collectivism comes out in the Fabian tract entitled “Twentieth Century Politics, a policy of National Efficiency,” by Mr. Sidney Webb. In this tract Mr. Webb gives an outline of what he considers should constitute the political programme of a really progressive reform party. After dealing separately with particular reforms, Mr. Webb passes on to consider ways and means of effecting them. And here he is beaten. For the life of him he cannot see where the impetus to carry them into effect is to come from. And so, in desperation, he proposes a measure to artificially stimulate political activity, which is worthy of “Punch,” but is quite wasted in a Fabian Tract.

Recognizing that the Local Government Board has always to be coercing its local authorities to secure the National minimum, Mr. Webb says: “for anything beyond that minimum, the wise minister would mingle premiums with his pressure. He would by his public speeches, by personal interviews with mayors and town clerks, and by the departmental publications, set on foot the utmost possible emulation among the various local governing bodies, as to which could make the greatest strides in municipal activity. We already have the different towns compared, quarter by quarter, in respect to their death rates, but at present only crudely, unscientifically, and perfunctorily. Why should not the Local Government Board avowedly put all the local governing bodies of each class into honorary competition with one another by an annual investigation of municipal efficiency, working out their statistical marks for excellence in drainage, water supply, paving, cleansing, watching and lighting, housing, hospital accommodation, medical service, sickness experience and mortality, and publicly classifying them all according to the result of the examination? Nay, a ministry keenly inspired with a passion for national efficiency, would call into play every possible incentive to local improvement. The King might give a ‘Shield of Honour’ to the local authority which had made the greatest progress in the year, together with a knighthood to the mayor and a Companionship of the Bath to the clerk, the engineer, and the medical officer of health. On the other hand, the six or eight districts which stood at the bottom of the list would be held up to public opprobrium, while the official report on their shortcomings might be sent by post to every local elector, in the hope that public discussion would induce the inhabitants to choose more competent administrators.” (Presumably Mr. Webb would accept Mr. Mallock’s definition of the modern conception of progress, as an improvement which can be tested by statistics, just as education is an improvement that can be tested by examinations.)

The most interesting of all the contradictions in which Collectivism has become involved, and which more than any other exposes the weakness of the position of its advocates is one which during the late war split the party in two; for while all Collectivists recognized that the war was a commercial one, waged in the interests of unscrupulous South African financiers, only part of them declared against it on the grounds of its manifest injustice, the remainder arguing that the best policy for Collectivists was to allow the war to run its course, for the reason that as internationalism and not nationalism was the condition of the future, a united South Africa would, notwithstanding the present injustice, hasten the Collectivist millennium.

Now both these positions are valid according to the theories of Collectivism. The first is a necessary deduction from the position Collectivists have assumed respecting the morality of trade. Recognizing the growth of capitalism to be the cause of the present evils in society, they were perfectly justified in opposing its encroachments. Yet, taking their stand on this ground, they come into collision with their own theory of social evolution, which teaches them that the growth of capitalistic control and of internationalism is a necessary step in the development of society towards the social millennium. While again, those who took their stand on the social evolution of Collectivism found themselves in the unfortunate position of having to compromise with all the evils which they set out to eradicate.

What, then, is the significance of Collectivism? Is it a product merely of the disease of Society, or a sign of health in the body politic? The answer is that it contains the elements of both. Collectivism came into existence to do a definite work, with the fulfilment of which it will assuredly disappear. As Liberalism appeared in opposition to the corrupt oligarchies of the eighteenth century, so Collectivism has come into existence to correct the evils which Liberalism brought with it—to dispel the \emph{laissez-faire} notions of the Manchester school—to expose the inhumanities of commercialism—to re-awaken the moral sense of society, and to restore to it the lost ideal of a corporate life. To Collectivism we are indebted for these ideas, and in their affirmation it has amply justified its existence. It may be, also, that the statistical method which it has pursued, though impossible from the point of view of social re-construction, was yet the only way of impressing certain broad truths on the national mind. It has indeed set forces in motion which may yet be turned in the direction of true social re-construction. But just as Liberalism failed because it had to use the plutocracy as the force with which to effect its purpose, so Collectivism must fail because it has had to make its appeal to the crowd.

Feeling comes nearer to truth than logic, and in the hesitation of the masses to respond to his appeal, the Collectivist may, if he will, see the condemnation of his own measures. The people would be right in neglecting this appeal; in so acting there is unconscious wisdom. The people feel instinctively that Government is not their affair; it leads them out of their depth, and with true inspiration hitherto they have refused to interfere where they cannot understand. They are right also in another and profounder sense. Not only is their indifference a sign that politics have moved out of contact with actuality, but they instinctively feel that Utopia does not lie along the road the Collectivist indicates; for in its appeal Collectivism made one great and fundamental error. It has sought to remedy the evils occasioned by the individual avarice of the few by an appeal to the avarice of the many—as if Satan could cast out Satan.

\footnotetext[1]{Collectivism is better known as Socialism. I prefer to use the term “Collectivism” because it is subject to less confusion. Socialism as a general term has been applied at times to a variety of movements aiming at the establishment of an ideal state. It has been associated with sects as different in aims and methods as are Communists, Anarchists, and Collectivists; while man as diverse and antagonistic in their views as Karl Marx, William Morris, Edward Bellamy, Robert Owen, Keir Hardie, Grant Allen and Bernard Shaw have called themselves Socialists. Collectivism is a definitely formulated scheme of reform, developed within the last twenty years, which, owing to the superior logic of its position, has beaten all rival Socialistic theories out of the field. In a word the term Collectivism may be said to indicate the generally accepted means of arriving at the Socialist ideal.

}\footnotetext[2]{“The Social Unrest; Studies in Labour and Socialist Movements,” by John Graham Brooks.

}\footnotetext[3]{In judging the probable effects of the Collectivist control of industry we must look to the way it would be likely to work in the average provincial town. We are not justified in regarding the L.C.C. as in any way typical. The circumstances which have made the L.C.C. what it is now are not likely to occur again. And there is every reason to believe that when the wave of municipal enthusiasm has gone that the L.C.C. may degenerate in the same way as the late London School Board.

}\footnotetext[4]{In this connection it may be interesting to observe that the abandonment of the international ideal of the Renaissance and an acceptance of national and local traditions underlies much of the success of the present architectural revival.

}\footnotetext[5]{Local taxation rose from £17,000,000 in 1869 to £40,000,000 in 1900, owing to increase of expenditure in Poor Law, Education, Police, Burial Boards, Street Improvements, Sewerage, Isolation Hospitals, Port Sanitary Authorities, Lunatic Asylums, Baths, Washhouses, Road-making, Lighting, etc.—H. T. Muggeridge, Pamphlet on the Anti-Municipal Conspiracy.

}\chapter{Social Evolution}
\label{chapter-2}
The underlying cause of this failure of Collectivism to fulfil the conditions required for the establishment of a sound social system is that in concentrating its attention too exclusively upon the material evils existing in Society it loses sight of the spiritual side of the problem; for indeed, rightly considered, the evils which the Collectivist seeks to eradicate are ultimately nothing but the more obtrusive symptoms of an internal spiritual disease. Religion, art and philosophy have in these latter days suffered a serious decline, and the social problem, as popularly understood, is the attendant symptom.

The truth of this is borne in upon us when we view the present state of things from the standpoint of social evolution. We may then see how the growth of this external material problem coincides at every point with an internal spiritual decline, which separating religion, art and philosophy from life, has plunged Society into the throes of materialism, with its concomitants of ugliness and money-making, the reckless pursuit of which throughout the nineteenth century has left us for our heritage the Rings, Trusts and Monopolies which exploit Society to-day. For had the spiritual forces in Society not dwindled into impotence, social evolution would not thus have tended towards the ignoble ideal of Collectivism, but towards that finer individualism upon which the Socialism of the future must be founded.

To understand how these things are related we must go back to the time of the Renaissance in Italy, when the effort was made to graft the ideas of antiquity upon the Christian nations of Europe. The civilization of the Middle Ages was undoubtedly a lapse from that of Paganism, in that the freedom of thought formerly permitted was everywhere stamped out by the dogmas of Christianity. Yet, strangely enough, though from one point of view this lapse is to be regretted, it achieved a useful work, for in addition to bracing the moral fibre it became the means of enlarging the experience of the race. If it put boundaries to the intellect, it thereby enlarged the boundaries of the imagination. For it was precisely because in the Middle Ages men had their minds at rest about the thousand and one doubts and difficulties which beset the pursuit of the intellectual life, that they were able to develop that sense of romantic beauty which enabled them to build the cathedrals and abbeys which cover Europe.

And so, without committing ourselves to the unlikely theory that the Middle Ages were in every respect an ideal age, and while certain that in many respects that time suffers in comparison with our own, I think we must admit its superiority in some directions. It was greater than our own in that it possessed a “sense of the large proportions of things,” and according to its lights it pursued perfection. For pursuit of religion and art were then the serious things of life, while commerce and politics, which have to-day usurped our best energies, were strictly subordinated to these attributes of perfection. The result of this condition of things, in its reaction upon Society, was that men found it possible to put into practice the dictum “Love thy neighbour as thyself”; and the principle of mutual aid became everywhere recognized in the structure of society. Each section of the community had its own appropriate duties to perform, while any confusion of function was jealously guarded against. In the cities the craftsmen and merchants were organized into guilds; the former for their mutual protection and education, and for the maintenance of fine standards of quality in production, and the latter for facilitating the exchange and distribution of merchandise. On the other hand, the land held in large fiefs under the feudal system by the nobility was formed and administered on a carefully organized system, while the political status of each individual was defined and guaranteed by feudal law and the universal code of morality supplied by the teachings, and enforced by the authority of the Roman Church.

Similarly, when we consider the external life of that age, what most impresses us is the marvellous and universal beauty of everything that has survived to our own time. The mediaeval period was not only great in its architecture, but the very humblest forms of craftsmanship, even the utensils, were beautiful. What a contrast to our day, where ugliness just as universal. It matters little where we look, in the city or the suburb, in the garden or in the house; at our dress or our furnishings; wherever modernity is to be found, vulgarity is also there. For this ugliness knows no exception save in the work of an insignificant and cultivated minority who are in conscious opposition to the present order of society. The Renaissance brought about this change by cutting at the roots of tradition\footnotemark[1] which hitherto had been the support of the Middle Ages.

The sense of a consecutive tradition has so completely disappeared from modern life, that it is difficult for most of us to realize what it means. To greater or lesser extent in the form or custom or habit it is always present with us in debased forms. Yet this is a different thing from that living tradition which survived until the Renaissance, the meaning of which will be best understood by considering its relation to the arts.

Tradition then, in relation to the arts, may be defined as a current language of design, and, indeed, design in the Middle Ages bears a striking resemblance to the language of speech, in that the faculty of design was not as it is to-day, the exclusive possession of a caste—a body of men who give prescriptions for the craftsman to dispense—but, like language, was a common possession of the whole people. Certain traditional ways of working, certain ideas of design and technique were universally recognized, so that when the craftsman was called upon to design he was not, like his modern successor, compelled to create something out of nothing, but had this tradition ready to hand as the vehicle of expression understood by all. It was thus that the arts and crafts of former times were identical—the artist was always a craftsman, while the craftsman was always an artist. In the production of architecture no architect was employed in the modern sense to conceive and supervise every detail, since as every craftsman was in some degree an artist, it was the practice of each craft to supply its own details and ornaments, the craftsman being subject only to such general control as was necessary to secure a unity of effect. Architecture was thus a great cooperative art, the expression of the national life and character.

We realize, perhaps, more fully what tradition means when we compare the conditions of craftsmanship in those days with those which obtain to-day. The modern craftsman, deprived of the guidance of a healthy tradition, is surrounded on all sides by forms which have persisted, though debased and vulgarized, while the thought which created them has been lost. Consequently, he uses them not merely without any perception of their meaning, but as he does not realize that they ever had any meaning, he has as much chance of making himself intelligible as a man whose speech is a hopeless jargon of all tongues, and who has lost the capacity of realizing that any word he uses has ever actually had a definite meaning.

In these circumstances the designer or craftsman of to-day has a task of far greater magnitude to perform in order to produce creditable work than had his predecessors. It is not merely a question of possessing good taste, since before he can design he must recover for himself a language of expression. He must, therefore, be not merely an artist, but an “etymologist of forms,” so to speak, in addition. How, then, can we wonder if little good work is produced?

Similarly we find the absence or degradation of tradition exercises its baneful influence in every department of life, for just as the craftsman cannot design beautifully because he has lost hold of a living tradition of design, so men are unnatural and inhuman because they have lost the art of right living, spontaneity and instinct having given place to conventions and fashions which exercise an intangible tyranny over their victims. Incidentally I would refer to the corroborative testimony of Mr. Bernard Shaw, who emphasizes the same truth in “Man and Superman.” Mr. Shaw observes that English critics disapproved of Zola’s works not because they considered them immoral, but because never having been taught to speak decently about such things, they were without a language by which to express their ideas.

To return to our subject: I said that the Renaissance, by cutting at the roots of tradition, brought about the changed state of things we see around us to-day. In seeking to liberate man from the fetters of the Middle Ages the Renaissance unfortunately destroyed what was really good and valuable.

Without minimizing in the least the ultimate benefits which the growth of the spirit of criticism stimulated by the Renaissance has in store for the human race, the development of that spirit has so far been attended with disastrous consequences. It is admitted that by undermining the authority of the Church and the Bible, criticism has largely destroyed the spirit of consecration to ideals, but it is not generally recognized that this same spirit operating upon the arts has brought about their decline, by separating them from life. First we see the gradual formation of canons of taste; then follows the growth of academies which impose rigid classical standards upon the people, and finally, tradition, which has hitherto been the source of vitality in the arts, is everywhere extinguished and a complete divorce is effected between Art and Life.

Art, ceasing to be the vehicle of expression for the whole people, now becomes a play-thing for the connoisseur and the \emph{dilettante}, hidden away in galleries and museums, while Life, having lost the power of refined expression, crystallizes into conventions and becomes ugly in all its manifestations.

Simultaneously with this separation of art from life comes the separation of the artist from the craftsman. The fine arts having turned their back upon their humble brethren, craftsmanship everywhere degenerates into manufacture—uniformity having supplanted variety as the ideal of production, machines are invented for multiplying wares. Factories are built to contain the machinery, and labour is organized for the purpose of working it, while universal markets arise through the desire of avaricious manufacturers to find some temporary escape from the evils of over-production. And now, when supply has got ahead of demand, comes a complete divorce between production and use—owing to the circumstance that under such conditions, speculation, and not human need, becomes the motive force of production. Business and money-making become the all-absorbing interest of life, while democracy takes its rise in the seething discontent engendered by the growth of such conditions simultaneously with the degeneration of the Guilds into close corporations, and the subsequent exclusion of the journeyman, who is thereby deprived of the position he formerly held in the social scheme.

Such would appear to be the true interpretation of social evolution—Religion, Art and Philosophy having separated themselves from life, business, money-making and politics, hitherto subordinated to the pursuit of these other attributes of perfection, become the all absorbing interests of life. To this reversal of the natural order of things is to be attributed the growth of the social problem.

What the future has in store for us it is indeed difficult to say. It would seem that the present system is doomed to collapse through its own internal rottenness. Commercialism will reap as it has sown—a social catastrophe is clearly its fit and proper harvest. For a century we have been putting off the evil consequences of unregulated production by dumping our surpluses in foreign markets; but this cannot continue indefinitely, for the problem which on a small scale should have been boldly faced a century ago, when machinery was first introduced, will have to be dealt with on a gigantic scale. The foreigner, who was once our customer, has now become our competitor; and so instead of expanding markets we have to face the problem of contracting markets. The appearance therefore of a large and ever-increasing unemployed class becomes inevitable. The probability is that this phenomenon will make its appearance in America, where industrial conditions are fast ripening for such a catastrophe. Until quite recently America was occupied not so much in the production of wares as in manufacture of machinery. It is obvious that when all this machinery becomes engaged in actual production the output available for exportation will be enormously increased; and it is stated on very good authority that the competition we have already experienced from America is as nothing in comparison with what we are likely to encounter during the next few years. Meanwhile, the growth of Trusts, Combines and Monopolies, by eliminating the waste consequent upon competition, tends in the same direction.

Under these circumstances, we shall be well advised to prepare for eventualities. Though unable to save existing society, it may yet be possible to build something out of its ruins.

\footnotetext[1]{“In a traditional art each product has a substance and content to which the greatest individual artists cannot hope to attain it is the result of organic process of thought and work. A great artist might make a little advance, a poor artist might stand a little behind, but the work as a whole was customary, and was shaped and perfected by a life-experience whose span was centuries.”–“Mediaeval Art,” by Professor W. R. Lethaby.

}\chapter{The Sphere of Political Reform}
\label{chapter-3}
In passing to the constructive side of our theme, it is first necessary to realize clearly that as commercialism, not competition, is the evil from which modern society suffers; the real battles of reform are to be fought in the industrial, not in the political arena. To abolish commercialism it is necessary to transfer the control of industry from the hands of the financier into those of the craftsman, and as this change is ultimately dependent upon such things as the recovery of a more scrupulous honesty in respect to our trade relationships, the restoration of living traditions of handicraft, and the emergence of nobler conceptions of life in general, it is evident that the nature of the reforms is such as to place the centre of gravity of the reform movement outside the sphere of politics.

At the same time it is well to remember that, though the solution is not a political one it has, nevertheless, a political aspect, for in this endeavour to reform industry the legislature may assist. Recognizing the truth that nobler conceptions of life are essential to the salvation of society, and that the desired change should be in the direction of simpler conditions of life, the legislature can greatly facilitate such a change by the wise expenditure of that portion of the surplus wealth of the nation which they would derive from the taxation of unearned incomes. In the long run it is the expenditure of surplus wealth which determines in what direction industrial energy shall be employed; and just as foolish expenditure is the forerunner of depression and decay, so wise expenditure imparts health and vigour to the body politic. “The vital question for individual and for nation,” as Ruskin said, “is not, how much do they make? but to what purpose do they spend?”

As to the way in which the expenditure of wealth could be used to facilitate the spiritual regeneration of society, the first condition of success is a more generous and magnanimous spirit than is customary to-day; in a word, we should not expect too much for our money, since, until the spirit of society is changed in this respect, there can be no possibility of returning to simpler conditions of life. Until then sweating, jerry-work, dishonesty and quackery will remain with us, and the producers will continue to be slave-driven.

The evil, moreover, does not end here. The attendant symptom of this pernicious system is that with our minds bent always upon making bargains, it comes about that less regard is paid to the intrinsic value than to the market value of things, and we thus create conditions under which the gulf separating the two is ever widening, until finally the anti-climax of the ideal of wealth accumulation is reached in the circumstance that it becomes daily more impossible to buy things worth possessing. To reverse this unnatural order, therefore, and to let our choice be determined by the intrinsic value than to the market value of things, is the second condition of successful expenditure.

There are two directions in which an immediate increase of expenditure is called for in the national interest. In the first place there can be no doubt that a serious attempt should be made to revive agriculture\footnotemark[1] in this country, for apart from its temporary commercial value, agriculture has an intrinsic value as a factor in the national life, in that it strengthens the economic position of the country at its base. Secondly, a substantial increase should be made in our national expenditure upon art, particularly by a more generous and sympathetic patronage of the humbler crafts; for not only would such expenditure tend to relieve the pressure of competition, but since the true root and basis of all art lies in the health and vigour of the handicrafts, a force would be definitely set in motion which would at once regenerate industry and restore beauty to life—industry and beauty being two of the most powerful factors in the spiritual regeneration of the race.

In answer to some who complained that Athens was over-adorned, even as a proud and vain woman tricks herself out with jewels, Pericles replied that “superfluous wealth should be laid out on such works as, when executed, would be eternal monuments of the glory of their city, works which, during their execution, would diffuse a universal plenty; for as so many kinds of labour and such a variety of instruments and materials were requisite to these undertakings, every art would be exerted, every hand employed, almost the whole city would be in pay, and be at the same time both adorned and supported by itself.” Such was the old-time solution of the unemployed problem; both the spiritual and material needs of the people are here provided for.

\footnotetext[1]{This issue will be found exhaustively dealt with in the recently issued Fabian Tract entitled “The Revival of Agriculture,” with which I feel perfectly in accord. It remains, however, to be said, that the principles underlying this tract are not in harmony with the formulated dogma to which members of the Fabian Society are asked to subscribe.

}\chapter{The Guild System}
\label{chapter-4}
The conclusion to be deduced from the the last chapter was that the wise expenditure of surplus wealth, and, indeed, all exercise of wisdom, demands that man be spiritually regenerated.

It is obvious that by spiritual regeneration something very different is meant from the morbid and sickly sentimentality which very often passes for spirituality to-day; rather must we be understood to mean the recovery by society of that “sense of the large proportion of things” as Pater calls it, which in all great ages of spiritual activity was in a greater or less degree the common possession of the whole people, and while giving a man a new scale of values may be said to completely change the individual nature. In this connexion it is well to remember that though in one sense the individual nature is unchangeable, the fact remains that the intellectual atmosphere which we breathe will determine the particular mode in which it will express itself; and that whereas a prejudiced and sectarian atmosphere, by refusing the higher nature its medium of expression, will encourage the expression of the lower nature, so a wider outlook on life, an atmosphere in which the nature and essential unity of things are more clearly discerned, will by transmuting values keep the selfish motives more effectually in subjection. It is thus that the recovery of the sense of the large proportion of things by the individual members of the community must precede all substantial reform. It is this sense which is the great socializer, making always for Collective action. There can be no Socialism without it.

No better example could be found of the way in which its absence militates against social reform than the common attitude of sociological thinkers towards the present proposal of re-establishing the Guild system in society. One and all of them, without further inquiry, dismiss Ruskin’s proposal as a harking back to Mediaevalism merely because the links which separated his proposals from practical politics were not in his day capable of being forged. In all this we see that characteristic failure of the modern mind to distinguish clearly between what is immediately practicable, and what must ultimately be brought to pass, and its incapacity to adjust the demands of the present to the needs of the future.

Tested by such principles the restoration of the Guilds will appear not merely reasonable but inevitable. Being social, religious, and political as well as industrial institutions, the Guilds postulated in their organization the essential unity of life. And so, just as it is certain that the reattainment of intellectual unity must precede the reorganization of society on a Co-operative basis, it is equally certain that the same or similar forms of social organization will be necessary again in the future.

For the present we shall regard them merely as political and industrial organizations, for these are the aspects which immediately concern us. The question of their restoration as religious and social organizations is outside the scope of the present volume, depending as it does upon the settlement of many theological and scientific questions which we do not feel qualified to discuss. To give the reader some idea of what the Guild system really was one cannot perhaps do better than quote from a lecture by Professor Lethaby on “Technical Education in the Building Trade” (for though this has particular reference to the building trades, the same conditions obtained in every trade), and to supplement this by adding the rules of the Cloth Weavers of Flanders as given by William Morris in “Architecture, Industry and Wealth.”

“In the Middle Ages,” says Professor Lethaby, “the masons’ and carpenters’ guilds were faculties or colleges of education in those arts, and every town was, so to say, a craft university. Corporations of masons, carpenters, and the like, were established in the town; each craft aspired to have a college hall. The universities themselves had been well named by a recent historian ‘Scholars’ Guilds,’ The guild which recognized all the customs of its trade guaranteed the relations of the apprentice and master craftsman with whom he was placed; but he was really apprenticed to the craft as a whole, and ultimately to the city, whose freedom he engaged to take up. He was, in fact, a graduate of his craft college and wore its robes. At a later stage the apprentice became a companion or a bachelor of his art, or by producing a masterwork, the thesis of his craft, he was admitted a master. Only then was he permitted to become an employer of labour or was admitted as one of the governing body of his college. As a citizen, City dignities were open to him. He might become the master in building some abbey or cathedral, or as king’s mason become a member of the royal household, the acknowledged great master of his time in mason-craft.) With such a system was it so very wonderful that the buildings of the Middle Ages, which were indeed wonderful, should have been produced?”

Let us now glance at the rules of the Cloth Weavers of Flanders. “No master to employ more than three journeymen in his workshop; no one under any pretence to have more than one workshop; the wages fixed per day, and the number of hours also; no work should be done on holidays; if piece-work (which was allowed) the price per yard fixed, but only so much and no more to be done in a day. No one allowed to buy wool privately, but at open sales duly announced. No mixing of wools allowed; the man who uses English wool (the best) not to have any other on his premises. English and foreign cloth not allowed to be sold. Workmen not belonging to the Commune not admitted unless hands fell short. Most of these rules and many others may be considered to have been made in the direct interest of the workmen. Now for the safeguards to the public. The workman must prove that he knows his craft duly; he serves as apprentice first, and then as journeyman, after which he is admitted as a master if he can manage capital enough to set up three looms besides his own, which of course he could generally do. Width of web is settled; colour of list according to quality; no work to be done in a frost or bad light. All cloth must be ‘walked’ or ‘fulled’ a certain time, and to a certain width, and so on and so on. Finally, every piece of cloth must stand the test of examination, and if it fall short, goes back to the maker, who is fined; if it comes up to the standard it is marked as satisfactory.”

The point to remember in all this is that the individual craftsman was privileged, and privileged he must be if he is to remain a conscientious producer. For privilege not only protected him from unscrupulous rivals but also secured him leisure at his work—a very necessary condition of good work. The framers of these regulations grasped one great sociological fact: which is not clearly understood to-day, namely, that rogues are very dangerous men and that it is impossible to keep control over them by granting that measure of liberty which permits of unfair competition. I say \emph{unfair} competition, because the Guilds did not aim at the suppression of competition, but at that particular form of it which we designate commercial competition. By preventing the lower competition for cheapness the plane of the struggle was raised and a competition of quality was the result. This was of course thoroughly healthy and stimulating, for it secured in the industrial struggle not the ‘survival of the fittest’ but the ‘survival of the best’ in the broadest sense of the word, as the general excellence of mediaeval craftsmanship abundantly proves.

That the Guild System is the only system of organization under which production can be healthy I am fully persuaded; but whether it will again be applied to distribution is perhaps open to question. The probability is that the distribution of raw material in the future will be undertaken by the State under some form of Collectivist administration with nationalized railways.\footnotemark[1] With regard to the nationalization of the land, this will probably be one of the last reforms we shall achieve. For if it be true, as Mr. Edward Carpenter has pointed out, that the crust of conventionality and artificiality in which the modern world is embedded is a change or growth coincident with the growth of property and the ideas flowing from it, does it not follow that so long as this crust of conventionality remains the nationalization of the land must remain impracticable, since, so long as the people are addicted to their present habits and modes of life they will rally to the support of a system the abolition of which would individually threaten their social existence.

As to the form which the Government of the future will take it is not improbable that the division of function between the Upper and Lower Chambers will continue, with this difference, that whereas the lower chamber would be elected by the people in their private capacity the members of the Upper Chamber would be nominated by the Guilds.

Such an arrangement would seem to secure for democracy what at present it appears to be incapable of securing for itself—the leadership of the best and wisest. Accurate thinking does not readily lend itself to platform oratory, and so it happens that owing to a disability to enforce their views at public meetings the community is deprived of the services of a large section of the most thoughtful members of the community. The creation, however, of an upper chamber whose members were the nominees of the Guilds would remedy this defect by removing oratory from the list of necessary qualifications for political life, and with the wisest at the helm the present anarchic tendencies of democracy would be checked; the principle of authority on a popular basis would be thereby established, while a balance of power between the various interests on the State would be automatically maintained.

Should this prediction prove to be a true one, and should Society again revert to the Guild system, we shall be in a position to realize their value more adequately. With our knowledge of the consequences of unfettered individual competition, society will be able to guard itself more securely against the growth of those evils of which we have had so bitter an experience. And so while the prospect of social salvation inspires us with hope, it may be well to remember that society is not to be saved by the establishment of any social \emph{régime}, since, until each individual member of society has sufficient moral courage to resist the temptation to pursue his own private ends at the expense of the common-weal, and possesses the mental outlook necessary to enable him at all times to know in what direction the best interests of the community lie, social institutions once established will tend inevitably to degenerate, for inasmuch as {[}all institutions are but the expression of national life and character, the integrity of the individual can alone secure the integrity of the State.

\footnotetext[1]{It may probably be found convenient to utilize the existing means of distribution as embodied in the Co-operative Society for supplying the ordinary necessities of life; this of course would not apply to such branches of production as the furnishing and building trades, where it is essential to true efficiency that no middleman interpose between the craftsman and his customer.

}\chapter{How the Guilds may be Restored}
\label{chapter-5}
Passing on now to consider the problem of ways and means of re-introducing the Guild System, the first fact we must grasp is that the Guilds cannot be re-established by further evolution upon the lines along which society is now travelling, but by the development of those forces which run counter to what maybe considered the normal line of social evolution.

Of these, the first force which will be instrumental in restoring the Guilds is the Trade Union movement. Already the unions with their elaborate organizations exercise many of the functions which were performed by the Guilds; such, for instance, as the regulation of wages and hours of labour, in addition to the more social duty of giving timely help to the sick and unfortunate. Like the Guilds, the Unions have grown from small beginnings, until they now control whole trades. Like the Guilds also, they are not political creations, but voluntary organizations which have arisen spontaneously to protect the weaker members of society against the oppression of the more powerful. In three respects only, as industrial organizations, are they differentiated from the Guilds. In the first place, they accept no responsibility for the quality of the wares they produce. Secondly, masters are not permitted to become members of these organizations; and thirdly, they do not possess monopolies in their separate trades.

Of course, these are very important differences—differences in fact which for the time being are insurmountable. The circumstance that modern industry is so completely in the grip of the financier and speculator is alone sufficient to prevent any speedy transformation of the Unions into Guilds, since so long as it exists it is difficult to see how masters and men could belong to the same organization. The question, therefore, which we require to answer is this: Will industry continue to be controlled by the financier, or are there grounds for supposing that the master-craftsman will supplant him in the future?

My answer to this question is, that we have very good grounds for supposing that the craftsman will supplant the financier. Speculation brings its own ruin. It is already ruining the workman, and in proportion as it succeeds in this it will undermine effective demand, and so ultimately destroy the very source of its dividends. This prediction is based on the assumption that society will quietly acquiesce in the operation of the speculator, but the probability being, as has already been shown, that a revolution will result, the ruin will be considerably hastened. Meanwhile, there are two agencies at work in modern society which are destined to supplant the large factory by the small workshop. The first of these is the increasing use which is made of electricity for the distribution of power at a cheap rate, and the second is the gradual raising of the standard of taste and craftsmanship.

Respecting these, it is easy to see that just as the introduction of steam power created the large factory by concentrating industry, so electricity, by facilitating the distribution of power, will render possible the small workshop in the future. It is true that the growth of the factory system preceded the introduction of steam power and machinery. This, however, in turn was preceded by a decline in craftsmanship which, by substituting uniformity for variety in the practice of industry, made such development possible. And so it may fairly be assumed that just in proportion as the standard of taste and craftsmanship is raised, the factory system will tend to disappear. The practice of good craftsmanship demands that care be taken with the quality of the work; it demands that work be done leisurely; that the worker shall receive a fair price for his work and that he shall have security of employment. All these things commercialism and the factory system deny him and must deny him, for the two are essentially antagonistic. The victory of the one must mean the death of the other.\footnotemark[1]

This brings us to the consideration of the second force which is preparing the way for the restoration of the Guilds, namely, the Arts and Crafts movement, which exists to promote the revival of handicraft. Recognizing that the true root and basis of all art lies in the handicrafts, and that under modern conditions the artist and craftsman have, to their mutual detriment, become fatally separated, the Arts and Crafts movement sought to remedy this defect by promoting their reunion.

Writing on the Revival of Handicrafts and Design,\footnotemark[2] Mr. Walter Crane says: “The movement indeed represents, in some sense, a revolt against the hard mechanical life and its insensibility to beauty (quite another thing to ornament). It is a protest against that so-called industrial progress which produces shoddy wares, the cheapness of which is paid for by the lives of their producers and the degradation of their users. It is a protest against the turning of men into machines, against artificial distinctions in art, and against making the immediate market value, or possibly of profit, the chief test of artistic merit. It also advances the claim of all and each to the common possession of beauty in things common and familiar, and would awaken the sense of this beauty, deadened and depressed as it now too often is, either on the one hand by luxurious superfluities, or on the other by the absence of the commonest necessities and the gnawing anxiety for the means of livelihood; not to speak of the every-day ugliness to which we have accustomed our eyes, confused by the flood of false taste or darkened by the hurried life of modern towns in which huge aggregations of humanity exist, equally removed from both art and nature, and their kindly and refining influences.

“It asserts, moreover, the value of the practice of handicraft as a good training for the faculties, and as a most valuable counter-action to that overstraining of purely mental effort under the fierce competitive conditions of the day; apart from the very wholesome and real pleasure in the fashioning of a thing with claims to art and beauty, the struggle with and triumph over technical necessities which refuse to be gainsaid. And, finally, thus claiming for man this primitive and common delight in common things made beautiful, it makes, through art, the great socializer for a common and kindred life, for sympathetic and healthy fellowship, and demands conditions under which your artist and craftsman shall be free.

“‘See how a great a matter a little fire kindle th.’ Some may think this is an extensive programme—a remote ideal for a purely artistic movement to touch. Yet if the revival of art and handicraft is not a mere theatrical and imitative impulse; if it is not merely to gratify a passing whim of fashion, or demand of commerce; if it has reality and roots of its own; if it is not merely a little glow of colour at the end of a sombre day—it can hardly mean less than what I have written. It must mean either the sunset or the dawn.”

We do not, of course, need to take this war-cry at its face value. It is one thing to declare a principle, it is another to reduce it to practice. And looking at the Arts and Crafts movement to-day it seems to resemble the sunset rather than the dawn. It cannot be denied that up to the present, while the movement has succeeded in popularizing the idea, it has for the most part failed to reduce it to practice. A favoured few, possessed of means or social advantages, have succeeded in establishing themselves before the public, but the number is comparatively insignificant. The majority, after struggling for a few years, have lost heart, and a depression of the movement has followed in consequence.

Sunset, however, is followed by dawn, and while we frankly recognize that the movement is suffering from a reaction, we are not justified in concluding that failure is its inevitable doom:

\begin{mdblockquote}
	Tasks in hours of insight willed

	Can be through hours of gloom fulfilled,


\end{mdblockquote}
says Matthew Arnold. The movement is fortifying itself upon more impregnable strongholds, for viewed from the inside it may be seen that the centre of gravity of the movement is being slowly transferred from artistic and \emph{dilettante} circles to the trade. Hitherto the movement has suffered from weakness in three directions. The first was its isolation from the trade; the second, the general absence of intellectual patronage—fashion having been the guiding and controlling influence with the vast majority of its patrons;–and the third has been lack of knowledge as to the sociological bearings of the movement, such as would have enabled it to direct its energies in the most effective way.

The gulf which has hitherto separated the movement from the trade shows a tendency to become bridged over. In many directions there are signs that the trade is being gradually leavened; the wave of feeling which created the Arts and Crafts movement has at length reached the workers, and there is good reason to believe that the first condition of widespread success—namely, the cooperation and goodwill of the trade—will ere long be attained.\footnotemark[3]

The patronage afforded to the crafts by fashionable circles, if it has not altogether ceased, is rapidly decreasing, and though, in the general absence of intelligent patronage the Arts and Crafts movement has every reason to be grateful to fashion for keeping the flame alive, we are persuaded that the withdrawal of such patronage will prove to be no evil, since, so long as the movement accustomed itself to look to fashion for its support, the work produced must necessarily be of an exotic nature, while the really valuable work which the movement stands for, namely, the restoration of beauty to life, is retarded.

The greatest weakness of all, however, is that hitherto the movement has never clearly understood its own sociological bearings—a defect which the present volume aims at remedying. In the long run I am persuaded that the movement will never be able to make much headway until it possesses a social theory which accords with its artistic philosophy—since until then it can never have a common meeting ground with the public; it will be unable to get support of the right sort, and without such support its value as a force in social reconstruction will be impaired.

I feel well advised in ranking the Arts and Crafts movement as one of the forces of social reconstruction, and that not merely because art is the eternal enemy of commercialism, but because of the peculiar relation in which it stands to modern society. This is the first time in history that art has progressed in advance of the age in which it is practised. Hitherto new movements in art have been preceded by popular movements which have prepared a public capable of understanding them. Thus, the Humanists in Italy prepared the way for the revival of Classical Architecture at the time of the Renaissance, and similarly the Anglo-Catholic movement provided a public for the Gothic Revivalists and the pre-Raphaelites. I cannot but believe that there is great significance in this fact and that art is one of the great forces of social reconstruction destined to roll back the wave of commercialism. At any rate, art is no longer able to follow the trend of the age, and sooner or later it will have to choose between being thrust out of society by the ever-increasing pressure of commercial conditions of existence, and definitely taking in hand the work of social reconstruction.

The obvious way in which this is to be accomplished is by joining hands with other reform movements. Instead of seeking to understand each other, reformers have attempted solutions of their own separate problems, regardless of the efforts of others to solve kindred difficulties, with the result that one and all have lost their way amid the expediencies and compromises of practical politics. Hence it would appear that the immediate work before us is the promotion of intellectual unity by a return to fundamentals, the result of which would be not only to unite the different sections of reform movements with each other, but to unite them with the public.

There are many signs that the tendency of modern thought is in this direction. One result of the failure of all reform movements to realize their dire6t intentions has been the growth of a general consensus of opinion that all substantial reform demands the spiritual regeneration of the people. On this point it is clear that sociologists, scientists, artists, philosophical thinkers, politicians and reformers are coming to some agreement. Moreover, it is becoming apparent to an ever increasing body of thinkers that gambling, drink, and many other social ills have their roots in modern industrial conditions; that so long as the majority are compelled to follow occupations which give no scope to the imagination and individuality of the worker, nobler conceptions of life, in other words, spiritual regeneration, are strangled at their roots; and that the cultivation of the aesthetic side of life is the great need of the day.

Meanwhile, abstract thought is tending in the same direction. What was ordinarily called philosophic materialism has of late years receded very much into the background, and in its place we find a restoration of belief in the immortality of the soul through the growing acceptance of the doctrines of re-incarnation and karma, and the tendency to admit the claims of mysticism. The necessity of a revival of religion of some kind is becoming very generally admitted. Consequently we do not hear so much of militant agnosticism as of a tendency to try and find out what are the really essential things in religion.

The discoveries of modern science are confirming this tendency. One has only to mention Sir William Crooke’s experiments with radiant light, the Rontgen rays, the N. rays, Dewar’s liquid air experiments, the Hertzian currents, the discovery of radium, and Mr. Butler Burke’s investigations into the origin of life, to show that the abandonment of the doctrine of the physical origin of life (so incompatible with spiritualistic conceptions) is not very far distant. To the lay mind, at any rate, the postulation by science of the existence of a universal consciousness interpenetrating all matter as the explanation of the contradictory results of investigations conducted by the chemist and the physicist, the astronomer and the geologist, implies such an abandonment. Hopeful as this tendency undoubtedly is, the prospects of unity are still more hopeful now that the establishment of a just standard of taste in art, in conformity with a philosophic conception of its nature, is within sight. After a century of experiment and failure, art is at last emerging from the cloud of darkness which through the nineteenth century enveloped it. It is difficult for those who are not professionally engaged in the arts to realize the enormous strides which have been made in architecture and the crafts during the last fifteen years, owing to the circumstance that so little good work finds its way into the streets of our great cities. Yet the advance is remarkable. The reason of this is that as a result of our experiments the fundamental principles of art are becoming more generally understood. The architect of to-day realizes that architecture is not a system of abstract proportions to be applied indifferently to all buildings and materials, but that, as already stated, the true root and basis of all art lies in healthy traditions of handicraft; that, indeed, it is impossible to detach design from the material to be used, since in its ultimate relationships design is an inseparable part of good quality. The discovery of this principle, which was foreshadowed by Ruskin in that famous chapter in the “Stones of Venice” entitled “The Nature of Gothic,” is rapidly rejuvenating modern art. Commencing with the establishment of traditions of handicraft, architecture by reaction is being regenerated. It is not unreasonable to expect that this new standard will gradually find its way into the finer arts of painting and sculpture. Painting and sculpture can never be healthy except when practised in subordination to architecture, and as patronage of the arts is now so grudgingly given, frequent opportunities for successful collaboration are not likely to be forthcoming.

We may safely anticipate that the new ideas now germinating in the arts will gradually find their way into other branches of activity. It may be true, perhaps, that the aestheticism of the connoisseur is often a very superficial thing. Nevertheless it is the stepping stone to higher attainments; for no man in the long run can study aesthetics apart from the realities they symbolize. The Gothic revival and the pre-Raphaelite movement at their inception may be regarded as in many respects superficial. Yet they have led to the discovery of truth in a hundred fields of research; indeed it is difficult to say for what they are not responsible. To them in the last analysis we owe the re-creation of the whole fabric of design, while indirectly they have re-created the past for us in a manner never understood before. Incidentally it may be pointed out that the forces they set in motion have not only supplied the key to the problems discussed in these pages, but have also supplied the facts necessary for their proper statement. While again it is to the aesthetic movement in literature that we owe the revival of interest in folk lore, symbolism and peasant life.

It will be thus, as element is added to element, that a soil will be prepared wherein new spiritual conceptions may take their rise. Ideas of spirituality have hitherto been associated with ideas of beauty. And just as spiritual truth is not to be expressed apart from the medium of beautiful form, so beauty of form is not ultimately to be detached from spiritual truth. It will be thus that the pursuit of beauty will tend to re-awaken and to give reality to the spiritual life. Not that the worship of beauty can ever be sufficient to constitute a religion, but that the seeking after beauty in all relationships of life (for society must pass through a state of self-conscious aestheticism ere beauty can resume its proper and subordinate function) is more likely to lead us into the vicinity of spiritual things than a breathless pursuit of riches and ugliness.

Such appear to be the main outlines of an intellectual unity to which we may reasonably look forward in the future; if, indeed, it can be called intellectual unity, for the unity which we anticipate will frankly recognize that the basis of all thought is emotional rather than intellectual, that thought is nothing more than the emotions become self-conscious—a conclusion which a modern writer has expressed in the striking phrase: “Reason can clear away error; it can give us no new light.”

Meanwhile the external conditions of modern society are cooperating to lift the masses out of the grooves in which they move and have their being. Rapid mechanical development has not lessened but increased the drudgery of the world; money-making has not, as our political economists prophesied, made the many rich, but has precipitated the masses into the most abject poverty the world has ever seen, while free trade and universal markets have not inaugurated an era of peace and goodwill among nations, but have plunged society into endless wars. Hence the majority of people to-day, feeling that the tendency of modern civilization is to add more to the sorrow than to the joy of life, are beginning to ask themselves what Carlyle and Ruskin were asking themselves fifty years ago—whither modern civilization goeth. And so it is not unreasonable to expect that the force which is to carry us back to the Guild system is now germinating in our midst. The failure of modern society to realize itself will result in an effort towards finding lost roads. The people will come to connect the Golden Age with the past again rather than with the future. And such a change is just what is wanted to unite the people with the intellect of the age. For it is the popular superstition exalting the present age at the expense of the past which, more than anything else, perhaps, separates ignorant from cultured people, who, being better informed, are, on the one hand, unable either to take part in popular activities, and on the other, to get a following themselves.

A reverence for the past, then, is the hope of the future. This is the testimony of history. Consider what the consciousness of a glorious past did for the Italians of the Renaissance. It was the hope of restoring the ancient splendour of the Senate and the Republic which created the force that gave birth to the great achievements of that era. And though this ambition was not realized, the ideal which it inspired in the national mind remained a force of reconstruction. In like manner a reverence among the English people for the achievements of the Middle Ages in architecture and the crafts, and of the Elizabethan era in literature, would become an influence coordinating a multitude of our national activities. The spirit of emulation which such a reverence would engender in society, by setting certain forces in motion, would add just those ingredients which are lacking and lead us out of the quagmire of materialism toward the realization of a happier and more beautiful life.

Evidence is not wanting that the change will be in the direction here indicated. It is only a century ago that Sir Walter Scott thought it necessary to apologize to his readers for his love of Gothic Architecture. Compare England to-day with the England of 1851, with its insolent belief in its own self-sufficiency. If external evidence be called for, take as a common instance the development of the trade in antique furniture and bric-a-brac, yet many of the men who set this force in motion are still with us. Indeed, the change of feeling which is now to be seen coming over the national mind appears to be nothing more than a change which was consummated in the world of art three-quarters of a century ago, for just as the eighteenth century architects thought they had reached a state of perfection equalled only by the ancients, so the mid-nineteenth century thought itself on the high road towards perfection. Subsequent experience, however, revealed the idea in each case to be illusory—a false development preparatory to accelerated decline, since, being essentially artificial, both had moved out of contact with actuality. And just as it was found necessary in the arts to seek a new source of inspiration in a study of Mediaevalism, so society by a reverence for the past may renew its lease of life. We live the life of the past to-day in our thoughts, to-morrow we may live it in reality. Thus hopes may be entertained that what has been may again be, and under new conditions with new possibilities, may be again in fuller measure and more complete perfection.

\footnotetext[1]{A possible objection to this is that the raising of the standard of taste will not affect the engineering trades. The answer is that the engineering trades will shrink immeasurably in the future. The growth of the engineering trades corresponds with the growth of artificial conditions of life, and as life in the future will be lived under simpler conditions, they will shrink proportionately.

}\footnotetext[2]{Arts and Crafts Essays. A collection of essays by members of the Arts and Crafts Society.

}\footnotetext[3]{In this connection special mention should be made of the admirable work done towards leavening the workers by the L.C.C. Technical Schools since Professor Lethaby was appointed Art Adviser to the Council.

}\chapter{Conclusion}
\label{chapter-6}
Having stated the direction in which social salvation may be sought, it remains for us to define more precisely the immediate work which is to be done; for, though it has been shown that the solution of our problems depends ultimately upon the regeneration of the spiritual life of the people, it is to be feared that such a generalization is insufficient as a guide to action, action being based rather upon what is sensuously apprehended in the concrete than upon what is intellectually comprehended in the abstract. To enter further into detail, therefore, it appears to us that the spiritual regeneration of society might be facilitated if the reform movement could be persuaded to concentrate its energies upon:

\begin{enumerate}
	\item The stimulation of right thinking upon social questions.


	\item The restoration of a spirit of reverence for the past.


	\item The dissemination of the principles of taste.


	\item The teaching of the elements of morality, especially in relation to commerce.


	\item The insistence upon the necessity of of personal sacrifice as a means to the salvation alike of the individual and of the State.



\end{enumerate}
Respecting these, the first has always been part of the reform programme, and therefore does not require elaborating. The second is recommended not in advocacy of a dogmatic or emotional reversion to obsolete models, but because a reverence for what was really good and valuable in the life of the past must not only form the basis of all rational culture, and thereby of clear thinking in sociology as in other topics, but because it would provide the best antidote to the iconoclasm of the present age, with its tendency to over-rate its own importance, thereby enabling the mind to gauge to what extent progress, and to what extent retrogression have taken place. In a word, it is the only way of restoring to the national mind the sense of the right proportion of things.

The dissemination of taste is necessary not only to teach people how to live reasonably, for taste is the best protection which the individual can possess against the love of ostentation and vulgar display, but it is a necessary condition of the abolition of commercialism. Until we can reform popular taste it is to be feared we shall have to submit to commercialism, since commercialism has its roots in the vulgarity of modern life on the one hand, and the general debasement of taste on the other. The need of the hour is not more schools of art, producing an unlimited supply of amateur artists with an exaggerated sense of their own capacities, but schools of taste which would teach them how little they know.

Again, it is necessary to teach the principles of morality in order to awaken people to the immorality of conventional morals.\footnotemark[1] This is especially true of the ethics of business, where precept and practice have drifted so far apart that there is great danger of the sense of honesty disappearing from amongst us.

Lastly, the insistence upon the necessity of personal sacrifice as a means to the salvation alike of the individual and of the State is particularly necessary to-day, because social reformers have ceased to insist upon it. Nevertheless, sacrifice, as hitherto, remains the law of life. To attain the higher good the lower must be sacrificed. Hence it is that just as for the individual salvation the higher pleasures may only be attained by the sacrifice of the lower, so the welfare of the State demands that the individual members forego personal advantages for the general well-being—that indeed they cultivate the spirit of those Spartan envoys, who, when asked whether they came with a public commission or on their own account, replied, “If successful, for the public; if unsuccessful, for ourselves.” What a contrast to the spirit of to-day—to the spirit which animates so many men of pious and progressive opinions, who shuffle off their personal responsibility wherever possible to the shoulders of the State. It is not by these, but by those who can realize their responsibilities and are prepared to make the necessary sacrifices, that reform will be achieved. To resist commercialism is in these days to resist the devil in the nearest and most obvious sense of the word, since until a man has successfully overcome its temptation he is incapable of really valuable work, for he will lack clearness of vision. A man may read and think as much as he pleases, but if he be without the courage to test his conclusions by practice he can possess no vital knowledge. His thoughts will lack precision. It is only by living his thoughts that they may be focussed to the surrounding circumstances of life. As was said of old, “If thine eye be single thine whole body shall be full of light.”

Such is a brief outline of the changes necessary to the spiritual regeneration of the people. In the meantime the constructive forces in society will be unable to make much headway. For as Carlyle says, “If you have much wisdom in your nation, you will get it faithfully collected; for the wise love wisdom, and will search for it as for life and salvation. If you have little wisdom you will get even that little ill-collected, trampled under foot, reduced as near as possible to annihilation/’ For the same reason until we can get some wisdom in the English nation, Trades Unionism, which we have come to regard as the new centre of order, will be too much on the defensive to work definitely for the salvation of society, while the revival of art and handicraft, upon which the abolition of commercialism in one sense depends, will be unable to break down the barriers which restrict its operations to a very limited sphere. It cannot be insisted upon too strongly that this reform cannot be achieved unless art reformers secure the good will and cooperation of the public. The experience of the Arts and Crafts movement has proved conclusively that unless the public can be persuaded to give organized support to the movement and are prepared to take a share in the sacrifices involved, professional effort can avail nothing.

And yet, do what we can, sooner or later we come into collision with the spiritual degradation of society. So long as people persist in their present extravagant habits of life their expenditure must always tend to get ahead of their incomes, and they will put up with the jerry and false in consequence. Modern life is an exact counterpart of modern art. It is fundamentally insincere, lacks restraint, worships shams. We have, as Carlyle says, “given up hope in the Everlasting and True, and placed hope in the Temporary, half or wholly false.” The further we inquire into the decadence of modern art the more we find that every aspect of it, yea, every line of it corresponds to some trait in the national character. It is the art of a people who have descended to every deception for the purpose of making money. Never did popular art more faithfully represent the national life than to-day.

According to the old conception, art, religion, ideas, integrity of work, the pursuit of perfection, were looked upon as the serious things of life, while business and money-making were subservient things, considered relatively unimportant, for men had a reverence for the truth and abhorrence for the false. And this we maintain is the right conception. Except a Society acknowledge the real things of life, and be true to them it cannot remain healthy. It is not economic causes outside our individual control which have given rise to the social problem, but the degradation of the spiritual which has followed the usurpation of life by business and money-making.

And this is the choice we are called upon to make. Are we content to continue living as a nation of Philistines, indifferent alike to poetry, religion, ideas and art, worshipping vulgar success, wasting our precious gifts in sordid speculative enterprises and our leisure in senseless luxury, dissipation and excitement? If so we shall continue to wallow in the troughs of commercialism, and no solution of our problems is possible, though every voice in the land should demand it. Unless individually we are prepared to live for the truth and to make sacrifices for it, Society will remain as at present at the mercy of the speculator, the sweater, the hustler, the mountebank and the adventurer. For there can be no remedy. More important to a nation than the acquisition of material riches is the welfare of its spiritual life. This is the lesson the social problem has to teach us; not until it is learned shall we emerge from the cloud of darkness in which we are enveloped.

\footnotetext[1]{The reader could not do better than read the Essay on the “Defence of Criminals,” in “Civilisation, its Cause and Cure,” by Mr. Edward Carpenter.

}

\end{document}
