% !TeX spellcheck = en_GB
 % Use  to include (non HTML-escape) variable foo instead of {{{foo}}}
\documentclass{book}

%% Pacake inclusion
% Unicode support if xelatex is used
\usepackage{fontspec}
\usepackage{xunicode}

\usepackage[english]{babel} % Language support
\usepackage{fancyhdr} % Headers

% Allows hyphenatations in \texttt
\usepackage[htt]{hyphenat}





% Included if the stdpage option if set to false
\usepackage[a5paper, top=2cm, bottom=1.5cm,
  left=2.5cm,right=1.5cm]{geometry} % Set dimensions/margins of the parge


\makeatletter
\date{}

% Redefine the \maketitle command, only for book class (not used if stdpage option is set to true)
\renewcommand{\maketitle}{
  % First page with only the title
  \thispagestyle{empty}
  \vspace*{\stretch{1}}
  
  \begin{center}
    {\Huge \@title   \\[5mm]}
  \end{center}
  \vspace*{\stretch{2}}
  
  \newpage
  % Empty left page
  \thispagestyle{empty}
  \cleardoublepage

  % Main title page, with author, title, subtitle, date
  \begin{center}  
    \thispagestyle{empty}
    \vspace*{\baselineskip}
    \rule{\textwidth}{1.6pt}\vspace*{-\baselineskip}\vspace*{2pt}
    \rule{\textwidth}{0.4pt}\\[\baselineskip]
    
    {\Huge\scshape \@title   \\[5mm]}
    {\Large }
    
    \rule{\textwidth}{0.4pt}\vspace*{-\baselineskip}\vspace{3.2pt}
    \rule{\textwidth}{1.6pt}\\[\baselineskip]

    \vspace*{4\baselineskip}

    {\Large \@author}
    \vfill
    
  \end{center}
  
  \pagebreak
  \newpage
  % Copyright page with author, version, and license
  \thispagestyle{empty}
  \null\vfill
  \noindent
  \begin{center}
    {\emph{\@title}, © \@author.\\[5mm]}
    {This work is free of known copyright restrictions.\\[5mm]}
  \end{center}
  \pagebreak
  \newpage
}


% Redefine headers
\pagestyle{fancy}
\fancyhead{}
\fancyhead[CO,CE]{\thepage}
\fancyfoot{}



%%%%%%%%%%%%%%%%%%%%%%%%%%%%%%%%%%%%%%%%%%%%%%%%%%%%%%%%%%%%%%%%%
% Command and environment definitions
%
% Here, commands are defined for all Markdown element (even if some
% of them do nothing in this template).
%
% If you want to change the rendering of some elements, this is probably
% what you should modify.
%
% Note that elements that already have a LaTeX semantic equivalent aren't redefined
% : if you want to redefine headers, you'll have to renew \chapter, \section, \subsection,
% ..., commands. If you want to change how emphasis is displayed, you'll have to renew
% the \emph command, for list the itemize one, for ordered list the enumerate one,
% for super/subscript the \textsuper/subscript ones.
%
%%%%%%%%%%%%%%%%%%%%%%%%%%%%%%%%%%%%%%%%%%%%%%%%%%%%%%%%%%%%%%%%%%%

% Strong
\newcommand\mdstrong[1]{\textbf{#1}}

% Code
\newcommand\mdcode[1]{\texttt{#1}}

% Rule
% Default impl : (displays centered asterisks)
\newcommand\mdrule{
  \nopagebreak
  {\vskip 1em}
  \nopagebreak
  \begin{center}
    ***
  \end{center}
  \nopagebreak
 {\vskip 1em}
 \nopagebreak
}

% Hardbreak
\newcommand\mdhardbreak{\\}

% Block quote$
\newenvironment{mdblockquote}{%
  \begin{quotation}
    \
}{%
  \end{quotation}
}


% Code block
%
% Only used if syntect is used for syntax highlighting is used, else
% the spverbatim environment is preferred.





\makeatother

\title{The Control and Distribution of Production}
\author{C. H. Douglas}

\begin{document}

% Redefine chapter and part names if they needs to be
% Needs to be after \begin{document} because babel



\maketitle

\setcounter{tocdepth}{0}
\setcounter{secnumdepth}{0}
\chapter*{Preface}
\label{chapter-0}
Certain of the chapters in this volume were first delivered as lectures before the Sociological Society, the Ruskin College at Oxford and the National Guilds League; whilst the others appeared in the pages of \emph{The New Age} and \emph{The English Review}, for which full acknowledgment for permission to reprint them here is made to their respective Editors.

\chapter{The Mechanism of Consumer Control}
\label{chapter-1}
No doubt to some members and guests of this Society much of the subject with which we are concerned to-night will be elementary, even if the method of approach to it is somewhat novel; but to others to whom the subject of Finance, which is an important component of it, is a mysterious and incomprehensible jungle, through which they feel they could never hope to find a way, I would make the following suggestions.

Money is only a mechanism by means of which we deal with things—it has no properties except those we choose to give to it. A phrase such as “There is no money in the country with which to do such and so” means simply nothing, unless we are also saying “The goods and services required to do this thing do not exist and cannot be produced, therefore it is useless to create the money equivalent of them.” For instance, it is simply childish to say that a country has no money for social betterment, or for any other purpose, when it has the skill, the men and the material and plant to create that betterment. The banks or the Treasury can create the money in five minutes, and are doing it every day, and have been doing it for centuries.

Secondly, you will hear a good deal to-night about credit, and I would ask you to bear most consistently in mind the two following definitions:-—

\emph{Real credit} is a correct estimate of the rate, or dynamic capacity, at which a community can deliver goods and services as demanded.

\emph{Financial credit} is ostensibly a device by which this capacity can be drawn upon. It is, however, actually a measure of the rate at which an organisation or individual can deliver money. The money may or may not represent goods and services.

I would also ask you to realise that the validity of the criticisms passed on the existing financial system does not rest to any considerable extent on the personal character, or the good or bad motives, of financiers. The motives of both sides of the Irish question, for example, may be of the most lofty, for all that I know to the contrary, and no one would suggest that there are not charming men on both sides; but one can hardly say that the result of their policy is happy, and that either side can be allowed to pursue a policy having such results, indefinitely, and the same line of reasoning can be applied to the existing financial system.

Before dealing with the subject described by the title of this address, I would therefore beg your indulgence for a short space of time in order to review briefly certain premises fundamental to the subject; because it has been found that even people very familiar with these matters are apt to raise vigorous objections which are really based on other and inconsistent premises unless they are placed in the limelight at once, and, as far as possible, simultaneously. If you disagree with these premises, you will of course disagree with our conclusions, but if you agree, and still dislike the conclusions, I hope you will tell us where the hiatus occurs and suggest another solution based on them.

Categorically, they are as follows:—-

1. Modern co-operative industry (all modern industry is co-operative) serves two purposes: it makes goods, and distributes purchasing power by means of which they are distributed.

2. The primary object of the overwhelming majority of persons who co-operate in industry is to get \emph{goods} with a minimum of discomfort, both of the right description, “right” being a matter of individual taste, and in the right quantity. It is not “employment,” and it is only “money” in so far as money is a means to these things.

If the system fails to achieve this end, it fails in its primary object and will break up, from the failure of the majority to co-operate.

3. If we insist that the distribution of the goods is entirely (Marxist) or chiefly (Capitalistic) dependent on the doing of work in connection with the production of them, then it follows that either (a) it takes all the available labour to provide the requisite amount of goods, or (6) an increasing number of persons cannot get the goods, or (c) goods or labour must be misapplied or wasted, purely for the purpose of distributing purchasing power.

We know that (a) is not true. If it were, the whole of modern progress would be a mere mockery. But, on the contrary, it is quite indisputable that, apart from many other factors making for real progress, production is practically proportionate to the dynamic energy applied to it, and the means developed during the past century by which solar dynamic energy (steam, water, oil-power, etc.) has been made available to the extent of thousands of times that due to human muscular energy (which yet, previous to this development, was able to secure for humanity a standard of life in many ways more tolerable than that existing to-day) is sufficient basis for such an assertion. Speaking as a technical man, I have no hesitation in saying that it is the programme of production and not the productive process which is chiefly at fault, and that where the productive process is working badly it is because of the inclusion of unnecessary labour in it.

(b) and (c) are true, as matters of both common and expert observation.

4. The system under which the whole of the world, not excluding Russia, carries on the production and distribution of goods and services is commonly called the Capitalistic system, which system, contrary to general opinion, has nothing, directly, to do with the relations of employers and employed, which are administrative relations. The fundamental premises of the Capitalistic system are, first, that all costs (purchasing power distributed to individuals during the productive process) should be added together, and recovered from the public, the consumer, in prices; and, second, that over and above that the price of an article is what it will fetch.

If you will give the foregoing premises your careful consideration, you will see that the existing economic system is breaking up, not so much from the attacks on it, which, on the whole, are neither very intelligent, nor very well directed, but because of the inherent incompatibility of its premises with the objective of industry and modern scientific progress as a whole.

The latter, taking the objective of industry as it finds it, endeavours, and fundamentally succeeds, in obtaining that objective with an ever-decreasing amount of human energy, by shifting the burden of civilisation from the backs of men on to the backs of machines; a process which, if unimpeded, must clearly result in freeing the human spirit for conquests at the moment beyond our wildest dreams.

The existing economic system, on the contrary, ably backed by the Marxian Socialist, takes as its motto that saying which I cannot help thinking proceeded rather from Saul of Tarsus than from the Apostle of Freedom–“if a man will not work, neither shall he eat”–and defining work as something the price of which can be included in costs and recovered in price.

It completely denies all recognition to the social nature of the heritage of civilisation, and by its refusal of purchasing power, except on terms, arrogates to a few persons selected by the system and not by humanity, the right to disinherit the indubitable heirs, the individuals who compose society.

May I emphasise this fact before passing on to more concrete arguments?–if wages and salaries, forming a portion of costs, and reappearing in prices, are to form the major portion of the purchasing power of Society, then modern scientific progress is the deadly enemy of Society, since it aims at replacing the persons who now obtain their living in this way, by machines and processes.

The prevalent assumption that human work is the foundation of purchasing power has more implications than it is possible to emphasise to-night; it is the root assumption of a world—philosophy which may yet bring civilisation to its death-grapple; but one result of it is that a man and a machine are, in the eyes of a cost-accountant, identical to the extent that both are an expense, a cost which must reappear in price, the man, however, being at this disadvantage as compared with a machine, that he has to bear his own maintenance and depreciation charges. Costs are a dispensation of purchasing power; and whether you are disciples of the “Cost” theory of prices, or of the “Supply and Demand” theory, you must admit that Capitalistic prices cannot be less than cost, over any considerable period of time.

If, therefore, a portion of the “costs” of production are allocated to machines, and yet reappear in ultimate prices, it is obvious that the costs (purchasing power) in individual hands are not sufficient to pay these prices.

I do not wish to pursue at great length this aspect of the subject to-night, because it has been elaborated in considerable detail in print and does not lend itself to platform discussion. But one consideration must be mentioned—the effect on the prices of ultimate products—those consumed by individuals—of the production of intermediate products—tools, factories, raw materials, etc. While, as has just been suggested, the flow of purchasing power to individuals through the media of wages, salaries, and, it may be added, dividends, is not sufficient to buy the total price-values created in the same time, it must be remembered that a great and increasing quantity of the total production of the world is not bought by individuals at all—it is bought and paid for by organisations, national or otherwise, and is of no use to individuals.

Now the costs of this production represent effective demand to individuals; and the second postulate of the present economic system is that \mdcode{average price = effective demand /\allowbreak{} goods in demand}.

Consequently, the more of these intermediate products we produce, under the present system, the higher rise the prices of goods for individual consumption; which is the reason why the cry for indiscriminate super-production is both inane and mischievous. You will see at once that if the above formula for price, under the so-called law of supply and demand, is correct, which I suppose is not disputed, then it is really immaterial whether more or less goods are made, and more or less money distributed—any quantity of goods \emph{less than sufficient} will absorb all the money available. And because the Capitalistic incentive to production is money, production stops when there is no more money.

You will see that, firstly, the existing system does not distribute the control of intermediate production to individuals at all; and, secondly, gives them no say whatever as to the quantity, quality or variety of ultimate products.

The distribution of purchasing power through the agency of the present volume of wages, salaries and dividends thus fails to distribute the product; and since when distribution stops production stops, the system would appear quite unworkable.

But we know, as a matter of observation, that, although the grinding and groaning of the machine is plainly audible evidence that it \emph{is} working very badly, it is working, and there must be something to account for the fact that distribution of a sort does take place. There are two things: export credit and loan credit.

Now I may say at once that I do not see how it is possible to conceive of an economic system capable of dealing with the modern productive system in which this credit factor in the total sum of purchasing power does not play a preponderating and increasing part. It is far better to arrive at conclusions of this sort inductively rather than deductively, and I will simply direct your attention to the present trade position in this country and in America. There is the plant; there is the raw material; there is labour; and there is real, though not effective, demand; but production is decreasing along a very steep curve.

Why? I do not suppose anyone here to-night is guileless enough to believe that it is all the fault of Labour. It would do the Labour extremists all the good in the world, and might modify their policy, if they could be brought to realise that Labour, while a necessary factor in production, is less and less a determining factor. The success of the various dilution measures carried through under the stress of war is quite convincing proof of that fact.

Nor is it Capital, in the ordinary sense of the word. A man who has sunk large sums of money in a manufacturing plant wants to manufacture, if lie can, because otherwise his plant is a dead loss to him.

There is no doubt whatever, and I do not suppose that anyone at all familiar with the subject would dispute the statement for a moment, that the present trade depression is directly and consciously caused by the concerted action of the banks in restricting credit facilities, and that such credit facilities as are granted have very little relation to public need; that, whatever else might have happened had this policy not been pursued, there would have been no trade depression at this time, any more than there was during the war; and that the banks, through their control of credit facilities, hold the volume of production at all times in the hollow of their hands. You will, of course, understand that no personal accusation is involved in this statement; the banks act quite automatically according to the rules of the game, and if the public is so foolish as to sanction these rules I do not see why it should complain.

I should like, however, to emphasise this point: if the civilised world continues to permit this centralised, irresponsible, anti-public control of the life-blood of production to continue, and at the same time the possibly well-meaning but ill-informed and dogmatic Syndicalist makes good what is in essence exactly the same claim in the administrative field, then the world, in no considerable time, will be faced with a tyranny besides which the crude efforts of the Spanish Inquisition may well retire into insignificance.

Let me repeat—the only true, sane origin of production is the real need or desire of the individual consumer. If we are to continue to have co-operative production, then that productive system must be subject to one condition only—that it delivers the goods where they are demanded. If any men, or body of men, by reason of their fortuitous position in that system, attempt to dictate the terms on which they will deliver the goods (not, be it noted, the terms on which they will work), then that is a tyranny, and the world has never tolerated a tyranny for very long.

There is, I think, a widespread idea that if agitators would only stop agitating, and reformers stop trying to reform, the world would settle down. For myself, I am quite convinced that both agitation and reformism are merely symptoms of a grave and quite possibly fatal disease in our social and economic system, and that unless an adequate remedy is administered there will be an irreparable breakdown. I am emphasising this lest anyone should imagine that mere \emph{laissez-faire} or, on the other hand, a vigorous suppression of symptoms is all that is necessary to cause things to “come right.”

The roots of this disease, then, are as follows:

\begin{enumerate}
	\item Wages, salaries, and dividends will not purchase total production. This difficulty is cumulative.


	\item The only sources of the purchasing power necessary to make up the difference are loan and export credits.


	\item All industrial nations are competing for export credits. The end of that is war.


	\item The major distribution of purchasing power to individuals is through the media of wages and salaries. The preponderating factor in production is improving process and the utilisation of solar energy.


	\item This latter tends to displace wages and salaries and the consequent distribution of the product to individuals. The credit factor in purchasing power thus increases in importance and dominates production.


	\item This production is consequently of a character demanded by those in control of credit and is capital production.


	\item The fundamental derivation of credit is from the community of individuals, and because individuals are ceasing to benefit by its use it is breaking down.



\end{enumerate}
If you have followed me so far you will see that there are two main and increasing defects in the present system—it makes the wrong things and so is colossally wasteful, and it does not satisfactorily distribute what it does make. The key to both of these is the control of credit.

I should like to direct your attention to the meaning which can be attached to the word “control.” We talk about the “public” control of this, that or the other. Is there any person in this room who has ever met the public, or knows in any clear-cut, tangible fashion, this alleged entity, the public, or really—if he or she is honest in the use of words—cares a broken rush about the public? Is it the public which wants better houses, better food, a wider life? I think not. When there is “unemployment” it is John Smith, Jane Smith and the Little Smiths who experiment with rationing. When there is a war it is Private, Lieutenant or Colonel Smith who loses an arm or whose wife places a wreath on the Cenotaph. I have not noticed that the name of the Public appears in the casualty lists of any of the nations engaged in the late war.

I do not suggest for a moment that there is not a real group-consciousness—I think that there is such a consciousness. But the ills from which we are suffering do not take effect on that plane of consciousness, they take effect on individuals; and if, as I have tried to indicate, the key to the solution of those ills is to be found in a modified control of credit, then that modification must be in favour of individuals. We can, I think, safely leave the group-consciousness to look after itself.

The problem, then, is to give to individuals such personal control of credit as will enable each of them, for himself or herself, to get from the machine of civilisation those things, now lacking, to the extent that the machine is capable of meeting the demand, and the answer is almost childishly simple—it is contained in the proposition that he ought to be able to buy those things with the money at his disposal, and that if he does not want to buy them, then he should not be made to pay for them.

If you will consider this matter in the light of everyday conditions in the world of business, you will find that the practical steps necessary to embody these principles in a practical mechanism resolve themselves into two groups—the control in the interest of the consumer of the credit issued to manufacturers, in order that those things shall be made which the ultimate consumer wants made—because the ultimate consumer should be the sole arbiter of the \emph{policy} of production, though not concerned with the processes by which his policy is materialised; and, secondly, that the credit, or purchasing power, in the hands of the consumer shall be adequate to enable him, if necessary, to draw on the maximum resources of the productive organisation; otherwise, it is clear, a part of those resources is ineffective.

As we have previously noticed, individuals in the modern world obtain their purchasing power through three sources—wages, salaries and dividends. This purchasing power is taken away from them through the medium of what we call prices, and it will be quite obvious to you that the first thing necessary is to make total purchasing power equal to total prices, a proposition which has no other known solution than by the addition of a credit issue to purchasing power. That is to say, \emph{we must give the consumer purchasing power which does not appear in prices}.

Please remember that prices contain not only production costs, but capital costs, and these latter are the increasing factor in both costs and prices. If we take them out of prices and distribute them as purchasing power, then prices bear the same relation to costs as does consumption to production. You will see that this is so if you remember that capital charges represent sums based on the credit value of tools, etc.

But, of course, this results in speedy bankruptcy to the producer who is selling under cost, unless we go a good deal further.

It must be borne in mind that, though we find that we require to eliminate these credit-capital charges from prices, the credit-capital is a real if intangible thing, and can be drawn upon, because tools, processes, solar power, etc., represent a real capacity to deliver goods and services. Therefore there must be something somewhere which stands in the position of trustee for the collective credit, and should administer it in the interests of the individuals. There is such an organ—it is the Treasury.

But the Treasury does not in normal times deal with manufacturers, it deals with the banks, and the banks are so-called private institutions which administer this collective credit for their own ends, and those ends are by no means similar to the ends of the community of individuals from whom the credit takes its rise.

If, therefore, we wish to solve the first half of the problem, that of the control, in the interest of the consumer, of the credit issued to manufacturers, we have to put control of the policy of the banks at the disposal of the consumer interest.

If, at the same time, we wish to ensure that the goods, when they \emph{are} produced, are distributed amongst the individuals in whose interest, \emph{ex hypothesi}, they were made, we have to get the credit purchasing power which attends the capacity to make and deliver them into the hands of those individuals. We can deal with this latter problem in two possible ways—either by a gift of Treasury “money” obtained by a creation of credit, or by reducing prices below cost to the individual consumer, and then making up this difference between price and cost by a Treasury issue to the producer. I hope you realise that the only basis for such a credit issue is the difference between what the productive organisation is called upon to deliver and what it could deliver if its capacity were stretched to the utmost.

The latter of the two foregoing alternatives is, I think, by far the more practicable, because it not only delivers the purchasing power at the moment that it is wanted—at the moment of purchase—but it is also far better adapted to the psychology of the present time. It is the method which has been embodied in the suggestions which Mr A. R. Orage and I have been endeavouring to bring to the notice of the public in the Draft Scheme for the Mining Industry.

This scheme has been fairly widely discussed, both here and in America, but there is one feature of it which will perhaps bear a little elaboration—the obvious traversing of all accepted Socialist policy in the provision not only for the continuance of dividends to present shareholders, but the wide extension of those dividends to still more shareholders. I will not take up your time with the philosophic basis of the proposal, although it has such a basis; but would merely draw your attention once again to the quite undeniable fact that there is simply not room in \emph{economic} industry—by which I mean industry financed from public credit—for more than a small and decreasing fraction of the available labour. The attempt to cram all this human energy into a function of society which has no need of it is neither more nor less than lunacy. But we have to recognise, as a matter of common sense, that to throw a large and inexperienced section of the population out of its usual pursuits suddenly, and without preparation, and with more spending power than it has the training to use, might have a number of unpleasant consequences. I do not believe for one moment in all the nonsense talked about work and drink being the only alternatives of the British working-man—it is a gross calumny; but a smooth and rapid transition stage is desirable, and that is provided in the scheme by the increasing substitution of wages by dividends. When this process had proceeded far enough we should have defeated also one of the worst features of the present system, which is unable to distribute goods made and stored, without making more goods, whether these are required or not, merely for the purpose of distributing purchasing power. You will no doubt ask what are the prospects of such a scheme as we are considering.

Well, in the first place, it has to be observed that the uncoordinated parts of it are coming into being with tremendous rapidity and, to those who have eyes to see, with irresistible momentum. In this country it is quite obvious that not only cannot the public debt (all issues of securities, whether to so-called private companies, local authorities or Governmental bodies, are public debt fundamentally) be reduced, but the business of the country cannot be carried on for a month without a continuous increase in it. The immediate effect of an attempt to restrict the flow is a slump in trade and an avalanche of business crises, which is only just beginning, but which will, unless I am very much mistaken, or war provides an alternative, proceed to lengths quite sufficient to establish the principle.

The mechanism is being forged. The Brotherhood of Locomotive Engineers in America has, on the first of this month, opened the doors of the first of a series of banks whose credit rests fundamentally on the railway services of the American continent, not on the cash in the vaults of the bank. The Confederation General de Travail is about to inaugurate a bank with a nominal capital of 25,000,000 francs on the same lines. These are the beginnings of the shifting of control.

The operations of these organisations will, in the first place, assist in raising prices—in fact, by enormously enhancing the economic power of Labour, will tend to raise them considerably. But as the toothache is the only agency which will drive the majority of people to a dentist, there will be posed thereby a plain issue—and to that issue I do not know any other reply than that I have endeavoured, so far as time has allowed, to put before you.

\chapter{The Control of Policy in Industry}
\label{chapter-2}
Your Principal has been flattering enough to suggest that you might be interested to listen for a short time to-night to certain ideas on the subject of the industrial problems which have been made public, for the most part, through the columns of \emph{The New Age}. Before proceeding to the concrete proposals, I should like, with your permission, to go over the philosophy of them very briefly.

In any undertaking in which men engage, to paraphrase the ever-green Sir W. S. Gilbert, there are always at least two fundamental aspects which demand recognition before success can possibly be expected to accrue to those engaged in it. These are that there must first be a clear, well-defined policy, which means that every person who has any right to be heard in the matter in hand shall agree as to the \emph{results} which he is willing to further with his support. And there must be somewhere resident in the venture some person or persons with expert knowledge as to the technical processes by which those results can be achieved with the materials (using the word in its broader sense) at the disposal of those associated together, and this person must have the confidence of the remainder.

I should like you to observe particularly that certain very important—in fact, quite fundamental—relationships proceed from these simple premises. The genesis of such an association is agreement that a certain result is desirable and a general belief that it can be attained—it is not at all necessary that all of those associated shall know how to attain the result, but it is vital that they shall be satisfied with it. We may imagine this association to be the community. Secondly, the person or persons who “know how,” who collectively we may call the producers, who will be empowered by the community to materialise the results of the agreed policy, stand fundamentally and unalterably on a basis of Service—it is their business to deliver the goods to order, not to make terms about them, because it is the basis of the whole arrangement that the general interest is best served by this relationship. (This applies, of course, to their simple function of producers, not to their comprehensive, all-embracing r61e of individuals.) Subject to this fundamental provision that they deliver the goods to order, it is no business of the controllers of policy, the community, how the producers deliver them—that is a matter for agreement amongst the producers.

The goods having been delivered to order, it is the business of the community, to whose order they were made, to dispose of them—not the business of the producers, who would never have been able to function without the consent of society.

Now in the present dissatisfaction with the productive system which is the outstanding feature of the present time there is a remarkable misdirection of attack—the battle front is aligned as between employer and employed, the so-called Capitalist and Labour, whereas the real cleavage is between “producer-distributor” (both controlled by the financier) and consumer—employers and employed forming the producer-distributor army, and the whole community, which includes the producers, forming the opposition. If anyone doubts this, a consideration of the facility with which Labour obtains increases of pay, just so long as these increases can be recovered from the public in the form of increased prices, will surely dispel the doubt. The position, therefore, is one of civil war of the gravest character—gravest because the “victory” of either side means the destruction of both.

Before proceeding to the consideration of the means available to meet this situation, it is necessary to be clear on the matter of policy.

There is no possible definition of a policy which is all-embracing in its acceptance other than the word “Freedom.” People only unite in wanting what they want. We shall never get one inch farther along the road to a final settlement of world problems until we make up our minds for good and all whether a man is in the largest sense more benefited by learning, through trial and error, what is good for him, or, on the other hand, whether he should be ruled in the way he should go by Authority.

Personally I am convinced that the former conclusion is inevitable. The dictatorship of the proletariat or any other \emph{comprehensive} dictatorship is intolerable and impracticable. Please note that I am referring to man as an individual—not as a producer. The technique of production is a matter of Law, not of Emotion and Desire, and I believe that a much more exacting discipline will be expected of those of all ranks who are privileged to serve the community in any capacity, and that the penalty of failure to live up to that discipline will be the loss of that privilege, which will be a much greater loss when no economic question enters into it.

It used to be a very common argument that the spur of economic necessity was ennobling to the character. Frankly, I don’t believe it. If you will, and I am sure you will, look at the question from a detached point of view, I think you will agree that the man who is engaged in “making money” is neither so pleasant or so broad-minded to deal with, nor so fundamentally efficient, as the man who, while yet exerting his capacity for useful effort to the utmost, is by fortune lifted above the necessity of considering his own economic advantage. The struggle to overcome difficulties is most unquestionably ennobling, but we have, I think, reached a stage when our attention may with advantage be diverted from the somewhat sordid struggle for mere existence.

We want, therefore, to put more and finally all people in this position, not to remove from it those who are already there, always assuming that the alternative exists; and to do that we want so to organise the machinery of production that it serves the single end of forming the most perfect instrument possible with which to carry out the policy of the community; and so to empower the community that individuals will submit themselves voluntarily to the discipline of the productive process, because in the first place they know that it is operated for production and so gains their primary ends with a minimum of exertion, and in the second place because of the interest and satisfaction of cooperative, coordinated effort. You will understand that the physical facts of production are such that, operated in this way, only a small proportion of the world’s population, working short hours, could find employment directly in the industrial process—a condition of affairs which is cumulative and reduces to an anachronism the complaint of the early Victorian Socialist against the idle rich, and to an absurdity the super-Industrialist cry for greater production at a cost of harder work. To anyone to whom this aspect of the case is unfamiliar, I would commend the works of Mr Thorstein Veblen on Capitalist Sabotage, or the more specialised conclusions of the late H. M. Gantt and his partner, Mr Walter Polakov. The present preoccupation of the financial system is to hide the enormous capacity for output which modern methods have placed at our disposal; and it is fairly successful in its efforts, so far.

So much for the philosophy of the subject. If you agree with it you will. see at once that the problem with which society has to grapple falls naturally and inevitably into certain lines. The \emph{primary} object of the whole industrial system should be the delivery to individuals, associated together as the public, or society, of the material goods and services they individually require. This demand of individuals, be it emphasised, is the absolute origin of all activity. Since men co-operate to satisfy this demand, which is complex in its nature, it is necessary to also combine the demand, and this combined demand of society is the policy, so far as it is economic, of society as a whole. The first part of the problem, then, consists in finding a mechanism which will impose this policy on the co-operating producers with the maximum effectiveness, which always means with the minimum of friction.

Now, if I have made my meaning clear, you will begin to see (willingly or otherwise!) that this has nothing to do with “workshop control by the workers”–in fact is in one sense the antithesis of it. It involves the assumption that the plant of civilisation belongs to the community, not to the operators, and the community can, or should, be able to appoint or dismiss anyone who in its discretion fails to use that plant to the best advantage. So far you might say this is pure State Socialism, but I think you will agree, if I make myself clear, that it is nothing like what is commonly so called. In this connection the following paragraph from \emph{The Threefold State}, by Dr Rudolf Steiner, a book which is attracting attention on the Continent, may be of interest:-—

“Modern socialism is absolutely justified 1 in demanding that the present-day methods, under which production is carried on for individual profit, should be replaced by others, under which production will be carried on for the sake of the common consumption. But it is just the person who most thoroughly recognises the justice of this demand who will find himself unable to concur in the conclusion which modern socialism deduces: That, therefore, the means of production must be transferred from private to communal ownership. Rather he will be forced to a conclusion that is quite different, namely: That whatever is privately produced by means of individual energies and talents must find its way to the community through the right channels.”

The radical difference—and I would commend it to your most serious consideration—is that State Socialism is based on the premise that, firstly, the control of policy is resident in administration, and, secondly, that it is possible to “socially” control administration, and, thirdly, that the State should be able to supply economic pressure to the individual; whereas I suggest to you that the control of policy is resident in credit (fundamentally, in the belief in the beneficial outcome of any line or action) and its financial derivations, of which money is one, while administration is a technical and expert matter not susceptible of being socialised, and, lastly, that the only possible method by which the highest civilisation can be reached is to make it impossible for either the State or any other body to apply economic pressure to any individual.

Any attempt either to socialise administration or to govern by economic coercion quite inevitably leads to centralised organisation and centralised credit, resulting in all the well-known phenomena of inefficiency inseparable from the attempted subordination of the human ego to the necessities of a non-human system. The difference is the recognition of the difference between beneficial ownership and administrative ownership. The managing director of the White Star Line was in beneficial ownership of the \emph{Titanic}, he controlled the credit of it; but his attempt to interfere in its administration destroyed the \&.

We can, then, for the moment leave the question of administration where it stands, the more so if you will consider that, however certain enthusiasts may endeavour to persuade you to the contrary, it is a well-recognised fact that it is impossible, in this country at any rate, to promote a strike of any magnitude on any basis but that of distribution—i.e. wages or prices—which only shows the general good sense of the British public.

It is not suggested that administration is faultless, but by deferring the consideration of it—for it is essentially a technical matter—we are free to concentrate on the primary requisite—the transfer of the control of the \emph{policy} of production into the hands of those for whom the whole productive process exists—the individuals who collectively form the public.

As has been stated, the control of policy is resident in credit—a word which is quite sufficient, I have no doubt, to excite your worst forebodings, but I assure you that in itself the matter is very simple. A credit instrument is something which will enable you to get what you want. If you are stranded without food on an island overrun with rabbits, a shot-cartridge is in all probability the most effective credit instrument with which to deal with the situation, but in more highly organised communities the instrument in most general use, and which typifies the rest, is what we call money. It differs from a cartridge chiefly in disappearing less noisily.

It is absolutely vital to realise that the essential part of money is the belief that through its agency you can satisfy your demands. Once this is agreed you will see that the control of the issue of something which embodies this belief is equivalent to the control of the policy of society. The belief, if well founded, is real credit, and its vehicle, financial credit, convertible into money.

There exists in civilised society in all countries to-day an institution whose business it is to issue money. This institution is called a bank. The banking business is in many respects the exact opposite of the Social Reform business—it is immensely powerful, talks very little, acts quickly, knows what it wants, chooses its employees wisely in its own interests.

When a bank allows a manufacturer an overdraft for the purpose of carrying out a contract or a production programme, it performs an absolutely vital function, without which production would stop. If you doubt this, consider for a moment the result of a rise in the bank rate of interest on loans and you will see that the power to choke off producers by taxing them at will is essentially similar to that exercised by governments on consumers by orthodox taxation, with the vital difference that in the first case a purely sectional interest is operating uncontrolled by society, whereas in the second case the power undoubtedly exists, though ineffective because misunderstood, to control it in the general interest.

Now the vital thing done by a bank in its financing aspect is to mobilise effective demand.

\emph{The effective demand is that of the public, based on the money of the public, and the willingness of producers to respond to economic orders; but the paramount policy which directs the mobilisation is anti-public, because it aims at depriving, with the greatest possible rapidity, the public of the means to make its demands effective; through the agency of prices.}

I would particularly ask you to note that there is no suggestion that \emph{bankers}, as human beings, are in the main actuated by any such anti-social policy—the system is such that they simply cannot help the result.

In order, then, to acquire public control of economic policy, we have to control the whole mechanism of effective demand—the rate at which its vehicle, financial credit, is issued, the conditions on which it is issued, and take such measures as will ensure that the public, from whom it arises, are penalised by withdrawal of the vehicle to the minimum possible extent. It must be obvious that the real limit of the rate at which something representing purchasing-power could be issued to the \emph{public} is equal to the maximum rate at which goods can be produced, whereas the “taking back” through prices of this purchasing-power should be the equivalent of the fraction of this potential production which \emph{is} delivered.

Let us imagine that wages, salaries and dividends, added together, were issued via the productive industries at a \emph{rate} representing the maximum possible production of ultimate products, and actual consumption was only one quarter of potential production. Then, clearly, the community would only have exercised one quarter of its potential demand. But the whole of the \emph{costs} of production—the issues of purchasing-power through the agencies of wages, salaries and dividends—would have to be allocated to the \emph{actual} production as at present, and if we charge the public with the whole cost of production their total effective demand is taken from them. But if we apply to the ascertained cost of production a fractional multiplier equal to the ratio of actual consumption to potential production, then we take back in prices that portion of the total purchasing-power which represents the actual energy draft on the productive resources of the community, and the price to the actual consumer would be, in the case above mentioned, 75\%, less than commercial cost.

If I have made myself clear you will see that credit-issue and price-making are the positive and negative aspects of the same thing, and we can only control the economic situation by controlling both of them—not one at a time, but both together, and in order to do this it is necessary to transfer the basis of the credit-system entirely away from \emph{currency}, on which it now rests, to \emph{useful productive capacity}. The issue of credit instruments will then not result in an expansion of money for the same or a diminishing amount of goods, which is inflation, but in an expansion of goods for the same or a diminishing amount of money, which is deflation.

I may perhaps be permitted to end on a graver note. The present maladministration of credit results in increasingly embittered struggles for markets. Unless it is remedied, war is inevitable—and the next, great war will destroy this civilisation.

\chapter{The Control of Production}
\label{chapter-3}
It has frequently and rightly been emphasised that the essence of any real progress towards a better condition of society resides in the acquisition of control of its functions by those who are affected by its structure; and it is well if somewhat vaguely recognised by the worker of all classes that this control is at present not resident in, but is external to, society itself, and that in consequence men and women, instead of rising to an ever superior control of circumstance, remain the slaves of a system they did not make and have not so far been able to alter in its fundamentals.

This system is assailed under the name of Capitalism; but of the millions who are convinced that by the destruction of Capitalism the Millennium will be achieved, not very many have yet awakened to the fact that Capitalism died an unhallowed death seventy-five years ago, more or less, and that the driving force of the system which, more than any other single cause, has produced the tangle of misery and unrest in which the world now welters is Creditism.

Credit is a real thing; it is the correct estimate of capacity to achieve, and the function and immense importance for good or evil of this real credit will be impressed on mankind with cumulative insistence in the difficult times ahead. But for the moment it is desirable to consider a narrower use of the word; one conveying, however, a sense with which it is more commonly associated—financial credit.

Financial credit is simply an estimate of the capacity to pay money—any sort of money which is legal or customary tender; it is not, for instance, an estimate of capital possessed; and its use as a driving-force through the creation of loan-credit is directly consequent on this definition. The British banking system has, since the Banking Act of 1844, based its operations on the ultimate liability to pay gold, but in actual fact the community, as a whole, has dethroned gold, and bases its acceptance of cheques and bills on its estimate of the bank credit of the individual or corporation issuing the document, and for practical purposes not at all on the likelihood that the bank will meet the document with gold. This bank credit simply consists of certain figures in a ledger combined with the willingness of the bank to manipulate those figures and at call to convert them into legal tender. What, then, is likely to induce a bank to increase the credit by the creation of loans, etc., of an applicant for that favour? The answer is contained in the definition: the capacity to pay money; and the credit will be extended absolutely and solely as the officials concerned are satisfied that this condition will be met. It is quite immaterial whether the judgment is based on existing “securities” or contemplated operations; the basis of bank credit to-day is simply and solely the capacity within an agreed time-limit, which may be long or short, to pay money.

Now apply the consideration of this to such a problem as control of the provision of decent housing for the miners at rents not exceeding 10\%, of the miners’ earnings. There are a number of idealists, who cannot be labelled otherwise than half-baked, who will say that it is a “sound business proposition” to house the miners properly at low rents. There are also a number of people by no means half-baked who are prepared to lose a little on housing to retain control of industry. That it is in the highest sense sound is unquestionable; but as to being a business proposition we suggest to those well-meaning people of the first class whose minds are above detail, that they go to the banks unsupported by security, and endeavour to borrow money for such a project.

We see, then, that it is purely a question of the financial effect likely to accrue from an enterprise which will induce the banks to back it with credit, and the use-value or inherent desirability of doing certain work is a by-product. But the deduction to be made from this is of transcendent importance—it is that to control industry in the interest of use-values you must back use-values with credit. And that means the control of credit. And in order to control credit the base on which it rests must be altered to meet the changed aspirations of society. The economic power of Labour is a potential power. By withholding it, Labour (using the term in its widest sense) can break down civilisation; but it cannot build it up again by any agency that the mind of man has yet conceived which does not involve the use of credit in some form or other. The community creates all the credit there is; there is nothing whatever to prevent the community entering into its own and dwelling therein except it shall be by sheer demonstrated inability to seize the opportunity which at this very moment lies open to it; an opportunity which if seized and used aright would within ten years reduce class-war to an absurdity and politics to the status of a disease.

\chapter{A Mechanical View of Economics}
\label{chapter-4}
Elsewhere an attempt has been made to show the dangerously false premises on which the New Unionist party bases all its hopes of Reconstruction. The keynote of the symphony we are to play under the conduct of Mr Lloyd George and the industrial federations behind him is production, production, yet more production; and by this simple remedy we are to change from a nation with a C3 population and many grievances into a band of busy B’s (or is it A1’s?) healthy, wealthy, happy and wise.

It is a simple little remedy—one wonders why we never thought of it before. You seize any unconsidered trifle of matter which may be lying about, preferably on your neighbour’s territory, and you make it into something else quite unspecified. You assert by a process of arithmetical legerdemain known as cost accounting that the value of the original matter which we may call \mdcode{a} is now \mdcode{a + (\allowbreak{}b + c)\allowbreak{} + (\allowbreak{}d + e)\allowbreak{}}, \mdcode{b} being labour, \mdcode{c} overhead charges, \mdcode{d} selling charges and \mdcode{e} profit, and that the “wealth” of the country is increased by this operation in respect of a sum equal to \mdcode{(\allowbreak{}b + c + d + e)\allowbreak{}}. With the aid of your banking system you now create credits which show that \mdcode{a} \emph{is} \mdcode{a + etc.\allowbreak{} -\allowbreak{} (\allowbreak{}x + y)\allowbreak{}} (where \mdcode{x} is loss in trading, etc., and \mdcode{y} is depreciation) and there you are—A1.

The chief objection to this otherwise fascinating idea is that despite a large body of most respectable and even highly paid accountants and bankers who will produce quantities of figures to prove that \mdcode{a} has now become \mdcode{(\allowbreak{}a + b, etc.\allowbreak{})\allowbreak{}} and that the wealth of the country has been increased, etc., etc., the facts do not, unfortunately, confirm their statements.

The power used in doing work on \mdcode{a} has been dissipated in heat and otherwise; the tools have been worn, the workmen have consumed food and clothes and have occupied houses, and \emph{what you have actually got is \mdcode{a} minus any ’portion of \mdcode{a} lost in conversion}; \mdcode{b}, \mdcode{c}, \mdcode{d}, \mdcode{e}, etc., are the price paid by the \emph{community} for the increased adaptability of \mdcode{a} to the needs of the community, which price must in the last event be paid for in energy. The question of the gain in adaptability depends on what you produce; but payment is inevitable.

Under the existing conditions probably no body of men has done more to crystallise the data on which we carry on the business of the world than has the accounting profession; but the utter confusion of thought which has undoubtedly arisen from the calm assumption of the book-keeper and the accountant that he and he alone was in a position to assign positive or negative values to the quantities represented by his figures is one of the outstanding curiosities of the industrial system; and the attempt to mould the activities of a great empire on such a basis is surely the final condemnation of an out-worn method.

While the effect of the concrete sum distributed as profit is overrated in the attacks made on the capitalistic system, and is far and increasingly less important than the overhead charges added to the value of the product in computing its factory cost, it is the dominant factor in the political aspect of the situation, because the equation of production is stated by the capitalist in a form which requires it to be solved in terms of selling price, while \mdcode{e}, the profit, is always a plus quantity.

Now the prime necessity of the situation, which is world-wide at this time, is to realise that in economics we are dealing with facts and not figures; and mechanical facts at that. The conversion of a bar of iron into a nut and bolt and its change in price from 2d. or 3d. to, say, Is. means absolutely nothing at all beyond the fact that we have transformed a certain amount of potential energy into work in the process of changing the bar of iron into a nut and bolt, and that an arbitrary and totally empirical measure of this potential energy in various forms is contained in the figures of cost. The factor which gives real character to the operation is the “inducement to produce.”

If the object of this use of material and energy is simply finance, we shall get a financial result of some sort—but two real things result in any case. First we have definitely decreased the energy potentially available for all other purposes, and, secondly, we have obtained simply a nut and bolt in return for a bar of iron and a definite amount of energy dissipated.

If by wealth is meant the original meaning attached to the word–“well-being”–the value in well-being to be attached to our bolt and nut depends entirely on its use for the promotion of well-being (unless we admire bolts and nuts as ornaments), and bears no relation whatever to the empirical process of giving values to \mdcode{a}, \mdcode{b} and \mdcode{c}, etc.

Let us particularise: The immediate necessity as to which all political parties are agreed is improved housing. The financier says; “Yes, you shall have money for housing as the result of building gunboats for Chile,” thereby implying that there is a chain of causation between gunboats for Chile and houses for Camberwell. Not only is there no such real chain of causation, but the building of gun-boats for Chile, or elsewhere, decreases the energy available to build those houses, and when the total available energy is utilised, as has been approximately the case during the war, and may easily be so again, not all the gunboats ever sold, no matter what the accounting figures attached to the transaction may indicate in added wealth to this country, will produce one house at Camberwell, or anywhere else. What is, of course, common to the two is the “inducement to produce,” but that may or may not be a sound inducement.

The matter is really very serious. The economic effect of charging all the waste in industry to the consumer so curtails his purchasing power that an increasing percentage of the product of industry must be exported. The effect of this on the worker is that he has to do many times the amount of work which should be necessary to keep him in the highest standard of living, as a result of an artificial inducement to produce things he does not want, which he cannot buy, and which are of no use to the attainment of his internal standard of well-being. “While the mechanism of the process is possibly too technical for his general comprehension, he has grasped the drift of the situation and shows every sign of a determination to make things interesting. On the other hand, we see a good sound reason for the capitalist’s hatred for internationalism; failing interplanetary commerce, he will have nowhere to export to, and will be faced with the horrible prospect of dividing up the world’s production amongst the individuals who live here. In which case a larger number of people than at present will agree that it is possible to overproduce gun-boats. Given this situation, what will be the result of a “strong” Coalition Government?

\chapter{Production and Prices}
\label{chapter-5}
It is admitted by almost everyone not utterly blind to the trend of public events that there is something seriously wrong in the world to-day. Hardly yet have the hospitals discharged the casualties of the first World War, yet the shadow of an even greater catastrophe is plain to those with eyes to see. Instead of the world for heroes to live in, one strike follows another to an inconclusive settlement; an apathetic public regards the conflict between “Capital” and “Labour” with a lack-lustre eye, repeating the while the \emph{clichés} of its particular brand of millionaire-owned newspaper.

Amongst the experts, various prescriptions for the disease of Society are propounded. These are:

\begin{enumerate}
	\item The Super-productionists, the “Capitalist” party, M\textasciicircum{}ho refuse to admit any fault in the social system. The keynote of their remedy is harder work and more of it.


	\item What may be called the ecclesiastical party; the keynote of their policy is “a change of heart.” Their attention is concentrated in hierarchical problems, administration, etc. The legal, military, bureaucratic mind is essentially of this type, and the Whitley Council, the Sankey Report, and the various committee schemes of the Fabian Society in this country, the Plumb scheme in America, etc., are examples of it. All these schemes are \emph{deductive} in character; they start with a theory of a different sort of society to the one we know, and assume that the problem is to change the world into that form. In consequence, all the solutions demand centralisation of administration; they involve a machinery by which individuals can be forced to do something—work, fight, etc.; the machine must be stronger than the man.



\end{enumerate}
Practically all socialist schemes, as well as Trust, Capitalist, Militarist, etc., schemes, are of this character–\emph{e.g.} the League of Nations, which is essentially ecclesiastical in origin, is probably the final instance of this.

It may be observed, however, that in the world in which things are actually done, not talked about, where bridges are built, engines are made, armies fight, we do not work that way. We do not sit down in London and say the Forth Bridge ought to be 500 yards long and 50 ft. high, and then make such a bridge and narrow down the Firth of Forth by about 75\%, and cut off the masts of every steamer 45 ft. above sea-level in order to make them pass under it. We measure the Firth, observe the ships, and make our structure fit our facts. Successful generals do not say, “The proper place to fight the battle is at X, I am not interested in what the other fellow is doing, I shall move all my troops there.”

The attempt to deal with one of the industrial and social difficulties existing at this time, which is embodied in these remarks, starts from this position therefore.

It does not attempt to suggest what people ought to want, but rather what they do want, and is arrived at not so much from any theory of political economy as from a fairly close acquaintance with what is actually happening in those spheres where production takes place and prices are fixed.

If we look at the problem of production from this point of view, the first thing we ask ourselves is, Why do we produce now? The answer to this is vital—it is to make money. Why do we want to make money? The answer is twofold. First, to get goods and services afterwards, to give expression, often perverted, to the creative instinct through power. Please note that these two are quite separate—whether a man has any recognisable creative instinct or not, he absolutely requires goods and services of some sort. We then have our problem stated; we have to inquire whether our present mechanism satisfies it, and if not, why not, and how can it be altered so that it does satisfy it.

Emphasising the fact that it is only half the problem, the only half I propose to deal with to-night, let us inquire to what extent we succeed in our primary object—that of obtaining goods and services when we produce for money under the existing economic system.

Production only takes place at present when at least two conditions are met, when the article produced meets with an effective demand—that is to say, when people with the means to pay are willing to buy, and when the price at which they are willing to buy is one at which the producers are willing to sell.

Now under the private capitalistic system the price at which the producer is willing to sell is the sum of all the expenses to which he has been put plus all the remuneration he can get called profit. \emph{The essential point to notice, however, is not the profit, hut that he cannot and will not produce unless his expenses on the average are more than covered.} These expenses may be of various descriptions, but they can all be resolved ultimately into labour charges of some sort (a fact which incidentally is responsible for the fallacy that labour, by which is meant the labour of the present population of the world, produces all wealth). Consider what this means. All past labour, represented by money charges, goes into cost and so into price. But a great part of the product of this labour—that part which represents consumption and depreciation—has become useless, and disappeared. Its money equivalent has also disappeared from the hand of the general public—a fact which is easily verifiable by comparing the wages paid in Industry with the sums deposited in the Savings Banks and elsewhere—but it still remains in price. So that if everyone had equal remuneration and equal purchasing power, and there were no other elements\textasciicircum{} the position would be one of absolute stagnation—it would be impossible to buy at any price at which it is possible to produce, and there would be no production. I may say that in spite of enormously modifying circumstances I believe that to be very much the case at present.

But there is a profound modifying factor, the factor of credit. Basing their operations fundamentally on faith—that faith which in sober truth moves mountains—the banks manufacture purchasing power by allowing overdrafts, and by other devices, to the entrepreneur class: in common phrase, the Capitalist. Now consider the position of this person. He has large purchasing power, but his personal consuming power is like that of any other human being: he requires food, clothes, lodging, etc.

If, as is increasingly the case, the personal Capitalist is replaced by a Trust, there is a somewhat larger personal consuming power, represented by the stockholders, but it is still incomparably below the purchasing power represented by credit. What happens? After exhausting the possibilities of luxuries, the organisation itself exercises the purchasing power and buys the goods and services which it itself consumes—machinery, raw material, etc. In consequence, the production which is stimulated—the production which we are asked to increase—is that which is required by the industrial machine, intermediate products or semi-manufactures, not that required by humanity. It is perfectly true that money is distributed in this process, but the ratio of this money to the price-value of human necessities—ultimate products—is constantly decreasing for the reasons shown, and the cost of living is therefore constantly rising.

Before turning to the examination of the remedy built upon this diagnosis it is necessary to emphasise a feature of our economic system which is vital to the condition in which we find ourselves–\emph{i.e.} that the wages, etc., system distributes goods and services through the same agency by which it produces goods and services—the productive system. In other words, it is quite immaterial how many commodities there are in the world, the general public cannot touch them without doing more work and producing more commodities. It is my own opinion, not lightly arrived at, that that is the condition of affairs in the world to-day—that there is little if any real shortage, but that production is hampered by prices, and the capitalists cannot drop prices without losing control. However that may be, this feature, in conjunction with those previously examined, has many far-reaching consequences—amongst others the feverish struggle for markets, which in turn has an overwhelmingly important bearing on Foreign Policy. To sum the whole matter up, the existing economic arrangements–

\begin{enumerate}
	\item Make credit the most important factor in effective demand;


	\item Base credit on the pursuit of a financial objective, and centralise it;


	\item This involves constantly expanding production;


	\item This must find an effective demand, which means export and more credit;


	\item Makes price a linear function of cost, and so limits distribution, largely to those with large credits;


	\item Therefore directs production into channels desired by those with the largest credits.



\end{enumerate}
A careful consideration of these factors will lead to the conclusion that loan-credit is the form of effective demand most suitable for stimulating semi-manufactures, plant, intermediate products, etc., and that “cash”-credit is required for ultimate products for real personal consumption. The control of production, therefore, is a problem of the control of loan-credit, while the distribution of ultimate products is a problem of the adjustment of prices to cash-credits. It is only with this latter that we are at present concerned.

We have already seen that the cash-credit provided by the whole of the money distributed by the industrial system, so far as it concerns the wage-earner, is only sufficient to provide a small surplus over the cost of the present standard of living, and that only by conditions of employment which the workers repudiate, and rightly repudiate. We cannot create a greater surplus by increasing wages, because the increase is reflected in a compound rise in prices. Keeping, for the moment, wages constant, we have to inquire what prices ought to be to ensure proper distribution.

Now the \emph{core of this problem is the fact that money which is distributed in respect of articles which do not come into the buying range of the persons to whom the money is distributed is not real money}–it is simply inflation of currency so far as those persons are concerned. The public does not buy machinery, industrial buildings, etc., for personal consumption at all. So that, as we have to distribute wages in respect of all these things, and we want to make these wages real money, we have to establish a relation between total production, represented by total wages, salaries, etc., and total ultimate consumption, so that whatever money a man receives, it is real purchasing power. This relation is the ratio which total production of all descriptions bears to \emph{total} consumption and depreciation.

The total money distributed represents total production. If prices are arranged as at present, so that this total will only buy a portion of the supply of ultimate products, then all intermediate products must be paid for in some other way. They are; they are paid for by internal and external (export) loan-credit.

If prices are arranged so that they bear the same relation to cost that consumption does to production, then every man’s money will buy him his average share of the total consumption, leaving him with a balance which represents his credit in respect of his share in—the production of intermediate products (semi-manufactures)–a share to which he is entitled, but which is now almost entirely controlled by the financier in partnership with the industrial price-fixer.

It is a little difficult to state with any accuracy what proportion of cost prices ought to be because of the distorting effect of waste, sabotage and aimless luxury.

I am making some rather tedious investigations into this, and I can only say that I am convinced that even now prices are five times too high, and that with proper direction of production this figure would be greatly exceeded.

\chapter{What is Capitalism?}
\label{chapter-6}
When two opposing forces of sufficient magnitude push transversely at either end of a plank—or a problem—it revolves: there is Revolution. When the forces are exhausted the revolution subsides, and the plank or problem remains in much the same position in space which it occupied before the forces acted on it. It is possible to conceive its molecules as being somewhat worn and giddy as a result of their rapid reorientation, but their environment is otherwise unchanged. If, however, the forces act through the centre of resistance, actual motion results; the object is shifted bodily by the greater force, without revolution.

In the first portion of this metaphor is to be found the explanation of the devastating inconclusiveness which dogs the steps of the constant and increasingly embittered controversy between the forces of what is called Capitalism and its antagonist Labour, and for a recent instance of the phenomenon it is not necessary to go further than the Coal Commission. During the earlier part of the inquiry it was made abundantly plain that an intolerable state of affairs existed in the coal industry. Mr Smillie’s attack was so well delivered, the evidence marshalled was so damning, that had the case been closed at that point the position of the miners, and with them Labour generally, would have been inconceivably strengthened. But, unfortunately in the general interest, the case was not closed there. The ground was immediately shifted to a discussion of the merits of private, as opposed to nationalised, administration.

Now I suppose it is a thankless task to say it, but the second question has about the same relation to the subject matter of the attack as has the strategy of a general to the pay of his troops. In consequence the issue now before the public is not whether the economic contract between the miners as members of the community, on the one hand, and the mining industry controlled by the colliery proprietors as producers for the community, on the other, is a bad and inequitable contract, but whether, under what is in essence the same contract, the miners’ scheme of organisation is a better scheme than the employers’. Personally I very much doubt it.

This is a matter which affects the general public quite as much as the miners themselves. It is fairly obvious that, recognising that Labour is determined to attack Capitalism, and having themselves no delusions about the real issue, the admirable brains behind the Capitalist organisation have decided, while providing just so much opposition as is necessary to register a protest, to allow an experiment on lines already discredited to be made at the expense of the consumer, in order that its stultification, which can be insured, will strengthen Finance elsewhere. Brer Rabbit, being in some danger, is betraying a special and exaggerated fear of the briar bush.

This is, of course, all very adroit: it shifts the opposing forces to the opposite ends of the plank. The question for the molecules—the general public—however, is whether they care about the resultant revolution. If not, then their concern is to bring the opposing forces into line—to see that Labour is attacking what Finance is really concerned to defend.

The general public is more likely to do this if it can be brought to realise that it is really as members of the community, not as artisans, that the attack is operating.

The whole tendency of Trade Unionist, just as much as Capitalistic, propaganda is to obscure this fact, and by so doing split the offensive, but the most superficial consideration of the root idea of the existing economic system will establish it.

\emph{“Capitalism” is not a system of administration at all; it is a system of fixing prices in relation to costs.} This is not to say, of course, that the personnel and methods of administration would not be profoundly affected and improved by a valid and radical modification of the “capitalistic”–\emph{i.e.} financial-system, but such changes would be effects and not causes.

The root problem of civilisation—not the only problem, but that which has to be disposed of before any other—is the problem of the provision of bed, board and clothes, and this affects the ordinary man in terms of effort. If he has to work hard and long hours to obtain a precarious existence, then for him civilisation fails. As the miner demonstrably had to work longer for a lower standard of life, measured in terms of purchasing power, than existed in the fourteenth century in England, then for him progress was not operative. But the reason he has to do these things is not at all that the coal mines are badly worked, although it is quite possible that they might be better worked, just as it is possible and excusable that the miners’ own efficiency is not so high as it might be under better conditions. The plain, simple English of the reason is that his wages will not buy him the things he wants. His own common-sense has consequently consistently been applied to the problem of raising his wages, but has for the most part stopped for want of technical knowledge at the recognition of the effect of this on prices.

In the December 1918 number of \emph{The English Review} it was pointed out in a short article entitled “The Delusion of Super-Production” that the sum of the wages, salaries and dividends distributed in respect of the world’s production was diminishingly able to buy that production \emph{at the prices which the capitalist is by his system forced to charge}. “Profiteering,” in the sense of charging exorbitant sums in excess of cost, is a mere excrescence on the system. If the producer could be imagined as making no profit at all, the difficulty would still exist, quite possibly in an exaggerated form. That is why the policy of more and yet more production at prices fixed on a basis of cost and profit is a mere aggravation of the prevailing difficulty. Because the available purchasing power would absorb a decreasing proportion of this production it must be either exported or wasted, and both of these lead straight to war, the supreme waster.

Now habits of thought are so powerful in their influence that at first sight a statement that the correct \emph{price} of an article may be a low percentage of its \emph{cost} is apt to induce both disbelief and ridicule. But if the matter be attacked from the other end, if it be realised that an article cannot be sold, nor can its exchange through export be sold, unless its average price is considerably less than cost; that if it cannot be sold the effort expended in making it is wasted; that if it is exported competitively every economic force is driving the community irresistibly towards war; it may then be agreed that it is worth while to consider whether the accepted principles of price making are so sacred that a world must be brought to ashes rather than that they should be analysed and revised.

The analysis has been made; and although the methods by which the results are arrived at are too technical for description in an article of this character, it may be said that the purchasing power of effort at this time should be certainly not less than five times its present return, and most probably very much more. In other words, with wages at their present level the cost of living ought to be one-fifth or less of what it is. The essential facts on which this statement is based are that production is overwhelmingly dependent on tool power and process; that tool power and process are a cultural inheritance belonging not to individuals but to the community, being largely the result of work done by persons now dead; and that in consequence the \emph{equitable} return for effort includes a dividend on this inheritance which is immeasurably larger than the direct payment. Just as the time-rate of production has diverged from that possible to a community without tools, processes or education, so to a corresponding degree has the present economic system become inequitable and unsound.

It is a matter of simple fact that men do not in the mass act together for ethical conceptions. That is why a strike can always be settled for the time on a money basis; and the only demand which will not be so disposed of is one which promises more purchasing power by its success than its opponents can in the nature of things dispose of, because such a demand will utterly divide them. But any demand which savours of the perpetuation and extension of a bureaucracy which is already highly unpopular will alienate not only the general public but the organised worker.

\chapter{The Question of Exports}
\label{chapter-7}
I have received two letters which seem to indicate some confusion of thought as to the bearing of a modified credit system on export trade. Both these letters quote statistics of wheat production and consumption with a view to throwing some doubt on our capacity to grow our own food. Now, ultimately, statistics are indispensable to sound practical politics, but to the writers of these letters, as well as to others who may be tempted to attack the problem on the basis of official statistics, it may be emphasised that it is nearly irrelevant to the primary issues whether this country can feed its population off its own acreage or not. It is quite arguable that it can; and it is also arguable that it would be bad business for it to try. These issues are:

\begin{enumerate}
	\item Are there inducements operating towards the best use of the land we have?


	\item If we export services (\emph{i.e.} the energy element of production) do we get the best real price for them?



\end{enumerate}
In regard to 1, and leaving out of the argument, for the moment, the indisputable fact that the acreage under wheat is steadily decreasing decade by decade, consider the position of the farmer. He, like everyone else at present, is in business to make money, not to deliver goods. It is quite true that he makes money by selling things, but he can easily make more money by selling less goods at a higher price than \emph{vice versa}.

Now wheat is one of a fairly small group of commodities over the price of which the individual producer has practically no control whatever. It is a graded homogeneous product bought in bulk by experts who have a strictly finite demand for it, and the price paid is under existing conditions purely fixed by financial supply and demand, whether un-fettered or artificially stimulated by rings, and is not directly based on cost. Normally, a given amount of foreign wheat is contracted for in this country—bought on “futures” by grain brokers whose price fixes a datum line for home-grown wheat. So long as wheat is in short supply as compared with the demand, the price rises, and everyone engaged in the grain trade, either as producer or dealer, may benefit, although no doubt most of the benefit goes to the dealer. The relation of the farmer to this situation must surely be plain. The one situation he must avoid at all costs is that produced by throwing grain on the market in any quantity which will bring down prices—that is to say, slacken the demand or competition to buy. His criterion of a satisfactory output, therefore, bears no relation to what amount of wheat the public requires, or what amount the land wiU produce, but rests fundamentally on, firstly, the operations of the grain brokers and, secondly, an estimate of what margin of profit can be extracted from the market by keeping it short of wheat without causing a secondary movement of grain from other markets. As transportation facilities improve, the proposition becomes less and less attractive to the farmer, who is driven more and more to the production of perishable goods, such as eggs, butter and milk, whose nature enables him to control the local market, or to the raising of stock on which the transportation charges and risks are heavy. The first prime question can therefore be answered quite confidently in the negative. In regard to the second point, let us assume that the magnitude, at any rate of our imports of foodstuffs, is a reasonable subject of discussion and policy. It is evident that there is a point at which it is debatable whether we should grow the last few million quarters of wheat required on land which may not be of the most suitable description, or whether it is sound business management to obtain this wheat by the exchange for it of manufactured goods—that is to say, by an export of economic energy. It does not take much consideration to see that the answer to this is purely quantitative: how much wheat are we to get for a given energy export?

Consider the present situation. It is true enough, as our super-industrialists and orthodox economists are always telling us, that imports are paid for by exports, but on the whole, they are content to leave it at that. They do not explain, for instance, how a population which most certainly cannot, and does not, buy its own total production for cash (if it could, there would be no necessity either for home or export credits, and no “unemployment” problem), can become able to buy the imports which are exchanged for the unpurchasable surplus. They do not, again, explain how a textile worker, paid wages for converting a bale of raw cotton worth, say, £20 into goods worth, say, £60 can benefit if in return for these manufactured goods two more bales of raw cotton at £40 are received—a condition common to trade booms. Nor do they generally publish the fact that English machinery is often sold to export agents abroad at far lower prices than those at which the same machinery can be obtained at home, or that it is possible to buy, in the bazaars of Bombay, a shirt made in Lancashire for a quarter the price at which the same shirt can be bought retail in Manchester.

The simple facts are that, under existing arrangements, our principal preoccupation is the provision of employment—the making of work. On this simple canon hangs the law and the profits. When, therefore, a locomotive is required for the Argentine, and assuming for the moment that it is in any sense sold in the open market, there is a competition, open to the industrial nations of the world, to \emph{sell} locomotives and to \emph{buy} wheat, with the usual and logical result that wheat appreciates in price in terms of locomotives, the industrial exporting country continually gives more, and the exporting agricultural country continually less, economic energy in every bargain.

That is the proposition in a nutshell. In order to make a bargain which is just–\emph{i.e.} judicious—the industrial nation must be restored to the position of a free, not a forced, seller, just as to restore social equilibrium inside the nation the individual must be put in the position of a free, not a forced, worker. The arrangements which would fulfil these desiderata are already sufficiently familiar in principle.

\chapter{Unemployment and Waste}
\label{chapter-8}
While it is necessary to bear in mind that the object of industry should not be employment, but rather the delivery of goods with a minimum expenditure of energy on their production, it is yet true that \emph{at the moment} unemployment does form a practical problem demanding alleviating treatment. The word is generally used to indicate labour unemployment, but it is practically impossible to have any considerable volume of labour unemployment without a capital unemployment representing many times the production value of the idle labour.

To the extent that private capitalism in the old sense can be said to exist, this is just as great an evil to the capitalist as to the manual worker, although its incidence may not be so personal or so immediately tragic. It penalises his initiative, depletes his reserves, and finally bankrupts him; and the whole of the process is eventually an injury distributed over the community in general, resulting in a deterioration of moral, as well as in the more material evil of a rise in prices.

It is particularly important to notice the wastefulness of the system. A demand backed by money arises in the community for a particular class of goods; an enterprising manufacturer puts down a plant “at his own expense,” as the misleading phrase goes (it is impossible for anyone to put down modern plant at the expense of other than the general consumer), and supplies the goods. This man is a public benefactor; he gives the public what it wants, and he gives it much quicker than it would be possible to get it by any other system, because one man can make a decision quicker than a dozen men, to say nothing of a Government Department. A trade slump comes; unemployment grows like a snowball, since every man thrown out of work is one man less receiving money, and therefore one man less in the market to buy goods; our manufacturer, though still willing and able to make his product, cannot sell it, and if this state of affairs continues for any length of time he is ruined. His business organisation is probably excellent, but it is broken up and bis plant dispersed, and when the trade revival comes a new plant and a new organisation has again to be constructed at the expense of the consumer.

Both the employer and the employed are so familiar with this cycle that both take steps which they imagine will protect them against its effects, but which in fact only make confusion worse confounded. During times of brisk trade the employer charges the highest price he can obtain, or, in other words, delivers the minimum of goods for the maximum of money, and embodies his large profits in invisible reserves, with the result that the consumer is left without any effective demand (demand backed by money) as soon as his wages cease. The worker, sensing this, does in his sphere precisely the same thing—he uses his trade combinations to obtain the maximum amount of money for the minimum amount of production, not realising that this money simply goes into the cost of the product, which has to be paid by the community of which he forms so large a part. Since, superficially, it seems vital to the interest of both of them to keep the process moving as long as possible, the manufacturer is driven to sell, by advertisement or otherwise, useless or inferior and quickly worn-out articles where he cannot make a handsome profit on durable and well-finished production, the life and usefulness of which operate in the truest sense towards labour-saving.

Consider, then, the position at the present time. It is certain that both employers and employed are willing and able to work \emph{on terms}; it is demonstrable without difficulty that the productive capacity of industry, with its labour, plant and organisation, greatly exceeds the consuming capacity of the nation, unless that consuming capacity is enormously and viciously inflated by waste, and especially the culminating waste of war; and yet it is patent that the needs of the individuals who comprise the community (whose collective needs are the only reason and justification for the existence of industry at all) are far, and even increasingly far, from being met. There is one possible explanation for this anomaly—the financial system, which ought to be an effective distributive mechanism for the whole possible production of society, is defective—it does not so arrange the prices of articles produced as to enable the extant purchasing power to acquire them.

Now without, for the moment, discussing the methods by which this defect can be remedied, let us imagine the remedy to be applied and consider its \emph{immediate} effect on the unemployment problem. There are still millions of persons wanting goods; the productive system can make these goods; the persons who want them can buy them, and those who make them can be paid for them.

It seems obvious that an enormous stimulation to production would be provided—a stimulation which no mere propaganda on its desirability has ever succeeded in evoking; and that the immediate effect of this would be a radical diminution of unemployment.

Consider now the policy actually being pursued at this moment by the Government and the financial powers to deal with the problem. They can be summarised in one sentence—the reduction of costs, and more especially labour costs. But labour costs are wages and form by far the most important item in the total purchasing power \emph{inside the country} available for the distribution of goods. Even supposing that retail prices were reduced in exact ratio to wage reductions, which is highly improbable or even impossible, how is the distribution of goods \emph{to people in this country}, which is the true object of British industry, thereby advantaged? As the prices fall by this method, so the amount of money to purchase also falls, and we are as badly off as before, with the added complication of the discontent evoked by the reduction of wages.

It would seem, then, that although a reduction of prices \emph{in relation to purchasing power} is not only vital in connection with the more fundamental problems of industry and society, but is the only effective method of dealing with the immediate problem of unemployment, we are not as a nation pursuing this policy, but rather one which, if not diametrically opposed to it, is yet wholly inapplicable to the situation. Is it impossible to obtain adequate recognition of fundamental remedies, and equally impossible to rouse the general public to a sense of the catastrophe towards which its passivity in the matter is hurrying it so swiftly?

\chapter{A Commentary on World Politics (I)}
\label{chapter-9}
Mr Balfour, in supporting the project of the League of Nations, stated with great impressiveness, and to an enthusiastic audience, that the League must come; there is no alternative. Now Mr Balfour is a statesman; a little \emph{passé} perhaps, but still a statesman as distinct from a politician. It is highly probable that we differ from him in nearly every fundamental conception of what society ought to be and could be, and in the means that can profitably be employed to induce such changes as are necessary. But we have no doubt whatever that Mr Balfour has a personal code from which he will not depart, and that included in that code is a refusal to state clearly and definitely as a fact that which he knows or even suspects to be false. We emphasise this point because it is necessary to a grasp of the difficulties and dangers with which this country in particular and the world in general is beset at this time. Mr Balfour, then, a representative of the best type of the old-fashioned statesman, puts forward a plea in support of a project involving the most tremendous consequences, and separating Great Britain from every fundamental canon of procedure not only of the past, but of the “platform” on which the war was fought (if we except empty phrases), and this course is recommended to his hearers, not by any reasoned or inductive process of argument or demonstration, but by the council of despair that, lacking any idea of the right thing to do, we must do this. It would be incredible, if it were not so clear that every statesman of every country in the world has either succumbed to panic, or retreated behind a barrier of phrases without concrete meaning or application to the course of events. Stripped of its verbiage and the mass of pious sentiment with which it is surrounded, what \emph{is} the League of Nations, as projected on the basis of existing social, political and economic systems? Its major premise is the avoidance of war, by the settlement of disputes at a centralised headquarters, backed ultimately by the logic of a position which centralises the final argument of force under an elected committee, operating by means of permanent officials. The first permanent officials have been appointed, and may broadly be said to represent the Ultramontane, or Temporal Power, section of Roman Catholic politics (said to be the only barrier between Europe and “anarchy”), by accommodation and in accordance with at least one section of High Finance. Consider this proposition, stripped of its sentiment, in the light of actual knowledge and observation of the working of such an organisation (quite apart from any question of personnel at all). The Post Office, for instance, is such an organisation. It is in theory a Department ruled over by a Political Minister responsible to an elected body, the House of Commons. Does anyone in their senses imagine that the Postmaster-General could carry any point of internal policy in the Post Office against the settled procedure of the Permanent Officials? Or that any attack by an individual from \emph{inside} the Post Office on a \emph{system} (as distinct from a person) which may press hardly on him has any chance of success? But, it may be argued, we are going to change all that. We are going to have democratically elected committees to deal with all such questions. Very well, let us consider the actual working of such a committee. A grievance comes before it and a decision is given which may quite reasonably not give satisfaction, and the committee is attacked for it. It is an honest decision honestly given, and the committee combines to resist the attack. Immediately a position is created in which the committee represents a vested interest, and acts not as a body of elected representatives, but as an Institution whose power must be consolidated, and whose dignity must be upheld. Anyone with practical knowledge of committees knows that this is what happens. It may be said that all this is simply an argument for anarchy (and it is \emph{the} argument for anarchy), but that is a mistaken view, as we hope to show.

Having got it firmly fixed in our minds that no conceivable change of heart has any bearing on the results of the arrangement we are discussing (we should imagine that from top to bottom, for instance, the Post Office is staffed with average kindly human beings), it is clearly vital to get some idea of where the difficulty does lie, since no difficulty is finally insuperable; and we have no hesitation in saying that the difficulty lies in the common confusion between organisation and administration. Organisation is a pure, if at present empirical, science; its relation to administration is the relation between the Theory of Structures and the Strength of Materials. No personality enters into sound principles of organisation at all; administration, on the other hand, which is an art, is wholly concerned with the satisfactory adjustment of individuality to organisation. The distinction is vital. Let us apply it to the projected League of Nations, which is first of all an organisation of some sort, and an organisation presupposes some objective. We have seen that \emph{the very core of the League of Nations’ idea is power, final and absolute; it is, therefore, an organisation expressly designed to eliminate administration by suppressing individuality}; to make the Machine finally supreme over the Man. And the alternative? Let us return to our \emph{corpus vile}, the Post Office.

Imagine the Post Office to be organised exactly as it is organised (though it is highly probable that its organisation could be improved). Its \emph{administration} admittedly is bad, for reasons, in our opinion, fundamentally unconnected with personnel. Leaving, we say, everything else exactly as it is for the moment, let us suppose a Regulation to be added to the few thousands which are now the chief exercise for the ingenuity of its staff, to the effect that all Post Office servants are at liberty to retire at will on a pension equal to their full salary. We admit that the traffic in St Martin’s-le-Grand would be congested for some time, but supposing this initial period to have been surmounted by a reasonably well thought-out transition policy, we have no doubt whatever that a staff would be found at work, having realised that creative activity is a luxury, if not a necessity, of existence. Our hypothetical arbitration committees, however, are now confronted with a new situation—they have to find a solution of problems submitted to them \emph{which will keep the complainant at work by preference, if it is to the advantage of the Post Office Service that he should he kept at work}. If he is a pure crank, the Post Office will be better without him; but if his ideas are sound—i.e. \emph{in the general interest of all concerned}–he will be in a position both to defy economic pressure and to apply collective interest to the solution of his difficulties. The illustration is crude, but it may serve. The conclusion of the matter is that association for the attainment of an objective inevitably becomes a tyranny (\emph{i.e.} an attack on individual initiative) unless it can be broken at any time, without incurring any penalty other than the loss of association itself.

Before, therefore, the League of Nations can be constituted as that League in the interest of Free Persons, which it pretends to be, but is not, we have to place the Machine at the disposal of the Man. Given that essential, we can design or alter the Machine with the single object that it shall be the best Machine with which to attain a result having the full approval of the individuals without whose co-operation it cannot work, and this will involve not one organisation but many organisations in which the relation of the individual to the organisation is not dissimilar to that of a man who is a director of several and possibly widely differing companies. If he does not approve of them he resigns, and if a sufficient number of persons resign and are not replaced, then the activities of that concern are clearly not desirable, and it goes out of existence. Now the project at present known as the League of Nations can be seen to be the converse of all this; if the individual or the nation does not approve of the objective of the League (which rests on a purely abstract and improbable assumption that its personnel not only represents the highest wisdom but an unearthly disinterestedness), then that individual or that nation is eliminated, so that in theory no effective will remains save that which reaches its highest expression in the apex of the perfect Pyramid of Power, which is its object. We repeat, therefore, that in this project is the greatest and probably the final attempt to enslave the world, an attempt which is exactly similar, and probably proceeds from exactly the same International source, as the attempt so recently failed, in which the German people were tools, blameworthy just to the extent that they allowed themselves to become tools; and we believe that while it must finally fail, the measure of the misery in which its trial would plunge the world is such as to dwarf the horrors of the years so recently endured. We do not, therefore, agree with Mr Balfour, either that the League of Nations must come, or that there is no alternative to it, and we trust that the community in whose hands may lie the power will not be so blinded by the fine words in which its description is enveloped as to miss the meaning of the thing which is behind them.

The most important report, issued by the United States Council of National Defence, entitled \emph{An Analysis of the High Cost of Living Problem}, is a document (we are sorry to say), as might be expected, incomparably in advance of any similar official pronouncement which has appeared in this country. After pointing out that the problem is so interrelated with others that its consideration opens up the entire field of reconstruction, it goes on to remark that it is neither a new problem nor (under existing circumstances) transitory in character. Proceeding, it explains, in an excellently concise manner, the form of currency inflation which is produced by the lavish distribution of money unrepresented by ultimate products in \emph{personal} demand (which is exactly the situation our super-producers are striving to foster, whether by ignorance or otherwise is immaterial), and remarks “with dismay on the general flood of misinformation, half complete information and undiluted ignorance which... pervades the land regarding our current economic situation.” We agree entirely with all this, and while the conclusions which the report draws as to the steps to be taken to deal with the situation are not so impressive (quite possibly for reasons over which the individuals who framed the report had little control), there is none of the glib claptrap about them which we are doomed to suffer in similar circumstances in this country. Compare all this with the solemn pronouncements of our only Mr G. H. Roberts. After admitting that food is, on the whole, abundant, and that its alleged scarcity has little to do with high prices (a piece of information he might have derived from us about a year ago), he admits sadly that “it is undoubtedly true that at the present time the increase in supply has not brought about the decline that was expected by many... in fact prices, so far from declining, have remained high, or shown some tendency to increase.”

After numbing his hearers with a mass of statistics to prove that we ought to be thankful that we are not worse off—most of which, when exchange is considered, prove exactly that we \emph{are} worse off than our neighbours—he goes on to make the original suggestion that the cure is more production for export. Let us paraphrase Mr Roberts, and explain him to himself, as he clearly requires explaining. He admits that there is a sufficiency of goods. He even allows it to be gathered that supply is being restricted by artificial means. He knows that the world is complaining of high prices. He knows, if he will keep quite quiet, and think for a few minutes, that prices ultimately represent work, man-hours of labour. As a remedy for a complaint that prices representing work are too high, although he is being asked to distribute goods which already exist in sufficient quantity, he recommends more work, much more work, to be applied to the making of unspecified articles, in order to export the result out of the community which has performed that work. Mr Roberts concludes by assuring the Conference that they may be confident that the Ministry of Food is doing everything in its power to keep down prices, and that the power of the Government is strictly limited in this respect.

We have no doubt Mr Roberts is entirely honest in making these latter statements, and, moreover, that, as distinct from his earlier remarks, he is entirely correct. Both the Ministry of Food and the ostensible Government, as a whole, are mere tools in the hands of the \emph{real} Governments, and Mr Roberts has probably found out by now, if he did not suspect it when he accepted office, that he is paid to do as he is told. If he really knew anything about the cause of high prices, and were determined to use his knowledge for the benefit of the country, he would not remain in office for ten days. But consider what he says. The Government–\emph{i.e.} the Ministers of the Crown—represent in theory the collective interest of the nation. They are always saying so, so it must be true. Is there any collective interest of the nation which is more immediate and more vital than that of food prices? If the Government has no power over prices–\emph{i.e.} if knowing that there is a sufficiency of the articles required, in existence, they cannot get those articles distributed without making an immense quantity of goods for other countries which are not asking for them, and whose population, in any event, these Ministers do not represent, if, in other words, they cannot affect or modify the most elementary functions of society—then who can modify them? And, if the real rulers of society are not in the Government, but behind the Government, who elected them, what interest do they represent, and what is the good of, say, Mr Roberts? We feel sure that Mr Roberts is convinced that it would be much better not to inquire too deeply into these matters, but at the same time he must recognise that he is certain to be asked about them sooner or later. We suggest, therefore, that the sooner the Hidden Governments of the world are brought out into the open, and a decision is obtained on points which really matter, the sooner we shall know what sort of a New-World-for-heroes we are likely to get. At the moment it requires heroism for any but Cabinet Ministers to live in it.

\chapter{A Commentary on World Politics (II)}
\label{chapter-10}
Readers of these pages who are also readers of \emph{The Daily Telegraph} will not have failed to notice the columns in the issue of that estimable journal signed by Sir Oswald Stoll. He asks in parenthesis, “Who can deny that we are on the verge of a great financial and economic crisis?” and goes on to say, “Hundreds of enterprises are held up by costs too high to admit of sane capitalisation! Thousands of enterprises necessary to keep the economic wheel revolving are on the brink of failure \emph{because they cannot buy cheaply or sell dearly enough}." (Our italics.) After observing that " Financiers’ finance, with its checkmates by rival groups, is ruining the country," and that “the aim of National Finance should be some prosperity for all Nationals, not all prosperity for some Internationals,” he points out the solution “...a true conception of National Finance and National Credit.” This is all very gratifying to our prescience, if not to our humanitarianism. We have been saying much the same sort of thing publicly for four and a half years, and we think it highly probable that the “sane Labour leaders” who preside at Eccleston Square and elsewhere will hear from the brainy fellows who comprise their official and unofficial general staff that Sir Oswald Stoll is in league with us, or \emph{vice versa}. But they will, if only on this occasion, be wrong—not only has that happy consummation still to be reached, but, in the meantime, while agreeing absolutely with all the quotations cited above, and many others which space forbids us to include, we disagree totally with the conclusion drawn from them—that “Production on the great scale will save us.” Now, some months ago, there appeared in the pages of \emph{Credit Power and Democracy} this statement: “In spite of the apparent lack of enthusiasm with which any attempt to examine the subject of credit and price control is apt to be received in the immediate present, there is no doubt whatever that its paramount importance will within a very short time be recognised, although perhaps not so quickly by British Labour as elsewhere. \emph{The real struggle is going to take place not as to the necessity of these controls, hut as to whether they shall he in the hands of the producer or the consumer.}" That is just exactly the point at which we join issue with Sir Oswald StoU, and the super-Productionists. The practical implication of their policy is a continuous rise in the level of prices of necessaries; we look to a continuous fall in such prices.

We believe it is no longer necessary to labour the point that whoever controls credit controls economic policy; and it follows as a simple syllogism that just to the extent that control of food, clothes and housing is control of society, so producer-control of credit means the enslavement of society to Industrialism; whereas the whole world now rocks to its social base in an effort to subdue the dragon of the industrial machine in order that men may be free. Any housewife who ordered from her tradesmen “as much as you can send me of everything” would be deserving of, and would receive, reprobation, even in a time of scarcity; but where the real capacity for supply is far in excess of any real demand, such an individual would be in danger of certification as insane. The public is the housewife, and its business is to order the right quantities of the right things in the right order, and to see that it gets them; not simply “More.” The productive system is easily capable of giving the public what it wants, if only producers can be salved from the unlimited task of giving the public what it doesn't want–\emph{e.g.} “employment.” The existing financial system exists by seeing that the public never gets quite enough of any one thing it wants; by constantly diverting the productive organisation before it has time to finish any one task; or else “sabotaging” the output; and while we require for this reason at the moment “more” of the fundamental necessaries of life, we do not require an indefinite amount even of these. As has so often been emphasised in the foregoing pages,; the whole problem fundamentally resolves itself into providing an organisation to get first: things first, with the minimum of trouble to everyone.

One of the vital means to that end is to throw overboard the superstition that “employment” is the inevitable condition antecedent to “pay,” and for this reason we welcome the support given by the numerous local Trades and Labour Councils to the resolution put forward by the Minimum Income League on the agenda of the Annual Labour Party Conference recently held at Scarborough. To students of the psychology alike of industrial and of world movements (which is, in essence, identical) it requires an effort to avoid cynicism at the similarity in the real aims of orthodox Socialism and ultra-Capitalism. The idolater of the State says: “I will make it impossible for you to live except you conform to my standard of conduct.” Lord Leverhulme, amongst others, says very little, but, being more capable, obtains world control of essential products, and lays down a policy both for his employees and those who must have his goods. Bismarck understood the situation perfectly when, in speaking of the German Socialist Party, he observed: “We march separately, but we conquer together.” The will-to-govern is identical in each case. Against this essentially insolent tyranny, the \emph{idea} underlying, \emph{inter alia}, the Minimum Income proposal, is the only defence, and we therefore congratulate its authors on the excellence of their achievement in planting it in somewhat difficult soil. But having said so much, we are bound to point out the ineffectiveness of the suggested \emph{mechanism}, which is based on the error, made in company with others such as Professor Bowley, who, we think, ought to know better, that the national income equals the sum of the price-values of the national production.

\emph{This would he true if all wages, salaries and dividends charged to production were used, at the instant they were earned, to buy the production in respect of which they are earned.} But they are not so used, and on this gap between production and delivery, which the complexity of modern co-operative production is widening, a mass of credit purchasing power is erected which never appears as income at all, and which is completely ignored by such proposals as that which we are considering. If A ordered a house off B, and B, having built it, lived in it for ten years and then insisted on charging his rent to A in a lump-sum addition to the price, A would probably complain; but when B put his overhead charges, the rent of his control of production, into the price of bricks for A’s garage, A seems to regard it as an act of God, or, alternatively, of the King’s enemies. Possibly he is right in both cases, but that does not alter the fact that A is being asked to pay, in prices, for something—viz. a period of use-value, past, and therefore destroyed and non-existent—of which the \emph{effective} purchasing power never was distributed either as wages, salaries, or dividends–\emph{i.e.} income—therefore income will not buy it. What may remain is the credit-value of this period of use, its assistance to future production, which may form a solid basis for a distribution of purchasing power possibly much in excess of the use-value charged in prices; but A gets none of this.

We admit the elusiveness of the argument; it is one of those conceptions which, like the differential co-efficient in mathematics, to which it has a strong family resemblance, comes suddenly rather than by intellectual explanation. But it is, without any doubt whatever, of the essence of the contract, and failing provision to deal with it, we are bound to agree with the dictum of an opponent of the scheme in the correspondence columns of \emph{The Times} who characterised the proposal (to pool 20\%, of every income, dividing the pool equally over the whole population) as being the heaviest direct tax on the poor ever invented. The Minimum Income League has a great cause to fight for, and we are confident that its progenitors, if they will concentrate on the problem, can so modify their proposals as to still further assist in gaining a great victory; and in any event we wish them luck. The fundamental point at issue will be still further brought into prominence by the next move in the strategy of the hard-shell Capitalists, which will be to concede unemployment maintenance to wage-earners in consideration of the removal of all restrictions upon output and the acceptance of payment by results—an arrangement which really means the formation of comprehensive low salary lists, plus a percentage commission on output. We are not so much concerned to point out that this arrangement makes men “slaves” to an imposed industrial policy, because a large number of human beings are slaves already and quite a number of them like it; but it is quite certainly a most ingenious device to keep them slaves, whether they like it or not, and we are sorry to see that the Building Guilds, which have been started in London as well as in Manchester, do not seem to grasp that fact, in their rather naive satisfaction at having incorporated the same principle in their constitution. The important point is not whether John Pushemup, of Messrs Cubitts, for example, builds houses or Mr James Articraft, of the London Building Guild, builds them—without knowing anything of the executive capacity of either, we do not know which is the right man for the job. But we do know that it makes very little difference to the result, after an initial short period, which organisation makes the rules, if either of them is in a position to lay down conditions to the public as to the use of the houses after they are built. That is exactly what this maintenance pay idea amounts to—that we shall all be nicely fed, watered, groomed and stabled if we will leave policy to the productive organisations; and the pity of it all is that it won’t work.

Some years ago one of the largest State-owned industrial organisations in this country imported from a commercial firm the idea of the suggestion box, into which any employee of any grade from the highest to the lowest was invited to place any proposal either for the smoother and more efficient running of the organisation or for improvement in the processes of manufacture. An elected Committee was set up to deal with the matter, and a fund, for which a Government grant was obtained, provided a source from which rewards, varying from a few shillings to several hundred pounds, could be paid. On the whole, the scheme was a failure. During the first year of its life a flood of suggestions, good, bad and indifferent, from the Selection Committee’s point of view, were submitted, many of them were paid for and some of them were acted upon. The second year showed a great falling off both in number and quality, and in subsequent years a mere trickle of, in general, impracticable proposals, usually emanating from new-comers, was the only output, and what was probably worse, the general run of workmen in the undertaking openly derided the plan as a scheme to “suck their brains.” (This in a “Nationalised” undertaking!) As a consequence of considerable familiarity with this and similar devices, we have no hesitation whatever in saying that the main cause of failure was not inadequacy of reward, or even dissatisfaction with the decisions of the Selection Committee, although both of these were alleged; but was rather a subconscious irritation at the complete impotence of the authors of the suggestions to superintend the process of giving their ideas a run. Now, each of these suggestions, where they were original, betrayed nascent initiative, and it is out of personal initiative that all progress of any description must come. In the case we have just instanced, it was possible to watch the strangling of initiative taking place; and the explanation was also obvious—-that the great mass of individuals will not risk economic disaster—the loss of their job—for the sake of an idea. But it is highly probable that many most valuable additions to the knowledge of industrial organisation and processes were thereby lost to the community and are daily so being lost; and only the grant of economic independence and the consequent freeing of personal initiative will stop this immensely important channel of social waste.

That estimable journal \emph{The Spectator} recently started a sort of symposium on the subject of “the Jewish Peril,” both the book which has recently been published under that name and the hypothetical thing itself. Most people are no doubt familiar with the general legend, if legend it be; it was the core of the Dreyfus case, which convulsed France some years ago, and is constantly reappearing in the guise of the Hidden Hand stories of various descriptions which crop up at any time of national crisis.

It presupposes the existence of great secret organisations bent on the acquisition of world-empire and the overthrow of their “enemies,” and directed by immensely wise men with all the power which an almost superhuman knowledge of psychology can bestow. Such an organisation would be capable of using Governments as its tools and the lives of men as the raw material for the fashioning of its projects. Like \emph{The Spectator}, we have no means of knowing how much of this idea is pure moonshine, or even whether the whole matter is a malignant stimulus to anti-Semitism; but, with that journal, we can understand that it \emph{might} have some foundation in fact, and that, as it puts the matter, we have a good many more Jews in important positions in this country than we deserve. And not only in this country, but in every country, certain ideas which are the gravest possible menace to humanity—ideas which can be traced through the propaganda of Collectivism to the idea of the Supreme, impersonal State, to which every individual must bow—seem to derive a good deal of their most active, intelligent support from Jewish sources, while at the same time a grim struggle is proceeding in the great international financial groups, many of which are purely Jewish, for the acquisition of key positions from which to control the World-State when formed. We are anxious not to be misunderstood. We do not believe for a single instant that the average British Jew would countenance such schemes for a single moment, but in view of the curiously circumstantial evidence which is put forward to support such theories, and the immense importance of the issues involved, we agree that it is very much better that as much daylight should be allowed to play on the matter as may be necessary to clear it up. The alternative will be an outbreak of popular fury in which the innocent will suffer with the guilty, if there be any such.

It is always difficult to know how much weight to attach to Press expressions of public opinion in the United States, and that difficulty is greatly enhanced at this time both by the immanence of the Presidential elections, and the selective censorship which our own Press exercises in its quotations. But there seems no reason to doubt the general truth of the impression which is conveyed both by them and by a perusal of the American political reviews, that anti-British feeling is steadily gaining ground, not only, and not even so much, in the eastern cities such as New York, Boston and Philadelphia, but in the Middle West and on the Pacific coast. Because of the constant flow of passenger traffic between Europe and the Eastern States, and the consequent tendency of European newspapers to quote American journals with which they are familiar, there is an impression prevalent that the centre of gravity of American action is resident along the Atlantic seaboard. Such an idea is probably far more mistaken than to imagine, for instance, that London opinion is British opinion. All through the Middle West, including such considerable cities as Chicago and Milwaukee, there is an actual numerical preponderance of people of definitely anti-British extraction—Milwaukee, in particular, is overwhelmingly German, while Chicago is politically in the hands of Irish emigrants largely of a generation having much greater and more solid grounds for hatred of British Governments than any which exist to-day. This population has on the whole not done well out of the war; it is hit by high prices, and irritated by all sorts of hindrances to peaceful progress, ranging from Labour troubles to a moribund railway system and a Negro problem. Such a soil is the perfect matrix of an international hatred, and the seed of such a hatred, already dormant, is being cultivated with a skill and assiduity which should command our attention, if not our admiration. AH sorts of misrepresentation both of fact and of policy, particularly in respect of Ireland, Egypt and India, are circulated with an utter disregard either of essential truth or contingent circumstances. On the Pacific Coast, where Japanese expansion is an obsession, our alliance with that country is a special reason for dislike, and is exploited to the utmost. This is not the place to examine at length the motives behind the persistent efforts to embitter the relations between Great Britain and the North American Republic—we have referred to some of them in previous issues—but to anyone who realises, as we do, the appalling horrors to which their success must lead, the situation is one to excite the gravest concern.

\chapter{A Commentary on World Politics (III)}
\label{chapter-11}
Sir Oswald Stoll returns to his attack on the system of credit-control by financial groups, and although, as we said in a previous comment, we are quite assured that the proposals he adumbrates in his campaign are \emph{by themselves} worse than useless as a cure for the present situation, we welcome the attention he cannot fail to attract to the problem as a whole. His text, in this case, is a quotation from a speech by the Chancellor of the Exchequer, in which the financiers are openly implored, as the controllers of the situation, to conserve “Capital” by which, from the context, it is obvious that the Chancellor means credit. It is a remarkable speech, and Sir Oswald is probably correct in stating that never before did a Chancellor of the Exchequer acknowledge in set terms the absolute control of the Government and the country by the financial community. Just think what it means. Two or three great groups of banks and issuing houses controlled by men, in many cases alien, and even anti-British alien, by birth and tradition, international in their interests and quite definitely anti-public in their policy, not elected and not subject to dismissal, able to set at naught the plans of governments; producing nothing, and yet controlling all production. We do not believe that there is a single considerable commercial organisation in this country or America, however apparently prosperous, which could live for two years against the active hostility of half-a-dozen of these men. To such. Ministers of the Crown are servants appointed to take orders, and dismissed if they are negligent in the execution of them; wars, famine and desolation are simply mechanisms by the aid of which their control is conserved. Consider the railway systems of America: twenty years ago giving promise of forming a model transportation system; to-day, looted, sacked and exhausted by one financial raid after another, they are almost \emph{in extremis}, and only maintain a service which is a mockery to technical capacity, by means of a grant from the public purse of a sum substantially equal to the original cost of their construction. Every one of the groups which were directly responsible for this result is represented in the city of London, and is included in the Chancellor’s reference, just as they are represented in Paris and Berlin, and probably Moscow. Meanwhile, Dean Inge deplores the failure of democracy, and the Labour Party agrees that what we really want is more production, and the building trade gets on with building more—factories.

We have always held that in America, where irresponsible financial control has been most blatant, and the results of it, taking natural resources into consideration, more obviously disastrous than even in Europe, there would come the first clear-cut and dangerous challenge to the system; and the modest little announcement which has crept into the London Press that representatives of a new party opposing the Democrats and Republicans alike, and not in sympathy with the Socialists, will meet at Chicago to nominate a Presidential candidate, is, if we mistake not, a justification of this opinion. The power behind this new movement is a composite one, involving the Non-Partisan League (which is definitely in possession of the machinery of government in North Dakota; is said to control Minnesota, and is steadily gaining ground in several other of the States of the Middle West), and a number of the Labour unions, including the Railroad Brotherhoods, who have revolted from the leadership of the egregious Mr Samuel Gompers, the latter individual undoubtedly one of the most valuable assets the trusts possess. “With these bodies are associated the Co-operative movement, the Consumers’ League, and several quasi-religious organisations for social service, the whole making up a body of opinion which, if time permits, is definitely powerful enough to carry the policy it represents—essentially that of the public control of credit and price-making—against any other single party of the Republic.

Mr Gompers, who is no mean politician, and is fully alive to the fact that his popularity is waning, has himself raised the credit issue with a demand for producer-control, coupling his platform with a political strategy which, consists in urging his followers to support any candidate, either Democrat or Republican, who will pledge himself to assist in obtaining it, thus forming, in intention, a coalition against the new party of Economic Democracy. How far American trade unionists will thus allow themselves to be spoofed only time can tell, but the result of this alignment should be a matter of most absorbing interest, not only in the United States, but in this country, for it is not too much to say that the peace of the world and the future of civilisation may be involved in the outcome of the struggle.

A deputation from the League to Abolish War, consisting, amongst others, of Mr G. N. Barnes, M.P., and Mr Frank Hodges, of the Miners’ Federation, waited on the Prime Minister in June 1920 for the purpose of urging upon him the necessity of an International Police Force to do the bidding of the League of Nations. It is, at first sight, a little difficult to understand the mentality of Labour representatives who, while professing to be profoundly dissatisfied with the existing state of society, and constantly concerned to accuse the Governments of all countries outside, perhaps, Russia, of acting in the interests of the Capitalist system which they condemn (an accusation which is probably justified), and of using the police and other armed forces for the purpose of buttressing their power, yet propose to set up a mechanism expressly designed to make revolt against such Government impossible.

We are not, for the moment, criticising the proposal itself; we are merely considering the support of it by official representatives of a party openly pledged to revolutionise society. Now, assuming, as we are quite willing to assume, that both Mr Barnes and Mr Hodges are perfectly honest both in their desire for a better state of society and for the abolition of war, and that not being merely irresponsible lunatics, they have some reasons for figuring on such a deputation, it becomes of interest to see how they reconcile the fact that revolutionary Labour is notoriously unable to capture existing police forces, with a desire to build up a police force which is \emph{ex hypothesi} incorruptible. We believe the reason to be twofold. In the first place, Mr Barnes and Mr Hodges no doubt believe that they represent a Labour Party which is coming into political power all the world over, and that therefore they will control this police force; and secondly, they confuse a remedial, if appalling, symptom, war, for the disease which causes it, much as one might ignorantly say that spots on the skin constitute measles.

Both of these reasons are demonstrably unsound; we will endeavour to show why. A modern Trade Union represents, not a body of individuals, but a monopoly more or less complete of an essential factor in production–\emph{i.e.} Labour—differing in no respect in principle from a trust monopoly of, say, sugar, and Mr Barnes and Mr Hodges, in so far as they represent or have represented trade unions, represent this functional monopoly just as truly as the chairman or secretary of the sugar trust represents a sugar monopoly, and the object of both is identical—viz. to exploit the public, the consumer. There is not anyone else to exploit; the employer is utterly incapable of carrying a 25\%, rise in wages for a month if lie does not recover it from the public, and, conversely, will joyously grant any percentage of wage increase if he is assured that prices will recoup him. The party, if it can be so called, which is undoubtedly coming into power in the next few years all over the world, therefore, is not the “Labour” Party which Mr Barnes and Mr Hodges represent, but a Public Party which will replace exploitation by co-operation and in consequence will deal just as faithfully with the abuse of a “Labour” monopoly as with any other trust, and which will represent the men and women who now form the constituents of the Labour Party, not as monopolists of a commodity, but as human beings anxious to gain their legitimate ends by the most convenient and comfortable methods. In an economic system constituted as is ours to-day, we do not in the very least blame any monopolists, organised Labour included, for exploiting their advantage to the utmost: that is the way the game is played, and that is the sure and certain method which will break up society as we know it, though it contains no promise of constructing a system to replace the ruins. But nothing is gained by idealising the process; and the Labour Party and its officials are just as much a part of the capitalistic system as, say, the Federation of British Industries, and \emph{qua} their representation of a monopoly, just as pernicious.

The second misconception, while one may have every sympathy with it, is none the less fatal to any effective remedial action. War, appalling orgy of waste and misery as every sane person must admit it to be, is not the greatest of all evils, although it may quite conceivably be a great enough evil to destroy this civilisation. A greater evil would be the unchecked operation in a helpless world of those causes of which war is an effect. That is exactly where what is commonly called pacifism makes its cardinal error—it is so concerned with the “rash” on the patient that it will go to any length to suppress it. Not that way lies a cure. The disease lies much deeper than the skin—is concerned with the vital processes of the body politic; and to avoid substituting a more lingering horror for the sharp fever of war, it is necessary to restore these vital processes to health and balanced functioning. Now although we have insisted that the financial organism is the region demanding the most instant attention, it is not the whole of the problem, although the successful reconstruction of it would probably render easy the solution of the remainder. That lust for domination which may perhaps be said to lie at the root of the major evils of anti-public finance also operates, by the substitution of the motive of fear for the motive of gain, through the mechanism of bureaucracy. The business man assists in an unsound policy through the lure of reward and through cupidity—the bureaucrat winks at intrigue because of a fear, born of experience, that his knuckles will be severely rapped if he doesn't. Merely to substitute cupidity (which, after all, as its name suggests, is only perverted \emph{affection}) by fear would be a sorry exchange. The problem is essentially a practical one, and we have no doubt whatever that the real inceptors of the deputation to which we refer (and who we feel positive do not comprise either Mr Barnes or Mr Frank Hodges) are actuated by motives which, whatever else they may not be, are almost luridly practical.

The proposition which has been aired during 1920 to increase railway rates another 40 per cent., making 90 per cent, permanent increase over pre-war rates, raises, in an interesting form, the question of the applicability to this particular case of what is known as the law of diminishing return. It is probably quite familiar to readers of these pages—it postulates that there is a certain maximum “load” which any mechanism, economic or otherwise, will carry: below and above this load the output drops away, finally reaching minima. Now it has long been an axiom with ) railway managers that it was impossible to base railway rates on cost; the only principle on which a railway system as a whole could be made to pay was to charge each separate class of traffic “what it would bear”–\emph{i.e.} the most, it would pay without revolt. The rich industry ' thus subsidised the less prosperous and the railway averaged their prosperity. This system had reached perfection before the war, and it is quite probable that in this country 5 per cent, increase in any one rate would have raised a storm. It is now proposed that rates shall rise not 5 per cent, but 90\%, and that at a time when there are not wanting interested persons (with whom we totally disagree) who would contend that the day of the railway is done and that road transport and aviation will carry the traffic of the immediate future. We say we disagree with such persons; but we do not mean by that to suggest that it is not possible by means of crazy finance to ruin a magnificent asset, in order that a few international credit-mongers may acquire control of national transport. The bearing on this of the law to which we refer will be plain: if the rates on British railways before the war were as high as could be borne, and they were, then any further rise means a decreasing return and the speedy bankruptcy of the whole railway system due to the use of alternative, though not fundamentally better, means of transport.

Just as it is quite erroneously said that threatened men live long, so there is a tendency as perverse as it is general to assume that it is only necessary to predict disaster of any description to form thereby a solid basis for optimism. For twenty years hundreds of men and women in this country, and thousands on the European continent, knew that, given the continuance of certain economic and political factors, war with Germany was just as inevitable as a chemical reaction. Certain social factors combined will produce certain social results, just as certainly as the combination by the aid of a spark of a mixture of oxygen and hydrogen will result in water. In 1900 the writer was told by an official of the German Foreign Office that there was not room in the world for a powerful Britain and a progressive Germany, and the reasons given, which required the major portion of a long and dull sea voyage for their discussion, were quite conclusive if the premises of the financial system were admitted. In spite of the organised efforts of Lord Roberts and others to drive the facts of the situation home to those persons most vitally affected, the members of the British public, it is quite certain that not 5 per cent, of the population of these islands regarded the question as anything but a political “stunt” run by a mixture of interested scaremongers and cranks with bees in their bonnets. Viewing the situation dispassionately in the light of events, it seems probable that the control of the organs of public information, and the general psychology of the peoples who were to be the victims of the coming disaster, were already so grouped as to make the late war, humanly speaking, inevitable; that any radical preventive propaganda, to have a reasonable prospect of success, must have been launched not much later than 1875, and must have taken effective steps, amongst other things, to deal with the capture and syndication of the public Press which marked the closing years of the nineteenth century. But the situation in regard to the disasters which threaten us now is profoundly altered, and we believe that it is a practical proposition to expect to bring such forces to bear on the situation as will suffice to avoid at any rate the full force of the blow which might otherwise destroy us. Amongst the differences on which legitimate optimism may be based is the increasing cynicism, common in every rank of society, in regard to the expression of beautiful sentiments unsupported by a live practical policy amenable to all the checks men apply to everyday affairs. The sob-stuff is losing its grip. By their fruits ye shall know them. Perhaps in some queer, perverted way President Wilson was indeed the saviour of the world, as he is said to have believed himself to be, when he heralded the entry of the United States into world politics by a series of speeches couched in the most silver eloquence, and embodying sentiments calculated to take the thoughts of men clean away from the facts of life; and then, in company with his fellow-conjurers, hatched out a treaty and a League of Nations expressly designed to reduce every one of these beautiful sentiments to a grinning mockery. “Open covenants openly arrived at”; Mr Lloyd George goes down to Lympne to discuss policy with Sir Philip Sassoon prior to reshuffling the destinies of peoples with M. Millerand; “self-determination”–and admittedly the ordinary everyday liberty of the subject fell during Mr Wilson’s administration of the United States Government to a lower level than that of Russia under the Tsar. The practical effect of this disillusionment is seen daily in operation; not so very long ago a rhetorical appeal for backing for the anti-Bolshevists was met by an unmistakably dry negative even from quarters which have no love for Lenin and Trotsky; the somewhat New-Jerusalemic tone of The Daily Herald is barely offset in its effect on its popularity by the realistic and detailed descriptions of the current prize-fights which form a feature of its otherwise pacifist pages. The general result of all this is to make it increasingly difficult to sweep a nation off its feet by a mere gust of emotion, and even if the change has not yet proceeded very far it is a most hopeful sign that it has begun.

The Food Controller’s monthly report issued in June 1920\footnotemark[1] showing a further serious rise in the price of food since January, was a grim comment on the attempt made to inaugurate what the French have christened \emph{la vague de baisse}, by the simple process of saying that it has arrived. So far from a wave of falling prices having reached either us or them, the level of prices steadily rose, and reached in this country 265 per cent, of the prices ruling for foodstuffs in July 1914, and there is every prospect that, with a possible temporary decline, it will continue to rise. The real cause can be stated in half-a-dozen words—the breakdown of credit; the disbelief in the reality of “money” as a good exchange for either goods or services; and there is nothing in the line being taken either by the Government or the large industrial combines; to show that they have either the will or the j understanding necessary to close the rapidly j widening gap between financial credit and real credit. It is now being allowed to transpire that the big manufacturers of the Midlands and the North are finding the way very hard indeed, their costs are such as to make their prices definitely non-commercial; and dark hints of the necessity of shutting down their plants for the purpose of bringing labour to its senses–\emph{i.e.} starving it into submission—are appearing in the columns of the Press. It is of course an open secret that this latter plan was concocted and agreed to by a ring of manufacturers in 1917 as being the inevitable result of concessions extracted from them during the war, but it was intended that it should be put into operation much earlier, and we very much doubt whether it may not now have results quite other than those expected by its inventors. What about the necessity for greater production as a cure for all evils? Surely if it is only production \emph{per se} that we want it is very reprehensible for the employer to consider for a single instant a policy which is not merely designed to limit production, but to stop it completely, when for so long we have been told that only greater production will save us. Can it be possible that the only production it is desired to increase is the production of money, and that if more money can be produced by making less goods we shall get less goods? The next few months should furnish an answer to that question.

\footnotetext[1]{This paragraph was written in June 1920, and is included for the purpose of showing the development (which took place almost exactly according to plan) of Financial Strategy

}\chapter{A Commentary on World Politics (IV)}
\label{chapter-12}
If Macaulay’s New Zealander, after musing on the more material remains of our social system as exemplified in the Houses of Parliament and the secretariats of Whitehall, should be driven to investigate the concepts of national organisation symbolised by them, it is fairly certain that nothing will astonish him more than the evidences he will find on every hand of the persistent and touching faith of this queer old people in what they call “representation.” He will find that this curious superstition (dating back to the earliest days of their history, when priests made a corner in deals with God and the dispensing of personal salvation became a close trust) persisted on even through the First World War when millions of persons who disliked war and held it in contempt as a moral and material anachronism allowed their representatives not merely to lead them into a war which had become inevitable but, almost without a protest, to throw away any poor consolation which might be derived from a real “war to end war.” He would note that at irregular and inappropriate intervals queer ceremonies called elections took place at which persons for the most part personally unknown to the electors were “returned” for the ostensible purpose of carrying out “reforms” which most of the electors neither understood nor cared about one fig. And he would further observe that these elected ones, once safely through the ceremony, at once became very superior persons, full of dignity and importance, and for the most part concerned with very intricate relations between the State and Borioboola-Gha. It seemed clear that these same electors never derived any benefit from these negotiations, or in fact and on the whole from more than the very minutest fraction of the activities of their representatives, while further it was quite plain that a small number of very opulent gentry of international sympathies, who were not elected, and represented no one but themselves, did in fact sway the whole deliberations of the elected assembly. Still this touching faith that some day they would elect the right men and all would be well seemed to sustain the people through a series of disappointments which would have daunted a less stubborn race. The New Zealander, who we must suppose to be an intelligent man, would, we think, conclude that this was a matter outside logic and reason, and only comparable to collective hypnotism. And he would be right.

In certain things this country in particular is under a spell. At the time of the Armistice there was not only not an unemployed man in this country, but there was hardly an unemployed woman or child over fourteen and under eighty. The wheat cultivation was increasing at a rate unknown for generations, shipbuilding was proceeding at such a rate that the destruction of war has been more than made good in two years, manufacturers were becoming rich, workmen were becoming manufacturers. Even the despised professional classes were for the most part able to eke out a precarious existence in either the fighting services, or if age or health precluded that, in ministering to the wants or patching the digestions of those who did well out of—a long way out of—the war. Production, which Mr Clynes will tell us is all we need to make us prosperous, reached heights far in excess of anything ever touched in history, even outstripping such destruction as Dante never dreamed of. Then peace, with the wings of a dove, burst upon us. Hardly had the last stretcher-case reached a casualty clearing station in a grim and haunted silence than a bleat of real anguish rose from these sheltered shores—not from the battered wrecks in hospital blue, the sad-eyed women in black, or even from the new poor, but from Lord Inchcape and other bankers. We were a poor nation—no homes for heroes for us. Perhaps, if we all worked harder than ever, and lived the simple life for twenty years or so, we might aspire to a few Nissen huts. As a preliminary to all of us working harder prices rose 50 per cent., and the unemployment figures rose from nil to the present figure of about three million. But further than that, the earnings, as apart from the wage rates of those still employed, fell also. On every hoarding may be seen auctioneers’ advertisements of eligible modern factories equipped with the finest tools to be sold at break-up prices, and manufacturers are beginning to compete with generals for eligible if undistinguished posts under the Holborn Borough Council. It is hardly to be wondered at that our warnings of a greater and more terrible war leave numbers of persons very cold, since only destruction on the largest scale, it seems to them, can provide a decent living for the survivors. Side by side with these happenings, which are plain for all to see, it cannot have escaped notice that every bank composing the charmed Circle of Five has pulled down its barns to build larger. The London City and Midland, to take one instance only, now has fifteen hundred branches, of which, at a guess, at least one half have been opened since 1914 in buildings of a solid magnificence appropriate to the temples of a great faith. Perhaps one of our readers with a taste for statistics will compile a table showing the percentage of corner sites occupied by banks as compared with those occupied by other undertakings. Has anyone during this time of industrial depression and labour distress noticed any bank premises for sale? Is there any possible room for doubt, not merely who did best out of the war, but is doing well out of the peace?

It might be noted from his article in \emph{The Daily Herald} of 24th March 1921, entitled “The Coal Crisis and the Nation’s Credit,” that Mr Frank Hodges “has been propounding up and down the country a scheme which is the only internal scheme calculated to help the mining industry out of its difficulties and consequently other industries out of theirs.” We wish Mr Hodges every success in his efforts, which aim at the use of national credit to enable coal to be sold below the cost of production, and we would offer him every assistance, technical and otherwise, to enable him to carry properly designed proposals of this character to a successful issue. His article in \emph{The Daily Herald} was, we think, admirable for the purpose for which it was intended, but we would suggest to him that a combination of his propaganda with a new and more effective form of “Direct Action” would be very—desirable at this time. He suggests that “the British Government” should either propose something better or put his scheme to the test of practice. We can assure Mr Hodges that the British Government, or that essential part of it which counts in matters of this sort, has no intention or desire to propose anything better—on the contrary, it has said in so many words that it is unalterably opposed to any proposition which involves the granting of a subsidy, and it is prepared to go to any amount of trouble and expense to prevent Mr Hodges making clear to any considerable number of persons how this proposal differs from one involving a subsidy. But if Mr Hodges will abandon the idea, so natural to ingenuous minds—we have had it ourselves—that the Government is struggling with a problem it does not understand and cannot solve, and ceasing his endeavours to enlighten it, will use the position entrusted to him to assist his constituents to dispense with Government acquiescence with his plan (and, of course, he must know that that is possible) we feel sure that he will be astonished at the quickened apprehension of Westminster.

At the time of writing (1921) the miners’ strike or lock-out, whichever it should be termed, has commenced, and according to the popular Press a number of pits are already irretrievably flooded. Lest the public should be in any doubt as to who pays for these little wrangles between the Coal Trust and the Labour Trust, the price of coal has been put up Is. per ton at once just to “larn us to be a twoad.” Our sympathies as between the two combatants are wholly with the Labour trust, because it contains more human beings, but they are a good deal more with the public than with either party, and we think we are not alone in the matter. It is quite time, we think, that the great trade unions should understand that the plea of the under dog, fighting against unfair odds of education and resources and injuring the bystander only because engaged in a life-and-death struggle, will not wash. The resources of, say, the Triple Alliance, are ample to put them in possession of every weapon in the hands of their antagonists—are, in fact, potentially far superior; and the fact that they are quite obviously incapable of striking a blow which the vile body of the public does not receive instead of the “Capitalists,” at whom it is aimed, might quite reasonably, and will, be adduced as a good sound reason that they are a public nuisance. That would be a superficial judgment, but we do suggest that clumsiness and ineptitude are now as inexcusable as real vice, and that the great causes of which the Trade Union and Labour movement claims to be the protagonist, and of which it is, in fact, the natural champion, should not be allowed much longer to be so compromised by mismanagement as has most unquestionably been the case during the past three years. We have said elsewhere that the British Labour Party in particular had an opportunity during the years 1914-1918 such as probably never before presented itself to any political party. That opportunity was missed thoroughly and completely, and the credit and power of the Labour Party is so damaged that it is quite possible that it may never recover. “The moving finger writes, and having writ moves on,” and not often, if ever, is a second innings vouchsafed to any side in a game of this magnitude. We see only one hope for the Labour Party; that it may, by a miraculous up-rush of leadership, renounce its absurd arrogance of all the virtues, and by truly representing the community, rather than a mere sectional interest, draw again to its aid—all those men of good-will in whatever station they may be found whose good offices it now seems so anxious to repel. If it will not do this, and do it soon, it will sink to the importance of the British Bolshevik Party, which is negligible except as a useful bogey, by the aid of which Mr Lloyd George can frighten old women of both sexes into voting for the Sassoon-Cassel-Zaharofi coalition.

\chapter{A Commentary on World Politics (V)}
\label{chapter-13}
If anyone is disposed to doubt our native genius for organisation we would direct his attention to the team-work of the Press on the subject of Mr Frank Hodges’ timid suggestion of a credit appropriation for the solution of the coal difficulty. In many keys, yet in perfect harmony, a shriek of horror has risen from organs professing all shades of political opinion yet united by the approach of a common danger to their financial masters. It is true that most of them, as newspapers, have no more knowledge of the processes of finance than is necessary to enable them to draw or cash a cheque, but, aided by some mysterious sense, none of them has failed to translate the proposal by exactly the same word “Subsidy.” And the concern of them for the poor taxpayer! Was there ever anything so touching? Happy the nation which has a Press so active and sensitive to the interests of its constituents. But there is more still to be done, and, without for the moment quibbling over the confusion involved in the misuse of words, we would direct the attention of Fleet Street to the great activity—supported from Downing Street and the city which is taking place in regard to various schemes for Export “Subsidies,” such as the Ter Meulen and Sir Edward Mountain proposals. Mr Hodges admits that a sum of £100,000,000 might be required for his purposes, but it is hardly denied that this sum would be represented by an increased distribution of coal in this country, since the increased purchasing power would not be reflected in an increased price for coal. That is to say, the “subsidy” would be represented by goods in this country. But the various Export “Subsidy” schemes contemplate the use of sums at least five times as large as that for which Mr Hodges is asking, and still taking Fleet Street’s word that a subsidy and a credit are the same thing, the distracted British tax-payer would be fleeced to ten times the extent to which Mr Hodges would subject him; he would not only have, \emph{ex hypothesi}, to find five hundred millions in taxation, but would be mulcted by a rise in the general level of prices due to the distribution of five hundred millions of money unrepresented by any increase of goods in this country. May we hope that, the point having been indicated, Sir Edward Mountain’s Export Subsidy Scheme will now receive the same candid and uniform treatment as that accorded to poor Mr Hodges?

Where Mr Hodges and the miners fail in strategy is that they do not seem to realise the fundamental weakness of their case as miners, and the immense strength of it as members of the public. We thoroughly recognise that the very worst and blackest aspect is put on their demands, but the elementary fact is that even as put by themselves there is nothing about them to compensate the public at large for the expense and inconvenience to which it is put by a strike. Outside the war profiteers, who, after all, are the merest tithe of the population and are congenitally selfish, there are very few classes in this country who are not far worse off financially than they were in 1914, and the classes who have lost most are those who, while saying least, think the more and exercise by far the most vital influence on affairs in a time of crisis. If instead of continually trumpeting their determination to raise their own standard of living, no matter who suffers in the process, the Triple Alliance would say, “We intend that the general standard of living in this country shall rise, and we mean to proceed, not by attacking anyone, but by assisting everyone—by first demanding a conference of all parties for the purpose of exploring every avenue which might lead to lowering the cost of living without lowering the income of anyone,” they would be invincible and would carry their own ends by a side wind. No altruism is required or is desirable—if every rich man in this country sold all that he hath and gave to the poor, the poor would only notice it for about three months, and after that would, under the conditions which the Labour movement has not so far challenged, starve to death through unemployment and tte failure of production, just as happened in Russia. As it is, a conviction is hardening in the country that, bad as things are, they would be simply intolerable if the Labour Party ever got into power. That psychology is disastrous, and when over some such issue as the present the Prime Minister decides to appeal to the country, it will result in his being returned with a majority which will be acclaimed as a mandate to put Labour exactly where Sir Alfred Mond and his confreres wish to see it.

Mr Hughes’ note to the Allied Powers, which may be considered as the first official pronouncement on Foreign Policy issued by the Harding Administration, is a sufficiently disquieting document. In the details of its comments on the Yap controversy and its demand for a share in the loot of Mesopotamia there is, of course, nothing new. The noteworthy content of the dispatch is the considered enunciation of a new Monroe doctrine embracing the whole world, and the intimation in effect that the other World Powers have been wasting their time in disposing (so far as they have disposed) of the problems contingent on the Peace Treaty—that nothing can be done without the acquiescence of the United States, and that as the United States has not acquiesced in what has been done, it is all null and void. Passing over the nice judicial point as to whether a nation which has been invited and has refused to take part in the deliberations which have led to the allotment of mandates and other little spoils of war is justified in objecting to the results when they are a more or less accomplished fact (because the only real sanction behind such an attitude is the will and the power to impose acceptance of it by force of arms), we may profitably consider exactly what is the position created by such a claim. If Washington is alone in making it, it is clear that the United States is claiming the position of the super-State, the ultimate arbiter of all things mundane. But if, on the other hand, she is only claiming a right which she is prepared to allow to others, then once again we are brought up against the question of sanctions. Suppose Montenegro should object to the future course of events in Mexico? Will Washington agree that all action in Mexico must be held up until Montenegro is placated? It requires some optimism to believe it.

The curious point about the old Monroe doctrine, which is not without interest in considering the new variant, is that probably more than anything else it has consistently handicapped the United States in her relations with South America to which it chiefly referred. While not above invoking it when occasion served, the peoples of the Latin republics derided it in conversation as a piece of unsolicited impertinence, and visited their resentment on the head of the unfortunate “Norte Americano,” both by trade discrimination against him and by direct personal dislike, with the result that, at any rate prior to 1914, he was easily the most unpopular national south of Panama. In itself there is, of course, no doubt that the Monroe doctrine was in the best interests of South America, and incidentally of this country, which always consistently supported it, but it is, nevertheless, incontestable that things being as they are, it was one of the ulterior forces concerned in the late war. Germany had acquired predominating commercial interests in Brazil, and only the Monroe doctrine and the British fleet stood between her and the annexation of a dominion larger than the United States and rich beyond the dreams of avarice—a country only held back by the incompetence and laziness of the Portuguese settlers. Presumably, although we have no information on the point, German interests in Brazil have suffered eclipse; it is certain that the United States have been making the most strenuous efforts to replace her not only in Brazil, but in the Argentine, where she was obtaining large financial power through her banking system; but the resentment of overlordship excited by the rather crude tactics of Washington is so strong that we may hazard a guess that our exporters are not doing very badly.

When a man is entirely destitute of knowledge and ideas in regard to the industrial situation, one of two pronouncements may safely be expected of him in regard to it. If he is of the traditional type of beef-eating Briton chiefly met with in country districts, who will endure anything if only he is not asked to think, he will probably bark out “Labour? D—d scoundrels! put ’em up against a wall and shoot ’em!” No one with a sense of humour ought to dislike this hearty ruffian, even if driven by uncontrollable impulse to throw a bucket of water over him. In the first place, he is no more responsible for his opinion than a terrier howling at Beethoven, and in the second place, however silly his method, his instinct is healthy—he wants a solution. The other person is in a different and, to us, much more contemptible category—he feels sure that all would be well if “both sides” would only show a little good-will. This man may not know it, but he is blasphemous. One of the most amazing features of the present situation is the steady bias towards good-will and reason met with everywhere—the prevalence of a subconscious feeling that an effort is being made to get honest men to fall out in order that thieves may break through and steal. It is particularly noticeable on the railways, where every grade seems anxious to discount the inconvenience it anticipates being forced to inflict on the public. The writer has been privileged to address various meetings up and down the country on Credit Reform proposals, at most of which have been present one or two unhappy-looking individuals whose ideals evidently did not agree with their digestions, or, perhaps, proceeded from them; but no one could mistake the isolation of their position. Most of these audiences either of so-called “masters” or “men” consisted of individuals actually grappling with the facts of industry, knowing the virtues, failings and common humanity of their neighbours, and well disposed to agree that a third party, Finance, understood by neither of them, might be the agency which for ever seemed to make agreement impossible. That there are small bodies of irreconcilables we agree; but if the main body of citizens had a sound lead we do not think that these warriors would count for very much.

There may be various opinions about Mr Lloyd George (known for obvious reasons, in political circles, as “The Goat”) as a Prime Minister, but it is impossible to deny him the very highest honours both as a strategist and as a political acrobat. His method of testing the electioneering temperature by calling out the reserves and imploring all loyal citizens to enlist in the Volunteer Defence Force during the late coal strike is very expensive to the tax-payer and very bad for the moral of the country, but gives him quite a fair idea of the votes he would get in the election he is doubtless considering. Having tested the temper of the country in this way, we may confidently expect him to go to the country at an early date and be returned to power with a substantial, even if slightly diminished, majority. In the unlikely event of his deciding that an election would be inopportune he will no doubt pose as the saviour of the country from the civil war we haven’t noticed. Either way, it all seems clear gain to Mr Lloyd George, and it is very, very clever. “Whether a little wisdom would not be worth more to the country, and to Mr Lloyd George himself, than all this agility is, of course, a matter on which one may hold strong opinions. It has always been incomprehensible to us that anyone could imagine that a body of men of the magnitude of, say, the Triple Alliance, beaten by starvation or force into accepting terms, distasteful to them, could fail to renew the struggle at the earliest possible moment; and we can only conclude that the International Financial Groups who precipitate these struggles do not really care how frequent they are—the cost of them is simply passed on to the public in prices, and the real authors of them not merely go completely untouched by the repeated tragedies, but from villas on the Riviera or elsewhere “glut” their love of power by contemplating the writhings of the world they have enslaved.

Dr Leighton Parkes, Rector of St Bartholomew’s Episcopal Church, New York, stirred up a hornets’ nest by stating that “he Roman Catholic Hierarchy in this country {[}United States{]} desires nothing more than to bring about a war with England, not only on account of the ancient grudge, but because England is the great Protestant country of Europe as we are in the Western Hemisphere.” We think Dr Leighton Parkes is to be congratulated on his plain speaking. It is quite certain that the fundamental difference between political Roman Catholicism and political Protestantism (all religions are the basis of political systems) is that the first is essentially authoritarian and the second is individualistic. There are thousands of English Roman Catholics who are such because they are attracted by the beauty and dignity of its ritual and the artistic impact of its code of life. But the simple fact remains that when stripped to its essentials the Roman claim is a claim for the surrender of individual judgment and, in any important crisis, of individual action. That is one reason why Roman Catholics are so successful in the army, and it is the great reason why the Hierarchy of Rome, as apart from the many delightful personages to be found in it, is a danger to peace, freedom and development, wherever it is entrenched.

\chapter{“The Moving Finger Writes...”}
\label{chapter-14}
It is now nearly three years since the first publication of the credit theory which has become, it is hoped, more familiar than seemed likely at that time. When that theory first saw the light of publicity the world, panting and enfeebled from the first world war, was threatened with social upheaval and torn with conflicting idealisms on the one hand, and a prey to the megalomaniacs of industry and finance on the other. “All power to the Soldiers and Workers Councils!” yelled the Left. “Increased production,” murmured Lord Inchcape, as he passed the plans for a few hundred new branch banks.

Well, they have all had their way. The greatest undivided unit of the world’s surface, a national territory which could accommodate comfortably the United States and the whole of non-Russian Europe within its boundaries and still have vast expanses unoccupied; an area which is probably far richer in potential resources than any other under one control, has been nominally ruled for more than four years by the first Workers Republic. In that short space of time millions of the class in whose interests it is alleged that the Soviet Republic was created have been reduced to a state of famine and misery far in excess of anything experienced under the corrupt and inefficient regime of the Tsar. The control of the individual worker over his life and destiny, so far from having increased, has become a mere mockery, and the only tolerable portions of Russia appear to be those in which the writ of the centralised despotism of Moscow does not run. A new era is opening; enter Herr Stinnes and Mr Hoover.

In Great Britain and America the working out of the dominant policy has been equally instructive if only less immediately disastrous. Following on the gigantic expansion of plant which took place during the war, the year 1919 and the early half of 1920 saw still more factory and real capital production, accompanied, for reasons explained many times in these pages, by a continuous rise in prices. Lord Inchcape has got his branch banks. Homes for heroes are still under strength.

In May 1920 the financial powers considered that the process had gone far enough and withheld further facilities, and in a period of less than eighteen months, of the many ambitious enterprises floated at the expense of the public in the immediate post-war period, probably 95 per cent, have come into the complete control of the Joint Stock Banks, and not one of the remainder can carry on for a year except with their permission. The banker, busily engaged just now in sorting out from his catch those specimens worth, from a banker’s point of view, preservation, condoles with the manufacturer and trader, who doesn’t quite know what has hit him. “Ah, my dear fellow,” you can hear him say, “if only those damned lazy scoundrels of yours had worked harder and consumed less! Wait a bit: sell your car and live quietly for a few months and we may be able to put you back into your own works as manager, and then you can teach the blighters what’s what. In fact we’ll take care that you do, if you want to hold your job. Good-morning, my \emph{dear} fellow."

(It will be noticed that while prices of retail or ultimate commodities rose during the period of credit inflation almost directly in proportion to that inflation, the stringency has failed signally to produce a corresponding fall: a result which confirms the credit theory that while prices can rise to any height under the stress of financial or effective demand, they cannot fall below costs, which include all credit issues, without bankruptcy of any entrepreneur who has not access to the general credit.)

Meanwhile the Labour Movement in this country and in America has met its Waterloo. Headed very vigorously and firmly away from one or two timid approaches to a consumers’ policy, such as the demand for a trifling reduction in the price of coal, and bound hand and foot to an economic theory identical with the capitalism it professes to attack, it is now firmly established in the public estimation as an anti-public interest. Endowed by the circumstances of the war with such an opportunity as no one political party ever had before or probably ever will have again, the Labour Party both in Parliament and out of it has proved to demonstration that because its structure is fundamentally identical with that of other political parties it moves more or less slowly along similar lines to those of its competitors, depending as to pace on the qualities of its personnel. They can change the pace but they cannot change the direction. That direction is merely to centralise or focus whatever power is resident in the interest for which the party stands, and it is patent that labour, simply as a component of the productive process, is fundamentally a dying interest.

Had the miners’ strike, or lock-out, occurred twenty-five years ago it would have paralysed this country and convulsed the world. How much of a ripple did it produce in 1921? If economic power precedes political power, as it does, how much influence will a purely Labour Party exercise in politics?

The factor most destructive of progress to the Labour Party and most useful to the forces opposed to its legitimate aspirations is its incorrigible abstraction from reality—an abstraction which is quite probably the result, amongst other things, of generations of “religious” instruction specifically directed to the preaching of “other worldliness,” and to that extent also an instance of the direction of Labour thought by financial influences. It is rampant in every sphere of Labour political action, from the lionising of Mr George Lansbury, an honest citizen who would like to apply his conception of the Sermon on the Mount to the game of cut-throat poker, to the instantaneous success of Mr Tawney’s title for his book, \emph{The Sickness of an Acquisitive Society}. I have not read that book, which is doubtless excellent, but its title suggests that the average man ought to work with the specific object of not getting what he works for—goods; a precisely parallel line of argument to that of the orthodox capitalist who insists that the major object of industry is to send goods away from those who made them, by export, or otherwise, so that “employment” may never fail. Put shortly, the psychology of the Labour Party is a psychology of failure. To be poor is to be virtuous; to be well off is to be wicked; and the objective of all action is to replace the wicked by the virtuous. As a result, the official Labour Party is almost irrevocably committed to a policy of attack, of levelling down, and is bound to be opposed, sooner or later, by everyone with any conception of the possibility of levelling up, as well as by those who have anything to lose.

It is no pleasant thing to have to criticise that party. There was a period when organised Labour appeared to be the hope of the world, but that hope is now very dim; not only from the causes just outlined, but because the power given to it by the circumstances of war has been dissipated. Not a single proposition of the capitalist system has been even challenged by it; every strike has been a fight for position in the system, a claim either that the office boy ought to be General Manager, or at any rate ought to control the General Manager, combined with lurid threats to the firm’s customers that, in the happy days to come, Labour would “larn” them what it was like to be an office boy. A very alluring programme. \emph{R.I.P.}

While the Labour Party has for all practical purposes devoted its attention to a mechanical and unreasoning claim to power on the grounds of virtue, the financiers have not been so immobile. So long as it was possible to keep the subject of credit away from public discussion it was done, and done well. But merely negative opposition, in the nature of things, being bound to fail, a positive line of action has been elaborated and is now well under way—the exploitation of public credit for export purposes. Apart from the Ter Meulen and Mountain schemes, the Government (\emph{i.e.} Zaharoff-Sassoon) proposals for dealing with “unemployment” are based fundamentally on an export credit scheme buttressed by relief works at home, the latter to be financed out of taxation.

Now it is our contention that the use and control of credit is absolutely the vital issue of the present era. It is a force and can be used like other forces to destroy or to build, and it is quite possible that in this Government proposal we are faced with a real crisis in the history of civilisation. If it is put into force, we are committed to a line of action diametrically opposed to that urged in the pages of this book, and it is therefore vital that it should be understood.

The proposal involves the pledging of public credit to the extent of (at first), say, £26,000,000. It should be particularly noted that Mr Lloyd George explicitly says: “It is not consumable goods of which the world stands in the most urgent need. What it stands mostly in need of is equipment to start its trade—machinery, transport; and short credits are of no use when you are dealing with heavy goods of that kind. We have come to the conclusion ... it is desirable that we should extend credit for five or even six years (\emph{Hear, hear!})"–\emph{Times} report, 20th October.

That is to say, although the productive capacity of the industrial nations was so enormous that it overtook the wastage of a four and a half years’ war in eighteen months, so that two and a half millions are unemployed in this country, and probably six millions in America, the energies of the nation are to be employed, not in obtaining the maximum benefit from its existing plant, but in producing still more plant to be exported in competition with countries similarly situated.

This £25,000,000, then, will be paid out in this country as wages, salaries and dividends, entirely unrepresented by any goods for which the general public has any demand whatever. The money so paid out, therefore, represents pure inflation, and, being unaccompanied by any method of dealing with prices, means the inevitable result of pure inflation—a rise in prices. In other words, the goods exported under these conditions are paid for by the general public through the agency of a general rise in prices, but not delivered to them, but the credits, if ever repaid, are repaid not to the general public, but to the banks who will finance these credits. And as at the same time these exports will be in fierce competition with similar goods from, say, America, preparations for the coming war will naturally be accelerated.



\end{document}
