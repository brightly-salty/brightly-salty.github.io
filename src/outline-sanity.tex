\documentclass{book}
\usepackage{fontspec}
\usepackage{xunicode}
\usepackage[english]{babel} 
\usepackage{fancyhdr}
\usepackage[htt]{hyphenat}
\usepackage[a5paper, top=2cm, bottom=1.5cm, left=2.5cm,right=1.5cm]{geometry}
\makeatletter
\date{}
\pagestyle{fancy}
\fancyhead{}
\fancyhead[CO,CE]{\thepage}
\fancyfoot{}
\makeatother
\title{The Outline of Sanity}
\author{G. K. Chesterton}
\begin{document}
\makeatletter
\renewcommand{\@chapapp}{Section}
\makeatother
\renewcommand{\partname}{Chapter}
\thispagestyle{empty}
\vspace*{\stretch{1}}
\begin{center}
	{\Huge \@title   \\[5mm]}
\end{center}
\vspace*{\stretch{2}}
\newpage
\thispagestyle{empty}
\cleardoublepage
\begin{center}
	\thispagestyle{empty}
	\vspace*{\baselineskip}
	\rule{\textwidth}{1.6pt}\vspace*{-\baselineskip}\vspace*{2pt}
	\rule{\textwidth}{0.4pt}\\[\baselineskip]
	{\Huge\scshape \@title   \\[5mm]}
	{\Large }
	\rule{\textwidth}{0.4pt}\vspace*{-\baselineskip}\vspace{3.2pt}
	\rule{\textwidth}{1.6pt}\\[\baselineskip]
	\vspace*{4\baselineskip}
	{\Large \@author}
	\vfill
\end{center}
\pagebreak
\newpage
\thispagestyle{empty}
\null\vfill
\noindent
\begin{center}
	{\emph{\@title}, © \@author.\\[5mm]}
	{This work is free of known copyright restrictions.\\[5mm]}
\end{center}
\pagebreak
\newpage
\setcounter{tocdepth}{0}
\setcounter{secnumdepth}{0}
\setcounter{chapter}{0}

\part{Some General Ideas}
\label{chapter-0}
\chapter{The Beginning of the Quarrel}
\label{chapter-1}
I have been asked to republish these notes—which appeared in a weekly paper—as a rough sketch of certain aspects of the institution of Private Property, now so completely forgotten amid the journalistic jubilations over Private Enterprise. The very fact that the publicists say so much of the latter and so little of the former is a measure of the moral tone of the times. A pickpocket is obviously a champion of private enterprise. But it would perhaps be an exaggeration to say that a pickpocket is a champion of private property. The point about Capitalism and Commercialism, as conducted of late, is that they have really preached the extension of business rather than the preservation of belongings; and have at best tried to disguise the pickpocket with some of the virtues of the pirate. The point about Communism is that it only reforms the pickpocket by forbidding pockets.

Pockets and possessions generally seem to me to have not only a more normal but a more dignified defence than the rather dirty individualism that talks about private enterprise. In the hope that it may possibly help others to understand it, I have decided to reproduce these studies as they stand, hasty and sometimes merely topical as they were. It is indeed very hard to reproduce them in this form, because they were editorial notes to a controversy largely conducted by others; but the general idea is at least present. In any case, “private enterprise” is no very noble way of stating the truth of one of the Ten Commandments. But there was at least a time when it was more or less true. The Manchester Radicals preached a rather crude and cruel sort of competition; but at least they practised what they preached. The newspapers now praising private enterprise are preaching the very opposite of anything that anybody dreams of practising. The practical tendency of all trade and business to-day is towards big commercial combinations, often more imperial, more impersonal, more international than many a communist commonwealth—things that are at least collective if not collectivist. It is all very well to repeat distractedly, “What are we coming to, with all this Bolshevism?” It is equally relevant to add, “What are we coming to, even without Bolshevism?” The obvious answer is—Monopoly. It is certainly not private enterprise. The American Trust is not private enterprise. It would be truer to call the Spanish Inquisition private judgment. Monopoly is neither private nor enterprising. It exists to prevent private enterprise. And that system of trust or monopoly, that complete destruction of property, would still be the present goal of all our progress, if there were not a Bolshevist in the world.

Now I am one of those who believe that the cure for centralization is decentralization. It has been described as a paradox. There is apparently something elvish and fantastic about saying that when capital has come to be too much in the hand of the few, the right thing is to restore it into the hands of the many. The Socialist would put it in the hands of even fewer people; but those people would be politicians, who (as we know) always administer it in the interests of the many. But before I put before the reader things written in the very thick of the current controversy, I foresee it will be necessary to preface them with these few paragraphs, explaining a few of the terms and amplifying a few of the assumptions. I was in the weekly paper arguing with people who knew the shorthand of this particular argument; but to be clearly understood, we must begin with a few definitions or, at least, descriptions. I assure the reader that I use words in quite a definite sense, but it is possible that he may use them in a different sense; and a muddle and misunderstanding of that sort does not even rise to the dignity of a difference of opinion.

For instance, Capitalism is really a very unpleasant word. It is also a very unpleasant thing. Yet the thing I have in mind, when I say so, is quite definite and definable; only the name is a very unworkable word for it. But obviously we must have some word for it. When I say “Capitalism,” I commonly mean something that may be stated thus: “That economic condition in which there is a class of capitalists, roughly recognizable and relatively small, in whose possession so much of the capital is concentrated as to necessitate a very large majority of the citizens serving those capitalists for a wage.” This particular state of things can and does exist, and we must have some word for it, and some way of discussing it. But this is undoubtedly a very bad word, because it is used by other people to mean quite other things. Some people seem to mean merely private property. Others suppose that capitalism must mean anything involving the use of capital. But if that use is too literal, it is also too loose and even too large. If the use of capital is capitalism, then everything is capitalism. Bolshevism is capitalism and anarchist communism is capitalism; and every revolutionary scheme, however wild, is still capitalism. Lenin and Trotsky believe as much as Lloyd George and Thomas that the economic operations of to-day must leave something over for the economic operations of to-morrow. And that is all that capital means in its economic sense. In that case, the word is useless. My use of it may be arbitrary, but it is not useless. If capitalism means private property, I am capitalist. If capitalism means capital, everybody is capitalist. But if capitalism means this particular condition of capital, only paid out to the mass in the form of wages, then it does mean something, even if it ought to mean something else.

The truth is that what we call Capitalism ought to be called Proletarianism. The point of it is not that some people have capital, but that most people only have wages because they do not have capital. I have made an heroic effort in my time to walk about the world always saying Proletarianism instead of Capitalism. But my path has been a thorny one of troubles and misunderstandings. I find that when I criticize the Duke of Northumberland for his Proletarianism, my meaning does not get home. When I say I should often agree with the Morning Post if it were not so deplorably Proletarian, there seems to be some strange momentary impediment to the complete communion of mind with mind. Yet that would be strictly accurate; for what I complain of, in the current defence of existing capitalism, is that it is a defence of keeping most men in wage dependence; that is, keeping most men without capital. I am not the sort of precision who prefers conveying correctly what he doesn’t mean, rather than conveying incorrectly what he does. I am totally indifferent to the term as compared to the meaning. I do not care whether I call one thing or the other by this mere printed word beginning with a “C,” so long as it is applied to one thing and not the other. I do not mind using a term as arbitrary as a mathematical sign, if it is accepted like a mathematical sign. I do not mind calling Property x and Capitalism y, so long as nobody thinks it necessary to say that x=y. I do not mind saying “cat” for capitalism and “dog” for distributism, so long as people understand that the things are different enough to fight like cat and dog. The proposal of the wider distribution of capital remains the same, whatever we call it, or whatever we call the present glaring contradiction of it. It is the same whether we state it by saying that there is too much capitalism in the one sense or too little capitalism in the other. And it is really quite pedantic to say that the use of capital must be capitalist. We might as fairly say that anything social must be Socialist; that Socialism can be identified with a social evening or a social glass. Which, I grieve to say, is not the case.

Nevertheless, there is enough verbal vagueness about Socialism to call for a word of definition. Socialism is a system which makes the corporate unity of society responsible for all its economic processes, or all those affecting life and essential living. If anything important is sold, the Government has sold it; if anything important is given, the Government has given it; if anything important is even tolerated, the Government is responsible for tolerating it. This is the very reverse of anarchy; it is an extreme enthusiasm for authority. It is in many ways worthy of the moral dignity of the mind; it is a collective acceptance of a very complete responsibility. But it is silly of Socialists to complain of our saying that it must be a destruction of liberty. It is almost equally silly of Anti-Socialists to complain of the unnatural and unbalanced brutality of the Bolshevist Government in crushing a political opposition. A Socialist Government is one which in its nature does not tolerate any true and real opposition. For there the Government provides everything; and it is absurd to ask a Government to provide an opposition.

You cannot go to the Sultan and say reproachfully, “You have made no arrangements for your brother dethroning you and seizing the Caliphate.” You cannot go to a medieval king and say, “Kindly lend me two thousand spears and one thousand bowmen, as I wish to raise a rebellion against you.” Still less can you reproach a Government which professes to set up everything, because it has not set up anything to pull down all it has set up. Opposition and rebellion depend on property and liberty. They can only be tolerated where other rights have been allowed to strike root, besides the central right of the ruler. Those rights must be protected by a morality which even the ruler will hesitate to defy. The critic of the State can only exist where a religious sense of right protects his claims to his own bow and spear; or at least, to his own pen or his own printing-press. It is absurd to suppose that he could borrow the royal pen to advocate regicide or use the Government printing-presses to expose the corruption of the Government. Yet it is the whole point of Socialism, the whole case for Socialism, that unless all printing-presses are Government printing-presses, printers may be oppressed. Everything is staked on the State’s justice; it is putting all the eggs in one basket. Many of them will be rotten eggs; but even then you will not be allowed to use them at political elections.

About fifteen years ago a few of us began to preach, in the old \emph{New Age} and \emph{New Witness,} a policy of small distributed property (which has since assumed the awkward but accurate name of Distributism), as we should have said then, against the two extremes of Capitalism and Communism. The first criticism we received was from the most brilliant Fabians, especially Mr. Bernard Shaw. And the form which that first criticism took was simply to tell us that our ideal was impossible. It was only a case of Catholic credulity about fairy-tales. The Law of Rent, and other economic laws, made it inevitable that the little rivulets of property should run down into the pool of plutocracy. In truth, it was the Fabian wit, and not merely the Tory fool, who confronted our vision with that venerable verbal opening, “If it were all divided up to-morrow—-”

Nevertheless, we had an answer even in those days, and though we have since found many others, it will clarify the question if I repeat this point of principle. It is true that I believe in fairy-tales—in the sense that I marvel so much at what does exist that I am the readier to admit what might. I understand the man who believes in the Sea Serpent on the ground that there are more fish in the sea than ever came out of it. But I do it the more because the other man, in his ardour for disproving the Sea Serpent, always argues that there are not only no snakes in Iceland, but none in the world. Suppose Mr. Bernard Shaw, commenting on this credulity, were to blame me for believing (on the word of some lying priest) that stones could be thrown up into the air and hang there suspended like a rainbow. Suppose he told me tenderly that I should not believe this Popish fable of the magic stones, if I had ever had the Law of Gravity scientifically explained to me. And suppose, after all this, I found he was only talking about the impossibility of building an arch. I think most of us would form two main conclusions about him and his school. First, we should think them very ill-informed about what is really meant by recognizing a law of nature. A law of nature can be recognized by resisting it, or out-manoeuvring it, or even using it against itself, as in the case of the arch. And second, and much more strongly, we should think them astonishingly ill-informed about what has already been done upon this earth.

Similarly, the first fact in the discussion of whether small properties can exist is the fact that they do exist. It is a fact almost equally unmistakable that they not only exist but endure. Mr. Shaw affirmed, in a sort of abstract fury, that “small properties will not stay small.” Now it is interesting to note here that the opponents of anything like a several proprietary bring two highly inconsistent charges against it. They are perpetually telling us that the peasant life in Latin or other countries is monotonous, is unprogressive, is covered with weedy superstitions, and is a sort of survival of the Stone Age. Yet even while they taunt us with its survival, they argue that it can never survive. They point to the peasant as a perennial stick-in-the-mud; and then refuse to plant him anywhere, on the specific ground that he would not stick. Now, the first of the two types of denunciation is arguable enough; but in order to denounce peasantries, the critics must admit that there are peasantries to denounce. And if it were true that they always tended rapidly to disappear, it would not be true that they exhibited those primitive customs and conservative opinions which they not only do, in fact, exhibit, but which the critics reproach them with exhibiting. They cannot in common sense accuse a thing at once of being antiquated and of being ephemeral. It is, of course, the dry fact, to be seen in broad daylight, that small peasant properties are not ephemeral. But anyhow, Mr. Shaw and his school must not say that arches cannot be built, and then that they disfigure the landscape. The Distributive State is not a hypothesis for him to demolish; it is a phenomenon for him to explain.

The truth is that the conception that small property evolves into Capitalism is a precise picture of what practically never takes place. The truth is attested even by facts of geography, facts which, as it seems to me, have been strangely overlooked. Nine times out of ten, an industrial civilization of the modern capitalist type does not arise, wherever else it may arise, in places where there has hitherto been a distributive civilization like that of a peasantry. Capitalism is a monster that grows in deserts. Industrial servitude has almost everywhere arisen in those empty spaces where the older civilization was thin or absent. Thus it grew up easily in the North of England rather than the South; precisely because the North had been comparatively empty and barbarous through all the ages when the South had a civilization of guilds and peasantries. Thus it grew up easily in the American continent rather than the European; precisely because it had nothing to supplant in America but a few savages, while in Europe it had to supplant the culture of multitudinous farms. Everywhere it has been but one stride from the mud-hut to the manufacturing town. Everywhere the mud-hut which really turned into the free farm has never since moved an inch towards the manufacturing town. Wherever there was the mere lord and the mere serf, they could almost instantly be turned into the mere employer and the mere employee. Wherever there has been the free man, even when he was relatively less rich and powerful, his mere memory has made complete industrial capitalism impossible. It is an enemy that has sown these tares, but even as an enemy he is a coward. For he can only sow them in waste places, where no wheat can spring up and choke them.

To take up our parable again, we say first that arches exist; and not only exist but remain. A hundred Roman aqueducts and amphitheatres are there to show that they can remain as long or longer than anything else. And if a progressive person informs us that an arch always turns into a factory chimney, or even that an arch always falls down because it is weaker than a factory chimney, or even that wherever it does fall down people perceive that they must replace it by a factory chimney—why, we shall be so audacious as to cast doubts on all these three assertions. All we could possibly admit is that the principle supporting the chimney is simpler than the principle of the arch; and for that very reason the factory chimney, like the feudal tower, can rise the more easily in a howling wilderness.

But the image has yet a further application. If at this moment the Latin countries are largely made our model in the matter of the small property, it is only in the sense in which they would have been, through certain periods of history, the only exemplars of the arch. There was a time when all arches were Roman arches; and when a man living by the Liffey or the Thames would know as little about them as Mr. Shaw knows about peasant proprietors. But that does not mean that we fight for something merely foreign, or advance the arch as a sort of Italian ensign; any more than we want to make the Thames as yellow as the Tiber, or have any particular taste in macaroni or malaria. The principle of the arch is human, and applicable to and by all humanity. So is the principle of well-distributed private property. That a few Roman arches stood in ruins in Britain is not a proof that arches cannot be built, but on the contrary, a proof that they can.

And now, to complete the coincidence or analogy, what is the principle of the arch? You can call it, if you like, an affront to gravitation; you will be more correct if you call it an appeal to gravitation. The principle asserts that by combining separate stones of a particular shape in a particular way, we can ensure that their very tendency to fall shall prevent them from falling. And though my image is merely an illustration, it does to a great extent hold even as to the success of more equalized properties. What upholds an arch is an equality of pressure of the separate stones upon each other. The equality is at once mutual aid and mutual obstruction. It is not difficult to show that in a healthy society the moral pressure of different private properties acts in exactly the same way. But if the other school finds the key or comparison insufficient, it must find some other. It is clear that no natural forces can frustrate the fact. To say that any law, such as that of rent, makes against it is true only in the sense that many natural laws make against all morality and the very essentials of manhood. In that sense, scientific arguments are as irrelevant to our case for property as Mr. Shaw used to say they were to his case against vivisection.

Lastly, it is not only true that the arch of property remains, it is true that the building of such arches increases, both in quantity and quality. For instance, the French peasant before the French Revolution was already indefinitely a proprietor; it has made his property more private and more absolute, not less. The French are now less than ever likely to abandon the system, when it has proved for the second, if not the hundredth time, the most stable type of prosperity in the stress of war. A revolution as heroic, and even more unconquerable, has already in Ireland disregarded alike the Socialist dream and the Capitalist reality, with a driving energy of which no one has yet dared to foresee the limits. So, when the round arch of the Romans and the Normans had remained for ages as a sort of relic, the rebirth of Christendom found for it a further application and issue. It sprang in an instant to the titanic stature of the Gothic; where man seemed to be a god who had hanged his worlds upon nothing. Then was unsealed again something of that ancient secret which had so strangely described the priest as the builder of bridges. And when I look to-day at some of the bridges which he built above the air, I can understand a man still calling them impossible, as their only possible praise.

What do we mean by that “equality of pressure” as of the stones in an arch? More will be said of this in detail; but in general we mean that the modern passion for incessant and restless buying and selling goes along with the extreme inequality of men too rich or too poor. The explanation of the continuity of peasantries (which their opponents are simply forced to leave unexplained) is that, where that independence exists, it is valued exactly as any other dignity is valued when it is regarded as normal to a man; as no man goes naked or is beaten with a stick for hire.

The theory that those who start reasonably equal cannot remain reasonably equal is a fallacy founded entirely on a society in which they start extremely unequal. It is quite true that when capitalism has passed a certain point, the broken fragments of property are very easily devoured. In other words, it is true when there is a small amount of small property; but it is quite untrue when there is a large amount of small property. To argue from what happened in the rush of big business and the rout of scattered small businesses to what must always happen when the parties are more on a level, is quite illogical. It is proving from Niagara that there is no such thing as a lake. Once tip up the lake and the whole of the water will rush one way; as the whole economic tendency of capitalist inequality rushes one way. Leave the lake as a lake, or the level as a level, and there is nothing to prevent the lake remaining until the crack of doom—as many levels of peasantry seem likely to remain until the crack of doom. This fact is proved by experience, even if it is not explained by experience; but, as a matter of fact, it is possible to suggest not only the experience but the explanation. The truth is that there is no economic tendency whatever towards the disappearance of small property, until that property becomes so very small as to cease to act as property at all. If one man has a hundred acres and another man has half an acre, it is likely enough that he will be unable to live on half an acre. Then there will be an economic tendency for him to sell his land and make the other man the proud possessor of a hundred and a half. But if one man has thirty acres and the other man has forty acres, there is no economic tendency of any kind whatever to make the first man sell to the second. It is simply false to say that the first man cannot be secure of thirty or the second man content with forty. It is sheer nonsense; like saying that any man who owns a bull terrier will be bound to sell it to somebody who owns a mastiff. It is like saying that I cannot own a horse because I have an eccentric neighbour who owns an elephant.

Needless to say, those who insist that roughly equalized ownership cannot exist, base their whole argument on the notion that it has existed. They have to suppose, in order to prove their point, that people in England, for instance, did begin as equals and rapidly reached inequality. And it only rounds off the humour of their whole position that they assume the existence of what they call an impossibility in the one case where it has really not occurred. They talk as if ten miners had run a race, and one of them became the Duke of Northumberland. They talk as if the first Rothschild was a peasant who patiently planted better cabbages than the other peasants. The truth is that England became a capitalist country because it had long been an oligarchical country. It would be much harder to point out in what way a country like Denmark need become oligarchical. But the case is even stronger when we add the ethical to the economic common sense. When there is once established a widely scattered ownership, there is a public opinion that is stronger than any law; and very often (what in modern times is even more remarkable) a law that is really an expression of public opinion. It may be very difficult for modern people to imagine a world in which men are not generally admired for covetousness and crushing their neighbours but I assure them that such strange patches of an earthly paradise do really remain on earth.

The truth is that this first objection of impossibility in the abstract flies flat in the face of all the facts of experience and human nature. It is not true that a moral custom cannot hold most men content with a reasonable status, and careful to preserve it. It is as if we were to say that because some men are more attractive to women than others, therefore the inhabitants of Balham under Queen Victoria could not possibly have been arranged on a monogamous model, with one man one wife. Sooner or later, it might be said, all females would be found clustering round the fascinating few, and nothing but bachelorhood be left for the unattractive many. Sooner or later the suburb must consist of a hundred hermitages and three harems. But this is not the case. It is not the case at present, whatever may happen if the moral tradition of marriage is really lost in Balham. So long as that moral tradition is alive, so long as stealing other people’s wives is reprobated or being faithful to a spouse is admired, there are limits to the extent to which the wildest profligate in Balham can disturb the balance of the sexes. So any land-grabber would very rapidly find that there were limits to the extent to which he could buy up land in an Irish or Spanish or Serbian village. When it is really thought hateful to take Naboth’s vineyard, as it is to take Uriah’s wife, there is little difficulty in finding a local prophet to pronounce the judgment of the Lord. In an atmosphere of capitalism the man who lays field to field is flattered; but in an atmosphere of property he is promptly jeered at or possibly stoned. The result is that the village has not sunk into plutocracy or the suburb into polygamy.

Property is a point of honour. The true contrary of the word “property” is the word “prostitution.” And it is not true that a human being will always sell what is sacred to that sense of self-ownership, whether it be the body or the boundary. A few do it in both cases; and by doing it they always become outcasts. But it is not true that a majority must do it; and anybody who says it is, is ignorant, not of our plans and proposals, not of anybody’s visions and ideals, not of distributism or division of capital by this or that process, but of the facts of history and the substance of humanity. He is a barbarian who has never seen an arch.

In the notes I have here jotted down it will be obvious, of course, that the restoration of this pattern, simple as it is, is much more complicated in a complicated society. Here I have only traced it in the simplest form as it stood, and still stands, at the beginning of our discussion. I disregard the view that such “reaction” cannot be. I hold the old mystical dogma that what Man has done, Man can do. My critics seem to hold a still more mystical dogma: that Man cannot possibly do a thing because he has done it. That is what seems to be meant by saying that small property is “antiquated.” It really means that all property is dead. There is nothing to be reached upon the present lines except the increasing loss of property by everybody, as something swallowed up into a system equally impersonal and inhuman, whether we call it Communism or Capitalism. If we cannot go back, it hardly seems worth while to go forward.

There is nothing in front but a flat wilderness of standardization either by Bolshevism or Big Business. But it is strange that some of us should have seen sanity, if only in a vision, while the rest go forward chained eternally to enlargement without liberty and progress without hope.

\chapter{The Peril of the Hour}
\label{chapter-2}
When we are for a moment satisfied, or sated, with reading the latest news of the loftiest social circles, or the most exact records of the most responsible courts of justice, we naturally turn to the serial story in the newspaper, called “Poisoned by Her Mother” or “The Mystery of the Crimson Wedding Ring,” in search of something calmer and more quietly convincing, more restful, more domestic, and more like real life. But as we turn over the pages, in passing from the incredible fact to the comparatively credible fiction, we are very likely to encounter a particular phrase on the general subject of social degeneracy. It is one of a number of phrases that seem to be kept in solid blocks in the printing-offices of newspapers. Like most of these solid statements, it is of a soothing character. It is like the headline of “Hopes of a Settlement,” by which we learn that things are unsettled; or that topic of the “Revival of Trade,” which it is part of the journalistic trade periodically to revive. The sentence to which I refer is to this effect: that the fears about social degeneracy need not disturb us, because such fears have been expressed in every age; and there are always romantic and retrospective persons, poets, and such riff-raff, who look back to imaginary “good old times.”

It is the mark of such statements that they seem to satisfy the mind; in other words, it is the mark of such thoughts that they stop us from thinking. The man who has thus praised progress does not think it necessary to progress any further. The man who has dismissed a complaint, as being old, does not himself think it necessary to say anything new. He is content to repeat this apology for existing things; and seems unable to offer any more thoughts on the subject. Now, as a matter of fact, there are a number of further thoughts that might be suggested by the subject. Of course, it is quite true that this notion of the decline of a state has been suggested in many periods, by many persons, some of them, unfortunately, poets. Thus, for instance, Byron, notoriously so moody and melodramatic, had somehow or other got it into his head that the Isles of Greece were less glorious in arts and arms in the last days of Turkish rule than in the days of the battle of Salamis or the Republic of Plato. So again Wordsworth, in an equally sentimental fashion, seems to insinuate that the Republic of Venice was not quite so powerful when Napoleon trod it out like a dying ember as when its commerce and art filled the seas of the world with a conflagration of colour. So many writers in the eighteenth and nineteenth centuries have even gone so far as to suggest that modern Spain played a less predominant part than Spain in the days of the discovery of America or the victory of Lepanto. Some, even more lacking in that Optimism which is the soul of commerce, have made an equally perverse comparison between the earlier and the later conditions of the commercial aristocracy of Holland. Some have even maintained that Tyre and Sidon are not quite so fashionable as they used to be; and somebody once said something about “the ruins of Carthage.”

In somewhat simpler language, we may say that all this argument has a very big and obvious hole in it. When a man says, “People were as pessimistic as you are in societies which were not declining, but were even advancing,” it is permissible to reply, “Yes, and people were probably as optimistic as you are in societies which really declined.” For, after all, there were societies which really declined. It is true that Horace said that every generation seemed to be worse than the last, and implied that Rome was going to the dogs, at the very moment when all the external world was being brought under the eagles. But it is quite likely that the last forgotten court poet, praising the last forgotten Augustulus at the stiff court of Byzantium, contradicted all the seditious rumours of social decline, exactly as our newspapers do, by saying that, after all, Horace had said the same thing. And it is also possible that Horace was right; that it was in his time that the turn was taken which led from Horatius on the bridge to Heracleius in the palace; that if Rome was not immediately going to the dogs, the dogs were coming to Rome, and their distant howling could first be heard in that hour of the uplifted eagles; that there had begun a long advance that was also a long decline, but ended in the Dark Ages. Rome had gone back to the Wolf.

I say this view is at least tenable, though it does not really represent my own; but it is quite sufficiently reasonable to refuse to be dismissed with the cheap cheerfulness of the current maxim. There has been, there can be, such a thing as social decline; and the only question is, at any given moment, whether Byzantium had declined or whether Britain is declining. In other words, we must judge any such case of alleged degeneracy on its own merits. It is no answer to say, what is, of course, perfectly true, that some people are naturally prone to such pessimism. We are not judging them, but the situation which they judged or misjudged. We may say that schoolboys have always disliked having to go to school. But there is such a thing as a bad school. We may say the farmers always grumble at the weather. But there is such a thing as a bad harvest. And we have to consider as a question of the facts of the case, and not of the feelings of the farmer, whether the moral world of modern England is likely to have a bad harvest.

Now the reasons for regarding the present problem of Europe, and especially of England, as most menacing and tragic, are entirely objective reasons; and have nothing to do with this alleged mood of melancholy reaction. The present system, whether we call it capitalism or anything else, especially as it exists in industrial countries, has already become a danger; and is rapidly becoming a death-trap. The evil is evident in the plainest private experience and in the coldest economic science. To take the practical test first, it is not merely alleged by the enemies of the system, but avowed by the defenders of it. In the Labour disputes of our time, it is not the employees but the employers who declare that business is bad. The successful business man is not pleading success; he is pleading bankruptcy. The case for Capitalists is the case against Capitalism. What is even more extraordinary is that its exponent has to fall back on the rhetoric of Socialism. He merely says that miners or railwaymen must go on working “in the interests of the public.” It will be noted that the capitalists now never use the argument of private property. They confine themselves entirely to this sort of sentimental version of general social responsibility. It is amusing to read the capitalist press on Socialists who sentimentally plead for people who are “failures.” It is now the chief argument of almost every capitalist in every strike that he is himself on the brink of failure.

I have one simple objection to this simple argument in the papers about Strikes and the Socialist peril. My objection is that their argument leads straight to Socialism. In itself it cannot possibly lead to anything else. If workmen are to go on working because they are the servants of the public, there cannot be any deduction except that they ought to be the servants of the public authority. If the Government ought to act in the interests of the public, and there is no more to be said, then obviously the Government ought to take over the whole business, and there is nothing else to be done. I do not think the matter is so simple as this; but they do. I do not think this argument for Socialism is conclusive. But according to the Anti-Socialists the argument for Socialism is quite conclusive. The public alone is to be considered, and the Government can do anything it likes so long as it considers the public. Presumably it can disregard the liberty of the employees and force them to work, possibly in chains. Presumably also it can disregard the property of the employers, and pay the proletariat for them, if necessary out of their trouser-pockets. All these consequences follow from the highly Bolshevist doctrine bawled at us every morning in the capitalist press. That is all they have to say; and if it is the only thing to be said, then the other is the only thing to be done.

In the last paragraph it is noted that if we were left to the logic of the leader-writers on the Socialist peril, they could only lead us straight to Socialism. And as some of us most heartily and vigorously refuse to be led to Socialism, we have long adopted the harder alternative called trying to think things out. And we shall certainly land in Socialism or in something worse called Socialism, or else in mere chaos and ruin, if we make no effort to see the situation as a whole apart from our immediate irritations. Now the capitalist system, good or bad, right or wrong, rests upon two ideas: that the rich will always be rich enough to hire the poor; and the poor will always be poor enough to want to be hired. But it also presumes that each side is bargaining with the other, and that neither is thinking primarily of the public. The owner of an omnibus does not run it for the good of all mankind, despite the universal fraternity blazoned in the Latin name of the vehicle. He runs it to make a profit for himself, and the poorer man consents to drive it in order to get wages for himself. Similarly, the omnibus-conductor is not filled with an abstract altruistic desire for the nice conduct of a crowded omnibus instead of a clouded cane. He does not want to conduct omnibuses because conduct is three-fourths of life. He is bargaining for the biggest wage he can get. Now the case for capitalism was that through this private bargain the public did really get served. And so for some time it did. But the only original case for capitalism collapses entirely, if we have to ask either party to go on for the good of the public. If capitalism cannot pay what will tempt men to work, capitalism is on capitalist principles simply bankrupt. If a tea-merchant cannot pay clerks, and cannot import tea without clerks, then his business is bust and there is an end of it. Nobody in the old capitalist conditions said the clerks were bound to work for less, so that a poor old lady might get a cup of tea.

So it is really the capitalist press that proves on capitalist principles that capitalism has come to an end. If it had not, it would not be necessary for them to make the social and sentimental appeals they do make. It would not be necessary for them to appeal for the intervention of the Government like Socialists. It would not have been necessary for them to plead the discomfort of passengers like sentimentalists or altruists. The truth is that everybody has now abandoned the argument on which the whole of the old capitalism was based: the argument that if men were left to bargain individually the public would benefit automatically. We have to find a new basis of some kind; and the ordinary Conservatives are falling back on the Communist basis without knowing it. Now I respectfully decline to fall back on the Communist basis. But I am sure it is perfectly impossible to continue to fall back on the old Capitalist basis. Those who try to do so tie themselves in quite impossible knots. The most practical and pressing affairs of the hour exhibit the contradiction day after day. For instance, when some great strike or lock-out takes place in a big business like that of the mines, we are always assured that no great saving could be achieved by cutting out private profits, because those private profits are now negligible and the trade in question is not now greatly enriching the few. Whatever be the value of this particular argument, it obviously entirely destroys the general argument. The general argument for capitalism or individualism is that men will not adventure unless there are considerable prizes in the lottery. It is what is familiar in all Socialistic debates as the argument of “the incentive of gain.” But if there is no gain, there is certainly no incentive. If royalty-owners and shareholders only get a little insecure or doubtful profit out of profiteering, it seems as if they might as well fall to the lowly estate of soldiers and servants of society. I have never understood, by the way, why Tory debaters are so very anxious to prove against Socialism that “State servants” must be incompetent and inert. Surely it might be left to others to point out the lethargy of Nelson or the dull routine of Gordon.

But this collapse of industrial individualism, which is not only a collapse but a contradiction (since it has to contradict all its own commonest maxims), is not only an accident of our condition, though it is most marked in our country. Anybody who can think in theories, those highly practical things, will see that sooner or later this paralysis in the system is inevitable. Capitalism is a contradiction; it is even a contradiction in terms. It takes a long time to box the compass, and a still longer time to see that it has done so; but the wheel has come full circle now. Capitalism is contradictory as soon as it is complete; because it is dealing with the mass of men in two opposite ways at once. When most men are wage-earners, it is more and more difficult for most men to be customers. For, the capitalist is always trying to cut down what his servant demands, and in doing so is cutting down what his customer can spend. As soon as his business is in any difficulties, as at present in the coal business, he tries to reduce what he has to spend on wages, and in doing so reduces what others have to spend on coal. He is wanting the same man to be rich and poor at the same time. This contradiction in capitalism does not appear in the earlier stages, because there are still populations not reduced to the common proletarian condition. But as soon as the wealthy as a whole are employing the wage-earners as a whole, this contradiction stares them in the face like an ironic doom and judgment. Employer and employee are simplified and solidified to the relation of Robinson Crusoe and Man Friday. Robinson Crusoe may say he has two problems: the supply of cheap labour and the prospect of trade with the natives. But as he is dealing in these two different ways with the same man, he will get into a muddle. Robinson Crusoe may possibly force Friday to work for nothing but his bare keep, the white man possessing all the weapons. As in the Geddes parallel, he may economize with an Axe. But he cannot cut down Friday’s salary to nothing and then expect Friday to give him gold and silver and orient pearls in return for rum and rifles. Now in proportion as capitalism covers the whole earth, links up large populations, and is ruled by centralized systems, the nearer and nearer approaches this resemblance to the lonely figures on the remote island. If the trade with the natives is really going down, so as to necessitate the wages of the natives also going down, we can only say that the case is rather more tragic if the excuse is true than if it is false. We can only say that Crusoe is now indeed alone, and that Friday is unquestionably unlucky.

I think it very important that people should understand that there is a principle at work behind the industrial troubles of England in our time; and that, whoever be right or wrong in any particular quarrel, it is no particular person or party who is responsible for our commercial experiment being faced with failure. It is a vicious circle into which wage-earning society will finally sink when it begins to lose profits and lower wages; and though some industrial countries are still rich enough to remain ignorant of the strain, it is only because their progress is incomplete; when they reach the goal they will find the riddle. In our own country, which concerns most of us most, we are already falling into that vicious circle of sinking wages and decreasing demand. And as I am going to suggest here, in however sketchy a manner, the line of escape from this slowly closing snare, and because I know some of the things that are commonly said about any such suggestion, I have a reason for reminding the reader of all these things at this stage.

“Safe! Of course it’s not safe! It’s a beggarly chance to cheat the gallows.” Such was the intemperate exclamation of Captain Wicks in the romance of Stevenson; and the same romancer has put a somewhat similar piece of candour into the mouth of Alan Breck Stewart. “But mind you, it’s no small thing! ye maun lie bare and hard... and ye shall sleep with your hand upon your weapons. Aye, man, ye shall taigle many a weary foot or we get clear. I tell ye this at the start, for it’s a life that I ken well. But if ye ask what other chance you have, I answer; Nane.”

And I myself am sometimes tempted to talk in this abrupt manner, after listening to long and thoughtful disquisitions throwing doubt on the detailed perfection of a Distributist State, as compared with the rich happiness and final repose that crowns the present Capitalist and Industrial State. People ask us how we should deal with the unskilled labour at the docks, and what we have to offer to replace the radiant popularity of Lord Devonport and the permanent industrial peace of the Port of London. Those who ask us what we shall do with the docks seldom seem to ask themselves what the docks will do with themselves, if our commerce steadily declines like that of so many commercial cities in the past. Other people ask us how we should deal with workmen holding shares in a business that might possibly go bankrupt. It never occurs to them to answer their own question, in a capitalist state in which business after business is going bankrupt. We have got to deal with the smallest and most remote possibilities of our more simple and static society, while they do not deal with the biggest and most blatant facts about their own complex and collapsing one. They are inquisitive about the details of our scheme, and wish to arrange beforehand a science of casuistry for all the exceptions. But they dare not look their own systems in the face, where ruin has become the rule. Other people wish to know whether a machine would be permitted to exist in this or that position in our Utopia; as an exhibit in a museum, or a toy in the nursery, or a “torture implement of the twentieth century” shown in the Chamber of Horrors. But those who ask us so anxiously how men are to work without machines do not tell us how machines are to work if men do not work them, or how either machines or men are to work if there is no work to do. They are so eager to discover the weak points in our proposal that they have not yet discovered any strong points in their own practice. Strange that our vain and sentimental vision should be so vivid to these realists that they can see its every detail; and that their own reality should be so vague to them that they cannot see it at all; that they cannot see the most obvious and overwhelming fact about it: that it is no longer there.

For it is one of the grim and even grisly jokes of the situation that the very complaint they always make of us is specially and peculiarly true of them. They are always telling us that we think we can bring back the past, or the barbarous simplicity and superstition of the past; apparently under the impression that we want to bring back the ninth century. But they do really think they can bring back the nineteenth century. They are always telling us that this or that tradition has gone for ever, that this or that craft or creed has gone for ever; but they dare not face the fact that their own vulgar and huckstering commerce has gone for ever. They call us reactionaries if we talk of a Revival of Faith or a Revival of Catholicism. But they go on calmly plastering their papers with the headline of a Revival of Trade. What a cry out of the distant past! What a voice from the tomb! They have no reason whatever for believing that there will be a revival of trade, except that their great-grandfathers would have found it impossible to believe in a decline of trade. They have no conceivable ground for supposing that we shall grow richer, except that our ancestors never prepared us for the prospect of growing poorer. Yet it is they who are always blaming us for depending on a sentimental tradition of the wisdom of our ancestors. It is they who are always rejecting social ideals merely because they were the social ideals of some former age. They are always telling us that the mill will never grind again the water that is past; without noticing that their own mills are already idle and grinding nothing at all—like ruined mills in some watery Early Victorian landscape suitable to their watery Early Victorian quotation. They are always telling us that we are fighting against the tide of time, as Mrs. Partington with a mop fought against the tide of the sea. And they cannot even see that time itself has made Mrs. Partington as antiquated a figure as Mother Shipton. They are always telling us that in resisting capitalism and commercialism we are like Canute rebuking the waves; and they do not even know that the England of Cobden is already as dead as the England of Canute. They are always seeking to overwhelm us in the water-floods, to sweep us away upon these weary and washy metaphors of tide and time; for all the world as if they could call back the rivers that have left our cities so far behind, or summon back the seven seas to their allegiance to the trident; or bridle again, with gold for the few and iron for the many, the roaring river of the Clyde.

We may well be tempted to the exclamation of Captain Wicks. We are not choosing between a possible peasantry and a successful commerce. We are choosing between a peasantry that might succeed and a commerce that has already failed. We are not seeking to lure men away from a thriving business to a sort of holiday in Arcadia or the peasant type of Utopia. We are trying to make suggestions about starting anew after a bankrupt business has really gone bankrupt. We can see no possible reason for supposing that English trade will regain its nineteenth-century predominance, except mere Victorian sentimentalism and that particular sort of lying which the newspapers call “optimism.” They taunt us for trying to bring back the conditions of the Middle Ages; as if we were trying to bring back the bows or the body-armour of the Middle Ages. Well, helmets have come back; and body-armour may come back; and bows and arrows will have to come back, a long time before there is any return of that fortunate moment on whose luck they live. It is quite as likely that the long bow will be found through some accident superior to the rifle as that the battleship will be able any longer to rule the waves without reference to the aeroplane. The commercial system implied the security of our commercial routes; and that implied the superiority of our national navy. Everybody who faces facts knows that aviation has altered the whole theory of that naval security. The whole huge horrible problem of a big population on a small island dependent on insecure imports is a problem quite as much for Capitalists and Collectivists as for Distributists. We are not choosing between model villages as part of a serene system of town-planning. We are making a sortie from a besieged city, sword in hand; a sortie from the ruin of Carthage. “Safe! Of course it’s not safe!” said Captain Wicks.

I think it is not unlikely that in any case a simpler social life will return; even if it return by the road of ruin. I think the soul will find simplicity again, if it be in the Dark Ages. But we are Christians and concerned with the body as well as the soul; we are Englishmen and we do not desire, if we can help it, that the English people should be merely the People of the Ruins. And we do most earnestly desire a serious consideration of whether the transition cannot be made in the light of reason and tradition; whether we cannot yet do deliberately and well what nemesis will do wastefully and without pity; whether we cannot build a bridge from these slippery downward slopes to freer and firmer land beyond, without consenting yet that our most noble nation must descend into that valley of humiliation in which nations disappear from history. For this purpose, with great conviction of our principles and with no shame of being open to argument about their application, we have called our companions to council.

\chapter{The Chance of Recovery}
\label{chapter-3}
Once upon a time, or conceivably even more than once, there was a man who went into a public-house and asked for a glass of beer. I will not mention his name, for various and obvious reasons; it may be libel nowadays to say this about a man; or it may lay him open to police prosecution under the more humane laws of our day. So far as this first recorded action is concerned, his name may have been anything: William Shakespeare or Geoffrey Chaucer or Charles Dickens or Henry Fielding, or any of those common names that crop up everywhere in the populace. The important thing about him is that he asked for a glass of beer. The still more important thing about him is that he drank it; and the most important thing of all is that he spat it out again (I regret to say) and threw the pewter mug at the publican. For the beer was abominably bad.

True, he had not yet submitted it to any chemical analysis; but, after he had drank a little of it, he felt an inward, a very inward, persuasion that there was something wrong about it. When he had been ill for a week, steadily getting worse all the time, he took some of the beer to the Public Analyst; and that learned man, after boiling it, freezing it, turning it green, blue, and yellow, and so on, told him that it did indeed contain a vast quantity of deadly poison. “To continue drinking it,” said the man of science thoughtfully, “will undoubtedly be a course attended with risks, but life is inseparable from risk. And before you decide to abandon it, you must make up your mind what Substitute you propose to put into your inside, in place of the beverage which at present (more or less) reposes there. If you will bring me a list of your selections in this difficult matter, I will willingly point out the various scientific objections that can be raised to all of them.”

The man went away, and became more and more ill; and indeed he noticed that nobody else seemed to be really well. As he passed the tavern, his eye chanced to fall upon various friends of his writhing in agony on the ground, and indeed not a few of them lying dead and stiff in heaps about the road. To his simple mind this seemed a matter of some concern to the community; so he hurried to a police court and laid before a magistrate a complaint against the inn. “It would indeed appear,” said the Justice of the Peace, “that the house you mention is one in which people are systematically murdered by means of poison. But before you demand so drastic a course as that of pulling it down or even shutting it up, you have to consider a problem of no little difficulty. Have you considered precisely what building you would Put In Its Place, whether a—-.” At this point I regret to say that the man gave a loud scream and was forcibly removed from the court announcing that he was going mad. Indeed, this conviction of his mental malady increased with his bodily malady; to such an extent that he consulted a distinguished Doctor of Psychology and Psycho-Analysis, who said to him confidentially, “As a matter of diagnosis, there can be no doubt that you are suffering from Bink’s Aberration; but when we come to treatment I may say frankly that it is very difficult to find anything to take the place of that affliction. Have you considered what is the alternative to madness—-?” Whereupon the man sprang up waving his arms and cried, “There is none. There is no alternative to madness. It is inevitable. It is universal. We must make the best of it.”

So making the best of it, he killed the doctor and then went back and killed the magistrate and the public analyst, and is now in an asylum, as happy as the day is long.

In the fable appearing above the case is propounded which is primarily necessary to see at the start of a sketch of social renewal. It concerned a gentleman who was asked what he would substitute for the poison that had been put into his inside, or what constructive scheme he had to put in place of the den of assassins that had poisoned him. A similar demand is made of those of us who regard plutocracy as a poison or the present plutocratic state as something like a den of thieves. In the parable of the poison it is possible that the reader may share some of the impatience of the hero. He will say that nobody would be such a fool as not to get rid of prussic acid or professional criminals, merely because there were differences of opinion about the course of action that would follow getting rid of them. But I would ask the reader to be a little more patient, not only with me but with himself; and ask himself why it is that we act with this promptitude in the case of poison and crime. It is not, even here, really because we are indifferent to the substitute. We should not regard one poison as an antidote to the other poison, if it made the malady worse. We should not set a thief to catch a thief, if it really increased the amount of thieving. The principle upon which we are acting, even if we are acting too quickly to think, or thinking too quickly to define, is nevertheless a principle that we could define. If we merely give a man an emetic after he has taken a poison, it is not because we think he can live on emetics any more than he can live on poisons. It is because we think that after he has first recovered from the poison, and then recovered from the emetic, there will come a time when he himself will think he would like a little ordinary food. That is the starting-point of the whole speculation, so far as we are concerned. If certain impediments are removed, it is not so much a question of what we would do as of what he would do. So if we save the lives of a number of people from the den of poisoners, we do not at that moment ask what they will do with their lives. We assume that they will do something a little more sensible than taking poison. In other words, the very simple first principle upon which all such reforms rest, is that there is some tendency to recovery in every living thing if we remove the pressure of an immediate peril or pain. Now at the beginning of all this rough outline of a social reform, which I propose to trace here, I wish to make clear this general principle of recovery, without which it will be unintelligible. We believe that if things were released they would recover; but we also believe (and this is very important in the practical question) that if things even begin to be released, they will begin to recover. If the man merely leaves off drinking the bad beer, his body will make some effort to recover its ordinary condition. If the man merely escapes from those who are slowly poisoning him, to some extent the very air he breathes will be an antidote to his poison.

As I hope to explain in the essays that follow, I think the question of the real social reform divides itself into two distinct stages and even ideas. One is arresting a race towards mad monopoly that is already going on, reversing that revolution and returning to something that is more or less normal, but by no means ideal; the other is trying to inspire that more normal society with something that is in a real sense ideal, though not necessarily merely Utopian. But the first thing to be understood is that any relief from the present pressure will probably have more moral effect than most of our critics imagine. Hitherto all the triumphs have been triumphs of plutocratic monopoly; all the defeats have been defeats of private property. I venture to guess that one real defeat of a monopoly would have an instant and incalculable effect, far beyond itself, like the first defeats in the field of a military empire like Prussia parading itself as invincible. As each group or family finds again the real experience of private property, it will become a centre of influence, a mission. What we are dealing with is not a question of a General Election to be counted by a calculating machine. It is a question of a popular movement, that never depends on mere numbers.

That is why we have so often taken, merely as a working model, the matter of a peasantry. The point about a peasantry is that it is not a machine, as practically every ideal social state is a machine; that is, a thing that will work only as it is set down to work in the pattern. You make laws for a Utopia; it is only by keeping those laws that it can be kept a Utopia. You do not make laws for a peasantry. You make a peasantry; and the peasants make the laws. I do not mean, as will be clear enough when I come to more detailed matters, that laws must not be used for the establishment of a peasantry or even for the protection of it. But I mean that the character of a peasantry does not depend on laws. The character of a peasantry depends on peasants. Men have remained side by side for centuries in their separate and fairly equal farms, without many of them losing their land, without any of them buying up the bulk of the land. Yet very often there was no law against their buying up the bulk of the land. Peasants could not buy because peasants would not sell. That is, this form of moderate equality, when once it exists, is not merely a legal formula; it is also a moral and psychological fact. People behave when they find themselves in that position as they do when they find themselves at home. That is, they stay there; or at least they behave normally there. There is nothing in abstract logic to prove that people cannot thus feel at home in a Socialist Utopia. But the Socialists who describe Utopias generally feel themselves in some dim way that people will not; and that is why they have to make their mere laws of economic control so elaborate and so clear. They use their army of officials to move men about like crowds of captives, from old quarters to new quarters, and doubtless to better quarters. But we believe that the slaves that we free will fight for us like soldiers.

In other words, all that I ask in this preliminary note is that the reader should understand that we are trying to make something that will run of itself. A machine will not run of itself. A man will run of himself; even if he runs into a good many things that he would have been wiser to avoid. When freed from certain disadvantages, he can to some extent take over the responsibility. All schemes of collective concentration have in them the character of controlling the man even when he is free; if you will, of controlling him to keep him free. They have the idea that the man will not be poisoned if he has a doctor standing behind his chair at dinner-time, to check the mouthfuls and measure the wine. We have the idea that the man may need a doctor when he is poisoned, but no longer needs him when he is unpoisoned. We do not say, as they possibly do say, that he will always be perfectly happy or perfectly good; because there are other elements in life besides the economic; and even the economic is affected by original sin. We do not say that because he does not need a doctor he does not need a priest or a wife or a friend or a God; or that his relations to these things can be ensured by any social scheme. But we do say that there is something which is much more real and much more reliable than any social scheme; and that is a society. There is such a thing as people finding a social life that suits them and enables them to get on reasonably well with each other. You do not have to wait till you have established that sort of society everywhere. It makes all the difference so soon as you have established it anywhere. So if I am told at the start: “You do not think Socialism or reformed Capitalism will save England; do you really think Distributism will save England?” I answer, “No; I think Englishmen will save England, if they begin to have half a chance.”

I am therefore in this sense hopeful; I believe that the breakdown has been a breakdown of machinery and not of men. And I fully agree, as I have just explained, that leaving work for a man is very different from leaving a plan for a machine. I ask the reader to realize this distinction, at this stage of the description, before I go on to describe more definitely some of the possible directions of reform. I am not at all ashamed of being ready to listen to reason; I am not at all afraid of leaving matters open to adjustment; I am not at all annoyed at the prospect of those who carry out these principles varying in many ways in their programmes. I am much too much in earnest to treat my own programme as a party programme; or to pretend that my private bill must become an Act of Parliament without any amendments. But I have a particular cause, in this particular case, for insisting in this chapter that there is a reasonable chance of escape; and for asking that the reasonable chance should be considered with reasonable cheerfulness. I do not care very much for that sort of American virtue which is now sometimes called optimism. It has too much of the flavour of Christian Science to be a comfortable thing for Christians. But I do feel, in the facts of this particular case, that there is a reason for warning people against a too hasty exhibition of pessimism and the pride of impotence. I do ask everybody to consider, in a free and open fashion, whether something of the sort here indicated cannot be carried out, even if it be carried out differently in detail; for it is a matter of the understanding of men. The position is much too serious for men to be anything but cheerful. And in this connection I would venture to utter a warning.

A man has been led by a foolish guide or a self-confident fellow-traveller to the brink of a precipice, which he might well have fallen over in the dark. It may well be said that there is nothing to be done but to sit down and wait for the light. Still, it might be well to pass the hours of darkness in some discussion about how it will be best for them to make their way backwards to more secure ground; and the recollection of any facts and the formulation of any coherent plan of travel will not be waste of time, especially if there is nothing else to do. But there is one piece of advice which we should be inclined to give to the guide who has misguided the simple stranger—especially if he is a really simple stranger, a man perhaps of rude education and elementary emotions. We should strongly advise him not to beguile the time by proving conclusively that it is impossible to go back, that there is no really secure ground behind, that there is no chance of finding the homeward path again, that the steps recently taken are irrevocable, and that progress must go forward and can never return. If he is a tactful man, in spite of his previous error, he will avoid this tone in conversation. If he is not a tactful man, it is not altogether impossible that before the end of the conversation, somebody will go over the precipice after all; and it will not be the simple stranger.

An army has marched across a wilderness, its column, in the military phrase, in the air; under a confident commander who is certain he will pick up new communications which will be far better than the old ones. When the soldiers are almost worn out with marching, and the rank and file of them have suffered horrible privations from hunger and exposure, they find they have only advanced unsupported into a hostile country; and that the signs of military occupation to be seen on every side are only those of an enemy closing round. The march is suddenly halted and the commander addresses his men. There are a great many things that he may say. Some may hold that he had much better say nothing at all. Many may hold that the less he says the better. Others may urge, very truly, that courage is even more needed for a retreat than for an advance. He may be advised to rouse his disappointed men by threatening the enemy with a more dramatic disappointment; by declaring that they will best him yet; that they will dash out of the net even as it is thrown, and that their escape will be far more victorious than his victory. But anyhow there is one kind of speech which the commander will not make to his men, unless he is much more of a fool than his original blunder proves him. He will not say: “We have now taken up a position which may appear to you very depressing; but I assure you it is nothing to the depression which you will certainly suffer as you make a series of inevitably futile attempts to improve it, or to fall back on what you may foolishly regard as a stronger position. I am very much amused at your absurd suggestions for getting back to our old communications; for I never thought much of your mangy old communications anyhow.” There have been mutinies in the desert before now; and it is possible that the general will not be killed in battle with the enemy.

A great nation and civilization has followed for a hundred years or more a form of progress which held itself independent of certain old communications, in the form of ancient traditions about the land, the hearth, or the altar. It has advanced under leaders who were confident, not to say cocksure. They were quite sure that their economic rules were rigid, that their political theory was right, that their commerce was beneficent, that their parliaments were popular, that their press was enlightened, that their science was humane. In this confidence they committed their people to certain new and enormous experiments; to making their own independent nation an eternal debtor to a few rich men; to piling up private property in heaps on the faith of financiers; to covering their land with iron and stone and stripping it of grass and grain; to driving food out of their own country in the hope of buying it back again from the ends of the earth; to loading up their little island with iron and gold till it was weighted like a sinking ship; to letting the rich grow richer and fewer and the poor poorer and more numerous; to letting the whole world be cloven in two with a war of mere masters and mere servants; to losing every type of moderate prosperity and candid patriotism, till there was no independence without luxury and no labour without ugliness; to leaving the millions of mankind dependent on indirect and distant discipline and indirect and distant sustenance, working themselves to death for they knew not whom and taking the means of life from they knew not where; and all hanging on a thread of alien trade which grew thinner and thinner. To the people who have been brought into this position many things may still be said. It will be right to remind them that mere wild revolt will make things worse and not better. It may be true to say that certain complexities must be tolerated for a time because they correspond to other complexities, and the two must be carefully simplified together. But if I may say one word to the princes and rulers of such a people, who have led them into such a pass, I would say to them as seriously as anything was ever said by man to men: “For God’s sake, for our sake, but, above all, for your own sake, do not be in this blind haste to tell them there is no way out of the trap into which your folly has led them; that there is no road except the road by which you have brought them to ruin; that there is no progress except the progress that has ended here. Do not be so eager to prove to your hapless victims that what is hapless is also hopeless. Do not be so anxious to convince them, now that you are at the end of your experiment, that you are also at the end of your resources. Do not be so very eloquent, so very elaborate, so very rational and radiantly convincing in proving that your own error is even more irrevocable and irremediable than it is. Do not try to minimize the industrial disease by showing it is an incurable disease. Do not brighten the dark problem of the coal-pit by proving it is a bottomless pit. Do not tell the people there is no way but this; for many even now will not endure this. Do not say to men that this alone is possible; for many already think it impossible to bear. And at some later time, at some eleventh hour, when the fates have grown darker and the ends have grown clearer, the mass of men may suddenly understand into what a blind alley your progress has led them. Then they may turn on you in the trap. And if they bore all else, they might not bear the final taunt that you can do nothing; that you will not even try to do anything. ‘What art thou, man, and why art thou despairing?’ wrote the poet. ‘God shall forgive thee all but thy despair.’ Man also may forgive you for blundering and may not forgive you for despairing.”

\chapter{On a Sense of Proportion}
\label{chapter-4}
Those of us who study the papers and the parliamentary speeches with proper attention must have by this time a fairly precise idea of the nature of the evil of Socialism. It is a remote Utopian dream impossible of fulfilment and also an overwhelming practical danger that threatens us at every moment. It is only a thing that is as distant as the end of the world and as near as the end of the street. All that is clear enough; but the aspect of it that arrests me at this moment is more especially the Utopian aspect. A person who used to write in the \emph{Daily Mail} paid some attention to this aspect; and represented this social ideal, or indeed almost any other social ideal, as a sort of paradise of poltroons. He suggested that “weaklings” wished to be protected from the strain and stress of our vigorous individualism, and so cried out for this paternal government or grandmotherly legislation. And it was while I was reading his remarks, with a deep and never-failing enjoyment, that the image of the Individualist rose before me; of the sort of man who probably writes such remarks and certainly reads them.

The reader refolds the \emph{Daily Mail} and rises from his intensely individualistic breakfast-table, where he has just dispatched his bold and adventurous breakfast; the bacon cut in rashers from the wild boar which but lately turned to bay in his back garden; the eggs perilously snatched from swaying nest and flapping bird at the top of those toppling trees which gave the house its appropriate name of Pine Crest. He puts on his curious and creative hat, built on some bold plan entirely made up out of his own curious and creative head. He walks outside his unique and unparalleled house, also built with his own well-won wealth according to his own well-conceived architectural design, and seeming by its very outline against the sky to express his own passionate personality. He strides down the street, making his own way over hill and dale towards the place of his own chosen and favourite labour, the workshop of his imaginative craft. He lingers on the way, now to pluck a flower, now to compose a poem, for his time is his own; he is an individual and a free man and not as these Communists. He can work at his own craft when he will, and labour far into the night to make up for an idle morning. Such is the life of the clerk in a world of private enterprise and practical individualism; such the manner of his free passage from his home. He continues to stride lightly along, until he sees afar off the picturesque and striking tower of that workshop in which he will, as with the creative strokes of a god...

He sees it, I say, afar off. The expression is not wholly accidental. For that is exactly the defect in all that sort of journalistic philosophy of individualism and enterprise; that those things are at present even more remote and improbable than communal visions. It is not the dreadful Bolshevist republic that is afar off. It is not the Socialistic State that is Utopian. In that sense, it is not even Utopia that is Utopian. The Socialist State may in one sense be very truly described as terribly and menacingly near. The Socialist State is exceedingly like the Capitalist State, in which the clerk reads and the journalist writes. Utopia is exactly like the present state of affairs, only worse.

It would make no difference to the clerk if his job became a part of a Government department to-morrow. He would be equally civilized and equally uncivic if the distant and shadowy person at the head of the department were a Government official. Indeed, it does make very little difference to him now, whether he or his sons and daughters are employed at the Post Office on bold and revolutionary Socialistic principles or employed at the Stores on wild and adventurous Individualist principles. I never heard of anything resembling civil war between the daughter at the Stores and the daughter in the Post Office. I doubt whether the young lady at the Post Office is so imbued with Bolshevist principles that she would think it a part of the Higher Morality to expropriate something without payment off the counter of the Stores. I doubt whether the young lady at the Stores shudders when she passes a red pillar box, seeing in it an outpost of the Red Peril.

What is really a long way off is this individuality and liberty the \emph{Daily Mail} praised. It is the tower that a man has built for himself that is seen in the distance. It is Private Enterprise that is Utopian, in the sense of something as distant as Utopia. It is Private Property that is for us an ideal and for our critics an impossibility. It is that which can really be discussed almost exactly as the writer in the \emph{Daily Mail} discusses Collectivism. It is that which some people consider a goal and some people a mirage. It is that which its friends maintain to be the final satisfaction of modern hopes and hungers, and its enemies maintain to be a contradiction to common sense and common human possibilities. All the controversialists who have become conscious of the real issue are already saying of our ideal exactly what used to be said of the Socialists’ ideal. They are saying that private property is too ideal not to be impossible. They are saying that private enterprise is too good to be true. They are saying that the idea of ordinary men owning ordinary possessions is against the laws of political economy and requires an alteration in human nature. They are saying that all practical business men know that the thing would never work, exactly as the same obliging people are always prepared to know that State management would never work. For they hold the simple and touching faith that no management except their own could ever work. They call this the law of nature; and they call anybody who ventures to doubt it a weakling. But the point to see is that, although the normal solution of private property for all is even now not very widely realized, in so far as it is realized by the rulers of the modern market (and therefore of the modern world) it is to this normal notion of property that they apply the same criticism as they applied to the abnormal notion of Communism. They say it is Utopian; and they are right. They say it is idealistic; and they are right. They say it is quixotic; and they are right. It deserves every name that will indicate how completely they have driven justice out of the world; every name that will measure how remote from them and their sort is the standard of honourable living; every name that will emphasize and repeat the fact that property and liberty are sundered from them and theirs, by an abyss between heaven and hell.

That is the real issue to be fought out with our serious critics; and I have written here a series of articles dealing more directly with it. It is the question of whether this ideal can be anything but an ideal; not the question of whether it is to be confounded with the present contemptible reality. It is simply the question of whether this good thing is really too good to be true. For the present I will merely say that if the pessimists are convinced of their pessimism, if the sceptics really hold that our social ideal is now banished for ever by mechanical difficulties or materialistic fate, they have at least reached a remarkable and curious conclusion. It is hardly stranger to say that man will have henceforth to be separated from his arms and legs, owing to the improved pattern of wheels, than to say that he must for ever say farewell to two supports so natural as the sense of choosing for himself and of owning something of his own. These critics, whether they figure as critics of Socialism or Distributism, are very fond of talking about extravagant stretches of the imagination or impossible strains upon human nature. I confess I have to stretch and strain my own human imagination and human nature very far, to conceive anything so crooked and uncanny as the human race ending with a complete forgetfulness of the possessive pronoun.

Nevertheless, as we say, it is with these critics we are in controversy. Distribution may be a dream; three acres and a cow may be a joke; cows may be fabulous animals; liberty may be a name; private enterprise may be a wild goose chase on which the world can go no further. But as for the people who talk as if property and private enterprise were the principles now in operation—those people are so blind and deaf and dead to all the realities of their own daily existence, that they can be dismissed from the debate.

In this sense, therefore, we are indeed Utopian; in the sense that our task is possibly more distant and certainly more difficult. We are more revolutionary in the sense that a revolution means a reversal: a reversal of direction, even if it were accompanied with a restraint upon pace. The world we want is much more different from the existing world than the existing world is different from the world of Socialism. Indeed, as has been already noted, there is not much difference between the present world and Socialism; except that we have left out the less important and more ornamental notions of Socialism, such additional fancies as justice, citizenship, the abolition of hunger, and so on. We have already accepted anything that anybody of intelligence ever disliked in Socialism. We have everything that critics used to complain of in the desolate utility and unity of Looking Backward. In so far as the world of Wells or Webb was criticized as a centralized, impersonal, and monotonous civilization, that is an exact description of existing civilization. Nothing has been left out but some idle fancies about feeding the poor or giving rights to the populace. In every other way the unification and regimentation is already complete. Utopia has done its worst. Capitalism has done all that Socialism threatened to do. The clerk has exactly the sort of passive functions and permissive pleasures that he would have in the most monstrous model village. I do not sneer at him; he has many intelligent tastes and domestic virtues in spite of the civilization he enjoys. They are exactly the tastes and virtues he could have as a tenant and servant of the State. But from the moment he wakes up to the moment he goes to sleep again, his life is run in grooves made for him by other people, and often other people he will never even know. He lives in a house that he does not own, that he did not make, that he does not want. He moves everywhere in ruts; he always goes up to his work on rails. He has forgotten what his fathers, the hunters and the pilgrims and the wandering minstrels, meant by finding their way to a place. He thinks in terms of wages; that is, he has forgotten the real meaning of wealth. His highest ambition is concerned with getting this or that subordinate post in a business that is already a bureaucracy. There is a certain amount of competition for that post inside that business; but so there would be inside any bureaucracy. This is a point that the apologists of monopoly often miss. They sometimes plead that even in such a system there may still be a competition among servants; presumably a competition in servility. But so there might be after Nationalization, when they were all Government servants. The whole objection to State Socialism vanishes, if that is an answer to the objection. If every shop were as thoroughly nationalized as a police station, it would not prevent the pleasing virtues of jealousy, intrigue, and selfish ambition from blooming and blossoming among them, as they sometimes do even among policemen.

Anyhow, that world exists; and to challenge that world may be called Utopian; to change that world may be called insanely Utopian. In that sense the name may be applied to me and those who agree with me, and we shall not quarrel with it. But in another sense the name is highly misleading and particularly inappropriate. The word “Utopia” implies not only difficulty of attainment but also other qualities attached to it in such examples as the Utopia of Mr. Wells. And it is essential to explain at once why they do not attach to our Utopia—if it is a Utopia.

There is such a thing as what we should call ideal Distributism; though we should not, in this vale of tears, expect Distributism to be ideal. In the same sense there certainly is such a thing as ideal Communism. But there is no such thing as ideal Capitalism; and there is no such thing as a Capitalist ideal. As we have already noticed (though it has not been noticed often enough), whenever the capitalist does become an idealist, and specially when he does become a sentimentalist, he always talks like a Socialist. He always talks about “social service” and our common interests in the whole community. From this it follows that in so far as such a man is likely to have such a thing as a Utopia, it will be more or less in the style of a Socialist Utopia. The successful financier can put up with an imperfect world, whether or no he has the Christian humility to recognize himself as one of its imperfections. But if he is called upon to conceive a perfect world, it will be something in the way of the pattern state of the Fabians or the I.L.P. He will look for something systematized, something simplified, something all on the same plan. And he will not get it; at least he will not get it from me. It is exactly from that simplification and sameness that I pray to be saved, and should be proud if I could save anybody. It is exactly from that order and unity that I call on the name of Liberty to deliver us.

We do not offer perfection; what we offer is proportion. We wish to correct the proportions of the modern state; but proportion is between varied things; and a proportion is hardly ever a pattern. It is as if we were drawing the picture of a living man and they thought we were drawing a diagram of wheels and rods for the construction of a Robot. We do not propose that in a healthy society all land should be held in the same way; or that all property should be owned on the same conditions; or that all citizens should have the same relation to the city. It is our whole point that the central power needs lesser powers to balance and check it, and that these must be of many kinds: some individual, some communal, some official, and so on. Some of them will probably abuse their privilege; but we prefer the risk to that of the State or of the Trust, which abuses its omnipotence.

For instance, I am sometimes blamed for not believing in my own age, or blamed still more for believing in my own religion. I am called medieval; and some have even traced in me a bias in favour of the Catholic Church to which I belong. But suppose we were to take a parallel from these things. If anyone said that medieval kings or modern peasant countries were to blame for tolerating patches of avowed Bolshevism, we should be rather surprised if we found that the remark really referred to their tolerating monasteries. Yet it is quite true in one sense that monasteries are devoted to Communism and that monks are all Communists. Their economic and ethical life is an exception to a general civilization of feudalism or family life. Yet their privileged position was regarded as rather a prop of social order. They give to certain communal ideas their proper and proportionate place in the State; and something of the same thing was true of the Common Land. We should welcome the chance of allowing any guilds or groups of a communal colour their proper and proportionate place in the State; we should be perfectly willing to mark off some part of the land as Common Land. What we say is that merely nationalizing all the land is like merely making monks of all the people; it is giving those ideals more than their proper and proportionate place in the State. The ordinary meaning of Communism is not that some people are Communists, but that all people are Communists. But we should not say, in the same hard and literal sense, that the meaning of Distributism is that all people are Distributists. We certainly should not say that the meaning of a peasant state is that all people are peasants. We should mean that it had the general character of a peasant state; that the land was largely held in that fashion and the law generally directed in that spirit; that any other institutions stood up as recognizable exceptions, as landmarks on that high tableland of equality.

If this is inconsistent, nothing is consistent; if this is unpractical, all human life in unpractical. If a man wants what he calls a flower-garden he plants flowers where he can, and especially where they will determine the general character of the landscape gardening. But they do not completely cover the garden; they only positively colour it. He does not expect roses to grow in the chimney-pots, or daisies to climb up the railings; still less does he expect tulips to grow on the pine, or the monkey tree to blossom like a rhododendron. But he knows perfectly well what he means by a flower-garden; and so does everybody else. If he does not want a flower-garden but a kitchen-garden, he proceeds differently. But he does not expect a kitchen-garden to be exactly like a kitchen. He does not dig out all the potatoes, because it is not a flower-garden and the potato has a flower. He knows the main thing he is trying to achieve; but, not being a born fool, he does not think he can achieve it everywhere in exactly the same degree, or in a manner equally unmixed with things of another sort. The flower-gardener will not banish nasturtiums to the kitchen-garden because some strange people have been known to eat them. Nor will the other class a vegetable as a flower because it is called a cauliflower. So, from our social garden, we should not necessarily exclude every modern machine any more than we should exclude every medieval monastery. And indeed the apologue is appropriate enough; for this is the sort of elementary human reason that men never lost until they lost their gardens: just as that higher reason that is more than human was lost with a garden long ago.

\setcounter{chapter}{0}\part{Some Aspects of Big Business}
\label{chapter-5}
\chapter{The Bluff of the Big Shops}
\label{chapter-6}
Twice in my life has an editor told me in so many words that he dared not print what I had written, because it would offend the advertisers in his paper. The presence of such pressure exists everywhere in a more silent and subtle form. But I have a great respect for the honesty of this particular editor; for it was, evidently as near to complete honesty as the editor of an important weekly magazine can possibly go. He told the truth about the falsehood he had to tell.

On both those occasions he denied me liberty of expression because I said that the widely advertised stores and large shops were really worse than little shops. That, it may be interesting to note, is one of the things that a man is now forbidden to say; perhaps the only thing he is really forbidden to say. If it had been an attack on Government, it would have been tolerated. If it had been an attack on God, it would have been respectfully and tactfully applauded. If I had been abusing marriage or patriotism or public decency, I should have been heralded in headlines and allowed to sprawl across Sunday newspapers. But the big newspaper is not likely to attack the big shop; being itself a big shop in its way and more and more a monument of monopoly. But it will be well if I repeat here in a book what I found it impossible to repeat in an article. I think the big shop is a bad shop. I think it bad not only in a moral but a mercantile sense; that is, I think shopping there is not only a bad action but a bad bargain. I think the monster emporium is not only vulgar and insolent, but incompetent and uncomfortable; and I deny that its large organization is efficient. Large organization is loose organization. Nay, it would be almost as true to say that organization is always disorganization. The only thing perfectly organic is an organism; like that grotesque and obscure organism called a man. He alone can be quite certain of doing what he wants; beyond him, every extra man may be an extra mistake. As applied to things like shops, the whole thing is an utter fallacy. Some things like armies have to be organized; and therefore do their very best to be well organized. You must have a long rigid line stretched out to guard a frontier; and therefore you stretch it tight. But it is not true that you must have a long rigid line of people trimming hats or tying bouquets, in order that they may be trimmed or tied neatly. The work is much more likely to be neat if it is done by a particular craftsman for a particular customer with particular ribbons and flowers. The person told to trim the hat will never do it quite suitably to the person who wants it trimmed; and the hundredth person told to do it will do it badly; as he does. If we collected all the stories from all the housewives and householders about the big shops sending the wrong goods, smashing the right goods, forgetting to send any sort of goods, we should behold a welter of inefficiency. There are far more blunders in a big shop than ever happen in a small shop, where the individual customer can curse the individual shopkeeper. Confronted with modern efficiency the customer is silent; well aware of that organization’s talent for sacking the wrong man. In short, organization is a necessary evil—which in this case is not necessary.

I have begun these notes with a note on the big shops because they are things near to us and familiar to us all. I need not dwell on other and still more entertaining claims made for the colossal combination of departments. One of the funniest is the statement that it is convenient to get everything in the same shop. That is to stay, it is convenient to walk the length of the street, so long as you walk indoors, or more frequently underground, instead of walking the same distance in the open air from one little shop to another. The truth is that the monopolists’ shops are really very convenient—to the monopolist. They have all the advantage of concentrating business as they concentrate wealth, in fewer and fewer of the citizens. Their wealth sometimes permits them to pay tolerable wages; their wealth also permits them to buy up better businesses and advertise worse goods. But that their own goods are better nobody has ever even begun to show; and most of us know any number of concrete cases where they are definitely worse. Now I expressed this opinion of my own (so shocking to the magazine editor and his advertisers) not only because it is an example of my general thesis that small properties should be revived, but because it is essential to the realization of another and much more curious truth. It concerns the psychology of all these things: of mere size, of mere wealth, of mere advertisement and arrogance. And it gives us the first working model of the way in which things are done to-day and the way in which (please God) they may be undone to-morrow.

There is one obvious and enormous and entirely neglected general fact to be noted before we consider the laws chiefly needed to renew the State. And that is the fact that one considerable revolution could be made without any laws at all. It does not concern any existing law, but rather an existing superstition. And the curious thing is that its upholders boast that it is a superstition. The other day I saw and very thoroughly enjoyed a popular play called \emph{It Pays to Advertise;} which is all about a young business man who tries to break up the soap monopoly of his father, a more old-fashioned business man, by the wildest application of American theories of the psychology of advertising. One thing that struck me as rather interesting about it was this. It was quite good comedy to give the old man and the young man our sympathy in turn. It was quite good farce to make the old man and the young man each alternately look a fool. But nobody seemed to feel what I felt to be the most outstanding and obvious points of folly. They scoffed at the old man because he was old; because he was old-fashioned; because he himself was healthy enough to scoff at the monkey tricks of their mad advertisements. But nobody really criticized him for having made a corner, for which he might once have stood in a pillory. Nobody seemed to have enough instinct for independence and human dignity to be irritated at the idea that one purse-proud old man could prevent us all from having an ordinary human commodity if he chose. And as with the old man, so it was with the young man. He had been taught by his American friend that advertisement can hypnotize the human brain; that people are dragged by a deadly fascination into the doors of a shop as into the mouth of a snake; that the subconscious is captured and the will paralysed by repetition; that we are all made to move like mechanical dolls when a Yankee advertiser says, “Do It Now.” But it never seemed to occur to anybody to resent this. Nobody seemed sufficiently alive to be annoyed. The young man was made game of because he was poor; because he was bankrupt; because he was driven to the shifts of bankruptcy; and so on. But he did not seem to know he was something much worse than a swindler, a sorcerer. He did not know he was by his own boast a mesmerist and a mystagogue; a destroyer of reason and will; an enemy of truth and liberty.

I think such people exaggerate the extent to which it pays to advertise; even if there is only the devil to pay. But in one sense this psychological case for advertising is of great practical importance to any programme of reform. The American advertisers have got hold of the wrong end of the stick; but it is a stick that can be used to beat something else besides their own absurd big drum. It is a stick that can be used also to beat their own absurd business philosophy. They are always telling us that the success of modern commerce depends on creating an atmosphere, on manufacturing a mentality, on assuming a point of view. In short, they insist that their commerce is not merely commercial, or even economic or political, but purely psychological. I hope they will go on saying it; for then some day everybody may suddenly see that it is true.

For the success of big shops and such things really is psychology; not to say psycho-analysis; or, in other words, nightmare. It is not real and, therefore, not reliable. This point concerns merely our immediate attitude, at the moment and on the spot, towards the whole plutocratic occupation of which such publicity is the gaudy banner. The very first thing to do, before we come to any of our proposals that are political and legal, is something that really is (to use their beloved word) entirely psychological. The very first thing to do is to tell these American poker-players that they do not know how to play poker. For they not only bluff, but they boast that they are bluffing. In so far as it really is a question of an instant psychological method, there must be, and there is, an immediate psychological answer. In other words, because they are admittedly bluffing, we can call their bluff.

I said recently that any practical programme for restoring normal property consists of two parts, which current cant would call destructive and constructive; but which might more truly be called defensive and offensive. The first is stopping the mere mad stampede towards monopoly, before the last traditions of property and liberty are lost. It is with that preliminary problem of resisting the world’s trend towards being more monopolist, that I am first of all dealing here. Now, when we ask what we can do, here and now, against the actual growth of monopoly, we are always given a very simple answer. We are told that we can do nothing. By a natural and inevitable operation the large things are swallowing the small, as large fish might swallow little fish. The trust can absorb what it likes, like a dragon devouring what it likes, because it is already the largest creature left alive in the land. Some people are so finally resolved to accept this result that they actually condescend to regret it. They are so convinced that it is fate that they will even admit that it is fatality. The fatalists almost become sentimentalists when looking at the little shop that is being bought up by the big company. They are ready to weep, so long as it is admitted that they weep because they weep in vain. They are willing to admit that the loss of a little toy-shop of their childhood, or a little tea-shop of their youth, is even in the true sense a tragedy. For a tragedy means always a man’s struggle with that which is stronger than man. And it is the feet of the gods themselves that are here trampling on our traditions; it is death and doom themselves that have broken our little toys like sticks; for against the stars of destiny none shall prevail. It is amazing what a little bluff will do in this world.

For they go on saying that the big fish eats the little fish, without asking whether little fish swim up to big fish and ask to be eaten. They accept the devouring dragon without wondering whether a fashionable crowd of princesses ran after the dragon to be devoured. They have never heard of a fashion; and do not know the difference between fashion and fate. The necessitarians have here carefully chosen the one example of something that is certainly not necessary, whatever else is necessary. They have chosen the one thing that does happen still to be free, as a proof of the unbreakable chains in which all things are bound. Very little is left free in the modern world; but private buying and selling are still supposed to be free; and indeed still are free; if anyone has a will free enough to use his freedom. Children may be driven by force to a particular school. Men may be driven by force away from a public-house. All sorts of people, for all sorts of new and nonsensical reasons, may be driven by force to a prison. But nobody is yet driven by force to a particular shop.

I shall deal later with some practical remedies and reactions against the rush towards rings and corners. But even before we consider these, it is well to have paused a moment on the moral fact which is so elementary and so entirely ignored. Of all things in the world, the rush to the big shops is the thing that could be most easily stopped—by the people who rush there. We do not know what may come later; but they cannot be driven there by bayonets just yet. American business enterprise, which has already used British soldiers for purposes of advertisement, may doubtless in time use British soldiers for purposes of coercion. But we cannot yet be dragooned by guns and sabres into Yankee shops or international stores. The alleged economic attraction, with which I will deal in due course, is quite a different thing: I am merely pointing out that if we came to the conclusion that big shops ought to be boycotted, we could boycott them as easily as we should (I hope) boycott shops selling instruments of torture or poisons for private use in the home. In other words, this first and fundamental question is not a question of necessity but of will. If we chose to make a vow, if we chose to make a league, for dealing only with little local shops and never with large centralized shops, the campaign could be every bit as practical as the Land Campaign in Ireland. It would probably be nearly as successful. It will be said, of course, that people will go to the best shop. I deny it; for Irish boycotters did not take the best offer. I deny that the big shop is the best shop; and I especially deny that people go there because it is the best shop. And if I be asked why, I answer at the end with the unanswerable fact with which I began at the beginning. I know it is not merely a matter of business, for the simple reason that the business men themselves tell me it is merely a matter of bluff. It is they who say that nothing succeeds like a mere appearance of success. It is they who say that publicity influences us without our will or knowledge. It is they who say that “It Pays to Advertise”; that is, to tell people in a bullying way that they must “Do It Now,” when they need not do it at all.

\chapter{A Misunderstanding About Method}
\label{chapter-7}
Before I go any further with this sketch, I find I must pause upon a parenthesis touching the nature of my task, without which the rest of it may be misunderstood. As a matter of fact, without pretending to any official or commercial experience, I am here doing a great deal more than has ever been asked of most of the mere men of letters (if I may call myself for the moment a man of letters) when they confidently conducted social movements or setup social ideals. I will promise that, by the end of these notes, the reader shall know a great deal more about how men might set about making a Distributive State than the readers of Carlyle ever knew about how they should set about finding a Hero King or a Real Superior. I think we can explain how to make a small shop or a small farm a common feature of our society better than Matthew Arnold explained how to make the State the organ of Our Best Self. I think the farm will be marked on some sort of rude map more clearly than the Earthly Paradise on the navigation chart of William Morris; and I think that in comparison with his \emph{News from Nowhere} this might fairly be called \emph{News from Somewhere}. Rousseau and Ruskin were often much more vague and visionary than I am; though Rousseau was even more rigid in abstractions, and Ruskin was sometimes very much excited about particular details. I need not say that I am not comparing myself to these great men; I am only pointing out that even from these, whose minds dominated so much wider a field, and whose position as publicists was much more respected and responsible, nothing was as a matter of fact asked beyond the general principles we are accused of giving. I am merely pointing out that the task has fallen to a very minor poet when these very major prophets were not required to carry out and complete the fulfilment of their own prophecies. It would seem that our fathers did not think it quite so futile to have a clear vision of the goal with or without a detailed map of the road; or to be able to describe a scandal without going on to describe a substitute. Anyhow, for whatever reason, it is quite certain that if I really were great enough to deserve the reproaches of the utilitarians, if I really were as merely idealistic or imaginative as they make me out, if I really did confine myself to describing a direction without exactly measuring a road, to pointing towards home or heaven and telling men to use their own good sense in getting there—if this were really all that I could do, it would be all that men immeasurably greater than I am were ever expected to do; from Plato and Isaiah to Emerson and Tolstoy.

But it is not all that I can do; even though those who did not do it did so much more. I can do something else as well; but I can only do it if it be understood what I am doing. At the same time I am well aware that, in explaining the improvement of so elaborate a society, a man may often find it very difficult to explain exactly what he is doing, until it is done. I have considered and rejected half a dozen ways of approaching the problem, by different roads that all lead to the same truth. I had thought of beginning with the simple example of the peasant; and then I knew that a hundred correspondents would leap upon me, accusing me of trying to turn all of them into peasants. I thought of beginning with describing a decent Distributive State in being, with all its balance of different things; just as the Socialists describe their Utopia in being, with its concentration in one thing. Then I knew a hundred correspondents would call me Utopian; and say it was obvious my scheme could not work, because I could only describe it when it was working. But what they would really mean by my being Utopian, would be this: that until that scheme was working, there was no work to be done. I have finally decided to approach the social solution in this fashion: to point out first that the monopolist momentum is not irresistible; that even here and now much could be done to modify it, much by anybody, almost everything by everybody. Then I would maintain that on the removal of that particular plutocratic pressure, the appetite and appreciation of natural property would revive, like any other natural thing. Then, I say, it will be worth while to propound to people thus returning to sanity, however sporadically, a sane society that could balance property and control machinery. With the description of that ultimate society, with its laws and limitations, I would conclude.

Now that may or may not be a good arrangement or order of ideas; but it is an intelligible one; and I submit with all humility that I have a right to arrange my explanations in that order, and no critic has a right to complain that I do not disarrange them in order to answer questions out of their order. I am willing to write him a whole Encyclopaedia of Distributism if he has the patience to read it; but he must have the patience to read it. It is unreasonable for him to complain that I have not dealt adequately with Zoology, State Provision For, under the letter B; or described the honourable social status of the Guild of the Xylographers while I am still dealing alphabetically with the Guild of Architects. I am willing to be as much of a bore as Euclid; but the critic must not complain that the forty-eighth proposition of the second book is not a part of the \emph{Pons Asinorum.} The ancient Guild of Bridge-Builders will have to build many such bridges.

Now from comments that have come my way, I gather that the suggestions I have already made may not altogether explain their own place and purpose in this scheme. I am merely pointing out that monopoly is not omnipotent even now and here; and that anybody could think, on the spur of the moment, of many ways in which its final triumph can be delayed and perhaps defeated. Suppose a monopolist who is my mortal enemy endeavours to ruin me by preventing me from selling eggs to my neighbours, I can tell him I shall live on my own turnips in my own kitchen-garden. I do not mean to tie myself to turnips; or swear never to touch my own potatoes or beans. I mean the turnips as an example; something to throw at him. Suppose the wicked millionaire in question comes and grins over my garden wall and says, “I perceive by your starved and emaciated appearance that you are in immediate need of a few shillings; but you can’t possibly get them,” I may possibly be stung into retorting, “Yes, I can. I could sell my first edition of \emph{Martin Chuzzlewit.}" I do not necessarily mean that I see myself already in a pauper’s grave unless I can sell \emph{Martin Chuzzlewit;} I do not mean that I have nothing else to suggest except selling \emph{Martin Chuzzlewit;} I do not mean to brag like any common politician that I have nailed my colours to the \emph{Martin Chuzzlewit} policy. I mean to tell the offensive pessimist that I am not at the end of my resources; that I can sell a book or even, if the case grows desperate, write a book. I could do a great many things before I came to definitely anti-social action like robbing a bank or (worse still) working in a bank. I could do a great many things of a great many kinds, and I give an example at the start to suggest that there are many more of them, not that there are no more of them. There are a great many things of a great many kinds in my house, besides the copy of a \emph{Martin Chuzzlewit.} Not many of them are of great value except to me; but some of them are of some value to anybody. For the whole point of a home is that it is a hodge-podge. And mine, at any rate, rises to that austere domestic ideal. The whole point of one’s own house is that it is not only a number of totally different things, which are nevertheless one thing, but it is one in which we still value even the things that we forget. If a man has burnt my house to a heap of ashes, I am none the less justly indignant with him for having burnt everything, because I cannot at first even remember everything he has burnt. And as it is with the household gods, so it is with the whole of that household religion, or what remains of it, to offer resistance to the destructive discipline of industrial capitalism. In a simpler society, I should rush out of the ruins, calling for help on the Commune or the King, and crying out, “Haro! a robber has burnt my house.” I might, of course, rush down the street crying in one passionate breath, “Haro! a robber has burnt my front door of seasoned oak with the usual fittings, fourteen window frames, nine curtains, five and a half carpets, 753 books, of which four were \emph{editions de luxe,} one portrait of my great-grandmother," and so on through all the items; but something would be lost of the fierce and simple feudal cry. And in the same way I could have begun this outline with an inventory of all the alterations I should like to see in the laws, with the object of establishing some economic justice in England. But I doubt whether the reader would have had any better idea of what I was ultimately driving at; and it would not have been the approach by which I propose at present to drive. I shall have occasion later to go into some slight detail about these things; but the cases I give are merely illustrations of my first general thesis: that we are not even at the moment doing everything that could be done to resist the rush of monopoly; and that when people talk as if nothing could now be done, that statement is false at the start; and that all sorts of answers to it will immediately occur to the mind.

Capitalism is breaking up; and in one sense we do not pretend to be sorry it is breaking up. Indeed, we might put our own point pretty correctly by saying that we would help it to break up; but we do not want it merely to break down. But the first fact to realize is precisely that; that it is a choice between its breaking up and its breaking down. It is a choice between its being voluntarily resolved into its real component parts, each taking back its own, and its merely collapsing on our heads in a crash or confusion of all its component parts, which some call communism and some call chaos. The former is the one thing all sensible people should try to procure. The latter is the one thing that all sensible people should try to prevent. That is why they are often classed together.

I have mainly confined myself to answering what I have always found to be the first question, “What are we to do now?” To that I answer, “What we must do now is to stop the other people from doing what they are doing now.” The initiative is with the enemy. It is he who is already doing things, and will have done them long before we can begin to do anything, since he has the money, the machinery, the rather mechanical majority, and other things which we have first to gain and then to use. He has nearly completed a monopolist conquest, but not quite; and he can still be hampered and halted. The world has woken up very late; but that is not our fault. That is the fault of all the fools who told us for twenty years that there could never be any Trusts; and are now telling us, equally wisely, that there can never be anything else.

There are other things I ask the reader to bear in mind. The first is that this outline is only an outline, though one that can hardly avoid some curves and loops. I do not profess to dispose of all the obstacles that might arise in this question, because so many of them would seem to many to be quite a different question. I will give one example of what I mean. What would the critical reader have thought, if at the very beginning of this sketch I had gone off into a long disputation about the Law of Libel? Yet, if I were strictly practical, I should find that one of the most practical obstacles. It is the present ridiculous position that monopoly is not resisted as a social force but can still be resented as a legal imputation. If you try to stop a man cornering milk, the first thing that happens will be a smashing libel action for calling it a corner. It is manifestly mere common sense that if the thing is not a sin it is not a slander. As things stand, there is no punishment for the man who does it; but there is a punishment for the man who discovers it. I do not deal here (though I am quite prepared to deal elsewhere) with all these detailed difficulties which a society as now constituted would raise against such a society as we want to constitute. If it were constituted on the principles I suggest, those details would be dealt with on those principles as they arose. For instance, it would put an end to the nonsense whereby men, who are more powerful than emperors, pretend to be private tradesmen suffering from private malice; it will assert that those who are in practice public men must be criticized as potential public evils. It would destroy the absurdity by which an “important case” is tried by a “special jury”; or, in other words, that any serious issue between rich and poor is tried by the rich. But the reader will see that I cannot here rule out all the ten thousand things that might trip us up; I must assume that a people ready to take the larger risks would also take the smaller ones.

Now this outline is an outline; in other words, it is a design, and anybody who thinks we can have practical things without theoretical designs can go and quarrel with the nearest engineer or architect for drawing thin lines on thin paper. But there is another and more special sense in which my suggestion is an outline; in the sense that it is deliberately drawn as a large limitation within which there are many varieties. I have long been acquainted, and not a little amused, with the sort of practical man who will certainly say that I generalize because there is no practical plan. The truth is that I generalize because there are so many practical plans. I myself know four or five schemes that have been drawn up, more or less drastically, for the diffusion of capital. The most cautious, from a capitalist standpoint, is the gradual extension of profit-sharing. A more stringently democratic form of the same thing is the management of every business (if it cannot be a small business) by a guild or group clubbing their contributions and dividing their results. Some Distributists dislike the idea of the workman having shares only where he has work; they think he would be more independent if his little capital were invested elsewhere; but they all agree that he ought to have the capital to invest. Others continue to call themselves Distributists because they would give every citizen a dividend out of much larger national systems of production. I deliberately draw out my general principles so as to cover as many as possible of these alternative business schemes. But I object to being told that I am covering so many because I know there are none. If I tell a man he is too luxurious and extravagant, and that he ought to economize in something, I am not bound to give him a list of his luxuries. The point is that he will be all the better for cutting down any of his luxuries. And my point is that modern society would be all the better for cutting up property by any of these processes. This does not mean that I have not my own favourite form; personally I prefer the second type of division given in the above list of examples. But my main business is to point out that any reversal of the rush to concentrate property will be an improvement on the present state of things. If I tell a man his house is burning down in Putney, he may thank me even if I do not give him a list of all the vehicles which go to Putney, with the numbers of all the taxicabs and the time-table of all the trams. It is enough that I know there are a great many vehicles for him to choose from, before he is reduced to the proverbial adventure of going to Putney on a pig. It is enough that any one of those vehicles is on the whole less uncomfortable than a house on fire or even a heap of ashes. I admit I might be called unpractical if impenetrable forests and destructive floods lay between here and Putney; it might then be as merely idealistic to praise Putney as to praise Paradise. But I do not admit that I am unpractical because I know there are half a dozen practical ways which are more practical than the present state of things. But it does not follow, in fact, that I do not know how to get to Putney. Here, for instance, are half a dozen things which would help the process of Distributism, apart from those on which I shall have occasion to touch as points of principle. Not all Distributists would agree with all of them; but all would agree that they are in the direction of Distributism. (1) The taxation of contracts so as to discourage the sale of small property to big proprietors and encourage the break-up of big property among small proprietors. (2) Something like the Napoleonic testamentary law and the destruction of primogeniture. (3) The establishment of free law for the poor, so that small property could always be defended against great. (4) The deliberate protection of certain experiments in small property, if necessary by tariffs and even local tariffs. (5) Subsidies to foster the starting of such experiments. (6) A league of voluntary dedication, and any number of other things of the same kind. But I have inserted this chapter here in order to explain that this is a sketch of the first principles of Distributism and not of the last details, about which even Distributists might dispute. In such a statement, examples are given as examples, and not as exact and exhaustive lists of all the cases covered by the rule. If this elementary principle of exposition be not understood I must be content to be called an unpractical person by that sort of practical man. And indeed in his sense there is something in his accusation. Whether or no I am a practical man, I am not what is called a practical politician, which means a professional politician. I can claim no part in the glory of having brought our country to its present promising and hopeful condition. Harder heads than mine have established the present prosperity of coal. Men of action, of a more rugged energy, have brought us to the comfortable condition of living on our capital. I have had no part in the great industrial revolution which has increased the beauties of nature and reconciled the classes of society; nor must the too enthusiastic reader think of thanking me for this more enlightened England, in which the employee is living on a dole from the State and the employer on an overdraft at the Bank.

\chapter{A Case in Point}
\label{chapter-8}
It is as natural to our commercial critics to argue in a circle as to travel on the Inner Circle. It is not mere stupidity, but it is mere habit; and it is not easy either to break into or to escape from that iron ring. When we say things can be done, we commonly mean either that they could be done by the mass of men, or else by the ruler of the State. I gave an example of something that could be done quite easily by the mass; and here I will give an example of something that could be done quite easily by the ruler. But we must be prepared for our critics beginning to argue in a circle and saying that the present populace will never agree or the present ruler act in that way. But this complaint is a confusion. We are answering people who call our ideal impossible in itself. If you do not want it, of course, you will not try to get it; but do not say that because you do not want it, it follows that you could not get it if you did want it. A thing does not become intrinsically impossible merely by a mob not trying to obtain it; nor does a thing cease to be practical politics because no politician is practical enough to do it.

I will start with a small and familiar example. In order to ensure that our huge proletariat should have a holiday, we have a law obliging all employers to shut their shops for half a day once a week. Given the proletarian principle, it is a healthy and necessary thing for a proletarian state; just as the saturnalia is a healthy and necessary thing for a slave state. Given this provision for the proletariat, a practical person will naturally say: “It has other advantages, too; it will be a chance for anybody who chooses to do his own dirty work; for the man who can manage without servants.” That degraded being who actually knows how to do things himself, will have a look in at last. That isolated crank, who can really work for his own living, may possibly have a chance to live. A man does not need to be a Distributist to say this; it is the ordinary and obvious thing that anybody would say. The man who has servants must cease to work his servants. Of course, the man who has no servants to work cannot cease to work them. But the law is actually so constructed that it forces this man also to give a holiday to the servants he has not got. He proclaims a saturnalia that never happens to a crowd of phantom slaves that have never been there. Now there is not a rudiment of reason about this arrangement. In every possible sense, from the immediate material to the abstract and mathematical sense, it is quite mad. We live in days of dangerous division of interests between the employer and the employed. Therefore, even when the two are not divided, but actually united in one person, we must divide them again into two parties. We coerce a man into giving himself something he does not want, because somebody else who does not exist might want it. We warn him that he had better receive a deputation from himself, or he might go on strike against himself. Perhaps he might even become a Bolshevist, and throw a bomb at himself; in which case he would have no other course left to his stern sense of law and order but to read the Riot Act and shoot himself. They call us unpractical; but we have not yet produced such an academic fantasy as this. They sometimes suggest that our regret for the disappearance of the yeoman or the apprentice is a mere matter of sentiment. Sentimental! We have not quite sunk to such sentimentalism as to be sorry for apprentices who never existed at all. We have not quite reached that richness of romantic emotion that we are capable of weeping more copiously for an imaginary grocer’s assistant than for a real grocer. We are not quite so maudlin yet as to see double when we look into our favourite little shop; or to set the little shopkeeper fighting with his own shadow. Let us leave these hard-headed and practical men of business shedding tears over the sorrows of a non-existent office boy, and proceed upon our own wild and erratic path, that at least happens to pass across the land of the living.

Now if so small a change as that were made to-morrow, it would make a difference: a considerable and increasing difference. And if any rash apologist of Big Business tells me that a little thing like that could make very little difference, let him beware. For he is doing the one thing which such apologists commonly avoid above all things: he is contradicting his masters. Among the thousand things of interest, which are lost in the million things of no interest, in the newspaper reports of Parliament and public affairs, there really was one delightful little comedy dealing with this point. Some man of normal sense and popular instincts, who had strayed into Parliament by some mistake or other, actually pointed out this plain fact: that there was no need to protect the proletariat where there was no proletariat to protect; and that the lonely shopkeeper might, therefore, remain in his lonely shop. And the Minister in charge of the matter actually replied, with a ghastly innocence, that it was impossible; for it would be unfair to the big shops. Tears evidently flow freely in such circles, as they did from the rising politician, Lord Lundy; and in this case it was the mere thought of the possible sufferings of the millionaires that moved him. There rose before his imagination Mr. Selfridge in his agony, and the groans of Mr. Woolworth, of the Woolworth Tower, thrilled through the kind hearts to which the cry of the sorrowing rich will never come in vain. But whatever we may think of the sensibility needed to regard the big store-owners as objects of sympathy, at any rate it disposes at a stroke of all the fashionable fatalism that sees something inevitable in their success. It is absurd to tell us that our attack is bound to fail; and then that there would be something quite unscrupulous in its so immediately succeeding. Apparently Big Business must be accepted because it is invulnerable, and spared because it is vulnerable. This big absurd bubble can never conceivably be burst; and it is simply cruel that a little pin-prick of competition can burst it.

I do not know whether the big shops are quite so weak and wobbly as their champion said. But whatever the immediate effect on the big shops, I am sure there would be an immediate effect on the little shops. I am sure that if they could trade on the general holiday, it would not only mean that there would be more trade for them, but that there would be more of them trading. It might mean at last a large class of little shopkeepers; and that is exactly the sort of thing that makes all the political difference, as it does in the case of a large class of little farmers. It is not in the merely mechanical sense a matter of numbers. It is a matter of the presence and pressure of a particular social type. It is not a question merely of how many noses are counted; but in the more real sense whether the noses count. If there were anything that could be called a class of peasants, or a class of small shopkeepers, they would make their presence felt in legislation, even if it were what is called class legislation. And the very existence of that third class would be the end of what is called the class war; in so far as its theory divides all men into employers and employed. I do not mean, of course, that this little legal alteration is the only one I have to propose; I mention it first because it is the most obvious. But I mention it also because it illustrates very clearly what I mean by the two stages: the nature of the negative and positive reform. If little shops began to gain custom and big shops began to lose it, it would mean two things, both indeed preliminary but both practical. It would mean that the mere centripetal rush was slowed down, if not stopped, and might at last change to a centrifugal movement. And it would mean that there were a number of new citizens in the State to whom all the ordinary Socialist or servile arguments were inapplicable. Now when you have got your considerable sprinkling of small proprietors, of men with the psychology and philosophy of small property, then you can begin to talk to them about something more like a just general settlement upon their own lines; something more like a land fit for Christians to live in. You can make them understand, as you cannot make plutocrats or proletarians understand, why the machine must not exist save as the servant of the man, why the things we produce ourselves are precious like our own children, and why we can pay too dearly for the possession of luxury by the loss of liberty. If bodies of men only begin to be detached from the servile settlements, they will begin to form the body of our public opinion. Now there are a large number of other advantages that could be given to the small man, which can be considered in their place. In all of them I presuppose a deliberate policy of favouring the small man. But in the primary example here given we can hardly even say that there is any question of favour. You make a law that slave-owners shall free their slaves for a day: the man who has no slaves is outside the thing entirely; he does not come under it in law, because he does not come into it in logic. He has been deliberately dragged into it; not in order that all slaves shall be free for a day, but in order that all free men shall be slaves for a lifetime. But while some of the expedients are only common justice to small property, and others are deliberate protection of small property, the point at the moment is that it will be worth while at the beginning to create small property though it were only on a small scale. English citizens and yeomen would once more exist; and wherever they exist they count. There are many other ways, which can be briefly described, by which the break-up of property can be encouraged on the legal and legislative side. I shall deal with some of them later, and especially with the real responsibility which Government might reasonably assume in a financial and economic condition which is becoming quite ludicrous. From the standpoint of any sane person, in any other society, the present problem of capitalist concentration is not only a question of law but of criminal law, not to mention criminal lunacy.

Of that monstrous megalomania of the big shops, with their blatant advertisements and stupid standardization, something is said elsewhere. But it may be well to add, in the matter of the small shops, that when once they exist they generally have an organization of their own which is much more self-respecting and much less vulgar. This voluntary organization, as every one knows, is called a Guild; and it is perfectly capable of doing everything that really needs to be done in the way of holidays and popular festivals. Twenty barbers would be quite capable of arranging with each other not to compete with each other on a particular festival or in a particular fashion, It is amusing to note that the same people who say that a Guild is a dead medieval thing that would never work are generally grumbling against the power of a Guild as a living modern thing where it is actually working. In the case of the Guild of the Doctors, for instance, it is made a matter of reproach in the newspapers, that the confederation in question refuses to “make medical discoveries accessible to the general public.” When we consider the wild and unbalanced nonsense that is made accessible to the general public by the public press, perhaps we have some reason to doubt whether our souls and bodies are not at least as safe in the hands of a Guild as they are likely to be in the hands of a Trust. For the moment the main point is that small shops can be governed even if they are not bossed by the Government. Horrible as this may seem to the democratic idealists of the day, they can be governed by themselves.

\chapter{The Tyranny of Trusts}
\label{chapter-9}
We have most of us met in literature, and even in life, a certain sort of old gentleman; he is very often represented by an old clergyman. He is the sort of man who has a horror of Socialists without any very definite idea of what they are. He is the man of whom men say that he means well; by which they mean that he means nothing. But this view is a little unjust to this social type. He is really something more than well-meaning; we might even go so far as to say that he would probably be right-thinking, if he ever thought. His principles would probably be sound enough if they were really applied; it is his practical ignorance that prevents him from knowing the world to which they are applicable. He might really be right, only he has no notion of what is wrong. Those who have sat under this old gentleman know that he is in the habit of softening his stern repudiation of the mysterious Socialists by saying that, of course, it is a Christian duty to use our wealth well, to remember that property is a trust committed to us by Providence for the good of others as well as ourselves, and even (unless the old gentleman is old enough to be a Modernist) that it is just possible that we may some day be asked a question or two about the abuse of such a trust. Now all this is perfectly true, so far as it goes, but it happens to illustrate in a rather curious way the queer and even uncanny innocence of the old gentleman. The very phrase that he uses, when he says that property is a trust committed to us by Providence, is a phrase which takes on, when it is uttered to the world around him, the character of an awful and appalling pun. His pathetic little sentence returns in a hundred howling echoes, repeating it again and again like the laughter of a hundred fiends in hell: “Property is a Trust.”

Now I could not more conveniently sum up what I meant by this first section than by taking this type of the dear old conservative clergyman, and considering the curious way in which he has been first caught napping, and then as it were knocked on the head. The first thing we have had to explain to him is expressed in that horrible pun about the Trust. While he has been crying out against imaginary robbers, whom he calls Socialists, he has been caught and carried away bodily by real robbers, whom he still could not even imagine. For the gangs of gamblers who make the great combines are really gangs of robbers, in the sense that they have far less feeling than anybody else for that individual responsibility for individual gifts of God which the old gentleman very rightly calls a Christian duty. While he has been weaving words in the air about irrelevant ideals, he has been caught in a net woven out of the very opposite words and notions: impersonal, irresponsible, irreligious. The financial forces that surround him are further away than anything else from the domestic idea of ownership with which, to do him justice, he himself began. So that when he still bleats faintly, “Property is a trust,” we shall reply firmly, “A trust is not property.”

And now I come to the really extraordinary thing about the old gentleman. I mean that I come to the queerest fact about the conventional or conservative type in modern English society. And that is the fact that the same society, which began by saying there was no such danger to avoid, now says that the danger cannot possibly be avoided. Our whole capitalist community has taken one huge stride from the extreme of optimism to the extreme of pessimism. They began by saying that there could not be Trusts in this country. They have ended by saying that there cannot be anything else except Trusts in this age. And in the course of calling the same thing impossible on Monday and inevitable on Tuesday, they have saved the life of the great gambler or robber twice over; first by calling him a fabulous monster, and second by calling him an almighty fate. Twelve years ago, when I talked of Trusts, people said: “There are no Trusts in England.” Now, when I say it, the same people say: “But how do you propose that England should escape from the Trusts?” They talk as if the Trusts had always been a part of the British Constitution, not to mention the Solar System. In short, the pun and parable with which I began this article have exactly and ironically come true. The poor old clergyman is now really driven to talk as if a Trust with a big T were something that had been bestowed on him by Providence. He is driven to abandon all that he originally meant by his own curious sort of Christian individualism, and hastily reconcile himself to something that is more like a sort of plutocratic collectivism. He is beginning, in a rather bewildered way, to understand that he must now say that monopoly and not merely private property is a part of the nature of things. The net had been thrown over him while he slept, because he never thought of such a thing as a net; because he would have denied the very possibility of anybody weaving such a net. But now the poor old gentleman has to begin to talk as if he had been born in the net. Perhaps, as I say, he has had a knock on the head; perhaps, as his enemies say, he was always just a little weak in the head. But, anyhow, now that his head is in the noose, or the net, he will often start preaching to us about the impossibility of escaping from nets and nooses that are woven or spun upon the wheel of the fates. In a word, I wish to point out that the old gentleman was much too heedless about getting into the net and is much too hopeless about getting out of it.

In short, I would sum up my general suggestions so far by saying that the chief danger to be avoided now, and the first danger to be considered now, is the danger of supposing the capitalist conquest more complete than it is. If I may use the terms of the Penny Catechism about the two sins against hope, the peril now is no longer the peril of presumption but rather of despair. It is not mere impudence like that of those who told us, without winking an eyelid, that there were no Trusts in England. It is rather mere impotence like that of those who tell us that England must soon be swallowed up in an earthquake called America. Now this sort of surrender to modern monopoly is not only ignoble, it is also panic-stricken and premature. It is not true that we can do nothing. What I have written so far has been directed to showing the doubtful and the terrified that it is not true that we can do nothing. Even now there is something that can be done, and done at once; though the things so to be done may appear to be of different kinds and even of degrees of effectiveness. Even if we only save a shop in our own street or stop a conspiracy in our own trade, or get a Bill to punish such conspiracies pressed by our own member, we may come in the nick of time and make all the difference.

To vary the metaphor to a military one, what has happened is that the monopolists have attempted an encircling movement. But the encircling movement is not yet complete. Unless we do something it will be complete; but it is not true to say that we can do nothing to prevent it being completed. We are in favour of striking out, of making sorties or sallies, of trying to pierce certain points in the line (far enough apart and chosen for their weakness), of breaking through the gap in the uncompleted circle. Most people around us are for surrender to the surprise; precisely because it was to them so complete a surprise. Yesterday they denied that the enemy could encircle. The day before yesterday they denied that the enemy could exist. They are paralysed as by a prodigy. But just as we never agreed that the thing was impossible, so we do not now agree that it is irresistible. Action ought to have been taken long ago; but action can still be taken now. That is why it is worth while to dwell on the diverse expedients already given as examples. A chain is as strong as its weakest link; a battle-line is as strong as its weakest man; an encircling movement is as strong as its weakest point, the point at which the circle may still be broken. Thus, to begin with, if anybody asks me in this matter, “What am I to do now?” I answer, “Do anything, however small, that will prevent the completion of the work of capitalist combination. Do anything that will even delay that completion. Save one shop out of a hundred shops. Save one croft out of a hundred crofts. Keep open one door out of a hundred doors; for so long as one door is open, we are not in prison. Throw up one barricade in their way, and you will soon see whether it is the way the world is going. Put one spoke in their wheel, and you will soon see whether it is the wheel of fate.” For it is of the essence of their enormous and unnatural effort that a small failure is as big as a big failure. The modern commercial combine has a great many points in common with a big balloon. It is swollen and yet it is swollen with levity; it climbs and yet it drifts; above all, it is full of gas, and generally of poison gas. But the resemblance most relevant here is that the smallest prick will shrivel the biggest balloon. If this tendency of our time received anything like a reasonably definite check, I believe the whole tendency would soon begin to weaken in its preposterous prestige. Until monopoly is monopolist it is nothing. Until the combine can combine everything, it is nothing. Ahab has not his kingdom so long as Naboth has his vineyard. Haman will not be happy in the palace while Mordecai is sitting in the gate. A hundred tales of human history are there to show that tendencies can be turned back, and that one stumbling-block can be the turning-point. The sands of time are simply dotted with single stakes that have thus marked the turn of the tide. The first step towards ultimately winning is to make sure that the enemy does not win, if it be only that he does not win everywhere. Then, when we have halted his rush, and perhaps fought it to a standstill, we may begin a general counter-attack. The nature of that counter-attack I shall next proceed to consider. In other words, I will try to explain to the old clergyman caught in the net (whose sufferings are ever before my eyes) what it will no doubt comfort him to know: that he was wrong from the first in thinking there could be no net; that he is wrong now in thinking there is no escape from the net; and that he will never know how wrong he was till he finds he has a net of his own, and is once more a fisher of men.

I began by enunciating the paradox that one way of supporting small shops would be to support them. Everybody could do it, but nobody can imagine it being done. In one sense nothing is so simple, and in another nothing is so hard. I went on to point out that without any sweeping change at all, the mere modification of existing laws would probably call thousands of little shops into life and activity. I may have occasion to return to the little shops at greater length; but for the moment I am only running rapidly through certain separate examples, to show that the citadel of plutocracy could even now be attacked from many different sides. It could be met by a concerted effort in the open field of competition. It could be checked by the creation or even correction of a large number of little laws. Thirdly, it could be attacked by the more sweeping operation of larger laws. But when we come to these, even at this stage, we also come into collision with larger questions.

The common sense of Christendom, for ages on end, has assumed that it was as possible to punish cornering as to punish coining. Yet to most readers to-day there seems a sort of vital contradiction, echoed in the verbal contradiction of saying, “Put not your trust in Trusts.” Yet to our fathers this would not seem even so much of a paradox as saying, “Put not your trust in princes,” but rather like saying, “Put not your trust in pirates.” But in applying this to modern conditions, we are checked first by a very modern sophistry.

When we say that a corner should be treated as a conspiracy, we are always told that the conspiracy is too elaborate to be unravelled. In other words, we are told that the conspirators are too conspiratorial to be caught. Now it is exactly at this point that my simple and childlike confidence in the business expert entirely breaks down. My attitude, a moment ago trustful and confiding, becomes disrespectful and frivolous. I am willing to agree that I do not know much about the details of business, but not that nobody could possibly ever come to know anything about them. I am willing to believe that there are people in the world who like to feel that they depend for the bread of life on one particular bounder, who probably began by making large profits on short weight. I am willing to believe that there are people so strangely constituted that they like to see a great nation held up by a small gang, more lawless than brigands but not so brave. In short, I am willing to admit that there may be people who trust in Trusts. I admit it with tears, like those of the benevolent captain in the Bab Ballads who said:

\begin{quotation}\
	“It’s human nature p’raps; if so,

	Oh, isn’t human nature low?"
\end{quotation}

I myself doubt whether it is quite so low as that; but I admit the possibility of this utter lowness; I admit it with weeping and lamentation. But when they tell me it would be impossible to find out whether a man is making a Trust or not—that is quite another thing. My demeanour alters. My spirits revive. When I am told that if cornering were a crime nobody could be convicted of that crime—then I laugh; nay, I jeer.

A murder is usually committed, we may infer, when one gentleman takes a dislike to the appearance of another gentleman in Piccadilly Circus at eleven o’clock in the morning; and steps up to the object of his distaste and dexterously cuts his throat. He then walks across to the kind policeman who is regulating the traffic, and draws his attention to the presence of the corpse on the pavement, consulting him about how to dispose of the encumbrance. That is apparently how these people expect financial crimes to be done, in order to be discovered. Sometimes indeed they are done almost as brazenly, in communities where they can safely be discovered. But the theory of legal impotence looks very extraordinary when we consider the sort of things that the police do discover. Look at the sort of murders they discover. An utterly ordinary and obscure man in some hole-and-corner house or tenement among ten thousand like it, washes his hands in a sink in a back scullery; the operation taking two minutes. The police can discover that, but they could not possibly discover the meeting of men or the sending of messages that turn the whole commercial world upside down. They can track a man that nobody has ever heard of to a place where nobody knew he was going, to do something that he took every possible precaution that nobody should see. But they cannot keep a watch on a man that everybody has heard of, to see whether he communicates with another man that everybody has heard of, in order to do something that nearly everybody knows he is trying all his life to do. They can tell us all about the movements of a man whose own wife or partner or landlady does not profess to know his movements; but they cannot tell when a great combination covering half the earth is on the move. Are the police really so foolish as this; or are they at once so foolish and so wise? Or if the police were as helpless as Sherlock Holmes thought them, what about Sherlock Holmes? What about the ardent amateur detective about whom all of us have read and some of us (alas!) have written. Is there no inspired sleuth to succeed where all the police have failed; and prove conclusively from a greasy spot on the tablecloth that Mr. Rockefeller is interested in oil? Is there no keen-faced man to infer from the late Lord Leverhulme buying up a crowd of soap-businesses that he was interested in soap? I feel inclined to write a new series of detective stories myself, about the discovery of these obscure and cryptic things. They would describe Sherlock Holmes with his monstrous magnifying-glass poring over a paper and making out one of the headlines letter by letter. They would show us Watson standing in amazement at the discovery of the Bank of England. My stories would bear the traditional sort of titles, such as “The Secret of the Skysign” and “The Mystery of the Megaphone” and “The Adventure of the Unnoticed Hoarding.”

What these people really mean is that they cannot imagine cornering being treated like coining. They cannot imagine attempted forestalling, or, indeed, any activity of the rich, coming into the realm of the criminal law at all. It would give them a shock to think of such men subjected to such tests. I will give one obvious example. The science of finger-prints is perpetually paraded before us by the criminologists when they merely want to glorify their not very glorious science. Finger-prints would prove as easily whether a millionaire had used a pen as whether a housebreaker had used a jemmy. They might show as clearly that a financier had used a telephone as that a burglar had used a ladder. But if we began to talk about taking the finger-prints of financiers, everybody would think it was a joke. And so it is: a very grim joke. The laughter that leaps up spontaneously at the suggestion is itself a proof that nobody takes seriously, or thinks of taking seriously, the idea of rich men and poor being equal before the law.

That is the reason why we do not treat Trust magnates and monopolists as they would be treated under the old laws of popular justice. And that is the reason why I take their case at this stage, and in this section of my remarks, along with such apparently light and superficial things as transferring custom from one shop to another. It is because in both cases it is a question wholly and solely of moral will; and not in the least, in any sense, a question of economic law. In other words, it is a lie to say that we cannot make a law to imprison monopolists, or pillory monopolists, or hang monopolists if we choose, as our fathers did before us. And in the same sense it is a lie to say that we cannot help buying the best advertised goods or going to the biggest shop or falling in, in our general social habits, with the general social trend. We could help it in a hundred ways; from the very simple one of walking out of a shop to the more ceremonial one of hanging a man on a gallows. If we mean that we do not want to help it, that may be very true, and even in some cases very right. But arresting a forestaller is as easy as falling off a log or walking out of a shop. Putting the log-roller in prison is no more impossible than walking out of the shop is impossible; and it is highly desirable for the health of this discussion that we should realize the fact from the first. Practically about half of the recognized expedients by which a big business is now made have been marked down as a crime in some community of the past; and could be so marked in a community of the future. I can only refer to them here in the most cursory fashion. One of them is the process against which the statesmen of the most respectable party rave day and night so long as they can pretend that it is only done by foreigners. It is called Dumping. There is a policy of deliberately selling at a loss to destroy another man’s market. Another is: a process against which the same statesmen of the same party actually have attempted to legislate, so long as it was confined to moneylenders. Unfortunately, however, it is not by any means confined to moneylenders. It is the trick of tying a poorer man up in a tangle of all sorts of obligations that he cannot ultimately discharge, except by selling his shop or business. It is done in one form by giving to the desperate things on the instalment plan or on long credit. All these conspiracies I would have tried as we try a conspiracy to overthrow the State or to shoot the King. We do not expect the man to write the King a post-card, telling him he is to be shot, or to give warning in the newspapers of the Day of Revolution. Such plots have always been judged in the only way in which they can be judged: by the use of common sense as to the existence of a purpose and the apparent existence of a plan. But we shall never have a real civic sense until it is once more felt that the plot of three citizens against one citizen is a crime, as well as the plot of one citizen against three. In other words, private property ought to be protected against private crime, just as public order is protected against private judgment. But private property ought to be protected against much bigger things than burglars and pickpockets. It needs protection against the plots of a whole plutocracy. It needs defence against the rich, who are now generally the rulers who ought to defend it. It may not be difficult to explain why they do not defend it. But anyhow, in all these cases, the difficulty is in imagining people wanting to do it; not in imagining people doing it. By all means let people say that they do not think the ideal of the Distributive State is worth the risk or even worth the trouble. But do not let them say that no human being in the past has ever taken any risk; or that no children of Adam are capable of taking any trouble. If they chose to take half as much risk to achieve justice as they have already taken to achieve degradation, if they toiled half as laboriously to make anything beautiful as they toiled to make everything ugly, if they had served their God as they have served their Pork King and their Petrol King, the success of our whole Distributive democracy would stare at the world like one of their flaming sky-signs and scrape the sky like one of their crazy towers.

\setcounter{chapter}{0}\part{Some Aspects of the Land}
\label{chapter-10}
\chapter{The Simple Truth}
\label{chapter-11}
All of us, or at least all those of my generation, heard in our youth an anecdote about George Stephenson, the discoverer of the Locomotive Steam-Engine. It was said that some miserable rustic raised the objection that it would be very awkward if a cow strayed on the railway line, whereupon the inventor replied, “It would be very awkward for the cow.” It is supremely characteristic of his age and school that it never seemed to occur to anybody that it might be rather awkward for the rustic who owned the cow.

Long before we heard that anecdote, however, we had probably heard another and more exciting anecdote called “Jack and the Beanstalk.” That story begins with the strange and startling words, “There once was a poor woman who had a cow.” It would be a wild paradox in modern England to imagine that a poor woman could have a cow; but things seem to have been different in ruder and more superstitious ages. Anyhow, she evidently would not have had a cow long in the sympathetic atmosphere of Stephenson and his steam-engine. The train went forward, the cow was killed in due course; and the state of mind of the old woman was described as the Depression of Agriculture. But everybody was so happy in travelling in trains and making it awkward for cows that nobody noticed that other difficulties remained. When wars or revolutions cut us off from cows, the industrialists discovered that milk does not come originally from cans. On this fact some of us have founded the idea that the cow (and even the miserable rustic) have a use in society, and have been prepared to concede her as much as three acres. But it will be well at this stage to repeat that we do not propose that every acre should be covered with cows; and do not propose to eliminate townspeople as they would eliminate rustics. On many minor points we might have to compromise with conditions, especially at first. But even my ideal, if ever I found it at last, would be what some call a compromise. Only I think it more accurate to call it a balance. For I do not think that the sun compromises with the rain when together they make a garden; or that the rose that grows there is a compromise between green and red. But I mean that even my Utopia would contain different things of different types holding on different tenures: that as in a medieval state there were some peasants, some monasteries, some common land, some private land, some town guilds, and so on, so in my modern state there would be some things nationalized, some machines owned corporately, some guilds sharing common profits, and so on, as well as many absolute individual owners, where such individual owners are most possible. But with these latter it is well to begin, because they are meant to give, and nearly always do give, the standard and tone of the society.

Among the things we have heard a thousand times is the statement that the English are a slow people, a cautious people, a conservative people, and so on. When we have heard a thing as many times as that, we generally either accept it as a truism, or suddenly see that it is quite untrue. And in this case it is quite untrue. The real peculiarity of England is that it is the only country on earth that has not got a conservative class. There are a large number, possibly a majority, of people who call themselves conservative. But the more they are examined, the less conservative they will appear. The commercial class that is in a special sense capitalist is in its nature the very opposite of conservative. By its own profession, it proclaims that it is perpetually using new methods and seeking for new markets. To some of us there seems to be something exceedingly stale about all that novelty. But that is because of the type of mind that is inventing, not because it does not mean to invent. From the biggest financier floating a company to the smallest tout peddling a sewing-machine, the same ideal prevails. It must always be a new company, especially after what has generally happened to the old company. And the sewing-machine must always be a new sort of sewing-machine, even if it is the sort that does not sew. But while this is obvious of the mere capitalist, it is equally true of the pure oligarch. Whatever else an aristocracy is, an aristocracy is never conservative. By its very nature it goes by fashion rather than by tradition. Men living a life of leisure and luxury are always eager for new things; we might fairly say they would be fools if they weren’t. And the English aristocrats are by no means fools. They can proudly claim to have played a great part in every stage of the intellectual progress that has brought us to our present ruin.

The first fact about establishing an English peasantry is that it is establishing, for the first time for many centuries, a traditional class. The absence of such a class will be found to be a very terrible fact, if the tug really becomes between Bolshevism and the historic ideal of property. But the converse is equally true and much more comforting. This difference in the quality means that the change will begin to be effective merely by quantity. I mean that we have not been concerned so much with the strength or weakness of a peasantry, as with presence or absence of a peasantry. As the society has suffered from its mere absence, so the society will begin to change by its mere presence. It will be a somewhat different England in which the peasant has to be considered at all. It will begin to alter the look of things, even when politicians think about peasants as often as they do about doctors. They have been known even to think about soldiers.

The primary case for the peasant is of a stark and almost savage simplicity. A man in England might live on the land, if he did not have rent to pay to the landlord and wages to pay to the labourer. He would therefore be better off, even on a small scale, if he were his own landlord and his own labourer. But there are obviously certain further considerations, and to my mind certain common misconceptions, to which the following notes refer roughly in their order. In the first place, of course, it is one thing to say that this is desirable, and another that it is desired. And in the first place, as will be seen, I do not deny that if it is to be desired, it can hardly be as a mere indulgence is desired; there will undoubtedly be required a certain spirit of effort and sacrifice for the sake of an acute national necessity, if we are to ask any landlord to do without rent or any farmer to do without assistance. But at least there really is a crisis and a necessity; to such an extent that the squire would often be only remitting a debt which he has already written off as a bad debt, and the employer only sacrificing the service of men who are already on strike. Still, we shall need the virtues that belong to a crisis; and it will be well to make the fact clear. Next, while there is all the difference between the desirable and the desired, I would point out that even now this normal life is more desired than many suppose. It is perhaps subconsciously desired; but I think it worth while to throw out a few suggestions that may bring it to the surface. Lastly, there is a misconception about what is meant by “living on the land”–and I have added some suggestions about how much more desirable it is than many suppose.

I shall consider these separate aspects of agricultural distributism more or less in the order in which I have just noted them; but here in the preliminary note I am concerned only with the primary fact. If we could create a peasantry we could create a conservative populace; and he would be a bold man who should undertake to tell us how the present industrial deadlock in the great cities is to produce a conservative populace. I am well aware that many would call the conservatism by coarser names; and say that peasants are stupid and stick-in-the-mud and tied to dull and dreary existence. I know it is said that a man must find it monotonous to do the twenty things that are done on a farm, whereas, of course, he always finds it uproariously funny and festive to do one thing hour after hour and day after day in a factory. I know that the same people also make exactly the contrary comment; and say it is selfish and avaricious for the peasant to be so intensely interested in his own farm, instead of showing, like the proletarians of modern industrialism, a selfless and romantic loyalty to somebody else’s factory, and an ascetic self-sacrifice in making profits for somebody else. Giving each of these claims of modern capitalism their due weight, it is still permissible to say that in so far as the peasant proprietor is certainly tenacious of the peasant property, is concentrated on the interest or content with the dullness, as the case may be, he does, in fact, constitute a solid block of private property which can be counted on to resist Communism; which is not only more than can be said of the proletariat, but is very much more than any capitalists say of them. I do not believe that the proletariat is honeycombed with Bolshevism (if honey be an apt metaphor for that doctrine), but if there is any truth in the newspaper fears on that subject it would certainly seem that large properties cannot prevent the thing happening, whereas small properties can. But, as a matter of fact, all experience is against the assertion that peasants are dreary and degraded savages, crawling about on all fours and eating grass like the beasts of the field. All over the world, for instance, there are peasant dances; and the dances of peasants are like dances of kings and queens. The popular dance is much more stately and ceremonial and full of human dignity than is the aristocratic dance. In many a modern countryside the countryfolk may still be found on high festivals wearing caps like crowns and using gestures like a religious ritual, while the castle or chateau of the lords and ladies is already full of people waddling about like monkeys to the noises made by negroes. All over Europe peasants have produced the embroideries and the handicrafts which were discovered with delight by artists when they had long been neglected by aristocrats. These people are not conservative merely in a negative sense; though there is great value in that which is negative when it is also defensive. They are also conservative in a positive sense; they conserve customs that do not perish like fashions, and crafts less ephemeral than those artistic movements which so very soon cease to move. The Bolshevists, I believe, have invented something which they call Proletarian Art, upon what principle I cannot imagine; save that they seem to have a mysterious pride in calling themselves a proletariat when they claim to be no longer proletarian. I rather think it is merely the reluctance of the half-educated to relinquish the use of a long word. Anyhow, there never has been in this world any such thing as Proletarian Art. But there has most emphatically been such a thing as Peasant Art.

I suppose that what is really meant is Communist Art; and that phrase alone will reveal much. I suppose a truly communal art would consist in a hundred men hanging on to one huge paint-brush like a battering-ram, and steering it across some vast canvas with the curves and lurches and majestic hesitations that would express, in darkly outlined forms, the composite mind of the community. Peasants have produced art because they were communal but not communist. Custom and a corporate tradition gave unity to their art; but each man was a separate artist. It is that satisfaction of the creative instinct in the individual that makes the peasantry as a whole content and therefore conservative. A multitude of men are standing on their own feet, because they are standing on their own land. But in our country, alas, the landowners have been standing upon nothing, except what they have trampled underfoot.

\chapter{Vows and Volunteers}
\label{chapter-12}
We have sometimes been asked why we do not admire advertisers quite so much as they admire themselves. One answer is that it is of their very nature to admire themselves. And it is of the very nature of our task that people must be taught to criticize themselves; or rather (preferably) to kick themselves. They talk about Truth in Advertising; but there cannot be any such thing in the sharp sense in which we need truth in politics. It is impossible to put in the cheery terms of “publicity” either the truth about how bad things are, or the truth about how hard it will be to cure them. No advertiser is so truthful as to say, “Do your best with our rotten old typewriter; we can’t get anything better just now.” But we have really got to say, “Do your best with your rotten old machine of production; don’t let it fall to pieces too suddenly.” We seldom see a gay and conspicuous hoarding inscribed, “You are in for a rough time if you use our new kitchen-range.” But we have really got to say to our friends, “You are in for a rough time if you start new farms on your own; but it is the right thing.” We cannot pretend to be offering merely comforts and conveniences. Whatever our ultimate view of labour-saving machinery, we cannot offer our ideal as a labour-saving machine. There is no more question of comfort than there is for a man in a fire, a battle, or a shipwreck. There is no way out of the danger except the dangerous way.

The sort of call that must be made on the modern English is the sort of call that is made before a great war or a great revolution. If the trumpet give an uncertain sound—but it must be unmistakably the sound of a trumpet. The megaphone of mere mercantile self-satisfaction is merely loud and not in the least clear. In its nature it is saying smooth things, even if it is roaring them; it is like one whispering soft nothings, even if its whisper is a horrible yell. How can advertisement bid men prepare themselves for a battle? How can publicity talk in the language of public spirit? It cannot say, “Buy land at Blinkington-on-Sea and prepare yourself for the battle with stones and thistles.” It cannot give a certain sound, like the old tocsin that rang for fire and flood, and tell the people of Puddleton that they are in danger of famine. To do men justice, no man did announce the needs of Kitchener’s Army like the comforts of the kitchen-range. We did not say to the recruits, “Spend your holiday at Mons.” We did not say, “Try our trenches; they are a treat.” We made some sort of attempt to appeal to better things. We have to make that appeal again; and in the face of worse things. It is this that is made so difficult by the whole tone of advertisement. For the next thing we have to consider is the need of independent individual action on a large scale. We want to make the need known, as the need for recruits was made known. Education was too commercial in origin, and has allowed itself to be largely swamped by commercial advertisement. It came too much from the town; and now it is nearly driven from the town. Education really meant the teaching of town things to country people who did not want to learn them. I suggest that education should now mean the teaching of country things to town people who do want to learn them. I quite admit it would be much better to begin at least with those who really want it. But I also maintain that there are really a great many people in town and country who do really want it.

Whether we look forward to an Agrarian Law or no, whether our notion of distribution is rigid or rough and ready, whether we believe in compensation or confiscation, whether we look for this law or that law, we ought not to sit down and wait for any law at all. While the grass grows the steed has got to show that he wants grass: the steed has got to explain that he is really a graminivorous quadruped. The fulfilment of parliamentary promises grows rather slower than grass; and if nothing is done before the completion of what is called a constitutional process, we shall be about as near to Distributism as a Labour politician is to Socialism. It seems to me first necessary to revive the medieval or moral method, and call for volunteers.

The English could do what the Irish did. They could make laws by obeying them. If we are, like the original Sinn Feiners, to anticipate legal change by social agreement, we want two sorts of volunteers, in order to make the experiment on the spot. We want to find out how many peasants there are, actual or potential, who would take over the responsibility of small farms, for the sake of self-sufficiency, of real property, and of saving England in a desperate hour. We want to know how many landlords there are who would now give or sell cheaply their land to be cut up into a number of such farms. Honestly, I think the landlord would have the best of the bargain. Or rather I think that the peasant would have the hardest and most heroic part of the bargain. Sometimes it would practically pay the landlord to chuck the land altogether, since he is paying out to something that does not pay him back. But in any case, everybody has got to realize that the situation is, in no cant phrases, one for heroic remedies. It is impossible to disguise that the man who gets the land, even more than the man who gives up the land, will have to be something of a hero. We shall be told that heroes do not grow on every hedgerow, that we cannot find enough to defend all our hedges. We raised three million heroes with the blast of a bugle but a few years ago; and the trumpet we hear to-day is in a more terrible sense the trump of doom.

We want a popular appeal for volunteers to save the land; exactly as volunteers in 1914 were wanted to save the country. But we do not want the appeal weakened by that weak-minded, that wearisome, that dismal and deplorable thing that the newspapers call Optimism. We are not asking babies to look pleasant while their photographs are taken; we are asking grown men to meet a crisis as grave as a great war. We are not asking people to cut a coupon out of a newspaper, but to carve a farm out of a trackless waste; and if it is to be successful, it must be faced in something of the stubborn spirit of the old fulfilment of a vow. St. Francis showed his followers the way to a greater happiness; but he did not tell them that a wandering and homeless life would mean Everything as Nice as Mother Makes It; nor did he advertise it on hoardings as a Home From Home. But we live in a time when it is harder for a free man to make a home than it was for a medieval ascetic to do without one.

The quarrel about the Limehouse slums was a working model of the problem—if we can talk of a working model of something that does not work, and something on which only a madman would model anything. The slum-dwellers actually and definitely say that they prefer their slums to the blocks of flats provided as a refuge from the slums. And they prefer them, it is stated, because the old homes had backyards in which they could pursue “their hobbies of bird-fancying and poultry-rearing.” When offered other opportunities on some scheme of allotment, they had the hideous depravity to say that they liked fences round their private yards. So awful and overwhelming is the Red torrent of Communism as it boils through the brains of the working classes.

Now, of course, it might conceivably be necessary, in some wild congestion and convulsion, for people’s houses to be piled on top of each other for ever, in the form of a tower of flats. And so it might be necessary for men to climb on other men’s shoulders in a flood or to get out of a chasm cloven by an earthquake. And it is logically conceivable, and even mathematically correct, that we might thin the crowds in the London streets, if we could thus arrange men vertically instead of horizontally. If there were only some expedient by which a man might walk about with another man standing above him, and another above that, and so on, it would save a great deal of jostling. Men are arranged like that in acrobatic performances; and a course of such acrobatics might be made compulsory in all the schools. It is a picture that pleases me very much, as a picture. I look forward (in spirit of art for art’s sake) to seeing such a living tower moving majestically down the Strand. I like to think of the time of true social organization, when all the clerks of Messrs. Boodle \& Bunkham shall no longer come up in their present random and straggling fashion, each from his little suburban villa. They shall not even, as in the immediate and intermediary stage of the Servile State, march in a well-drilled column from the dormitory in one part of London, to the emporium in the other. No, a nobler vision has arisen before me into the very heights of heaven. A toppling pagoda of clerks, one balanced on the top of another, moves down the street, perhaps making acrobatic patterns in the air as it moves, to illustrate the perfect discipline of its social machinery. All that would be very impressive; and it really would, among other things, economize space. But if one of the men near the top of that swaying tower were to say that he hoped some day to be able to revisit the earth, I should sympathize with his sense of exile. If he were to say that it is natural to man to walk on the earth, I should find myself in agreement with his school of philosophy. If he were to say that it was very difficult to look after chickens in that acrobatic attitude and altitude, I should think his difficulty a real one. At first it might be retorted that bird-fancying would be even more appropriate to such an airy perch, but in practice those birds would be very fancy birds. Finally, if he said that keeping chickens that laid eggs was a worthy and valuable social work, much more worthy and valuable than serving Messrs. Boodle \& Bunkham with the most perfect discipline and organization, I should agree with that sentiment most of all.

Now the whole of our modern problem is very difficult, and though in one way the agricultural part of it is much the simplest, in another way it is by no means the least difficult. But this Limehouse affair is a vivid example of how we make the difficulty more difficult. We are told again and again that the slum-dwellers of the big towns cannot merely be turned loose on the land, that they do not want to go on the land, that they have no tastes or turn of thought that could make them by any process into a people interested in the land, that they cannot be conceived as having any pleasures except town pleasures, or even any discontents except the Bolshevism of the towns. And then when a whole crowd of them want to keep chickens, we force them to live in flats. When a whole crowd of them want to have fences, we laugh and order them off into communal barracks. When a whole population wishes to insist on palings and enclosures and the traditions of private property, the authorities act as if they were suppressing a Red riot. When these very hopeless slum-dwellers do actually set all their hopes on a rural occupation, which they can still practise even in the slums, we tear them away from that occupation and call it improving their condition. You pick a man up who has his head in a hen-coop, forcibly set him on giant stilts a hundred feet high where he cannot reach the ground, and then say you have saved him from misery. And you add that a man like that can only live on stilts and would never be interested in hens.

Now the very first question that is always asked of those advocating our sort of agricultural reconstruction is this question, which is fundamental because it is psychological. Whatever else we may or may not need for a peasantry, we do certainly need peasants. In the present mixture and muddle of more or less urbanized civilization, have we even the first elements or the first possibilities? Have we peasants, or even potential peasants? Like all questions of this sort, it cannot be answered by statistics. Statistics are artificial even when they are not fictitious, for they always assume the very fact which a moral estimate must always deny; they assume that every man is one man. They are based on a sort of atomic theory that the individual is really individual, in the sense of indivisible. But when we are dealing professedly with the proportion of different loves or hates or hopes or hungers, this is so far from being a fact that can be assumed, it is the very first that must be denied. It is denied by all that deeper consideration which wise men used to call spiritual, but which fools were frightened out of calling spiritual, till they ventured to say it in Greek and call it psychical or psychological. In one sense the highest spirituality insists, of course, that one man is one. But in the sense here involved, the spiritual view has always been that one man was at least two, and the psychological view has shown some taste for turning him into half a dozen. It is no good, therefore, to discuss the number of peasants who are nothing else but peasants. Very probably there are none at all. It is no good asking how many complete and compact yeomen or yokels are waiting all ready in smock-frocks or blouses, their spades and hay-forks clutched in their hand, in the neighbourhood of Brompton or Brixton; waiting for us to give the signal to rush back to the land. If anybody is such a fool as to expect that sort of thing, the fool is not to be found in our small political party. When we are dealing with a matter of this kind, we are dealing with different elements in the same class, or even in the same man. We are dealing with elements which should be encouraged or educated or (if we must bring the word in somewhere) evolved. We have to consider whether there are any materials out of which to make peasants to make a peasantry, if we really choose to try. Nowhere in these notes have I suggested that there is the faintest possibility of it being done, if we do not choose to try.

Now, using words in this sensible sense, I should maintain that there is a very large element still in England that would like to return to this simpler sort of England. Some of them understand it better than others, some of them understand themselves better than others; some would be prepared for it as a revolution; some only cling to it very blindly as a tradition; some have never thought of it as anything but a hobby; some have never heard of it and feel it only as a want. But the number of people who would like to get out of the tangle of mere ramifications and communications in the town, and get back nearer to the roots of things, where things are made directly out of nature, I believe to be very large. It is probably not a majority, but I suspect that even now it is a very large minority. A man does not necessarily want this more than everything else at every moment of his life. No sane person expects any movement to consist entirely of such monomaniacs. But a good many people want it a good deal. I have formed that impression from experience, which is of all things the most difficult to reproduce in controversy. I guess it from the way in which numberless suburbans talk about their gardens. I guess it from the sort of things that they really envy in the rich; one of the most notable of which is merely empty space. I notice it in all the element that desires the country, even if it defaces the country. I notice it in the profound popular interest everywhere, especially in England, in the breeding or training of any kind of animal. And if I wanted a supreme, a symbolic, a triumphant example of all that I mean, I could find it in the case I have quoted of these men living in the most miserable slums of Limehouse, and reluctant to leave them because it would mean leaving behind a rabbit in a rabbit-hutch or a chicken in a hen-coop.

Now if we were really doing what I suggest, or if we really knew what we were doing, we should seize on these slum dwellers as if they were infant prodigies or (even more lucrative) monsters to be exhibited in a fair. We should see that such people have a natural genius for such things. We should encourage them in such things. We should educate them in such things. We should see in them the seed and living principle of a real spontaneous revival of the countryside. I repeat that it would be a matter of proportion and therefore of tact. But we should be on their side, being confident that they are on our side and on the side of the countryside. We should reconstruct our popular education so as to help these hobbies. We should think it worth while to teach people the things they are so eager to teach themselves. We should teach them; we might even, in a burst of Christian humility, occasionally allow them to teach us. What we do is to bundle them out of their houses, where they do these things with difficulty, and drag them shrieking to new and unfamiliar places where they cannot do them at all. This example alone would show how much we are really doing for the rural reconstruction of England.

Though much could be done by volunteers, and by a voluntary bargain between the man who really could do the work and the man who frequently cannot get the rent, there is nothing in our social philosophy that forbids the use of the State power where it can be used. And either by the State subsidy or some large voluntary fund, it seems to me that it would still be possible at least to give the other man something as good as the rent that he does not get. In other words, long before our Communists come to the controversial ethics of confiscation, it seems to me within the resources of civilization to enable Brown to buy from Smith what is now of very little value to Smith and might be of very great value to Brown. I know the current complaint against subsidy, and the general argument that applies equally to subscription; but I do think that a subsidy to restore agriculture would find more repayment in the future than a subsidy to patch up the position of coal; just as I think that in its turn more defensible than half a hundred salaries that we pay to a mob of nobodies for plaguing the poor with sham science and petty tyranny. But there are, as I have already hinted, other ways by which even the State could help in the matter. So long as we have State education, it seems a pity that it can never at any moment be determined by the needs of the State. If the immediate need of the State is to pay some attention to the existence of the earth, there really seems no reason why the eyes of the schoolmasters and schoolboys, staring at the stars, should not be turned in the direction of that planet. At present we have education, not indeed for angels, but rather for aviators. They do not even understand a man’s wish to remain tied to the ground. There is in their ideal an insanity that may be truly called unearthly.

Now I suggest such a peasantry of volunteers primarily as a nucleus, but I think it will be a nucleus of attraction. I think it will stand up not only as a rock but as a magnet. In other words, as soon as it is admitted that it can be done, it will become important when a number of other things can no longer be done. When trade is increasingly bad, this will be counted better even by those who count it a second best. When we speak of people leaving the countryside and flocking to the towns, we are not judging the case fairly. Something may be allowed for a social type that would always prefer cinemas and picture post cards even to property and liberty. But there is nothing conclusive in the fact that people prefer to go without property and liberty, with a cinema, to going without property and liberty without a cinema. Some people may like the town so much that they would rather be sweated in the town than free in the country. But nothing is proved by the mere fact that they would rather be sweated in the town than sweated in the country. I believe, therefore, that if we created even a considerable patch of peasantry, the patch would grow. People would fall back on it as they retired from the declining trades. At present the patch is not growing, because there is no patch to grow; people do not even believe in its existence, and can hardly believe in its extension.

So far, I merely propose to suggest that many peasants would now be ready to work alone on the land, though it would be a sacrifice; that many squires would be ready to let them have the land, though it would be a sacrifice; that the State (and for that matter any other patriotic corporation) could be called upon to help either or both in these actions, that it might not be an intolerable or impossible sacrifice. In all this I would remind the reader that I am only dealing with immediately practicable action and not with an ultimate or complete condition; but it seems to me that something of this sort might be set about almost at once. I shall next proceed to consider a misunderstanding about how a group of peasants could live on the land.

\chapter{The Real Life on the Land}
\label{chapter-13}
We offer one among many proposals for undoing the evil of capitalism, on the ground that ours is the only one that really is a proposal for undoing it. The others are all proposals for overdoing it. The natural thing to do with a wrong operation is to reverse it. The natural action, when property has fallen into fewer hands, is to restore it to more numerous hands. If twenty men are fishing in a river in such a crowd that their fishing-lines all get entangled into one, the normal operation is to disentangle them, and sort them out so that each fisherman has his own fishing-line. No doubt a collectivist philosopher standing on the bank might point out that the interwoven lines were now practically a net; and might be trailed along by a common effort so as to drag the river-bed. But apart from his scheme being doubtful in practice, it insults the intellectual instincts even in principle. It is not putting things right to take a doubtful advantage of their being wrong; and it does not even sound like a sane design to exaggerate an accident. Socialism is but the completion of the capitalist concentration; yet that concentration was itself effected blindly like a blunder. Now this naturalness, in the idea of undoing what was ill done would appeal, I think, to many natural people who feel the long-winded sociological schemes to be quite unnatural. For that reason I suggest in this section that many ordinary men, landlords and labourers, Tories and Radicals, would probably help us in this task, if it were separated from party politics and from the pride and pedantry of the intellectuals.

But there is another aspect in which the task is both more easy and more difficult. It is more easy because it need not be crushed by complexities of cosmopolitan trade. It is harder because it is a hard life to live apart from them. A Distributist for whose work (on a little paper defaced, alas, with my own initials) I have a very lively gratitude, once noted a truth often neglected. He said that living on the land was quite a different thing from living by carting things off it. He proved, far more lucidly than I could, how practical is the difference in economics. But I should like to add here a word about a corresponding distinction in ethics. For the former, it is obvious that most arguments about the inevitable failure of a man growing turnips in Sussex are arguments about his failing to sell them, not about his failing to eat them. Now as I have already explained, I do not propose to reduce all citizens to one type, and certainly not to one turnip-eater. In a greater or less degree, as circumstances dictated, there would doubtless be people selling turnips to other people; perhaps even the most ardent turnip-eater would probably sell some turnips to some people. But my meaning will not be clear if it be supposed that no more social simplification is needed than is implied in selling turnips out of a field instead of top-hats out of a shop. It seems to me that a great many people would be only too glad to live on the land, when they find the only alternative is to starve in the street. And it would surely modify the modern enormity of unemployment, if any large number of people were really living on the land, not merely in the sense of sleeping on the land but of feeding on the land. There will be many who maintain that this would mean a very dull life compared with the excitements of dying in a workhouse in Liverpool; just as there are many who insist that the average woman is made to drudge in the home, without asking whether the average man exults in having to drudge in the office. But passing over the fact that we may soon be faced with a problem at least as prosaic as a famine, I do not admit that such a life is necessarily or entirely prosaic. Rustic populations, largely self-supporting, seem to have amused themselves with a great many mythologies and dances and decorative arts; and I am not convinced that the turnip-eater always has a head like a turnip or that the top-hat always covers the brain of a philosopher. But if we look at the problem from the point of view of the community as a whole, we shall note other and not uninteresting things. A system based entirely on the division of labour is in one sense literally half-witted. That is, each performer of half of an operation does really use only half of his wits. It is not a question in the ordinary sense of intellect, and certainly not in the sense of intellectualism. But it is a question of integrity, in the strict sense of the word. The peasant does live, not merely a simple life, but a complete life. It may be very simple in its completeness, but the community is not complete without that completeness. The community is at present very defective because there is not in the core of it any such simple consciousness; any one man who represents the two parties to a contract. Unless there is, there is nowhere a full understanding of those terms: self-support, self-control, self-government. He is the only unanimous mob and the only universal man. He is the one half of the world which does know how the other half lives.

Many must have quoted the stately tag from Virgil which says, “Happy were he who could know the causes of things,” without remembering in what context it comes. Many have probably quoted it because the others have quoted it. Many, if left in ignorance to guess whence it comes, would probably guess wrong. Everybody knows that Virgil, like Homer, ventured to describe boldly enough the most secret councils of the gods. Everybody knows that Virgil, like Dante took his hero into Tartarus and the labyrinth of the last and lowest foundations of the universe. Every one knows that he dealt with the fall of Troy and the rise of Rome, with the laws of an empire fitted to rule all the children of men, with the ideals that should stand like stars before men committed to that awful stewardship. Yet it is in none of these connections, in none of these passages, that he makes the curious remark about human happiness consisting in a knowledge of causes. He says it, I fancy, in a pleasantly didactic poem about the rules for keeping bees. Anyhow, it is part of a series of elegant essays on country pursuits, in one sense, indeed, trivial, but in another sense almost technical. It is in the midst of these quiet and yet busy things that the great poet suddenly breaks out into the great passage, about the happy man whom neither kings nor mobs can cow; who, having beheld the root and reason of all things, can even hear under his feet, unshaken, the roar of the river of hell.

And in saying this, the poet certainly proves once more the two great truths: that a poet is a prophet, and that a prophet is a practical man. Just as his longing for a deliverer of the nations was an unconscious prophecy of Christ, so his criticism of town and country is an unconscious prophecy of the decay that has come on the world through falling away from Christianity. Much may be said about the monstrosity of modern cities; it is easy to see and perhaps a little too easy to say. I have every sympathy with some wild-haired prophet who should lift up his voice in the streets to proclaim the Burden of Brompton in the manner of the Burden of Babylon. I will support (to the extent of sixpence, as Carlyle said) any old man with a beard who will wave his arms and call down fire from heaven upon Bayswater. I quite agree that lions will howl in the high places of Paddington; and I am entirely in favour of jackals and vultures rearing their young in the ruins of the Albert Hall. But in these cases, perhaps, the prophet is less explicit than the poet. He does not tell us exactly what is wrong with the town; but merely leaves it to our own delicate intuitions, to infer from the sudden appearance of wild unicorns trampling down our gardens, or a shower of flaming serpents shooting over our heads through the sky like a flight of arrows, or some such significant detail, that there probably is something wrong. But if we wish in another mood to know intellectually what it is that is wrong with the city, and why it seems to be driving on to dooms quite as unnatural and much more ugly, we shall certainly find it in that profound and piercing irrelevancy of the Latin line.

What is wrong with the man in the modern town is that he does not know the causes of things; and that is why, as the poet says, he can be too much dominated by despots and demagogues. He does not know where things come from; he is the type of the cultivated Cockney who said he liked milk out of a clean shop and not a dirty cow. The more elaborate is the town organization, the more elaborate even is the town education, the less is he the happy man of Virgil who knows the causes of things. The town civilization simply means the number of shops through which the milk does pass from the cow to the man; in other words, it means the number of opportunities of wasting the milk, of watering the milk, of poisoning the milk, and of swindling the man. If ever he protests against being poisoned or swindled, he will certainly be told that it is no good crying over spilt milk; or, in other words, that it is reactionary sentimentalism to attempt to undo what is done or to restore what is perished. But he does not protest very much, because he cannot; and he cannot because he does not know enough about the causes of things—about the primary forms of property and production, or the points where man is nearest to his natural origins.

So far the fundamental fact is clear enough; and by this time this side of the truth is even fairly familiar. A few people are still ignorant enough to talk about the ignorant peasant. But obviously in the essential sense it would be far truer to talk about the ignorant townsman. Even where the townsman is equally well employed, he is not in this sense equally well informed. Indeed, we should see this simple fact clearly enough, if it concerned almost anything except the essentials of our life. If a geologist were tapping with a geological hammer on the bricks of a half-built house, and telling the bricklayers what the clay was and where it came from, we might think him a nuisance; but we should probably think him a learned nuisance. We might prefer the workman’s hammer to the geologist’s hammer; but we should admit that there were some things in the geologist’s head that did not happen to be in the workman’s head. Yet the yokel, or young man from the country, really would know something about the origin of our breakfasts, as does the professor about the origin of our bricks. Should we see a grotesque medieval monster called a pig hung topsy-turvy from a butcher’s hook, like a huge bat from a branch, it will be the young man from the country who will soothe our fears and still our refined shrieks with some account of the harmless habits of this fabulous animal, and by tracing the strange and secret connection between it and the rashers on the breakfast table. If a thunderbolt or meteoric stone fell in front of us in the street, we might have more sympathy with the policeman who wanted to remove it from the thoroughfare than with the professor who wished to stand in the middle of the thoroughfare, lecturing on the constituent elements of the comet or nebula of which it was a flying fragment. But though the policeman might be justified in exclaiming (in the original Greek) “What are the Pleiades to me?” even he would admit that more information about the soil and strata of the Pleiades can be obtained from a professor than from a policeman. So if some strange and swollen monstrosity called a vegetable marrow surprises us like a thunderbolt, let us not imagine that it is so strange to the man who grows marrows as it is to us, merely because his field and work seem to be as far away as the Pleiades. Let us recognize that he is, after all, a specialist on these mysterious marrows and prehistoric pigs; and treat him like a learned man come from a foreign university. England is now such a long way off from London that its emissaries might at least be received with the respect due to distinguished visitors from China or the Cannibal Islands. But, anyhow, we need no longer talk of them as merely ignorant, in talking of the very thing of which we are ignorant ourselves. One man may think the peasant’s knowledge irrelevant, as another may think the professor’s irrelevant; but in both cases it is knowledge; for it is knowledge of the causes of things.

Most of us realize in some sense that this is true; but many of us have not yet realized that the converse is also true. And it is that other truth, when we have understood it, that leads to the next necessary point about the full status of the peasant. And the point is this: that the peasant also will have but a partial experience if he grows things in the country solely in order to sell them to the town. Of course, it is only a joke to represent either the ignorance of town or country as being so grotesque as I have suggested for the sake of example. The townsman does not really think that milk is rained from the clouds or that rashers grow on trees, even when he is a little vague about vegetable marrows. He knows something about it; but not enough to make his advice of much value. The rustic does not really think that milk is used as whitewash or marrows as bolsters, even if he never actually sees them used. But if he is a mere producer and not a consumer of them, his position does become as partial as that of any Cockney clerk; nearly as narrow and even more servile. Given the wonderful romance of the vegetable marrow, it is a bad thing that the peasant should only know the beginning of the story, as it is a bad thing that the clerk should only know the end of it.

I insert here this general suggestion for a particular reason. Before we come to the practical expediency of the peasant who consumes what he produces (and the reason for thinking it, as Mr. Heseltine has urged, much more practicable than the method by which he only sells what he produces), I think it well to point out that this course, while it is more expedient, is not a mere surrender to expediency. It seems to me a very good thing, in theory as well as practice, that there should be a body of citizens primarily concerned in producing and consuming and not in exchanging. It seems to me a part of our ideal, and not merely a part of our compromise, that there should be in the community a sort of core not only of simplicity but of completeness. Exchange and variation can then be given their reasonable place; as they were in the old world of fairs and markets. But there would be somewhere in the centre of civilization a type that was truly independent; in the sense of producing and consuming within its own social circle. I do not say that such a complete human life stands for a complete humanity. I do not say that the State needs only the man who needs nothing from the State. But I do say that this man who supplies his own needs is very much needed. I say it largely because of his absence from modern civilization, that modern civilization has lost unity. It is nobody’s business to note the whole of a process, to see where things come from and where they go to. Nobody follows the whole winding course of the river of milk as it flows from the cow to the baby. Nobody who is in at the death of the pig is responsible for realizing that the proof of the pig is in the eating. Men throw marrows at other men like cannon balls; but they do not return to them like boomerangs. We need a social circle in which things constantly return to those that threw them; and men who know the end and the beginning and the rounding of our little life.

\setcounter{chapter}{0}\part{Some Aspects of Machinery}
\label{chapter-14}
\chapter{The Wheel of Fate}
\label{chapter-15}
The evil we are seeking to destroy clings about in corners especially in the form of catch-phrases by which even the intelligent can easily be caught. One phrase, which we may hear from anybody at any moment, is the phrase that such and such a modern institution has “come to stay.” It is these half-metaphors that tend to make us all half-witted. What is precisely meant by the statement that the steam-engine or the wireless apparatus has come to stay? What is meant, for that matter, even by saying that the Eiffel Tower has come to stay? To begin with, we obviously do not mean what we mean when we use the words naturally; as in the expression, “Uncle Humphrey has come to stay.” That last sentence may be uttered in tones of joy, or of resignation, or even of despair; but not of despair in the sense that Uncle Humphrey is really a monument that can never be moved. Uncle Humphrey did come; and Uncle Humphrey will presumably at some time go; it is even possible (however painful it may be to imagine such domestic relations) that in the last resort he should be made to go. The fact that the figure breaks down, even apart from the reality it is supposed to represent, illustrates how loosely these catch-words are used. But when we say, “The Eiffel Tower has come to stay,” we are still more inaccurate. For, to begin with, the Eiffel Tower has not come at all. There was never a moment when the Eiffel Tower was seen striding towards Paris on its long iron legs across the plains of France, as the giant in the glorious nightmare of Rabelais came to tower over Paris and carry away the bells of Notre-Dame. The figure of Uncle Humphrey seen coming up the road may possibly strike as much terror as any walking tower or towering giant; and the question that may leap into every mind may be the question of whether he has come to stay. But whether or no he has come to stay he has certainly come. He has willed; he has propelled or precipitated his body in a certain direction; he has agitated his own legs; it is even possible (for we all know what Uncle Humphrey is like) that he has insisted on carrying his own portmanteau, to show the lazy young dogs what he can still do at seventy-three.

Now suppose that what had really happened was something like this; something like a weird story of Hawthorne or Poe. Suppose we ourselves had actually manufactured Uncle Humphrey; had put him together, piece by piece, like a mechanical doll. Suppose we had so ardently felt at the moment the need of an uncle in our home life that we had constructed him out of domestic materials, like a Guy for the fifth of November. Taking, it may be, a turnip from the kitchen-garden to represent his bald and venerable head; permitting the water-butt, as it were, to suggest the lines of his figure; stuffing a pair of trousers and attaching a pair of boots, we could produce a complete and convincing uncle of whom any family might be proud. Under those conditions, it might be graceful enough to say, in the merely social sense and as a sort of polite fiction, “Uncle Humphrey has come to stay.” But surely it would be very extraordinary if we afterwards found the dummy relative was nothing but a nuisance, or that his materials were needed for other purposes—surely it would be very extraordinary if we were then forbidden to take him to pieces again; if every effort in that direction were met with the resolute answer, “No, no; Uncle Humphrey has come to stay.” Surely we should be tempted to retort that Uncle Humphrey never came at all. Suppose all the turnips were wanted for the self-support of the peasant home. Suppose the water-butts were wanted; let us hope for the purpose of holding beer. Suppose the male members of the family refused any longer to lend their trousers to an entirely imaginary relative. Surely we should then see through the polite fiction that led us to talk as if the uncle had “come,” had come with an intention, had remained with a purpose, and all the rest. The thing we made did not come, and certainly did not come to do anything, either to stay or to depart.

Now no doubt most people even in the logical city of Paris would say that the Eiffel Tower has come to stay. And no doubt most people in the same city rather more than a hundred years before would have said that the Bastille had come to stay. But it did not stay; it left the neighbourhood quite abruptly. In plain words, the Bastille was something that man had made and, therefore, man could unmake. The Eiffel Tower is something that man has made and man could unmake; though perhaps we may think it practically probable that some time will elapse before man will have the good taste or good sense or even the common sanity to unmake it. But this one little phrase about the thing “coming” is alone enough to indicate something profoundly wrong about the very working of men’s minds on the subject. Obviously a man ought to be saying, “I have made an electric battery. Shall I smash it, or shall I make another?” Instead of that, he seems to be bewitched by a sort of magic and stand staring at the thing as if it were a seven-headed dragon; and he can only say, “The electric battery has come. Has it come to stay?”

Before we begin any talk of the practical problem of machinery, it is necessary to leave off thinking like machines. It is necessary to begin at the beginning and consider the end. Now we do not necessarily wish to destroy every sort of machinery. But we do desire to destroy a certain sort of mentality. And that is precisely the sort of mentality that begins by telling us that nobody can destroy machinery. Those who begin by saying that we cannot abolish the machine, that we must use the machine, are themselves refusing to use the mind.

The aim of human polity is human happiness. For those holding certain beliefs it is conditioned by the hope of a larger happiness, which it must not imperil. But happiness, the making glad of the heart of man, is the secular test and the only realistic test. So far from this test, by the talisman of the heart, being merely sentimental, it is the only test that is in the least practical. There is no law of logic or nature or anything else forcing us to prefer anything else. There is no obligation on us to be richer, or busier, or more efficient, or more productive, or more progressive, or in any way worldlier or wealthier, if it does not make us happier. Mankind has as much right to scrap its machinery and live on the land, if it really likes it better, as any man has to sell his old bicycle and go for a walk, if he likes that better. It is obvious that the walk will be slower; but he has no duty to be fast. And if it can be shown that machinery has come into the world as a curse, there is no reason whatever for our respecting it because it is a marvellous and practical and productive curse. There is no reason why we should not leave all its powers unused, if we have really come to the conclusion that the powers do us harm. The mere fact that we shall be missing a number of interesting things would apply equally to any number of impossible things. Machinery may be a magnificent sight, but not so magnificent as a Great Fire of London; yet we resist that vision and avert our eyes from all that potential splendour. Machinery may not yet be at its best; and perhaps lions and tigers will never be at their best, will never make their most graceful leaps or show all their natural splendours, until we erect an amphitheatre and give them a few live people to eat. Yet that sight also is one which we forbid ourselves, with whatever austere self-denial. We give up so many glorious possibilities, in our stern and strenuous and self-sacrificing preference for having a tolerable time. Happiness, in a sense, is a hard taskmaster. It tells us not to get entangled with many things that are much more superficially attractive than machinery. But, anyhow, it is necessary to clear our minds at the start of any mere vague association or assumption to the effect that we must go by the quickest train or cannot help using the most productive instrument. Granted Mr. Penty’s thesis of the evil of machinery, as something like the evil of black magic, and there is nothing in the least unpractical about Mr. Penty’s proposal that it should simply stop. A process of invention would cease that might have gone further. But its relative imperfection would be nothing compared with the rudimentary state in which we have left such scientific instruments as the rack and the thumbscrew. Those rude implements of torture are clumsy compared with the finished products that modern knowledge of physiology and mechanics might have given us. Many a talented torturer is left in obscurity by the moral prejudices of modern society. Nay, his budding promise is now nipped even in childhood, when he attempts to develop his natural genius on the flies or the tail of the dog. Our own strong sentimental bias against torture represses his noble rage and freezes the genial current of his soul. But we reconcile ourselves to this; though it be undoubtedly the loss of a whole science for which many ingenious persons might have sought out many inventions. If we really conclude that machinery is hostile to happiness, then it is no more inevitable that all ploughing should be done by machinery than it is inevitable that a shop should do a roaring trade on Ludgate Hill by selling the instruments of Chinese tortures.

Let it be clearly understood that I note this only to make the primary problem clear; I am not now saying, nor perhaps should I ever say, that machinery has been proved to be practically poisonous in this degree. I am only stating, in answer to a hundred confused assumptions, the only ultimate aim and test. If we can make men happier, it does not matter if we make them poorer, it does not matter if we make them less productive, it does not matter if we make them less progressive, in the sense of merely changing their life without increasing their liking for it. We of this school of thought may or may not get what we want; but it is at least necessary that we should know what we are trying to get. And those who are called practical men never know what they are trying to get. If machinery does prevent happiness, then it is as futile to tell a man trying to make men happy that he is neglecting the talents of Arkwright, as to tell a man trying to make men humane that he is neglecting the tastes of Nero.

Now it is exactly those who have the clarity to imagine the instant annihilation of machines who will probably have too much common sense to annihilate them instantly. To go mad and smash machinery is a more or less healthy and human malady, as it was in the Luddites. But it was really owing to the ignorance of the Luddites, in a very different sense from that spoken of scornfully by the stupendous ignorance of the Industrial Economists. It was blind revolt as against some ancient and awful dragon, by men too ignorant to know how artificial and even temporary was that particular instrument, or where was the seat of the real tyrants who wielded it. The real answer to the mechanical problem for the present is of a different sort; and I will proceed to suggest it, having once made clear the only methods of judgment by which it can be judged. And having begun at the right end, which is the ultimate spiritual standard by which a man or a machine is to be valued, I will now begin at the other end; I might say at the wrong end; but it will be more respectful to our practical friends to call it the business end.

If I am asked what I should immediately do with a machine, I have no doubt about the sort of practical programme that could be a preliminary to a possible spiritual revolution of a much wider sort. In so far as the machine cannot be shared, I would have the ownership of it shared; that is, the direction of it shared and the profits of it shared. But when I say “shared” I mean it in the modern mercantile sense of the word “shares.” That is, I mean something divided and not merely something pooled. Our business friends bustle forward to tell us that all this is impossible; completely unconscious, apparently, that all this part of the business exists already. You cannot distribute a steam-engine, in the sense of giving one wheel to each shareholder to take home with him, clasped in his arms. But you not only can, but you already do distribute the ownership and profit of the steam-engine; and you distribute it in the form of private property. Only you do not distribute it enough, or to the right people, or to the people who really require it or could really do work for it. Now there are many schemes having this normal and general character; almost any one of which I should prefer to the concentration presented by capitalism or promised by communism. My own preference, on the whole, would be that any such necessary machine should be owned by a small local guild, on principles of profit-sharing, or rather profit-dividing: but of real profit-sharing and real profit-dividing, not to be confounded with capitalist patronage.

Touching the last point, it may be well to say in passing that what I say about the problem of profit-sharing is in that respect parallel to what I say also about the problem of emigration. The real difficulty of starting it in the right way is that it has so often been started in the wrong way; and especially in the wrong spirit. There is a certain amount of prejudice against profit-sharing, just as there is a certain amount of prejudice against emigration, in the industrial democracy of to-day. It is due in both cases to the type and especially the tone of the proposals. I entirely sympathize with the Trade Unionist who dislikes a certain sort of condescending capitalist concession; and the spirit which gives every man a place in the sun which turns out to be a place in Port Sunlight. Similarly, I quite sympathize with Mr. Kirkwood when he resented being lectured about emigration by Sir Alfred Mond, to the extent of saying, “The Scots will leave Scotland when the German Jews leave England.” But I think it would be possible to have a more genuinely egalitarian emigration, with a positive policy of self-government for the poor, to which Mr. Kirkwood might be kind; and I think that profit-sharing that began at the popular end, establishing first the property of a guild and not merely the caprice of an employer, would not contradict any true principle of Trades Unions. For the moment, however, I am only saying that something could be done with what lies nearest to us; quite apart from our general ideal about the position of machinery in an ideal social state. I understand what is meant by saying that the ideal in both cases depends upon the wrong ideals. But I do not understand what our critics mean by saying that it is impossible to divide the shares and profits in a machine among definite individuals. Any healthy man in any historical period would have thought it a project far more practicable than a Milk Trust.

\chapter{The Romance of Machinery}
\label{chapter-16}
I have repeatedly asked the reader to remember that my general view of our potential future divides itself into two parts. First, there is the policy of reversing, or even merely of resisting, the modern tendency to monopoly or the concentration of capital. Let it be noted that this is a policy because it is a direction, if pursued in any degree. In one sense, indeed, he who is not with us is against us; because if that tendency is not resisted, it will prevail. But in another sense anyone who resists it at all is with us; even if he would not go so far in the reversal as we should. In trying to reverse the concentration at all, he is helping us to do what nobody has done yet. He will be setting himself against the trend of his age, or at least of recent ages. And a man can work in our direction, instead of the existing and contrary direction, even with the existing and perhaps contrary machinery. Even while we remain industrial, we can work towards industrial distribution and away from industrial monopoly. Even while we live in town houses, we can own town houses. Even while we are a nation of shopkeepers, we can try to own our shops. Even while we are the workshop of the world, we can try to own our tools. Even if our town is covered with advertisements, it can be covered with different advertisements. If the mark of our whole society is the trade-mark, it need not be the same trade-mark. In short, there is a perfectly tenable and practicable policy of resisting mercantile monopoly even in a mercantile state. And we say that a great many people ought to support us in that, who might not agree with our ultimate ideal of a state that should not be mercantile—or rather a state that should not be entirely mercantile. We cannot call on England as a nation of peasants, as France or Serbia is a nation of peasants. But we can call on England that has been a nation of shopkeepers to resist being turned into one big Yankee store.

That is why in beginning here the discussion of machinery I pointed out, first, that in the ultimate sense we are free to destroy machinery; and second, that in the immediate sense it is possible to divide the ownership of machinery. And I should say myself that even in a healthy state there would be some ownership of machinery to divide. But when we come to consider that larger test, we must say something about the definition of machinery, and even the ideal of machinery. Now I have a great deal of sympathy with what I may call the sentimental argument for machinery. Of all the critics who have rebuked us, the man I like best is the engineer who says: “But I do like machinery—just as you like mythology. Why should I have my toys taken away any more than you?” And of the various positions that I have to meet, I will begin with his. Now on a previous page I said I agreed with Mr. Penty that it would be a human right to abandon machinery altogether. I will add here that I do not agree with Mr. Penty in thinking machinery like magic—a mere malignant power or origin of evils. It seems to me quite as materialistic to be damned by a machine as saved by a machine. It seems to me quite as idolatrous to blaspheme it as to worship it. But even supposing that somebody, without worshipping it, is yet enjoying it imaginatively and in some sense mystically, the case as we state it still stands.

Nobody would be more really unsuitable to the machine age than a man who really admired machines. The modern system presupposes people who will take mechanism mechanically; not people who will take it mystically. An amusing story might be written about a poet who was really appreciative of the fairy-tales of science, and who found himself more of an obstacle in the scientific civilization than if he had delayed it by telling the fairy-tales of infancy. Suppose whenever he went to the telephone (bowing three times as he approached the shrine of the disembodied oracle and murmuring some appropriate form of words such as \emph{vox et praeterea nihil}), he were to act as if he really valued the significance of the instrument. Suppose he were to fall into a trembling ecstasy on hearing from a distant exchange the voice of an unknown young woman in a remote town, were to linger upon the very real wonder of that momentary meeting in mid-air with a human spirit whom he would never see on earth, were to speculate on her life and personality, so real and yet so remote from his own, were to pause to ask a few personal questions about her, just sufficient to accentuate her human strangeness, were to ask whether she also had not some sense of this weird psychical \emph{tête-à-tête,} created and dissolved in an instant, whether she also thought of those unthinkable leagues of valley and forest that lay between the moving mouth and the listening ear—suppose, in short, he were to say all this to the lady at the Exchange who was just about to put him on to 666 Upper Tooting. He would be really and truly expressing the sentiment, “Wonderful thing, the telephone!”; and, unlike the thousands who say it, he would actually mean it. He would be really and truly justifying the great scientific discoveries and doing honour to the great scientific inventors. He would indeed be the worthy son of a scientific age. And yet I fear that in a scientific age he would possibly be misunderstood, and even suffer from lack of sympathy. I fear that he would, in fact, be in practice an opponent of all that he desired to uphold. He would be a worse enemy of machinery than any Luddite smashing machines. He would obstruct the activities of the telephone exchange, by praising the beauties of the telephone, more than if he had sat down, like a more normal and traditional poet, to tell all those bustling business people about the beauties of a wayside flower.

It would of course be the same with any adventure of the same luckless admiration. If a philosopher, when taken for the first time for a ride in a motor-car, were to fall into such an enthusiasm for the marvel that he insisted on understanding the whole of the mechanism on the spot, it is probable that he would have got to his destination rather quicker if he had walked. If he were, in his simple zeal, to insist on the machine being taken to pieces in the road, that he might rejoice in the inmost secrets of its structure, he might even lose his popularity with the garage taxi-driver or chauffeur. Now we have all known children, for instance, who did really in this fashion want to see wheels go round. But though their attitude may bring them nearest to the kingdom of heaven, it does not necessarily bring them nearer to the end of the journey. They are admiring motors; but they are not motoring—that is, they are not necessarily moving. They are not serving that purpose which motoring was meant to serve. Now as a matter of fact this contradiction has ended in a congestion; and a sort of stagnant state of the spirit in which there is rather less real appreciation of the marvels of man’s invention than if the poet confined himself to making a penny whistle (on which to pipe in the woods of Arcady) or the child confined himself to making a toy bow or a catapult. The child really is happy with a beautiful happiness every time he lets fly an arrow. It is by no means certain that the business man is happy with a beautiful happiness every time he sends off a telegram. The very name of a telegram is a poem, even more magical than the arrow; for it means a dart, and a dart that writes. Think what the child would feel if he could shoot a pencil-arrow that drew a picture at the other end of the valley or the long street. Yet the business man but seldom dances and claps his hands for joy, at the thought of this, whenever he sends a telegram.

Now this has a considerable relevancy to the real criticism of the modern mechanical civilization. Its supporters are always telling us of its marvellous inventions and proving that they are marvellous improvements. But it is highly doubtful whether they really feel them as improvements. For instance, I have heard it said a hundred times that glass is an excellent illustration of the way in which something becomes a convenience for everybody. “Look at glass in windows,” they say; “that has been turned into a mere necessity; yet that also was once a luxury.” And I always feel disposed to answer, “Yes, and it would be better for people like you if it were still a luxury; if that would induce you to look at it, and not only to look through it. Do you ever consider how magical a thing is that invisible film standing between you and the birds and the wind? Do you ever think of it as water hung in the air or a flattened diamond too clear to be even valued? Do you ever feel a window as a sudden opening in a wall? And if you do not, what is the good of glass to you?” This may be a little exaggerated, in the heat of the moment, but it is really true that in these things invention outstrips imagination. Humanity has not got the good out of its own inventions; and by making more and more inventions, it is only leaving its own power of happiness further and further behind.

I remarked in an earlier part of this particular meditation that machinery was not necessarily evil, and that there were some who valued it in the right spirit, but that most of those who had to do with it never had a chance of valuing it at all. A poet might enjoy a clock as a child enjoys a musical-box. But the actual clerk who looks at the actual clock, to see that he is just in time to catch the train for the city, is no more enjoying machinery than he is enjoying music. There may be something to be said for mechanical toys; but modern society is a mechanism and not a toy. The child indeed is a good test in these matters; and illustrates both the fact that there is an interest in machinery and the fact that machinery itself generally prevents us from being interested. It is almost a proverb that every little boy wants to be an engine-driver. But machinery has not multiplied the number of engine-drivers, so as to allow all little boys to drive engines. It has not given each little boy a real engine, as his family might give him a toy engine. The effect of railways on a population cannot be to produce a population of engine-drivers. It can only produce a population of passengers; and of passengers a little too like packages. In other words, its only effect on the visionary or potential engine-driver is to put him inside the train, where he cannot see the engine, instead of outside the train where he can. And though he grows up to the greatest and most glorious success in life, and swindles the widow and orphan till he can travel in a first-class carriage specially reserved, with a permanent pass to the International Congress of Cosmopolitan World Peace for Wire-Pullers, he will never perhaps enjoy a railway train again, he will never even see a railway train again, as he saw it when he stood as a ragged urchin and waved wildly from a grassy bank at the passage of the Scotch Express.

We may transfer the parable from engine-drivers to engineers. It may be that the driver of the Scotch Express hurls himself forward in a fury of speed because his heart is in the Highlands, his heart is not here; that he spurns the Border behind him with a gesture and hails the Grampians before him with a cheer. And whether or no it is true that the engine-driver’s heart is in the Highlands, it is sometimes true that the little boy’s heart is in the engine. But it is by no means true that passengers as a whole, travelling behind engines as a whole, enjoy the speed in a positive sense, though they may approve of it in a negative sense. I mean that they wish to travel swiftly, not because swift travelling is enjoyable, but because it is not enjoyable. They want it rushed through; not because being behind the railway-engine is a rapture, but because being in the railway-carriage is a bore. In the same way, if we consider the joy of engineers, we must remember that there is only one joyful engineer to a thousand bored victims of engineering. The discussion that raged between Mr. Penty and others at one time threatened to resolve itself into a feud between engineers and architects. But when the engineer asks us to forget all the monotony and materialism of a mechanical age because his own science has some of the inspiration of an art, the architect may well be ready with a reply. For this is very much as if architects were never engaged in anything but the building of prisons and lunatic asylums. It is as if they told us proudly with what passionate and poetical enthusiasm they had themselves reared towers high enough for the hanging of Haman or dug dungeons impenetrable enough for the starving of Ugolino.

Now I have already explained that I do not propose anything in what some call the practical way, but should rather be called the immediate way, beyond the better distribution of the ownership of such machines as are really found to be necessary. But when we come to the larger question of machinery in a fundamentally different sort of society, governed by our philosophy and religion, there is a great deal more to be said. The best and shortest way of saying it is that instead of the machine being a giant to which the man is a pygmy, we must at least reverse the proportions until man is a giant to whom the machine is a toy. Granted that idea, and we have no reason to deny that it might be a legitimate and enlivening toy. In that sense it would not matter if every child were an engine-driver or (better still) every engine-driver a child. But those who were always taunting us with unpracticality will at least admit that this is not practical.

I have thus tried to put myself fairly in the position of the enthusiast, as we should always do in judging of enthusiasms. And I think it will be agreed that even after the experiment a real difference between the engineering enthusiasm and older enthusiasms remains as a fact of common sense. Admitting that the man who designs a steam-engine is as original as the man who designs a statue, there is an immediate and immense difference in the effects of what they design. The original statue is a joy to the sculptor; but it is also in some degree (when it is not too original) a joy to the people who see the statue. Or at any rate it is meant to be a joy to other people seeing it, or there would be no point in letting it be seen. But though the engine may be a great joy to the engineer and of great use to the other people, it is not, and it is not meant to be, in the same sense a great joy to the other people. Nor is this because of a deficiency in education, as some of the artists might allege in the case of art. It is involved in the very nature of machinery; which, when once it is established, consists of repetitions and not of variations and surprises. A man can see something in the limbs of a statue which he never saw before; they may seem to toss or sweep as they never did before; but he would not only be astonished but alarmed if the wheels of the steam-engine began to behave as they never did before. We may take it, therefore, as an essential and not an accidental character of machinery that it is an inspiration for the inventor but merely a monotony for the consumer.

This being so, it seems to me that in an ideal state engineering would be the exception, just as the delight in engines is the exception. As it is, engineering and engines are the rule; and are even a grinding and oppressive rule. The lifelessness which the machine imposes on the masses is an infinitely bigger and more obvious fact than the individual interest of the man who makes machines. Having reached this point in the argument, we may well compare it with what may be called the practical aspect of the problem of machinery. Now it seems to me obvious that machinery, as it exists to-day, has gone almost as much beyond its practical sphere as it has beyond its imaginative sphere. The whole of industrial society is founded on the notion that the quickest and cheapest thing is to carry coals to Newcastle; even if it be only with the object of afterwards carrying them from Newcastle. It is founded on the idea that rapid and regular transit and transport, perpetual interchange of goods, and incessant communication between remote places, is of all things the most economical and direct. But it is not true that the quickest and cheapest thing, for a man who has just pulled an apple from an apple tree, is to send it in a consignment of apples on a train that goes like a thunderbolt to a market at the other end of England. The quickest and cheapest thing for a man who has pulled a fruit from a tree is to put it in his mouth. He is the supreme economist who wastes no money on railway journeys. He is the absolute type of efficiency who is far too efficient to go in for organization. And though he is, of course, an extreme and ideal case of simplification, the case for simplification does stand as solid as an apple tree. In so far as men can produce their own goods on the spot, they are saving the community a vast expenditure which is often quite out of proportion to the return. In so far as we can establish a considerable proportion of simple and self-supporting people, we are relieving the pressure of what is often a wasteful as well as a harassing process. And taking this as a general outline of the reform, it does appear true that a simpler life in large areas of the community might leave machinery more or less as an exceptional thing; as it may well be to the exceptional man who really puts his soul into it.

There are difficulties in this view; but for the moment I may well take as an illustration the parallel of the particular sort of modern engineering which moderns are very fond of denouncing. They often forget that most of their praise of scientific instruments applies most vividly to scientific weapons. If we are to have so much pity for the unhappy genius who has just invented a new galvanometer, what about the poor genius who has just invented a new gun? If there is a real imaginative inspiration in the making of a steam-engine, is there not imaginative interest in the making of a submarine? Yet many modern admirers of science would be very anxious to abolish these machines altogether; even in the very act of telling us that we cannot abolish machines at all. As I believe in the right of national self-defence, I would not abolish them altogether. But I think they may give us a hint of how exceptional things may be treated exceptionally. For the moment I will leave the progressive to laugh at my absurd notion of a limitation of machines, and go off to a meeting to demand the limitation of armaments.

\chapter{The Holiday of the Slave}
\label{chapter-17}
I have sometimes suggested that industrialism of the American type, with its machinery and mechanical hustle, will some day be preserved on a truly American model; I mean in the manner of the Red Indian Reservation. As we leave a patch of forest for savages to hunt and fish in, so a higher civilization might leave a patch of factories for those who are still at such a stage of intellectual infancy as really to want to see the wheels go round. And as the Red Indians could still, I suppose, tell their quaint old legends of a red god who smoked a pipe or a red hero who stole the sun and moon, so the simple folk in the industrial enclosure could go on talking of their own Outline of History and discussing the evolution of ethics, while all around them a more mature civilization was dealing with real history and serious philosophy. I hesitate to repeat this fancy here; for, after all, machinery is their religion, or at any rate superstition, and they do not like it to be treated with levity. But I do think there is something to be said for the notion of which this fancy might stand as a sort of symbol; for the idea that a wiser society would eventually treat machines as it treats weapons, as something special and dangerous and perhaps more directly under a central control. But however this may be, I do think the wildest fancy of a manufacturer kept at bay like a painted barbarian is much more sane than a serious scientific alternative now often put before us. I mean what its friends call the Leisure State, in which everything is to be done by machinery. It is only right to say a word about this suggestion in comparison with our own.

In practice we already know what is meant by a holiday in a world of machinery and mass production. It means that a man, when he has done turning a handle, has a choice of certain pleasures offered to him. He can, if he likes, read a newspaper and discover with interest how the Crown Prince of Fontarabia landed from the magnificent yacht \emph{Atlantis} amid a cheering crowd; how certain great American millionaires are making great financial consolidations; how the Modern Girl is a delightful creature, in spite of (or because of) having shingled hair or short skirts; and how the true religion, for which we all look to the Churches, consists of sympathy and social progress and marrying, divorcing, or burying everybody without reference to the precise meaning of the ceremony. On the other hand, if he prefers some other amusement, he may go to the Cinema, where he will see a very vivid and animated scene of the crowds cheering the Crown Prince of Fontarabia after the arrival of the yacht \emph{Atlantis;} where he will see an American film featuring the features of American millionaires, with all those resolute contortions of visage which accompany their making of great financial consolidations; where there will not be lacking a charming and vivacious heroine, recognizable as a Modern Girl by her short hair and short skirts; and possibly a kind and good clergyman (if any) who explains in dumb show, with the aid of a few printed sentences, that true religion is social sympathy and progress and marrying and burying people at random. But supposing the man’s tastes to be detached from the drama and from the kindred arts, he may prefer the reading of fiction; and he will have no difficulty in finding a popular novel about the doubts and difficulties of a good and kind clergyman slowly discovering that true religion consists of progress and social sympathy, with the assistance of a Modern Girl whose shingled hair and short skirts proclaim her indifference to all fine distinctions about who should be buried and who divorced; nor, probably, will the story fail to contain an American millionaire making vast financial consolidations, and certainly a yacht and possibly a Crown Prince. But there are yet other tastes that are catered for under the conditions of modern publicity and pleasure-seeking. There is the great institution of wireless or broadcasting; and the holiday-maker, turning away from fiction, journalism, and film drama, may prefer to “listenin” to a programme that will contain the very latest news of great financial consolidations made by American millionaires; which will most probably contain little lectures on how the Modern Girl can crop her hair or abbreviate her skirts; in which he can hear the very accents of some great popular preacher proclaiming to the world that revelation of true religion which consists of sympathy and social progress rather than of dogma and creed; and in which he will certainly hear the very thunder of cheering which welcomes His Royal Highness the Crown Prince of Fontarabia when he lands from the magnificent yacht \emph{Atlantis.} There is thus indeed a very elaborate and well-ordered choice placed before him, in the matter of the means of entertainment.

But even the rich variety of method and approach unfolded before us in this alternative seems to some to cover a certain secret and subtle element of monotony. Even here the pleasure-seeker may have that weird psychological sensation of having known the same thing before. There seems to be something recurrent about the type of topic; suggestive of something rigid about the type of mind. Now I think it very doubtful whether it is really a superior mind. If the pleasure-seeker himself were really a pleasure-maker for himself, if he were forced to amuse himself instead of being amused, if he were, in short, obliged to sit down in an old tavern and talk—I am really very doubtful about whether he would confine his conversation entirely to the Crown Prince of Fontarabia, the shingling of hair, the greatness of certain rich Yankees, and so on; and then begin the same round of subjects all over again. His interests might be more local, but they would be more lively; his experience of men more personal but more mixed; his likes and dislikes more capricious but not quite so easily satisfied. To take a parallel, modern children are made to play public-school games, and will doubtless soon be made to listen to the praise of the millionaires on the wireless and in the newspaper. But children left to themselves almost invariably invent games of their own, dramas of their own, often whole imaginary kingdoms and commonwealths of their own. In other words, they produce; until the competition of monopoly kills their production. The boy playing at robbers is not liberated but stunted by learning about American crooks, all of one pattern less picturesque than his own. He is psychologically undercut, undersold, dumped upon, frozen out, flooded, swamped, and ruined; but not emancipated.

Inventions have destroyed invention. The big modern machines are like big guns dominating and terrorizing a whole stretch of country, within the range of which nothing can raise its head. There is far more inventiveness to the square yard of mankind than can ever appear under that monopolist terror. The minds of men are not so much alike as the motor-cars of men, or the morning papers of men, or the mechanical manufacture of the coats and hats of men. In other words, we are not getting the best out of men. We are certainly not getting the most individual or the most interesting qualities out of men. And it is doubtful whether we ever shall, until we shut off this deafening din of megaphones that drowns their voices, this deathly glare of limelight which kills the colours of their complexions, this plangent yell of platitudes which stuns and stops their minds. All this sort of thing is killing thoughts as they grow, as a great white death-ray might kill plants as they grow. When, therefore, people tell me that making a great part of England rustic and self-supporting would mean making it rude and senseless, I do not agree with them; and I do not think they understand the alternative or the problem. Nobody wants all men to be rustics even in normal times; it is very tenable that some of the most intelligent would turn to the towns even in normal times. But I say the towns themselves are the foes of intelligence, in these times; I say the rustics themselves would have more variety and vivacity than is really encouraged by these towns. I say it is only by shutting off this unnatural noise and light that men’s minds can begin again to move and to grow. Just as we spread paving-stones over different soils without reference to the different crops that might grow there, so we spread programmes of platitudinous plutocracy over souls that God made various, and simpler societies have made free.

If by machinery saving labour, and therefore producing leisure, be meant the machinery that now achieves what is called mass production, I cannot see any vital value in the leisure; because there is in that leisure nothing of liberty. The man may only work for an hour with his machine-made tools, but he can only run away and play for twenty-three hours with machine-made toys. Everything he handles has to come from a huge machine that he cannot handle. Everything must come from something to which, in the current capitalist phrase, he can only lend “a hand.” Now as this would apply to intellectual and artistic toys as well as to merely material toys, it seems to me that the machine would dominate him for a much longer time than his hand had to turn the handle. It is practically admitted that much fewer men are needed to work the machine. The answer of the mechanical collectivists is that though the machine might give work to the few, it could give food to the many. But it could only give food to the many by an operation that had to be presided over by the few. Or even if we suppose that some work, subdivided into small sections, were given to the many, that system of rotation would have to be ruled by a responsible few; and some fixed authority would be needed to distribute the work as much as to distribute the food. In other words, the officials would very decidedly be permanent officials. In a sense all the rest of us might be intermittent or occasional officials. But the general character of the system would remain; and whatever else it is like, nothing can make it like a population pottering about in its own several fields or practising small creative crafts in its own little workshops. The man who has helped to produce a machine-made article may indeed leave off working, in the sense of leaving off turning one particular wheel. He may have an opportunity to do as he likes, in so far as he likes using what the system likes producing. He may have a power of choice—in the sense that he may choose between one thing it produces and another thing it produces. He may choose to pass his leisure hours in sitting in a machine-made chair or lying on a machine-made bed or resting in a machine-made hammock or swinging on a machine-made trapeze. But he will not be in the same position as a man who carves his own hobby-horse out of his own wood or his own hobby out of his own will. For that introduces another principle or purpose; which there is no warrant for supposing will coexist with the principle or purpose of using all the wood so as to save labour or simplifying all the wills so as to save leisure. If our ideal is to produce things as rapidly and easily as possible, we must have a definite number of things that we desire to produce. If we desire to produce them as freely and variously as possible, we must not at the same time try to produce them as quickly as possible. I think it most probable that the result of saving labour by machinery would be then what it is now, only more so: the limitation of the type of thing produced; standardization.

Now it may be that some of the supporters of the Leisure State have in mind some system of distributed machinery, which shall really make each man the master of his machine; and in that case I agree that the problem becomes different and that a great part of the problem is resolved. There would still remain the question of whether a man with a free soul would want to use a machine upon about three-quarters of the things for which machines are now used. In other words, there would remain the whole problem of the craftsman in the sense of the creator. But I should agree that if the small man found his small mechanical plant helpful to the preservation of his small property, its claim would be very considerable. But it is necessary to make it clear, that if the holidays provided for the mechanic are provided as mechanically as at present, and with the merely mechanical alternative offered at present, I think that even the slavery of his labour would be light compared to the grinding slavery of his leisure.

\chapter{The Free Man and the Ford Car}
\label{chapter-18}
I am not a fanatic; and I think that machines may be of considerable use in destroying machinery. I should generously accord them a considerable value in the work of exterminating all that they represent. But to put the truth in those terms is to talk in terms of the remote conclusion of our slow and reasonable revolution. In the immediate situation the same truth may be stated in a more moderate way. Towards all typical things of our time we should have a rational charity. Machinery is not wrong; it is only absurd. Perhaps we should say it is merely childish, and can even be taken in the right spirit by a child. If, therefore, we find that some machine enables us to escape from an inferno of machinery, we cannot be committing a sin though we may be cutting a silly figure, like a dragoon rejoining his regiment on an old bicycle. What is essential is to realize that there is something ridiculous about the present position, something wilder than any Utopia. For instance, I shall have occasion here to note the proposal of centralized electricity, and we could justify the use of it so long as we see the joke of it. But, in fact, we do not even see the joke of the waterworks and the water company. It is almost too broadly comic that an essential of life like water should be pumped to us from nobody knows where, by nobody knows whom, sometimes nearly a hundred miles away. It is every bit as funny as if air were pumped to us from miles away, and we all walked about like divers at the bottom of the sea. The only reasonable person is the peasant who owns his own well. But we have a long way to go before we begin to think about being reasonable.

There are at present some examples of centralization of which the effects may work for decentralization. An obvious case is that recently discussed in connection with a common plant of electricity. I think it is broadly true that if electricity could be cheapened, the chances of a very large number of small independent shops, especially workshops, would be greatly improved. At the same time, there is no doubt at all that such dependence for essential power on a central plant is a real dependence, and is therefore a defect in any complete scheme of independence. On this point I imagine that many Distributists might differ considerably; but, speaking for myself, I am inclined to follow the more moderate and provisional policy that I have suggested more than once in this place. I think the first necessity is to make sure of any small properties obtaining any success in any decisive or determining degree. Above all, I think it is vital to create the experience of small property, the psychology of small property, the sort of man who is a small proprietor. When once men of that sort exist, they will decide, in a manner very different from any modern mob, how far the central power-house is to dominate their own private house, or whether it need dominate at all. They will perhaps discover the way of breaking up and individualizing that power. They will sacrifice, if there is any need to sacrifice, even the help of science to the hunger for possession. So that I am disposed at the moment to accept any help that science and machinery can give in creating small property, without in the least bowing down to such superstitions where they only destroy it. But we must keep in mind the peasant ideal as the motive and the goal; and most of those who offer us mechanical help seem to be blankly ignorant of what we regard it as helping. A well-known name will illustrate both the thing being done and the man being ignorant of what he is doing.

The other day I found myself in a Ford car, like that in which I remember riding over Palestine, and in which, (I suppose) Mr. Ford would enjoy riding over Palestinians. Anyhow, it reminded me of Mr. Ford, and that reminded me of Mr. Penty and his views upon equality and mechanical civilization. The Ford car (if I may venture on one of those new ideas urged upon us in newspapers) is a typical product of the age. The best thing about it is the thing for which it is despised; that it is small. The worst thing about it is the thing for which it is praised; that it is standardized. Its smallness is, of course, the subject of endless American jokes, about a man catching a Ford like a fly or possibly a flea. But nobody seems to notice how this popularization of motoring (however wrong in motive or in method) really is a complete contradiction to the fatalistic talk about inevitable combination and concentration. The railway is fading before our eyes—birds nesting, as it were, in the railway signals, and wolves howling, so to speak, in the waiting-room. And the railway really was a communal and concentrated mode of travel like that in a Utopia of the Socialists. The free and solitary traveller is returning before our very eyes; not always (it is true) equipped with scrip or scallop, but having recovered to some extent the freedom of the King’s highway in the manner of Merry England. Nor is this the only ancient thing such travel has revived. While Mugby Junction neglected its refreshment-rooms, Hugby-in-the-Hole has revived its inns. To that limited extent the Ford motor is already a reversion to the free man. If he has not three acres and a cow, he has the very inadequate substitute of three hundred miles and a car. I do not mean that this development satisfies my theories. But I do say that it destroys other people’s theories; all the theories about the collective thing as a thing of the future and the individual thing as a thing of the past. Even in their own special and stinking way of science and machinery, the facts are very largely going against their theories.

Yet I have never seen Mr. Ford and his little car really and intelligently praised for this. I have often, of course, seen him praised for all the conveniences of what is called standardization. The argument seems to be more or less to this effect. When your car breaks down with a loud crash in the middle of Salisbury Plain, though it is not very likely that any fragments of other ruined motor cars will be lying about amid the ruins of Stonehenge, yet if they are, it is a great advantage to think that they will probably be of the same pattern, and you can take them to mend your own car. The same principle applies to persons motoring in Tibet, and exulting in the reflection that if another motorist from the United States did happen to come along, it would be possible to exchange wheels or footbrakes in token of amity. I may not have got the details of the argument quite correct; but the general point of it is that if anything goes wrong with parts of a machine, they can be replaced with identical machinery. And anyhow the argument could be carried much further; and used to explain a great many other things. I am not sure that it is not the clue to many mysteries of the age. I begin to understand, for instance, why magazine stories are all exactly alike; it is ordered so that when you have left one magazine in a railway carriage in the middle of a story called “Pansy Eyes,” you may go on with exactly the same story in another magazine under the title of “Dandelion Locks.” It explains why all leading articles on The Future of the Churches are exactly the same; so that we may begin reading the article in the \emph{Daily Chronicle} and finish it in the \emph{Daily Express.} It explains why all the public utterances urging us to prefer new things to old never by any chance say anything new; they mean that we should go to a new paper-stall and read it in a new newspaper. This is why all American caricatures repeat themselves like a mathematical pattern; it means that when we have torn off a part of the picture to wrap up sandwiches, we can tear off a bit of another picture and it will always fit in. And this is also why American millionaires all look exactly alike; so that when the bright, resolute expression of one of them has led us to do serious damage to his face with a heavy blow of the fist, it is always possible to mend it with noses and jaw-bones taken from other millionaires, who are exactly similarly constituted.

Such are the advantages of standardization; but, as may be suspected, I think the advantages are exaggerated; and I agree with Mr. Penty in doubting whether all this repetition really corresponds to human nature. But a very interesting question was raised by Mr. Ford’s remarks on the difference between men and men; and his suggestion that most men preferred mechanical action or were only fitted for it. About all those arguments affecting human equality, I myself always have one feeling, which finds expression in a little test of my own. I shall begin to take seriously those classifications of superiority and inferiority, when I find a man classifying himself as inferior. It will be noted that Mr. Ford does not say that he is only fitted to mind machines; he confesses frankly that he is too fine and free and fastidious a being for such tasks. I shall believe the doctrine when I hear somebody say: “I have only got the wits to turn a wheel.” That would be real, that would be realistic, that would be scientific. That would be independent testimony that could not easily be disputed. It is exactly the same, of course, with all the other superiorities and denials of human equality that are so specially characteristic of a scientific age. It is so with the men who talk about superior and inferior races; I never heard a man say: “Anthropology shows that I belong to an inferior race.” If he did, he might be talking like an anthropologist; as it is, he is talking like a man, and not infrequently like a fool. I have long hoped that I might some day hear a man explaining on scientific principles his own unfitness for any important post or privilege, say: “The world should belong to the free and fighting races, and not to persons of that servile disposition that you will notice in myself; the intelligent will know how to form opinions, but the weakness of intellect from which I so obviously suffer renders my opinions manifestly absurd on the face of them: there are indeed stately and godlike races—but look at me! Observe my shapeless and fourth-rate features! Gaze, if you can bear it, on my commonplace and repulsive face!” If I heard a man making a scientific demonstration in that style, I might admit that he was really scientific. But as it invariably happens, by a curious coincidence, that the superior race is his own race, the superior type is his own type, and the superior preference for work the sort of work he happens to prefer—I have come to the conclusion that there is a simpler explanation.

Now Mr. Ford is a good man, so far as it is consistent with being a good millionaire. But he himself will very well illustrate where the fallacy of his argument lies. It is probably quite true that, in the making of motors, there are a hundred men who can work a motor and only one man who can design a motor. But of the hundred men who could work a motor, it is very probable that one could design a garden, another design a charade, another design a practical joke or a derisive picture of Mr. Ford. I do not mean, of course, in anything I say here, to deny differences of intelligence, or to suggest that equality (a thing wholly religious) depends on any such impossible denial. But I do mean that men are nearer to a level than anybody will discover by setting them all to make one particular kind of run-about clock. Now Mr. Ford himself is a man of defiant limitations. He is so indifferent to history, for example, that he calmly admitted in the witness-box that he had never heard of Benedict Arnold. An American who has never heard of Benedict Arnold is like a Christian who has never heard of Judas Iscariot. He is rare. I believe that Mr. Ford indicated in a general way that he thought Benedict Arnold was the same as Arnold Bennett. Not only is this not the case, but it is an error to suppose that there is no importance in such an error. If he were to find himself, in the heat of some controversy, accusing Mr. Arnold Bennett of having betrayed the American President and ravaged the South with an Anti-American army, Mr. Bennett might bring an action. If Mr. Ford were to suppose that the lady who recently wrote revelations in the \emph{Daily Express} was old enough to be the widow of Benedict Arnold, the lady might bring an action. Now it is not impossible that among the workmen whom Mr. Ford perceives (probably quite truly) to be only suited to the mechanical part of the construction of mechanical things, there might be a man who was fond of reading all the history he could lay his hands on; and who had advanced step by step, by painful efforts of self-education, until the difference between Benedict Arnold and Arnold Bennett was quite clear in his mind. If his employer did not care about the difference, of course, he would not consult him about the difference, and the man would remain to all appearance a mere cog in the machine; there would be no reason for finding out that he was a rather cogitating cog. Anybody who knows anything of modern business knows that there are any number of such men who remain in subordinate and obscure positions because their private tastes and talents have no relation to the very stupid business in which they are engaged. If Mr. Ford extends his business over the Solar System, and gives cars to the Martians and the Man in the Moon, he will not be an inch nearer to the mind of the man who is working his machine for him, and thinking about something more sensible. Now all human things are imperfect; but the condition in which such hobbies and secondary talents do to some extent come out is the condition of small independence. The peasant almost always runs two or three sideshows and lives on a variety of crafts and expedients. The village shopkeeper will shave travellers and stuff weasels and grow cabbages and do half a dozen such things, keeping a sort of balance in his life like the balance of sanity in the soul. The method is not perfect; but it is more intelligent than turning him into a machine in order to find out whether he has a soul above machinery.

Upon this point of immediate compromise with machinery, therefore, I am inclined to conclude that it is quite right to use the existing machines in so far as they do create a psychology that can despise machines; but not if they create a psychology that respects them. The Ford car is an excellent illustration of the question; even better than the other illustration I have given of an electrical supply for small workshops. If possessing a Ford car means rejoicing in a Ford car, it is melancholy enough; it does not bring us much farther than Tooting or rejoicing in a Tooting tramcar. But if possessing a Ford car means rejoicing in a field of corn or clover, in a fresh landscape and a free atmosphere, it may be the beginning of many things—and even the end of many things. It may be, for instance, the end of the car and the beginning of the cottage. Thus we might almost say that the final triumph of Mr. Ford is not when the man gets into the car, but when he enthusiastically falls out of the car. It is when he finds somewhere, in remote and rural corners that he could not normally have reached, that perfect poise and combination of hedge and tree and meadow in the presence of which any modern machine seems suddenly to look an absurdity; yes, even an antiquated absurdity. Probably that happy man, having found the place of his true home, will proceed joyfully to break up the car with a large hammer, putting its iron fragments for the first time to some real use, as kitchen utensils or garden tools. That is using a scientific instrument in the proper way; for it is using it as an instrument. The man has used modern machinery to escape from modern society; and the reason and rectitude of such a course commends itself instantly to the mind. It is not so with the weaker brethren who are not content to trust Mr. Ford’s car, but also trust Mr. Ford’s creed. If accepting the car means accepting the philosophy I have just criticized, the notion that some men are born to make cars, or rather small bits of cars, then it will be far more worthy of a philosopher to say frankly that men never needed to have cars at all. It is only because the man had been sent into exile in a railway-train that he has to be brought back home in a motor-car. It is only because all machinery has been used to put things wrong that some machinery may now rightly be used to put things right. But I conclude upon the whole that it may so be used; and my reason is that which I considered on a previous page under the heading of “The Chance of Recovery.” I pointed out that our ideal is so sane and simple, so much in accord with the ancient and general instincts of men, that when once it is given a chance anywhere it will improve that chance by its own inner vitality because there is some reaction towards health whenever disease is removed. The man who has used his car to find his farm will be more interested in the farm than in the car; certainly more interested than in the shop where he once bought the car. Nor will Mr. Ford always woo him back to that shop, even by telling him tenderly that he is not fitted to be a lord of land, a rider of horses, or a ruler of cattle; since his deficient intellect and degraded anthropological type fit him only for mean and mechanical operations. If anyone will try saying this (tenderly, of course) to any considerable number of large farmers, who have lived for some time on their own farms with their own families, he will discover the defects of the approach.

\setcounter{chapter}{0}\part{A Note on Emigration}
\label{chapter-19}
\chapter{The Need of a New Spirit}
\label{chapter-20}
Before closing these notes, with some words on the colonial aspect of democratic distribution, it will be well to make some acknowledgment of the recent suggestion of so distinguished a man as Mr. John Galsworthy. Mr. Galsworthy is a man for whom I have the very warmest regard; for a human being who really tries to be fair is something very like a monster and miracle in the long history of this merry race of ours. Sometimes, indeed, I get a little exasperated at being so persistently excused. I can imagine few things more annoying, to a free-born and properly constituted Christian, than the thought that if he did choose to wait for Mr. Galsworthy behind a wall, knock him down with a brick, jump on him with heavy boots, and so on, Mr. Galsworthy would still faintly gasp that it was only the fault of the System; that the System made bricks and the System heaved bricks and the System went about wearing heavy boots, and so on. As a human being, I should feel a longing for a little human justice, after all that inhuman mercy. But these feelings do not interfere with the other feelings I have, of something like enthusiasm, for something that can only be called beautiful in the fair-mindedness of a study like “The White Monkey.” It is when this attitude of detachment is applied not to the judgment of individuals but of men in bulk, that the detachment begins to savour of something unnatural. And in Mr. Galsworthy’s last political pronouncement the detachment amounts to despair. At any rate, it amounts to despair about this earth, this England, about which I am certainly not going to despair yet. But I think it might be well if I took this opportunity of stating what I, for one, at least feel about the different claims here involved.

It may be debated whether it is a good or a bad thing for England that England has an Empire. It may be debated, at least as a matter of true definition, whether England has an Empire at all. But upon one point all Englishmen ought to stand firm, as a matter of history, of philosophy, and of logic. And that is that it has been, and is, a question of our owning an Empire and not of an Empire owning us.

There is sense in being separated from Americans on the principles of George Washington, and sense in being attached to Americans on the principles of George the Third. But there is no sense in being out-voted and swamped by Americans in the name of the Anglo-Saxon race. The Colonies were by origin English. They owe us that much; if it be only the trivial circumstance, so little valued by modern thought, that without their maker they could never have existed at all. If they choose to remain English, we thank them very sincerely for the compliment. If they choose not to remain English, but to turn into something else, we think they are within their rights. But anyhow England shall remain English. They shall not first turn themselves into something else, and then turn us into themselves. It may have been wrong to be an Empire, but it does not rob us of our right to be a nation.

But there is another sense in which those of our school would use the motto of “England First.” It is in the sense that our first step should be to discover how far the best ethical and economic system can be fitted into England, before we treat it as an export and cart it away to the ends of the earth. The scientific or commercial character, who is sure he has found an explosive that will blow up the solar system or a bullet that will kill the man in the moon, always makes a great parade of saying that he offers it first to his own country, and only afterwards to a foreign country. Personally, I cannot conceive how a man can bring himself in any case to offer such a thing to a foreign country. But then I am not a great scientific and commercial genius. Anyhow, such as our little notion of normal ownership is, we certainly do not propose to offer it to any foreign country, or even to any colony, before we offer it to our own country. And we do think it highly urgent and practical to find out first how much of it can really be carried out in our own country. Nobody supposes that the whole English population could live on the English land. But everybody ought to realize that immeasurably more people could live on it than do live on it; and that if such a policy did establish such a peasantry, there would be a recognizable narrowing of the margin of men left over for the town and the colonies. But we would suggest that these ought really to be left over, and dealt with as seems most desirable, after the main experiment has been made where it matters most. And what most of us would complain of in the emigrationists of the ordinary sort is that they seem to think first of the colony and then of what must be left behind in the country; instead of thinking first of the country and then of what must overflow into the colony.

People talk about an optimist being in a hurry; but it seems to me that a pessimist like Mr. Galsworthy is very much in a hurry. He has not tried the obvious reform on England, and, finding it fail, gone into exile to try it elsewhere. He is trying the obvious reform everywhere except where it is most obvious. And in this I think he has a subconscious affinity to people much less reasonable and respectable than himself. The pessimists have a curious way of urging us to counsels of despair as the only solution of a problem they have not troubled to solve. They declare solemnly that some unnatural thing would become necessary if certain conditions existed; and then somehow assume from that that they exist. They never think of attempting to prove that they exist, before they prove what follows from their existence. This is exactly the sort of plunging and premature pessimism, for instance, that people exhibit about Birth Control. Their desire is towards destruction; their hope is for despair; they eagerly anticipate the darkest and most doubtful predictions. They run with eager feet before and beyond the lingering and inconveniently slow statistics; like as the hart pants for the water-brooks they thirst to drink of Styx and Lethe before their hour; even the facts they show fall far short of the faith that they see shining beyond them; for faith is the substance of things hoped for, the evidence of things not seen.

If I do not compare the critic in question with the doctors of this dismal perversion, still less do I compare him with those whose motives are merely self-protective and plutocratic. But it must also be said that many rush to the expedient of emigration, just as many rush to the expedient of Birth Control, for the perfectly simple reason that it is the easiest way in which the capitalists can escape from their own blunder of capitalism. They lured men into the town with the promise of greater pleasures; they ruined them there and left them with only one pleasure; they found the increase it produced at first convenient for labour and then inconvenient for supply; and now they are ready to round off their experiment in a highly appropriate manner, by telling them that they must have no families, or that their families must go to the modern equivalent of Botany Bay. It is not in that spirit that we envisage an element of colonization; and so long as it is treated in that spirit we refuse to consider it. I put first the statement that real colonial settlement must be not only stable but sacred. I say the new home must be not only a home but a shrine. And that is why I say it must be first established in England, in the home of our fathers and the shrine of our saints, to be a light and an ensign to our children.

I have explained that I cannot content myself with leaving my own nationality out of my own normal ideal; or leaving England as the mere tool-house or coal-cellar of other countries like Canada or Australia—or, for that matter, Argentina. I should like England also to have a much more rural type of redistribution; nor do I think it impossible. But when this is allowed for, nobody in his five wits would dream of denying that there is a real scope and even necessity for emigration and colonial settlement. Only, when we come to that, I have to draw a line rather sharply and explain something else, which is by no means inconsistent with my love of England, but I fear is not so likely to make me loved by Englishmen. I do not believe, as the newspapers and national histories always tell me to believe, that we have “the secret” of this sort of successful colonization and need nothing else to achieve this sort of democratic social construction. I ask for nothing better than that a man should be English in England. But I think he will have to be something more than English (or at any rate something more than “British”) if he is to create a solid social equality outside England. For something is needed for that solid social creation which our colonial tradition has not given. My reasons for holding this highly unpopular opinion I will attempt to suggest; but the fact that they are rather difficult to suggest is itself an evidence of their unfamiliarity and of that narrowness which is neither national nor international, but only imperial.

I should very much like to be present at a conversation between Mr. Saklatvala and Dean Inge. I have a great deal of respect for the real sincerity of the Dean of St. Paul’s, but his subconscious prejudices are of a strange sort. I cannot help having a feeling that he might have a certain sympathy with a Socialist so long as he was not a Christian Socialist. I do not indeed pretend to any respect for the ordinary sort of broad-mindedness which is ready to embrace a Buddhist but draws the line at a Bolshevist. I think its significance is very simple. It means welcoming alien religions when they make us feel comfortable, and persecuting them when they make us feel uncomfortable. But the particular reason I have at the moment for entertaining this association of ideas is one that concerns a larger matter. It concerns, indeed, what is commonly called the British Empire, which we were once taught to reverence largely because it was large. And one of my complaints against that common and rather vulgar sort of imperialism is that it did not really secure even the advantages of largeness. As I have said, I am a nationalist; England is good enough for me. I would defend England against the whole European continent. With even greater joy would I defend England against the whole British Empire. With a romantic rapture would I defend England against Mr. Ramsay MacDonald when he had become King of Scotland; lighting again the watch fires of Newark and Carlisle and sounding the old tocsins of the Border. With equal energy would I defend England against Mr. Tim Healy as King of Ireland, if ever the gross and growing prosperity of that helpless and decaying Celtic stock became positively offensive. With the greatest ecstasy of all would I defend England against Mr. Lloyd George as King of Wales. It will be seen, therefore, that there is nothing broad-minded about my patriotism; most modern nationality is not narrow enough for me.

But putting aside my own local affections, and looking at the matter in what is called a larger way, I note once more that our Imperialism does not get any of the good that could be got out of being large. And I was reminded of Dean Inge, because he suggested some time ago that the Irish and the French Canadians were increasing in numbers, not because they held the Catholic view of the family, but because they were a backward and apparently almost barbaric stock which naturally (I suppose he meant) increased with the blind luxuriance of a jungle. I have already remarked on the amusing trick of having it both ways which is illustrated in this remark. So long as savages are dying out, we say they are dying out because they are savages. When they are inconveniently increasing, we say they are increasing because they are savages. And from this it is but a simple logical step to say that the countrymen of Sir Wilfred Laurier or Senator Yeats are savages because they are increasing. But what strikes me most about the situation is this: that this spirit will always miss what is really to be learnt by covering any large and varied area. If French Canada is really a part of the British Empire, it would seem that the Empire might at least have served as a sort of interpreter between the British and the French. The Imperial statesman, if he had really been a statesman, ought to have been able to say, “It is always difficult to understand another nation or another religion; but I am more fortunately placed than most people. I know a little more than can be known by self-contained and isolated states like Sweden or Spain. I have more sympathy with the Catholic faith or the French blood because I have French Catholics in my own Empire.” Now it seems to me that the Imperial statesman never has said this; never has even been able to say it; never has even tried or pretended to be able to say it. He has been far narrower than a nationalist like myself, engaged in desperately defending Offa’s Dyke against a horde of Welsh politicians. I doubt if there was ever a politician who knew a word more of the French language, let alone a word more of the Latin Mass, because he had to govern a whole population that drew its traditions from Rome and Gaul. I will suggest in a moment how this enormous international narrowness affects the question of a peasantry and the extension of the natural ownership of land. But for the moment it is important to make the point clear about the nature of that narrowness. And that is why some light might be thrown on it in that tender, that intimate, that heart-to-heart talk between Mr. Saklatvala and the Dean of St. Paul’s. Mr. Saklatvala is a sort of parody or extreme and extravagant exhibition of the point; that we really know nothing at all about the moral and philosophical elements that make up the Empire. It is quite obvious, of course, that he does not represent Battersea. But have we any way of knowing to what extent he represents India? It seems to me not impossible that the more impersonal and indefinite doctrines of Asia do form a soil for Bolshevism. Most of the eastern philosophy differs from the western theology in refusing to draw the line anywhere; and it would be a highly probable perversion of that instinct to refuse to draw the line between \emph{meum} and \emph{tuum.} I do not think the Indian gentleman is any judge of whether we in the West want to have a hedge round our fields or a wall round our gardens. And as I happen to hold that the very highest human thought and art consists almost entirely in drawing the line somewhere, though not in drawing it anywhere, I am completely confident that in this the western tendency is right and the eastern tendency is wrong. But, in any case, it seems to me that a rather sharp lesson to us is indicated in these two parallel cases of the Indian who grows into a Bolshevist in our dominions without our being able to influence his growth, and the French Canadian who remains a peasant in our dominions without our getting any sort of advantage out of his stability.

I do not profess to know very much about the French Canadians; but I know enough to know that most of the people who talk at large about the Empire know even less than I do. And the point about them is that they generally do not even try to know any more. The very vague picture that they always call up, of colonists doing wonders in all the corners of the world, never does, in fact, include the sort of thing that French Canadians can do, or might possibly show other people how to do. There is about all this fashionable fancy of colonization a very dangerous sort of hypocrisy. People tried to use the Over-seas Dominion as Eldorado while still using it as Botany Bay. They sent away people that they wanted to get rid of, and then added insult to injury by representing that the ends of the earth would be delighted to have them. And they called up a sort of fancy portrait of a person whose virtues and even vices were entirely suitable for founding an Empire, though apparently quite unsuitable for founding a family. The very language they used was misleading. They talked of such people as settlers; but the very last thing they ever expected them to do was to settle. They expected of them a sort of indistinct individualistic breaking of new ground, for which the world is less and less really concerned to-day. They sent an inconvenient nephew to hunt wild bisons in the streets of Toronto; just as they had sent any number of irrepressible Irish exiles to war with wild Redskins in the streets of New York. They incessantly repeated that what the world wants is pioneers, and had never even heard that what the world wants is peasants. There was a certain amount of sincere and natural sentiment about the wandering exile inheriting our traditions. There was really no pretence that he was engaged in founding his own traditions. All the ideas that go with a secure social standing were absent from the very discussion; no one thought of the continuity, the customs, the religion, or the folklore of the future colonist. Above all, nobody ever conceived him as having any strong sense of private property. There was in the vague idea of his gaining something for the Empire always, if anything, the idea of his gaining what belonged to somebody else. I am not now discussing how wrong it was or whether it could in some cases be right; I am pointing out that nobody ever entertained the notion of the other sort of right; the special right of every man to his own. I doubt whether a word could be quoted emphasizing it even from the healthiest adventure story or the jolliest Jingo song. I quite appreciate all there is in such songs or stories that is really healthy or jolly. I am only pointing out that we have badly neglected something; and are now suffering from the neglect. And the worst aspect of the neglect was that we learnt nothing whatever from the peoples that were actually inside the Empire which we wished to glorify: nothing whatever from the Irish; nothing whatever from the French Canadian; nothing whatever even from the poor Hindoos. We have now reached a crisis in which we particularly require these neglected talents; and we do not even know how to set about learning them. And the explanation of this blunder, as of most blunders, is in the weakness which is called pride: in other words, it is in the tone taken by people like the Dean of St. Paul’s.

Now there will be needed a large element of emigration in the solution of re-creating a peasantry in the modern world. I shall have more to say about the elements of the idea in the next section. But I believe that any scheme of the sort will have to be based on a totally different and indeed diametrically opposite spirit and principle to that which is commonly applied to emigration in England to-day. I think we need a new sort of inspiration, a new sort of appeal, a new sort of ordinary language even, before that solution will even help to solve anything. What we need is the ideal of Property, not merely of Progress—especially progress over other people’s property. Utopia needs more frontiers, not less. And it is because we were weak in the ethics of property on the edges of Empire that our own society will not defend property as men defend a right. The Bolshevist is the sequel and punishment of the Buccaneer.

\chapter{The Religion of Small Property}
\label{chapter-21}
We hear a great deal nowadays about the disadvantages of decorum, especially from those who are always telling us that women in the last generation were helpless and impotent, and then proceed to prove it by describing the tremendous and towering tyranny of Mrs. Grundy. Rather in the same way, they insist that Victorian women were especially soft and submissive. And it is rather unfortunate for them that, even in order to say so, they have to introduce the name of Queen Victoria. But it is more especially in connection with the indecorous in art and literature that the question arises, and it is now the fashion to argue as if there were no psychological basis for reticence at all. That is where the argument should end; but fortunately these thinkers do not know how to get to the end of an argument. I have heard it argued that there is no more harm in describing the violation of one Commandment than of another; but this is obviously a fallacy. There is at least a case in psychology for saying that certain images move the imagination to the weakening of the character. There is no case for saying that the mere contemplation of a kit of burglar’s tools would inflame us all with a desire to break into houses. There is no possibility of pretending that the mere sight of means to murder our maiden aunt with a poker does really make the ill deed done. But what strikes me as most curious about the controversy is this: that while our fiction and journalism is largely breaking down the prohibitions for which there really was a logical case, in the consideration of human nature, they still very largely feel the pressure of prohibitions for which there was never any case at all. And the most curious thing about the criticism we hear directed against the Victorian Age is that it is never directed against the most arbitrary conventions of that age. One of these, which I remember very vividly in my youth, was the convention that there is something embarrassing or unfair about a man mentioning his religion. There was something of the same feeling about his mentioning his money. Now these things cannot possibly be defended by the same psychological argument as the other. Nobody is moved to madness by the mere sight of a church spire, or finds uncontrollable emotions possess him at the thought of an archdeacon’s hat. Yet there is still enough of that really irrational Victorian convention lingering in our life and literature to make it necessary to offer a defence, if not an apology, whenever an argument depends upon this fundamental fact in life.

Now when I remark that we want a type of colonization rather represented by the French Canadians, there are probably still a number of sly critics who would point the finger of detection at me and cry, as if they had caught me in something very naughty, “You believe in the French Canadians because they are Catholics”; which is in one sense not only true, but very nearly the whole truth. But in another sense it is not true at all; if it means that I exercise no independent judgment in perceiving that this is really what we do want. Now when this difficulty and misunderstanding arises, there is only one practical way of meeting it in the present state of public information, or lack of information. It is to call what is generally described as an impartial witness; though it is quite probable that he is far less impartial than I am. What is really important about him is that, if he were partial, he would be partial on the other side.

The dear old \emph{Daily News,} of the days of my youth, on which I wrote happily for many years and had so many good and admirable friends, cannot be accused as yet as being an organ of the Jesuits. It was, and is, as every one knows, the organ of the Nonconformists. Dr. Clifford brandished his teapot there when he was selling it in order to demonstrate, by one symbolical act, that he had long been a teetotaller and was now a Passive Resister. We may be pardoned for smiling at this aspect of the matter; but there are many other aspects which are real and worthy of all possible respect. The tradition of the old Puritan ideal does really descend to this paper; and multitudes of honest and hard-thinking Radicals read it in my youth and read it still.

I therefore think that the following remarks which appeared recently in the \emph{Daily News,} in an article by Mr. Hugh Martin, writing from Toronto, are rather remarkable. He begins by saying that the Anglo-Saxon has got too proud to bend his back; but the curious thing is that he goes on to suggest, almost in so many words, that the backs of the French Canadians are actually strengthened, not only by being bent over rustic spades, but even by being bent before superstitious altars. I am very anxious not to do my impartial witness an unfair damage in the matter; so I may be excused if I quote his own words at some little length. After saying that the Anglo-Saxons are drawn away to the United States, or at any rate to the industrial cities, he remarks that the French are of course very numerous in Quebec and elsewhere, but that it is not here that the notable development is taking place, and that Montreal, being a large city, is showing signs of the slackening to be seen in other large cities.

“Now look at the other picture. The race that is going ahead is the French race. . . . In Quebec, where there are nearly 2,000,000 Canadians of French origin in a population of 2,350,000, that might have been expected. But as a matter of fact it is not in Quebec that the French are making good most conspicuously . . . nor in Nova Scotia and New Brunswick is the comparative success of the French stock most marked. They are doing splendidly on the land and raising prodigious families. A family of twelve is quite common, and I could name several cases where there have been twenty, who all lived. The day may come when they will equal or outnumber the Scotch, but that is some way ahead. If you want to see what French stock can still achieve, you should go to the northern part of this province of Ontario. It is doing pioneer work. It is bending its back as men did in the old days. It is multiplying and staying on the soil. It is content to be happy without being rich.

“Though I am not a religious man myself, I must confess I think religion has a good deal to do with it. These French Canadians are more Catholic than the Pope. You might call a good many of them desperately ignorant and desperately superstitious. They seem to me to be a century behind the times and a century nearer happiness.”

These seem to me, I repeat, to be rather remarkable words; remarkable if they appeared anywhere, arresting and astonishing when they appear in the traditional paper of the Manchester Radicals and the nineteenth-century Nonconformists. The words are splendidly straightforward and unaffected in their literary form; they have a clear ring of sincerity and experience, and they are all the more convincing because they are written by somebody who does not share my own desperate ignorance and desperate superstition. But he proceeds to suggest a reason, and incidentally to make his own independence in the matter quite clear.

“Apart from the fact that their women bear an incredible number of children, you have this other consequence of their submission to the priest, that a social organism is created, which is of incalculable value in the backwoods. The church, the school, the curé, hold each little group together as a unit. Do not think for a moment that I believe a general spread of Catholicism would turn us back into a pioneer people. One might just as reasonably recommend a return to early Scottish Protestantism. I merely record the fact that the simplicity of these people is proving their salvation and is one of the most hopeful things in Canada to-day.”

Of course, there are a good many things of an incidental kind that a person with my views might comment on in that passage. I might go off at a gallop on the highly interesting comparison with early Scottish Protestantism. Very early Scottish Protestantism, like very early English Protestantism, consisted chiefly of loot. But if we take it as referring to the perfectly pure and sincere enthusiasm of many Covenanters or early Calvinists, we come upon the contrast that is the point of the whole matter. Early Puritanism was pure Puritanism; but the purer it is the more early it seems. We cannot imagine it as a good thing and also a modern thing. It might have been one of the most honest things in Scotland then. But nobody would be found calling it one of the most hopeful things in Canada to-day. If John Knox appeared to-morrow in the pulpit of St. Giles, he would be a stickit minister. He would be regarded as a raving savage because of his ignorance of German metaphysics. That comparison does not meet the extraordinary case of the thing that is older than Knox and yet also newer than Knox. Or again, I might point out that the common connotation of “submission to the priest” is misleading, even if it is true. It is like talking of the Charge of the Light Brigade as the submission to Lord Raglan. It is still more like talking about the storming of Jerusalem as the submission to the Count of Bouillon. In one sense it is quite true; in another it is very untrue. But I have not the smallest desire here to disturb the impartiality of my witness. I have not the smallest intention of using any of the tortures of the Inquisition to make him admit anything that he did not wish to admit. The admission as it stands seems to me very remarkable; not so much because it is a tribute to Frenchmen as colonists as because it is a tribute to colonists as pious and devout people. But what concerns me most of all in the general discussion of my own theme is the insistence on stability. They are staying on the soil; they are a social organism; they are held together as a unit. That is the new note which I think is needed in all talk of colonization, before it can again be any part of the hope of the world.

A recent description of the Happy Factory, as it exists in America or will exist in Utopia, rose from height to height of ideality until it ended with a sort of hush, as of the ultimate opening of the heavens, and these words about the workman, “He turns out for his homeward journey like a member of the Stock Exchange.” Any attempt to imagine humanity in its final perfection always has about it something faintly unreal, as being too good for this world; but the visionary light that breaks from the cloud, in that last phrase, accentuates clearly the contrast which is to be drawn between such a condition and that of the labour of common men. Adam left Eden as a gardener; but he will set out for his homeward journey like a member of the Stock Exchange. St. Joseph was a carpenter; but he will be raised again as a stockbroker. Giotto was a shepherd; for he was not yet worthy to be a stockbroker. Shakespeare was an actor; but he dreamed day and night of being a stockbroker. Burns was a ploughman; but if he sang at the plough, how much more appropriately he would have sung in the Stock Exchange. It is assumed in this kind of argument that all humanity has consciously or unconsciously hoped for this consummation; and that if men were not brokers, it was because they were not able to broke. But this remarkable passage in Sir Ernest Benn’s exposition has another application besides the obvious one. A stockbroker in one sense really is a very poetical figure. In one sense he is as poetical as Shakespeare, and his ideal poet, since he does give to airy nothing a local habitation and a name. He does deal to a great extent in what economists (in their poetical way) describe as imaginaries. When he exchanges two thousand Patagonian Pumpkins for one thousand shares in Alaskan Whale Blubber, he does not demand the sensual satisfaction of eating the pumpkin or need to behold the whale with the gross eye of flesh. It is quite possible that there are no pumpkins; and if there is somewhere such a thing as a whale, it is very unlikely to obtrude itself upon the conversation in the Stock Exchange. Now what is the matter with the financial world is that it is a great deal too full of imagination, in the sense of fiction. And when we react against it, we naturally in the first place react into realism. When the stockbroker homeward plods his weary way and leaves the world to darkness and Sir Ernest Benn, we are disposed to insist that it is indeed he who has the darkness and we who have the daylight. He has not only the darkness but the dreams, and all the unreal leviathans and unearthly pumpkins pass before him like a mere scroll of symbols in the dreams of the Old Testament. But when the small proprietor grows pumpkins, they really are pumpkins, and sometimes quite a large pumpkin for quite a small proprietor. If he should ever have occasion to grow whales (which seems improbable) they would either be real whales or they would be of no use to him. We naturally grow a little impatient, under these conditions, when people who call themselves practical scoff at the small proprietor as if he were a minor poet. Nevertheless, there is another side to the case, and there is a sense in which the small proprietor had better be a minor poet, or at least a mystic. Nay, there is even a sort of queer paradoxical sense in which the stockbroker is a man of business.

It is to that other side of small property, as exemplified in the French Canadians, and an article on them in the \emph{Daily News,} that I devoted my last remarks. The really practical point in that highly interesting statement is, that in this case, being progressive is actually identified with being what is called static. In this case, by a strange paradox, a pioneer is really d settler. In this case, by a still stranger paradox, a settler is a person who really settles. It will be noted that the success of the experiment is actually founded on a certain power of striking root; which we might almost call rapid tradition, as others talk of rapid transit. And indeed the ground under the pioneer’s feet can only be made solid by being made sacred. It is only religion that can thus rapidly give a sort of accumulated power of culture and legend to something that is crude or incomplete. It sounds like a joke to say that baptizing a baby makes the baby venerable; it suggests the old joke of the baby with spectacles who died an enfeebled old dotard at five. Yet it is profoundly true that something is added that is not only something to be venerated, but something partly to be venerated for its antiquity—that is, for the unfathomable depth of its humanity. In a sense a new world can be baptized as a new baby is baptized, and become a part of an ancient order not merely on the map but in the mind. Instead of crude people merely extending their crudity, and calling that colonization, it would be possible for people to cultivate the soil as they cultivate the soul. But for this it is necessary to have a respect for the soil as well as for the soul; and even a reverence for it, as having some associations with holy things. But for that purpose we need some sense of carrying holy things with us and taking them home with us; not merely the feeling that holiness may exist as a hope. In the most exalted phrase, we need a real presence. In the most popular phrase, we need something that is always on the spot.

That is, we want something that is always on the spot, and not only beyond the horizon. The pioneer instinct is beginning to fail, as a well-known traveller recently complained, but I doubt whether he could tell us the reason. It is even possible that he will not understand it, in one radiant burst of joyful comprehension, if I tell him that I am all in favour of a wild-goose chase, so long as he really believes that the wild goose is the bird of paradise; but that it is necessary to hunt it with the hounds of heaven. If it be barely possible that this does not seem quite clear to him, I will explain that the traveller must possess something as well as pursue something, or he will not even know what to pursue. It is not enough always to follow the gleam: it is necessary sometimes to rest in the glow; to feel something sacred in the glow of the camp fire as well as the gleam of the polar star. And that same mysterious and to some divided voice, which alone tells that we have here no abiding city, is the only voice which within the limits of this world can build up cities that abide.

As I said at the beginning of this section, it is futile to pretend that such a faith is not a fundamental of the true change. But its practical relation to the reconstruction of property is that, unless we understand this spirit, we cannot now relieve congestion with colonization. People will prefer the mere nomadism of the town to the mere nomadism of the wilderness. They will not tolerate emigration if it merely means being moved on by the politicians as they have been moved on by the policemen. They will prefer bread and circuses to locusts and wild honey, so long as the forerunner does not know for what God he prepares the way.

But even if we put aside for the moment the strictly spiritual ideals involved in the change, we must admit that there are secular ideals involved which must be positive and not merely comparative, like the ideal of progress. We are sometimes taunted with setting against all other Utopias what is in truth the most impossible Utopia; with describing a Merry Peasant who cannot exist except on the stage, with depending on a China Shepherdess who never was seen except on the mantelpiece. If we are indeed presenting impossible portraits of an ideal humanity, we are not alone in that. Not only the Socialists but also the Capitalists parade before us their imaginary and ideal figures, and the Capitalists if possible more than the Socialists. For once that we read of the last Earthly Paradise of Mr. Wells, where men and women move gracefully in simple garments and keep their tempers in a way in which we in this world sometimes find difficult (even when we are the authors of Utopian novels), for once that we see the ideal figure of that vision, we see ten times a day the ideal figure of the commercial advertisers. We are told to “Be Like This Man,” or to imitate an aggressive person pointing his finger at us in a very rude manner for one who regards himself as a pattern to the young. Yet it is entirely an ideal portrait; it is very unlikely (we are glad to say) that any of us will develop a chin or a finger of that obtrusive type. But we do not blame either the Capitalists or the Socialists for setting up a type or talismanic figure to fix the imagination. We do not wonder at their presenting the perfect person for our admiration; we only wonder at the person they admire. And it is quite true that, in our movement as much as any other, there must be a certain amount of this romantic picture-making. Men have never done anything in the world without it; but ours is much more of a reality as well as a romance than the dreams of the other romantics. There cannot be a nation of millionaires, and there has never yet been a nation of Utopian comrades; but there have been any number of nations of tolerably contented peasants. In this connection, however, the point is that if we do not directly demand the religion of small property, we must at least demand the poetry of small property. It is a thing about which it is definitely and even urgently practical to be poetical. And it is those who blame us for being poetical who do not really see the practical problem.

For the practical problem is the goal. The pioneer notion has weakened like the progressive notion, and for the same reason. People could go on talking about progress so long as they were not merely thinking about progress. Progressives really had in their minds some notion of a purpose in progress; and even the most practical pioneer had some vague and shadowy idea of what he wanted. The Progressives trusted the tendency of their time, because they did believe, or at least had believed, in a body of democratic doctrines which they supposed to be in process of establishment. And the pioneers and empire-builders were filled with hope and courage because, to do them justice, most of them did at least in some dim way believe that the flag they carried stood for law and liberty, and a higher civilization. They were therefore in search of something and not merely in search of searching. They subconsciously conceived an end of travel and not endless travelling; they were not only breaking through a jungle but building a city. They knew more or less the style of architecture in which it would be built, and they honestly believed it was the best style of architecture in the world. The spirit of adventure has failed because it has been left to adventurers. Adventure for adventure’s sake became like art for art’s sake. Those who had lost all sense of aim lost all sense of art and even of accident. The time has come in every department, but especially in our department, to make once again vivid and solid the aim of political progress or colonial adventure. Even if we picture the goal of the pilgrimage as a sort of peasant paradise, it will be far more practical than setting out on a pilgrimage which has no goal. But it is yet more practical to insist that we do not want to insist only on what are called the qualities of a pioneer; that we do not want to describe merely the virtues that achieve adventures. We want men to think, not merely of a place which they would be interested to find, but of a place where they would be contented to stay. Those who wish merely to arouse again the social hopes of the nineteenth century must offer not an endless hope, but the hope of an end. Those who wish to continue the building of the old colonial idea must leave off telling us that the Church of Empire is founded entirely on the rolling stone. For it is a sin against the reason to tell men that to travel hopefully is better than to arrive; and when once they believe it, they travel hopefully no longer.

\setcounter{chapter}{0}\part{A Summary}
\label{chapter-22}
I once debated with a learned man who had a curious fancy for arranging the correspondence in mathematical patterns; first a thousand words each and then a hundred words each—and then altering them all to another pattern. I accepted as I would always accept a challenge, especially an apparent appeal for fairness, but I was tempted to tell him how utterly unworkable this mechanical method is for a living thing like argument. Obviously a man might need a thousand words to reply to ten words. Suppose I began the philosophic dialogue by saying, “You strangle babies.” He would naturally reply, “Nonsense—I never strangled any babies.” And even in that obvious ejaculation he has already used twice as many words as I have. It is impossible to have real debate without digression. Every definition will look like a digression. Suppose somebody puts to me some journalistic statement, say, “Spanish Jesuits denounced in Parliament.” I cannot deal with it without explaining to the journalist where I differ from him about the atmosphere and implication of each term in turn. I cannot answer quickly if I am just discovering slowly that the man suffers from a series of extraordinary delusions: as (1) that Parliament is a popular representative assembly; (2) that Spain is an effete and decadent country; or (3) that a Spanish Jesuit is a sort of soft-footed court chaplain; whereas it was a Spanish Jesuit who anticipated the whole democratic theory of our day, and actually hurled it as a defiance against the divine right of kings. Each of these explanations would have to be a digression, and each would be necessary. Now in this book I am well aware that there are many digressions that may not at first sight seem to be necessary. For I have had to construct it out of what was originally a sort of controversial causerie; and it has proved impossible to cut down the causerie and only leave the controversy. Moreover, no man can controvert with many foes without going into many subjects, as every one knows who has been heckled. And on this occasion I was, I am happy to say, being heckled by many foes who were also friends. I was discharging the double function of writing essays and of talking over the tea-table, or preferably over the tavern table. To turn this sort of mixture of a gossip and a gospel into anything like a grammar of Distributism has been quite impossible. But I fancy that, even considered as a string of essays, it appears more inconsequent than it really is; and many may read the essays without quite seeing the string. I have decided, therefore, to add this last essay merely in order to sum up the intention of the whole; even if the summary be only a recapitulation. I have had a reason for many of my digressions, which may not appear until the whole is seen in some sort of perspective; and where the digression has no such justification, but was due to a desire to answer a friend or (what is even worse) a disposition towards idle and unseemly mirth, I can only apologize sincerely to the scientific reader and promise to do my best to make this final summary as dull as possible.

If we proceed as at present in a proper orderly fashion, the very idea of property will vanish. It is not revolutionary violence that will destroy it. It is rather the desperate and reckless habit of not having a revolution. The world will be occupied, or rather is already occupied, by two powers which are now one power. I speak, of course, of that part of the world that is covered by our system, and that part of the history of the world which will last very much longer than our time. Sooner or later, no doubt, men would rediscover so natural a pleasure as property. But it might be discovered after ages, like those ages filled with pagan slavery. It might be discovered after a long decline of our whole civilization. Barbarians might rediscover it and imagine it was a new thing.

Anyhow, the prospect is a progress towards the complete combination of two combinations. They are both powers that believe only in combination; and have never understood or even heard that there is any dignity in division. They have never had the imagination to understand the idea of Genesis and the great myths: that Creation itself was division. The beginning of the world was the division of heaven and earth; the beginning of humanity was the division of man and woman. But these flat and platitudinous minds can never see the difference between the creative cleavage of Adam and Eve and the destructive cleavage of Cain and Abel. Anyhow, these powers or minds are now both in the same mood; and it is a mood of disliking all division, and therefore all distribution. They believe in unity, in unanimity, in harmony. One of these powers is State Socialism and the other is Big Business. They are already one spirit; they will soon be one body. For, disbelieving in division, they cannot remain divided; believing only in combination, they will themselves combine. At present one of them calls it Solidarity and the other calls it Consolidation. It would seem that we have only to wait while both monsters are taught to say Consolidarity. But, whatever it is called, there will be no doubt about the character of the world which they will have made between them. It is becoming more and more fixed and familiar. It will be a world of organization, or syndication, of standardization. People will be able to get hats, houses, holidays, and patent medicines of a recognized and universal pattern; they will be fed, clothed, educated, and examined by a wide and elaborate system; but if you were to ask them at any given moment whether the agency which housed or hatted them was still merely mercantile or had become municipal, they probably would not know, and they possibly would not care.

Many believe that humanity will be happy in this new peace; that classes can be reconciled and souls set at rest. I do not think things will be quite so bad as that. But I admit that there are many things which may make possible such a catastrophe of contentment. Men in large numbers have submitted to slavery; men submit naturally to government, and perhaps even especially to despotic government. But I take it as obvious to any intelligent person that this government will be something more than despotic. It is the very essence of the Trust that it has the power, not only to extinguish military rivalry or mob rebellion as has the State, but also the power to crush any new custom or costume or craft or private enterprise that it does not choose to like. Militarism can only prevent people from fighting; but monopoly can prevent them from buying or selling anything except the article (generally the inferior article) having the trade mark of the monopoly. If anything can be inferred from history and human nature, it is absolutely certain that the despotism will grow more and more despotic, and that the article will grow more and more inferior. There is no conceivable argument from psychology, by which it can be pretended that people preserving such a power, generation after generation, would not abuse it more and more, or neglect everything else more and more. We know what far less rigid rule has become, even when founded by spirited and intelligent rulers. We can darkly guess the effect of larger powers in the hands of lesser men. And if the name of Caesar came at last to stand for all that we call Byzantine, exactly what degree of dullness are we to anticipate when the name of Harrod shall sound even duller than it does? If China passed into a proverb at last for stiffness and monotony after being nourished for centuries by Confucius, what will be the condition of the brains that have been nourished for centuries by Callisthenes?

I leave out there the particular case of my own country, where we are threatened not with a long decline, but rather with an unpleasantly rapid collapse. But taking monopolist capitalism in a country where it is still in the vulgar sense successful, as in the United States, we only see more clearly, and on a more colossal scale, the long and descending perspectives that point down to Byzantium or Pekin. It is perfectly obvious that the whole business is a machine for manufacturing tenth-rate things, and keeping people ignorant of first-rate things. Most civilized systems have declined from a height; but this starts on a low level and in a flat place; and what it would be like when it had really crushed all its critics and rivals and made its monopoly watertight for two hundred years, the most morbid imagination will find it hard to imagine. But whatever the last stage of the story, no sane man any longer doubts that we are seeing the first stages of it. There is no longer any difference in tone and type between collectivist and ordinary commercial order; commerce has its officialism and communism has its organization. Private things are already public in the worst sense of the word; that is, they are impersonal and dehumanized. Public things are already private in the worst sense of the word; that is, they are mysterious and secretive and largely corrupt. The new sort of Business Government will combine everything that is bad in all the plans for a better world. There will be no eccentricity; no humour; no noble disdain of the world. There will be nothing but a loathsome thing called Social Service; which means slavery without loyalty. This Service will be one of the ideals. I forgot to mention that there will be ideals. All the wealthiest men in the movement have made it quite clear that they are in possession of a number of these little comforts. People always have ideals when they can no longer have ideas.

The philanthropists in question will probably be surprised to learn that some of us regard this prospect very much as we should regard the theory that we are to be evolved back into apes. We therefore consider whether it is even yet conceivable to restore that long-forgotten thing called Self-Government: that is, the power of the citizen in some degree to direct his own life and construct his own environment; to eat what he likes, to wear what he chooses, and to have (what the Trust must of necessity deny him) a range of choice. In these notes upon the notion, I have been concerned to ask whether it is possible to escape from this enormous evil of simplification or centralization, and what I have said is best summed up under two heads or in two parallel statements. They may seem to some to contradict each other, but they really confirm each other.

First, I say that this is a thing that could be done by people. It is not a thing that can be done to people. That is where it differs from nearly all Socialist schemes as it does from plutocratic philanthropy. I do not say that I, regarding this prospect with hatred and contempt, can save them from it. I say that they can save me from it, and themselves from it, if they also regard it with hatred and contempt. But it must be done in the spirit of a religion, of a revolution, and (I will add) of a renunciation. They must want to do it as they want to drive invaders out of a country or to stop the spread of a plague. And in this respect our critics have a curious way of arguing in a circle. They ask why we trouble to denounce what we cannot destroy; and offer an ideal we cannot attain. They say we are merely throwing away dirty water before we can get clean; or rather that we are merely analysing the animalculae in the dirty water, while we do not even venture to throw it away. Why do we make men discontented with conditions with which they must be content? Why revile an intolerable slavery that must be tolerated? But when we in turn ask why our ideal is impossible or why the evil is indestructible, they answer in effect, “Because you cannot persuade people to want it destroyed.” Possibly; but, on their own showing, they cannot blame us because we try. They cannot say that people do not hate plutocracy enough to kill it; and then blame us for asking them to look at it enough to hate it. If they will not attack it until they hate it, then we are doing the most practical thing we can do, in showing it to be hateful. A moral movement must begin somewhere; but I do most positively postulate that there must be a moral movement. This is not a financial flutter or a police regulation or a private bill or a detail of book-keeping. It is a mighty effort of the will of man, like the throwing off of any other great evil, or it is nothing. I say that if men will fight for this they may win; I have nowhere suggested that there is any way of winning without fighting.

Under this heading I have considered in their place, for instance, the possibility of an organized boycott of big shops. Undoubtedly it would be some sacrifice to boycott big shops; it would be some trouble to seek out small shops. But it would be about a hundredth part of the sacrifice and trouble that has often been shown by masses of men making some patriotic or religious protest—when they really wanted to protest. Under the same general rule, I have remarked that a real life on the land, men not only dwelling on the land but living off it, would be an adventure involving both stubbornness and abnegation. But it would not be half so ascetic as the sort of adventure which it is a commonplace to attribute to colonists and empire-builders; it is nothing to what has been normally shown by millions of soldiers and monks. Only it is true that monks have a faith, that soldiers have a flag, and that even empire-builders were presumably under the impression that they could assist the Empire. But it does not seem to me quite inconceivable, in the varieties of religious experience, that men might take as much notice of earth as monks do of heaven; that people might really believe in the spades that create as well as in the swords that destroy; and that the English who have colonized everywhere else might begin to colonize England.

Having thus admitted, or rather insisted, that this thing cannot be done unless people do really think it worth doing, I then proceeded to suggest that, even in these different departments, there are more people who think it worth doing than is noticed by the people who do not think it worth noticing. Thus, even in the crowds that throng the big shops, you do in fact hear a vast amount of grumbling at the big shops—not so much because they are big as because they are bad. But these real criticisms are disconnected, while the unreal puffs and praises are connected, like any other conspiracy. When the millionaire owning the stores is criticized, it is by his customers. When he is handsomely complimented, it is by himself. But when he is cursed, it is in the inner chamber; when he is praised (by himself) it is proclaimed from the house-tops. That is what is meant by publicity—a voice loud enough to drown any remarks made by the public.

In the case of the land, as in the case of the shops, I went on to point out that there is, if not a moral agitation, at least the materials of a moral agitation. Just as a discontent with the shops lingers even among those who are shopping, so a desire for the land lingers even in those who are hardly allowed to walk on the ground. I gave the instance of the slum population of Limehouse, who were forcibly lifted into high flats, bitterly lamenting the loss of the funny little farmyards they had constructed for themselves in the corners of their slum. It seems absurd to say of a country that none of its people could be countrymen, when even its cockneys try to be countrymen. I also noted that, in the case of the country, there is now a general discontent, in landlords as well as tenants. Everything seems to point to a simpler life of one man one field, free as far as possible of the complications of rent and labour, especially when the rent is so often unpaid or unprofitable, and the labourers are so often on strike or on the dole. Here again there may often be a million individuals feeling like this; but the million has not become a mob; for a mob is a moral thing. But I will never be so unpatriotic as to suggest that the English could never conduct an agrarian war in England as the Irish did in Ireland. Generally, therefore, under this first principle, the thing would most certainly have to be preached rather like a Crusade; but it is quite untrue and unhistorical to say, as a rule, that when once the Crusade is preached, there are no Crusaders.

And my second general principle, which may seem contradictory but is confirmatory, is this. I think the thing would have to be done step by step and with patience and partial concessions. I think this, not because I have any faith whatever in the silly cult of slowness that is sometimes called evolution, but because of the peculiar circumstances of the case. First, mobs may loot and burn and rob the rich man, very much to his spiritual edification and benefit. They may not unnaturally do it, almost absentmindedly, when they are thinking of something else, such as a dislike of Jews or Huguenots. But it would never do for us to give very violent shocks to the sentiment of property, even where it is very ill-placed or ill-proportioned; for that happens to be the very sentiment we are trying to revive. As a matter of psychology, it would be foolish to insult even an unfeminine feminist in order to awaken a delicate chivalry towards females. It would be unwise to use a sacred image as a club with which to thump an Iconoclast and teach him not to touch the holy images. Where the old-fashioned feeling of property is still honest, I think it should be dealt with by degrees and with some consideration. Where the sense of property does not exist at all, as in millionaires, it might well be regarded rather differently; there it would become a question of whether property procured in certain ways is property at all. As for the case of cornering and making monopolies in restraint of trade, that falls under the first of my two principles. It is simply a question of whether we have the moral courage to punish what is certainly immoral. There is no more doubt about these operations of high finance than there is about piracy on the high seas. It is merely a case of a country being so disorderly and ill-governed that it becomes infested with pirates. I have, therefore, in this book treated of Trusts and Anti-Trust Law as a matter, not merely for the popular protest of a boycott or a strike, but for the direct action of the State against criminals. But when the criminals are stronger than the State, any attempt to punish them will be certainly called a rebellion and may rightly be called a Crusade.

Recurring to the second principle, however, there is another and less abstract reason for recognizing that the goal must be reached by stages. I have here had to consider several things that may bring us a stage nearer to Distributism, even if they are in themselves not very satisfactory to ardent or austere Distributists. I took the examples of a Ford car, which may be made by mass production but is used for individual adventure; for, after all, a private car is more private than a train or a tram. I also took the example of a general supply of electricity, which might lead to many little workshops having a chance for the first time. I do not claim that all Distributists would agree with me in my decision here; but on the whole I am inclined to decide that we should use these things to break up the hopeless block of concentrated capital and management, even if we urge their abandonment when they have done their work. We are concerned to produce a particular sort of men, the sort of men who will not worship machines even if they use machines. But it is essential to insist at every stage that we hold ourselves free not only to cease worshipping machines, but to cease using them. It was in this connection that I criticized certain remarks of Mr. Ford and the whole of that idea of standardization which he may be said to represent. But everywhere I recognize a difference between the methods we may use to produce a saner society and the things which that saner society might itself be sane enough to do. For instance, a people who had really found out what fun it is to make things would never want to make most of them with a machine. Sculptors do not want to turn a statue out with a lathe or painters to print off a picture as a pattern, and a craftsman who was really capable of making pots or pans would be no readier to condescend to what is called manufacturing them. It is odd, by the way, that the very word “manufacture” means the opposite of what it is supposed to mean. It is itself a testimony to a better time when it did not mean the work of a modern factory. In the strict meaning of words, a sculptor does manufacture a statue, and a factory worker does not manufacture a screw.

But, anyhow, a world in which there were many independent men would probably be a world in which there were more individual craftsmen. When we have created anything like such a world, we may trust it to feel more than the modern world does the danger of machinery deadening creation, and the value of what it deadens. And I suggested that such a world might very well make special provision about machines, as we all do about weapons; admitting them for particular purposes, but keeping watch on them in particular ways.

But all that belongs to the later stage of improvement, when the commonwealth of free men already exists; I do not think it inconsistent with using any instruments that are innocent in themselves in order to help such citizens to find a footing. I have also noted that just as I do not think machinery an immoral instrument in itself, so I do not think State action an immoral instrument in itself. The State might do a great deal in the first stages, especially by education in the new and necessary crafts and labours, by subsidy or tariff to protect distributive experiments and by special laws, such as the taxation of contracts. All these are covered by what I call the second principle, that we may use intermediate or imperfect instruments; but it goes along with the first principle, that we must be perfect not only in our patience, but in our passion and our enduring indignation.

Lastly, there are the ordinary and obvious problems like that of population, and in that connection I fully concede that the process may sooner or later involve an element of emigration. But I think the emigration must be undertaken by those who understand the new England, and not by those who want to escape from it or from the necessity of it. Men must realize the new meaning of the old phrase, “the sacredness of private property.” There must be a spirit that will make the colonist feel at home and not abroad. And there, I admit, there is a difficulty; for I confess I know only one thing that will thus give to a new soil the sanctity of something already old and full of mystical affections. And that thing is a shrine—the real presence of a sacramental religion.

Thus, unavoidably, I end on the note of another controversy—a controversy that I have no idea of pursuing here. But I should not be honest if I did not mention it, and whatever be the case in that connection it is impossible to deny that there is a doctrine behind the whole of our political position. It is not necessarily the doctrine of the religious authority which I myself receive; but it cannot be denied that it must in a sense be religious. That is to say, it must at least have some reference to an ultimate view of the universe and especially of the nature of man. Those who are thus ready to see property atrophied would ultimately be ready to see arms and legs amputated. They really believe that these could become extinct organs like the appendix. In other words, there is indeed a fundamental difference between my own view and that vision of man as a merely intermediate and changing thing—a Link, if not a Missing Link. The creature, it is claimed, once went on four legs and now goes on two legs. The obvious inference would be that the next stage of evolution will be for a man to stand on one leg. And this will be of very great value to the capitalist or bureaucratic powers that are now to take charge of him. It will mean, for one thing, that only half the number of boots need be supplied to the working classes. It will mean that all wages will be of a one-legged sort. But I would testify at the end, as at the beginning, that I believe in Man standing on two legs and requiring two boots, and that I desire them to be his own boots. You may call it conservative to want this. You may call it revolutionary to attempt to get it. But if that is conservative, I am conservative; if that is revolutionary, I am revolutionary—but too democratic to be evolutionary, anyhow.

The thing behind Bolshevism and many other modern things is a new doubt. It is not merely a doubt about God; it is rather specially a doubt about Man. The old morality, the Christian religion, the Catholic Church, differed from all this new mentality because it really believed in the rights of men. That is, it believed that ordinary men were clothed with powers and privileges and a kind of authority. Thus the ordinary man had a right to deal with dead matter, up to a given point; that is the right of property. Thus the ordinary man had a right to rule the other animals within reason; that is the objection to vegetarianism and many other things. The ordinary man had a right to judge about his own health, and what risks he would take with the ordinary things of his environment; that is the objection to Prohibition and many other things. The ordinary man had a right to judge of his children’s health, and generally to bring up children to the best of his ability; that is the objection to many interpretations of modern State education. Now in these primary things in which the old religion trusted a man, the new philosophy utterly distrusts a man. It insists that he must be a very rare sort of man to have any rights in these matters; and when he is the rare sort, he has the right to rule others even more than himself. It is this profound scepticism about the common man that is the common point in the most contradictory elements of modern thought. That is why Mr. Bernard Shaw wants to evolve a new animal that shall live longer and grow wiser than man. That is why Mr. Sidney Webb wants to herd the men that exist like sheep, or animals much more foolish than man. They are not rebelling against an abnormal tyranny; they are rebelling against what they think is a normal tyranny—the tyranny of the normal. They are not in revolt against the King. They are in revolt against the Citizen. The old revolutionist, when he stood on the roof (like the revolutionist in The Dynamiter) and looked over the city, used to say to himself, “Think how the princes and nobles revel in their palaces; think how the captains and cohorts ride the streets and trample on the people.” But the new revolutionist is not brooding on that. He is saying, “Think of all those stupid men in vulgar villas or ignorant slums. Think how badly they teach their children; think how they do the wrong thing to the dog and offend the feelings of the parrot.” In short, these sages, rightly or wrongly, cannot trust the normal man to rule in the home, and most certainly do not want him to rule in the State. They do not really want to give him any political power. They are willing to give him a vote, because they have long discovered that it need not give him any power. They are not willing to give him a house, or a wife, or a child, or a dog, or a cow, or a piece of land, because these things really do give him power.

Now we wish it to be understood that our policy is to give him power by giving him these things. We wish to insist that this is the real moral division underlying all our disputes, and perhaps the only one really worth disputing. We are far from denying, especially at this time, that there is much to be said on the other side. We alone, perhaps, are likely to insist in the full sense that the average respectable citizen ought to have something to rule. We alone, to the same extent and for the same reason, have the right to call ourselves democratic. A republic used to be called a nation of kings, and in our republic the kings really have kingdoms. All modern governments, Prussian or Russian, all modern movements, Capitalist or Socialist, are taking away that kingdom from the king. Because they dislike the independence of that kingdom, they are against property. Because they dislike the loyalty of that kingdom, they are against marriage.

It is therefore with a somewhat sad amusement that I note the soaring visions that accompany the sinking wages. I observe that the social prophets are still offering the homeless something much higher and purer than a home, and promising a supernormal superiority to people who are not allowed to be normal. I am quite content to dream of the old drudgery of democracy, by which as much as possible of a human life should be given to every human being; while the brilliant author of The First Men in the Moon will doubtless be soon deriding us in a romance called The Last Men on the Earth. And indeed I do believe that when they lose the pride of personal ownership they will lose something that belongs to their erect posture and to their footing and poise upon the planet. Meanwhile I sit amid droves of overdriven clerks and underpaid workmen in a tube or a tram; I read of the great conception of Men Like Gods and I wonder when men will be like men.



\end{document}
