\documentclass{book}
\usepackage{fontspec}
\usepackage{xunicode}
\usepackage[english]{babel}
\usepackage{fancyhdr}
\usepackage[htt]{hyphenat}
\usepackage[a5paper, top=2cm, bottom=1.5cm, left=2.5cm,right=1.5cm]{geometry}
\makeatletter
\date{}
\pagestyle{fancy}
\fancyhead{}
\fancyhead[CO,CE]{\thepage}
\fancyfoot{}
\makeatother
\title{Utopia of Usurers}
\author{G. K. Chesterton}
\begin{document}
\thispagestyle{empty}
\vspace*{\stretch{1}}
\begin{center}
	{\Huge \@title   \\[5mm]}
\end{center}
\vspace*{\stretch{2}}
\newpage
\thispagestyle{empty}
\cleardoublepage
\begin{center}
	\thispagestyle{empty}
	\vspace*{\baselineskip}
	\rule{\textwidth}{1.6pt}\vspace*{-\baselineskip}\vspace*{2pt}
	\rule{\textwidth}{0.4pt}\\[\baselineskip]
	{\Huge\scshape \@title   \\[5mm]}
	{\Large }
	\rule{\textwidth}{0.4pt}\vspace*{-\baselineskip}\vspace{3.2pt}
	\rule{\textwidth}{1.6pt}\\[\baselineskip]
	\vspace*{4\baselineskip}
	{\Large \@author}
	\vfill
\end{center}
\pagebreak
\newpage
\thispagestyle{empty}
\null\vfill
\noindent
\begin{center}
	{\emph{\@title}, © \@author.\\[5mm]}
	{This work is free of known copyright restrictions.\\[5mm]}
\end{center}
\pagebreak
\newpage
\setcounter{tocdepth}{0}
\setcounter{secnumdepth}{0}

\chapter{Art and Advertisement}
\label{chapter-0}
I propose, subject to the patience of the reader, to devote two or three articles to prophecy. Like all healthy-minded prophets, sacred and profane, I can only prophesy when I am in a rage and think things look ugly for everybody. And like all healthy-minded prophets, I prophesy in the hope that my prophecy may not come true. For the prediction made by the true soothsayer is like the warning given by a good doctor. And the doctor has really triumphed when the patient he condemned to death has revived to life. The threat is justified at the very moment when it is falsified. Now I have said again and again (and I shall continue to say again and again on all the most inappropriate occasions) that we must hit Capitalism, and hit it hard, for the plain and definite reason that it is growing stronger. Most of the excuses which serve the capitalists as masks are, of course, the excuses of hypocrites. They lie when they claim philanthropy; they no more feel any particular love of men than Albu felt an affection for Chinamen. They lie when they say they have reached their position through their own organising ability. They generally have to pay men to organise the mine, exactly as they pay men to go down it. They often lie about their present wealth, as they generally lie about their past poverty. But when they say that they are going in for a “constructive social policy,” they do not lie. They really are going in for a constructive social policy. And we must go in for an equally destructive social policy; and destroy, while it is still half-constructed, the accursed thing which they construct.

\section{The Example of the Arts}
Now I propose to take, one after another, certain aspects and departments of modern life, and describe what I think they will be like in this paradise of plutocrats, this Utopia of gold and brass in which the great story of England seems so likely to end. I propose to say what I think our new masters, the mere millionaires, will do with certain human interests and institutions, such as art, science, jurisprudence, or religion—unless we strike soon enough to prevent them. And for the sake of argument I will take in this article the example of the arts.

Most people have seen a picture called “Bubbles,” which is used for the advertisement of a celebrated soap, a small cake of which is introduced into the pictorial design. And anybody with an instinct for design (the caricaturist of the \emph{Daily Herald}, for instance), will guess that it was not originally a part of the design. He will see that the cake of soap destroys the picture as a picture; as much as if the cake of soap had been used to scrub off the paint. Small as it is, it breaks and confuses the whole balance of objects in the composition. I offer no judgment here upon Millais’s action in the matter; in fact, I do not know what it was. The important point for me at the moment is that the picture was not painted for the soap, but the soap added to the picture. And the spirit of the corrupting change which has separated us from that Victorian epoch can be best seen in this: that the Victorian atmosphere, with all its faults, did not permit such a style of patronage to pass as a matter of course. Michael Angelo may have been proud to have helped an emperor or a pope; though, indeed, I think he was prouder than they were on his own account. I do not believe Sir John Millais was proud of having helped a soap-boiler. I do not say he thought it wrong; but he was not proud of it. And that marks precisely the change from his time to our own. Our merchants have really adopted the style of merchant princes. They have begun openly to dominate the civilisation of the State, as the emperors and popes openly dominated in Italy. In Millais’s time, broadly speaking, art was supposed to mean good art; advertisement was supposed to mean inferior art. The head of a black man, painted to advertise somebody’s blacking, could be a rough symbol, like an inn sign. The black man had only to be black enough. An artist exhibiting the picture of a Negro was expected to know that a black man is not so black as he is painted. He was expected to render a thousand tints of grey and brown and violet: for there is no such thing as a black man just as there is no such thing as a white man. A fairly clear line separated advertisement from art.

\section{The First Effect}
I should say the first effect of the triumph of the capitalist (if we allow him to triumph) will be that that line of demarcation will entirely disappear. There will be no art that might not just as well be advertisement. I do not necessarily mean that there will be no good art; much of it might be, much of it already is, very good art. You may put it, if you please, in the form that there has been a vast improvement in advertisements. Certainly there would be nothing surprising if the head of a Negro advertising Somebody’s Blacking nowadays were finished with as careful and subtle colours as one of the old and superstitious painters would have wasted on the negro king who brought gifts to Christ. But the improvement of advertisements is the degradation of artists. It is their degradation for this clear and vital reason: that the artist will work, not only to please the rich, but only to increase their riches; which is a considerable step lower. After all, it was as a human being that a pope took pleasure in a cartoon of Raphael or a prince took pleasure in a statuette of Cellini. The prince paid for the statuette; but he did not expect the statuette to pay him. It is my impression that no cake of soap can be found anywhere in the cartoons which the Pope ordered of Raphael. And no one who knows the small-minded cynicism of our plutocracy, its secrecy, its gambling spirit, its contempt of conscience, can doubt that the artist-advertiser will often be assisting enterprises over which he will have no moral control, and of which he could feel no moral approval. He will be working to spread quack medicines, queer investments; and will work for Marconi instead of Medici. And to this base ingenuity he will have to bend the proudest and purest of the virtues of the intellect, the power to attract his brethren, and the noble duty of praise. For that picture by Millais is a very allegorical picture. It is almost a prophecy of what uses are awaiting the beauty of the child unborn. The praise will be of a kind that may correctly be called soap; and the enterprises of a kind that may truly be described as Bubbles.

\chapter{Letters and the New Laureates}
\label{chapter-1}
In these articles I only take two or three examples of the first and fundamental fact of our time. I mean the fact that the capitalists of our community are becoming quite openly the kings of it. In my last (and first) article, I took the case of Art and advertisement. I pointed out that Art must be growing worse—merely because advertisement is growing better. In those days Millais condescended to Pears’ soap. In these days I really think it would be Pears who condescended to Millais. But here I turn to an art I know more about, that of journalism. Only in my case the art verges on artlessness.

The great difficulty with the English lies in the absence of something one may call democratic imagination. We find it easy to realise an individual, but very hard to realise that the great masses consist of individuals. Our system has been aristocratic: in the special sense of there being only a few actors on the stage. And the back scene is kept quite dark, though it is really a throng of faces. Home Rule tended to be not so much the Irish as the Grand Old Man. The Boer War tended not to be so much South Africa as simply “Joe.” And it is the amusing but distressing fact that every class of political leadership, as it comes to the front in its turn, catches the rays of this isolating lime-light; and becomes a small aristocracy. Certainly no one has the aristocratic complaint so badly as the Labour Party. At the recent Congress, the real difference between Larkin and the English Labour leaders was not so much in anything right or wrong in what he said, as in something elemental and even mystical in the way he suggested a mob. But it must be plain, even to those who agree with the more official policy, that for Mr. Havelock Wilson the principal question was Mr. Havelock Wilson; and that Mr. Sexton was mainly considering the dignity and fine feelings of Mr. Sexton. You may say they were as sensitive as aristocrats, or as sulky as babies; the point is that the feeling was personal. But Larkin, like Danton, not only talks like ten thousand men talking, but he also has some of the carelessness of the colossus of Arcis; “Que mon nom soit fletri, que la France soit libra.”

\section{A Dance of Degradation}
It is needless to say that this respecting of persons has led all the other parties a dance of degradation. We ruin South Africa because it would be a slight on Lord Gladstone to save South Africa. We have a bad army. because it would be a snub to Lord Haldane to have a good army. And no Tory is allowed to say “Marconi” for fear Mr. George should say “Kynoch.” But this curious personal element, with its appalling lack of patriotism, has appeared in a new and curious form in another department of life; the department of literature, especially periodical literature. And the form it takes is the next example I shall give of the way in which the capitalists are now appearing, more and more openly, as the masters and princes of the community.

I will take a Victorian instance to mark the change; as I did in the case of the advertisement of “Bubbles.” It was said in my childhood, by the more apoplectic and elderly sort of Tory, that W. E. Gladstone was only a Free Trader because he had a partnership in Gilbey’s foreign wines. This was, no doubt, nonsense; but it had a dim symbolic, or mainly prophetic, truth in it. It was true, to some extent even then, and it has been increasingly true since, that the statesman was often an ally of the salesman; and represented not only a nation of shopkeepers, but one particular shop. But in Gladstone’s time, even if this was true, it was never the whole truth; and no one would have endured it being the admitted truth. The politician was not solely an eloquent and persuasive bagman travelling for certain business men; he was bound to mix even his corruption with some intelligible ideals and rules of policy. And the proof of it is this: that at least it was the statesman who bulked large in the public eye; and his financial backer was entirely in the background. Old gentlemen might choke over their port, with the moral certainty that the Prime Minister had shares in a wine merchant’s. But the old gentleman would have died on the spot if the wine merchant had really been made as important as the Prime Minister. If it had been Sir Walter Gilbey whom Disraeli denounced, or Punch caricatured; if Sir Walter Gilbey’s favourite collars (with the design of which I am unacquainted) had grown as large as the wings of an archangel; if Sir Walter Gilbey had been credited with successfully eliminating the British Oak with his little hatchet; if, near the Temple and the Courts of Justice, our sight was struck by a majestic statue of a wine merchant; or if the earnest Conservative lady who threw a gingerbread-nut at the Premier had directed it towards the wine merchant instead, the shock to Victorian England would have been very great indeed.

\section{Haloes for Employers}
Now something very like that is happening; the mere wealthy employer is beginning to have not only the power but some of the glory. I have seen in several magazines lately, and magazines of a high class, the appearance of a new kind of article. Literary men are being employed to praise a big business man personally, as men used to praise a king. They not only find political reasons for the commercial schemes—that they have done for some time past—they also find moral defences for the commercial schemers. They describe the capitalist’s brain of steel and heart of gold in a way that Englishmen hitherto have been at least in the habit of reserving for romantic figures like Garibaldi or Gordon. In one excellent magazine Mr. T. P. O’Connor, who, when he likes, can write on letters like a man of letters, has some purple pages in praise of Sir Joseph Lyons—the man who runs those tea-shop places. He incidentally brought in a delightful passage about the beautiful souls possessed by some people called Salmon and Gluckstein. I think I like best the passage where he said that Lyons’s charming social accomplishments included a talent for “imitating a Jew.” The article is accompanied with a large and somewhat leering portrait of that shopkeeper, which makes the parlour-trick in question particularly astonishing. Another literary man, who certainly ought to know better, wrote in another paper a piece of hero-worship about Mr. Selfridge. No doubt the fashion will spread, and the art of words, as polished and pointed by Ruskin or Meredith, will be perfected yet further to explore the labyrinthine heart of Harrod; or compare the simple stoicism of Marshall with the saintly charm of Snelgrove.

Any man can be praised—and rightly praised. If he only stands on two legs he does something a cow cannot do. If a rich man can manage to stand on two legs for a reasonable time, it is called self-control. If he has only one leg, it is called (with some truth) self-sacrifice. I could say something nice (and true) about every man I have ever met. Therefore, I do not doubt I could find something nice about Lyons or Self ridge if I searched for it. But I shall not. The nearest postman or cab-man will provide me with just the same brain of steel and heart of gold as these unlucky lucky men. But I do resent the whole age of patronage being revived under such absurd patrons; and all poets becoming court poets, under kings that have taken no oath, nor led us into any battle.

\chapter{Unbusinesslike Business}
\label{chapter-2}
The fairy tales we were all taught did not, like the history we were all taught, consist entirely of lies. Parts of the tale of “Puss in Boots” or “Jack and the Beanstalk” may strike the realistic eye as a little unlikely and out of the common way, so to speak; but they contain some very solid and very practical truths. For instance, it may be noted that both in “Puss in Boots” and “Jack and the Beanstalk,” if I remember aright, the ogre was not only an ogre but also a magician. And it will generally be found that in all such popular narratives, the king, if he is a wicked king, is generally also a wizard. Now there is a very vital human truth enshrined in this. Bad government, like good government, is a spiritual thing. Even the tyrant never rules by force alone; but mostly by fairy tales. And so it is with the modern tyrant, the great employer. The sight of a millionaire is seldom, in the ordinary sense, an enchanting sight: nevertheless he is in his way an enchanter. As they say in the gushing articles about him in the magazines, he is a fascinating personality. So is a snake. At least he is fascinating to rabbits; and so is the millionaire to the rabbit-witted sort of people that ladies and gentlemen have allowed themselves to become. He does, in a manner, cast a spell, such as that which imprisoned princes and princesses under the shapes of falcons or stags. He has truly turned men into sheep, as Circe turned them into swine.

Now, the chief of the fairy tales, by which he gains this glory and glamour, is a certain hazy association he has managed to create between the idea of bigness and the idea of practicality. Numbers of the rabbit-witted ladies and gentlemen do really think, in spite of themselves and their experience, that so long as a shop has hundreds of different doors and a great many hot and unhealthy underground departments (they must be hot; this is very important), and more people than would be needed for a man-of-war, or crowded cathedral, to say: “This way, madam,” and “The next article, sir,” it follows that the goods are good. In short, they hold that the big businesses are businesslike. They are not. Any housekeeper in a truthful mood, that is to say, any housekeeper in a bad temper, will tell you that they are not. But housekeepers, too, are human, and therefore inconsistent and complex; and they do not always stick to truth and bad temper. They are also affected by this queer idolatry of the enormous and elaborate; and cannot help feeling that anything so complicated must go like clockwork. But complexity is no guarantee of accuracy—in clock-work or in anything else. A clock can be as wrong as the human head; and a clock can stop, as suddenly as the human heart.

But this strange poetry of plutocracy prevails over people against their very senses. You write to one of the great London stores or emporia, asking, let us say, for an umbrella. A month or two afterwards you receive a very elaborately constructed parcel, containing a broken parasol. You are very pleased. You are gratified to reflect on what a vast number of assistants and employees had combined to break that parasol. You luxuriate in the memory of all those long rooms and departments and wonder in which of them the parasol that you never ordered was broken. Or you want a toy elephant for your child on Christmas Day; as children, like all nice and healthy people, are very ritualistic. Some week or so after Twelfth Night, let us say, you have the pleasure of removing three layers of pasteboards, five layers of brown paper, and fifteen layers of tissue paper and discovering the fragments of an artificial crocodile. You smile in an expansive spirit. You feel that your soul has been broadened by the vision of incompetence conducted on so large a scale. You admire all the more the colossal and Omnipresent Brain of the Organiser of Industry, who amid all his multitudinous cares did not disdain to remember his duty of smashing even the smallest toy of the smallest child. Or, supposing you have asked him to send you some two rolls of cocoa-nut matting: and supposing (after a due interval for reflection) he duly delivers to you the five rolls of wire netting. You take pleasure in the consideration of a mystery: which coarse minds might have called a mistake. It consoles you to know how big the business is: and what an enormous number of people were needed to make such a mistake.

That is the romance that has been told about the big shops; in the literature and art which they have bought, and which (as I said in my recent articles) will soon be quite indistinguishable from their ordinary advertisements. The literature is commercial; and it is only fair to say that the commerce is often really literary. It is no romance, but only rubbish.

The big commercial concerns of to-day are quite exceptionally incompetent. They will be even more incompetent when they are omnipotent. Indeed, that is, and always has been, the whole point of a monopoly; the old and sound argument against a monopoly. It is only because it is incompetent that it has to be omnipotent. When one large shop occupies the whole of one side of a street (or sometimes both sides), it does so in order that men may be unable to get what they want; and may be forced to buy what they don’t want. That the rapidly approaching kingdom of the Capitalists will ruin art and letters, I have already said. I say here that in the only sense that can be called human, it will ruin trade, too.

I will not let Christmas go by, even when writing for a revolutionary paper necessarily appealing to many with none of my religious sympathies, without appealing to those sympathies. I knew a man who sent to a great rich shop for a figure for a group of Bethlehem. It arrived broken. I think that is exactly all that business men have now the sense to do.

\chapter{The War on Holidays}
\label{chapter-3}
The general proposition, not always easy to define exhaustively, that the reign of the capitalist will be the reign of the cad—that is, of the unlicked type that is neither the citizen nor the gentleman—can be excellently studied in its attitude towards holidays. The special emblematic Employer of to-day, especially the Model Employer (who is the worst sort) has in his starved and evil heart a sincere hatred of holidays. I do not mean that he necessarily wants all his workmen to work until they drop; that only occurs when he happens to be stupid as well as wicked. I do not mean to say that he is necessarily unwilling to grant what he would call “decent hours of labour.” He may treat men like dirt; but if you want to make money, even out of dirt, you must let it lie fallow by some rotation of rest. He may treat men as dogs, but unless he is a lunatic he will for certain periods let sleeping dogs lie.

But humane and reasonable hours for labour have nothing whatever to do with the idea of holidays. It is not even a question of ten-hours day and eight-hours day; it is not a question of cutting down leisure to the space necessary for food, sleep and exercise. If the modern employer came to the conclusion, for some reason or other, that he could get most out of his men by working them hard for only two hours a day, his whole mental attitude would still be foreign and hostile to holidays. For his whole mental attitude is that the passive time and the active time are alike useful for him and his business. All is, indeed, grist that comes to his mill, including the millers. His slaves’ still serve him in unconsciousness, as dogs still hunt in slumber. His grist is ground not only by the sounding wheels of iron, but by the soundless wheel of blood and brain. His sacks are still filling silently when the doors are shut on the streets and the sound of the grinding is low.

\section{The Great Holiday}
Now a holiday has no connection with using a man either by beating or feeding him. When you give a man a holiday you give him back his body and soul. It is quite possible you may be doing him an injury (though he seldom thinks so), but that does not affect the question for those to whom a holiday is holy. Immortality is the great holiday; and a holiday, like the immortality in the old theologies, is a double-edged privilege. But wherever it is genuine it is simply the restoration and completion of the man. If people ever looked at the printed word under their eye, the word “recreation” would be like the word “resurrection,” the blast of a trumpet.

A man, being merely useful, is necessarily incomplete, especially if he be a modern man and means by being useful being “utilitarian.” A man going into a modern club gives up his hat; a man going into a modern factory gives up his head. He then goes in and works loyally for the old firm to build up the great fabric of commerce (which can be done without a head), but when he has done work he goes to the cloak-room, like the man at the club, and gets his head back again; that is the germ of the holiday. It may be urged that the club man who leaves his hat often goes away with another hat; and perhaps it may be the same with the factory hand who has left his head. A hand that has lost its head may affect the fastidious as a mixed metaphor; but, God pardon us all, what an unmixed truth! We could almost prove the whole case from the habit of calling human beings merely “hands” while they are working; as if the hand were horribly cut off, like the hand that has offended; as if, while the sinner entered heaven maimed, his unhappy hand still laboured laying up riches for the lords of hell. But to return to the man whom we found waiting for his head in the cloak-room. It may be urged, we say, that he might take the wrong head, like the wrong hat; but here the similarity ceases. For it has been observed by benevolent onlookers at life’s drama that the hat taken away by mistake is frequently better than the real hat; whereas the head taken away after the hours of toil is certainly worse: stained with the cobwebs and dust of this dustbin of all the centuries.

\section{The Supreme Adventure}
All the words dedicated to places of eating and drinking are pure and poetic words. Even the word “hotel” is the word hospital. And St. Julien, whose claret I drank this Christmas, was the patron saint of innkeepers, because (as far as I can make out) he was hospitable to lepers. Now I do not say that the ordinary hotel-keeper in Piccadilly or the Avenue de l’Opera would embrace a leper, slap him on the back, and ask him to order what he liked; but I do say that hospitality is his trade virtue. And I do also say it is well to keep before our eyes the supreme adventure of a virtue. If you are brave, think of the man who was braver than you. If you are kind, think of the man who was kinder than you.

That is what was meant by having a patron saint. That is the link between the poor saint who received bodily lepers and the great hotel proprietor who (as a rule) receives spiritual lepers. But a word yet weaker than “hotel” illustrates the same point—the word “restaurant.” There again you have the admission that there is a definite building or statue to “restore”; that ineffaceable image of man that some call the image of God. And that is the holiday; it is the restaurant or restoring thing that, by a blast of magic, turns a man into himself.

This complete and reconstructed man is the nightmare of the modern capitalist. His whole scheme would crack across like a mirror of Shallot, if once a plain man were ready for his two plain duties—ready to live and ready to die. And that horror of holidays which marks the modern capitalist is very largely a horror of the vision of a whole human being: something that is not a “hand” or a “head for figures.” But an awful creature who has met himself in the wilderness. The employers will give time to eat, time to sleep; they are in terror of a time to think.

To anyone who knows any history it is wholly needless to say that holidays have been destroyed. As Mr. Belloc, who knows much more history than you or I, recently pointed out in the “Pall Mall Magazine,” Shakespeare’s title of “Twelfth Night: or What You Will” simply meant that a winter carnival for everybody went on wildly till the twelfth night after Christmas. Those of my readers who work for modern offices or factories might ask their employers for twelve days’ holidays after Christmas. And they might let me know the reply.

\chapter{The Church of the Servile State}
\label{chapter-4}
I confess I cannot see why mere blasphemy by itself should be an excuse for tyranny and treason; or how the mere isolated fact of a man not believing in God should be a reason for my believing in Him.

But the rather spinsterish flutter among some of the old Freethinkers has put one tiny ripple of truth in it; and that affects the idea which I wish to emphasise even to monotony in these pages. I mean the idea that the new community which the capitalists are now constructing will be a very complete and absolute community; and one which will tolerate nothing really independent of itself. Now, it is true that any positive creed, true or false, would tend to be independent of itself. It might be Roman Catholicism or Islam or Materialism; but, if strongly held, it would be a thorn in the side of the Servile State. The Muslim thinks all men immortal: the Materialist thinks all men mortal. But the Muslim does not think the rich Sinbad will live forever; but the poor Sinbad will die on his deathbed. The Materialist does not think that Mr. Haeckel will go to heaven, while all the peasants will go to pot, like their chickens. In every serious doctrine of the destiny of men, there is some trace of the doctrine of the equality of men. But the capitalist really depends on some religion of inequality. The capitalist must somehow distinguish himself from human kind; he must be obviously above it—or he would be obviously below it. Take even the least attractive and popular side of the larger religions to-day; take the mere vetoes imposed by Islam on Atheism or Catholicism. The Muslim veto upon intoxicants cuts across all classes. But it is absolutely necessary for the capitalist (who presides at a Licensing Committee, and also at a large dinner), it is absolutely necessary for \emph{him}, to make a distinction between gin and champagne. The Atheist veto upon all miracles cuts across all classes. But it is absolutely necessary for the capitalist to make a distinction between his wife (who is an aristocrat and consults crystal gazers and star gazers in the West End), and vulgar miracles claimed by gipsies or travelling showmen. The Catholic veto upon usury, as defined in dogmatic councils, cuts across all classes. But it is absolutely necessary to the capitalist to distinguish more delicately between two kinds of usury; the kind he finds useful and the kind he does not find useful. The religion of the Servile State must have no dogmas or definitions. It cannot afford to have any definitions. For definitions are very dreadful things: they do the two things that most men, especially comfortable men, cannot endure. They fight; and they fight fair.

Every religion, apart from open devil worship, must appeal to a virtue or the pretence of a virtue. But a virtue, generally speaking, does some good to everybody. It is therefore necessary to distinguish among the people it was meant to benefit those whom it does benefit. Modern broad-mindedness benefits the rich; and benefits nobody else. It was meant to benefit the rich; and meant to benefit no-body else. And if you think this unwarranted, I will put before you one plain question. There are some pleasures of the poor that may also mean profits for the rich: there are other pleasures of the poor which cannot mean profits for the rich? Watch this one contrast, and you will watch the whole creation of a careful slavery.

In the last resort the two things called Beer and Soap end only in a froth. They are both below the high notice of a real religion. But there is just this difference: that the soap makes the factory more satisfactory, while the beer only makes the workman more satisfied. Wait and see if the Soap does not increase and the Beer decrease. Wait and see whether the religion of the Servile State is not in every case what I say: the encouragement of small virtues supporting capitalism, the discouragement of the huge virtues that defy it. Many great religions, Pagan and Christian, have insisted on wine. Only one, I think, has insisted on Soap. You will find it in the New Testament attributed to the Pharisees.

\chapter{Science and the Eugenists}
\label{chapter-5}
The key fact in the new development of plutocracy is that it will use its own blunder as an excuse for further crimes. Everywhere the very completeness of the impoverishment will be made a reason for the enslavement; though the men who impoverished were the same who enslaved. It is as if a highwayman not only took away a gentleman’s horse and all his money, but then handed him over to the police for tramping without visible means of subsistence. And the most monstrous feature in this enormous meanness may be noted in the plutocratic appeal to science, or, rather, to the pseudo-science that they call Eugenics.

The Eugenists get the ear of the humane but rather hazy cliques by saying that the present “conditions” under which people work and breed are bad for the race; but the modern mind will not generally stretch beyond one step of reasoning, and the consequence which appears to follow on the consideration of these “conditions” is by no means what would originally have been expected. If somebody says: “A rickety cradle may mean a rickety baby,” the natural deduction, one would think, would be to give the people a good cradle, or give them money enough to buy one. But that means higher wages and greater equalisation of wealth; and the plutocratic scientist, with a slightly troubled expression, turns his eyes and pince-nez in another direction. Reduced to brutal terms of truth, his difficulty is this and simply this: More food, leisure, and money for the workman would mean a better workman, better even from the point of view of anyone for whom he worked. But more food, leisure, and money would also mean a more independent workman. A house with a decent fire and a full pantry would be a better house to make a chair or mend a clock in, even from the customer’s point of view, than a hovel with a leaky roof and a cold hearth. But a house with a decent fire and a full pantry would also be a better house in which to \emph{refuse} to make a chair or mend a clock—a much better house to do nothing in—and doing nothing is sometimes one of the highest of the duties of man. All but the hard-hearted must be torn with pity for this pathetic dilemma of the rich man, who has to keep the poor man just stout enough to do the work and just thin enough to have to do it. As he stood gazing at the leaky roof and the rickety cradle in a pensive manner, there one day came into his mind a new and curious idea—one of the most strange, simple, and horrible ideas that have ever risen from the deep pit of original sin.

The roof could not be mended, or, at least, it could not be mended much, without upsetting the capitalist balance, or, rather, disproportion in society; for a man with a roof is a man with a house, and to that extent his house is his castle. The cradle could not be made to rock easier, or, at least, not much easier, without strengthening the hands of the poor household, for the hand that rocks the cradle rules the world—to that extent. But it occurred to the capitalist that there was one sort of furniture in the house that could be altered. The husband and wife could be altered. Birth costs nothing, except in pain and valour and such old-fashioned things; and the merchant need pay no more for mating a strong miner to a healthy fishwife than he pays when the miner mates himself with a less robust female whom he has the sentimentality to prefer. Thus it might be possible, by keeping on certain broad lines of heredity, to have some physical improvement without any moral, political, or social improvement. It might be possible to keep a supply of strong and healthy slaves without coddling them with decent conditions. As the mill-owners use the wind and the water to drive their mills, they would use this natural force as something even cheaper; and turn their wheels by diverting from its channel the blood of a man in his youth. That is what Eugenics means; and that is all that it means.

Of the moral state of those who think of such things it does not become us to speak. The practical question is rather the intellectual one: of whether their calculations are well founded, and whether the men of science can or will guarantee them any such physical certainties. Fortunately, it becomes clearer every day that they are, scientifically speaking, building on the shifting sand. The theory of breeding slaves breaks down through what a democrat calls the equality of men, but which even an oligarchist will find himself forced to call the similarity of men. That is, that though it is not true that all men are normal, it is overwhelmingly certain that most men are normal. All the common Eugenic arguments are drawn from extreme cases, which, even if human honour and laughter allowed of their being eliminated, would not by their elimination greatly affect the mass. For the rest, there remains the enormous weakness in Eugenics, that if ordinary men’s judgment or liberty is to be discounted in relation to heredity, the judgment of the judges must be discounted in relation to their heredity. The Eugenic professor may or may not succeed in choosing a baby’s parents; it is quite certain that he cannot succeed in choosing his own parents. All his thoughts, including his Eugenic thoughts, are, by the very principle of those thoughts, flowing from a doubtful or tainted source. In short, we should need a perfectly Wise Man to do the thing at all. And if he were a Wise Man he would not do it.

\chapter{The Evolution of the Prison}
\label{chapter-6}
I have never understood why it is that those who talk most about evolution, and talk it in the very age of fashionable evolutionism, do not see the one way in which evolution really does apply to our modern difficulty. There is, of course, an element of evolutionism in the universe; and I know no religion or philosophy that ever entirely ignored it. Evolution, popularly speaking, is that which happens to unconscious things. They grow unconsciously; or fade unconsciously; or rather, some parts of them grow and some parts of them fade; and at any given moment there is almost always some presence of the fading thing, and some incompleteness in the growing one. Thus, if I went to sleep for a hundred years, like the Sleeping Beauty (I wish I could), I should grow a beard—unlike the Sleeping Beauty. And just as I should grow hair if I were asleep, I should grow grass if I were dead. Those whose religion it was that God was asleep were perpetually impressed and affected by the fact that he had a long beard. And those whose philosophy it is that the universe is dead from the beginning (being the grave of nobody in particular) think that is the way that grass can grow. In any case, these developments only occur with dead or dreaming things. What happens when everyone is asleep is called Evolution. What happens when everyone is awake is called Revolution.

There was once an honest man, whose name I never knew, but whose face I can almost see (it is framed in Victorian whiskers and fixed in a Victorian neck-cloth), who was balancing the achievements of France and England in civilisation and social efficiencies. And when he came to the religious aspect he said that there were more stone and brick churches used in France; but, on the other hand, there are more sects in England. Whether such a lively disintegration is a proof of vitality in any valuable sense I have always doubted. The sun may breed maggots in a dead dog; but it is essential for such a liberation of life that the dog should be unconscious or (to say the least of it) absent-minded. Broadly speaking, you may call the thing corruption, if you happen to like dogs. You may call it evolution, if you happen to like maggots. In either case, it is what happens to things if you leave them alone.

\section{The Evolutionists’ Error}
Now, the modern Evolutionists have made no real use of the idea of evolution, especially in the matter of social prediction. They always fall into what is (from their logical point of view) the error of supposing that evolution knows what it is doing. They predict the State of the future as a fruit rounded and polished. But the whole point of evolution (the only point there is in it) is that no State will ever be rounded and polished, because it will always contain some organs that outlived their use, and some that have not yet fully found theirs. If we wish to prophesy what will happen, we must imagine things now moderate grown enormous; things now local grown universal; things now promising grown triumphant; primroses bigger than sunflowers, and sparrows stalking about like flamingoes.

In other words, we must ask what modern institution has a future before it? What modern institution may have swollen to six times its present size in the social heat and growth of the future? I do not think the Garden City will grow: but of that I may speak in my next and last article of this series. I do not think even the ordinary Elementary School, with its compulsory education, will grow. Too many unlettered people hate the teacher for teaching; and too many lettered people hate the teacher for not teaching. The Garden City will not bear much blossom; the young idea will not shoot, unless it shoots the teacher. But the one flowering tree on the estate, the one natural expansion which I think will expand, is the institution we call the Prison.

\section{Prisons for All}
If the capitalists are allowed to erect their constructive capitalist community, I speak quite seriously when I say that I think Prison will become an almost universal experience. It will not necessarily be a cruel or shameful experience: on these points (I concede certainly for the present purpose of debate) it may be a vastly improved experience. The conditions in the prison, very possibly, will be made more humane. But the prison will be made more humane only in order to contain more of humanity. I think little of the judgment and sense of humour of any man who can have watched recent police trials without realising that it is no longer a question of whether the law has been broken by a crime; but, now, solely a question of whether the situation could be mended by an imprisonment. It was so with Tom Mann; it was so with Larkin; it was so with the poor atheist who was kept in gaol for saying something he had been acquitted of saying: it is so in such cases day by day. We no longer lock a man up for doing something; we lock him up in the hope of his doing nothing. Given this principle, it is evidently possible to make the mere conditions of punishment more moderate, or–(more probably) more secret. There may really be more mercy in the Prison, on condition that there is less justice in the Court. I should not be surprised if, before we are done with all this, a man was allowed to smoke in prison, on condition, of course, that he had been put in prison for smoking.

Now that is the process which, in the absence of democratic protest, will certainly proceed, will increase and multiply and replenish the earth and subdue it. Prison may even lose its disgrace for a little time; it will be difficult to make it disgraceful when men like Larkin can be imprisoned for no reason at all, just as his celebrated ancestor was hanged for no reason at all. But capitalist society, which naturally does not know the meaning of honour, cannot know the meaning of disgrace: and it will still go on imprisoning for no reason at all. Or rather for that rather simple reason that makes a cat spring or a rat run away.

It matters little whether our masters stoop to state the matter in the form that every prison should be a school; or in the more candid form that every school should be a prison. They have already fulfilled their servile principle in the case of the schools. Everyone goes to the Elementary Schools except the few people who tell them to go there. I prophesy that (unless our revolt succeeds) nearly everyone will be going to Prison, with a precisely similar patience.

\chapter{The Lash for Labour}
\label{chapter-7}
If I were to prophesy that two hundred years hence a grocer would have the right and habit of beating the grocer’s assistant with a stick, or that shop girls might be flogged, as they already can be fined, many would regard it as rather a rash remark. It would be a rash remark. Prophecy is always unreliable; unless we except the kind which is avowedly irrational, mystical and supernatural prophecy. But relatively to nearly all the other prophecies that are being made around me to-day, I should say my prediction stood an exceptionally good chance. In short, I think the grocer with the stick is a figure we are far more likely to see than the Superman or the Samurai, or the True Model Employer, or the Perfect Fabian Official, or the citizen of the Collectivist State. And it is best for us to see the full ugliness of the transformation which is passing over our Society in some such abrupt and even grotesque image at the end of it. The beginnings of a decline, in every age of history, have always had the appearance of being reforms. Nero not only fiddled while Rome was burning, but he probably really paid more attention to the fiddle than to the fire. The Roi Soleil, like many other \emph{soleils}, was most splendid to all appearance a little before sunset. And if I ask myself what will be the ultimate and final fruit of all our social reforms, garden cities, model employers, insurances, exchanges, arbitration courts, and so on, then, I say, quite seriously, “I think it will be labour under the lash.”

\section{The Sultan and the Sack}
Let us arrange in some order a number of converging considerations that all point in this direction. (1) It is broadly true, no doubt, that the weapon of the employer has hitherto been the threat of dismissal, that is, the threat of enforced starvation. He is a Sultan who need not order the bastinado, so long as he can order the sack. But there are not a few signs that this weapon is not quite so convenient and flexible a one as his increasing rapacities require. The fact of the introduction of fines, secretly or openly, in many shops and factories, proves that it is convenient for the capitalists to have some temporary and adjustable form of punishment besides the final punishment of pure ruin. Nor is it difficult to see the common-sense of this from their wholly inhuman point of view. The act of sacking a man is attended with the same disadvantages as the act of shooting a man: one of which is that you can get no more out of him. It is, I am told, distinctly annoying to blow a fellow creature’s brains out with a revolver and then suddenly remember that he was the only person who knew where to get the best Russian cigarettes. So our Sultan, who is the orderer of the sack, is also the bearer of the bow-string. A school in which there was no punishment, except expulsion, would be a school in which it would be very difficult to keep proper discipline; and the sort of discipline on which the reformed capitalism will insist will be all of the type which in free nations is imposed only on children. Such a school would probably be in a chronic condition of breaking up for the holidays. And the reasons for the insufficiency of this extreme instrument are also varied and evident. The materialistic Sociologists, who talk about the survival of the fittest and the weakest going to the wall (and whose way of looking at the world is to put on the latest and most powerful scientific spectacles, and then shut their eyes), frequently talk as if a workman were simply efficient or non-efficient, as if a criminal were reclaimable or irreclaimable. The employers have sense enough at least to know better than that. They can see that a servant may be useful in one way and exasperating in another; that he may be bad in one part of his work and good in another; that he may be occasionally drunk and yet generally indispensable. Just as a practical school-master would know that a schoolboy can be at once the plague and the pride of the school. Under these circumstances small and varying penalties are obviously the most convenient things for the person keeping order; an underling can be punished for coming late, and yet do useful work when he comes. It will be possible to give a rap over the knuckles without wholly cutting off the right hand that has offended. Under these circumstances the employers have naturally resorted to fines. But there is a further ground for believing that the process will go beyond fines before it is completed.

(2) The fine is based on the old European idea that everybody possesses private property in some reasonable degree; but not only is this not true to-day, but it is not being made any truer, even by those who honestly believe that they are mending matters. The great employers will often do something towards improving what they call the “conditions” of their workers; but a worker might have his conditions as carefully arranged as a racehorse has, and still have no more personal property than a racehorse. If you take an average poor seamstress or factory girl, you will find that the power of chastising her through her property has very considerable limits; it is almost as hard for the employer of labour to tax her for punishment as it is for the Chancellor of the Exchequer to tax her for revenue. The next most obvious thing to think of, of course, would be imprisonment, and that might be effective enough under simpler conditions. An old-fashioned shop-keeper might have locked up his apprentice in his coal-cellar; but his coal-cellar would be a real, pitch dark coal-cellar, and the rest of his house would be a real human house. Everybody (especially the apprentice) would see a most perceptible difference between the two. But, as I pointed out in the article before this, the whole tendency of the capitalist legislation and experiment is to make imprisonment much more general and automatic, while making it, or professing to make it, more humane. In other words, the hygienic prison and the servile factory will become so uncommonly like each other that the poor man will hardly know or care whether he is at the moment expiating an offence or merely swelling a dividend. In both places there will be the same sort of shiny tiles. In neither place will there be any cell so unwholesome as a coal-cellar or so wholesome as a home. The weapon of the prison, therefore, like the weapon of the fine, will be found to have considerable limitations to its effectiveness when employed against the wretched reduced citizen of our day. Whether it be property or liberty you cannot take from him what he has not got. You cannot imprison a slave, because you cannot enslave a slave.

\section{The Barbarous Revival}
 (3) Most people, on hearing the suggestion that it may come to corporal punishment at last (as it did in every slave system I ever heard of, including some that were generally kindly, and even successful), will merely be struck with horror and incredulity, and feel that such a barbarous revival is unthinkable in the modern atmosphere. How far it will be, or need be, a revival of the actual images and methods of ruder times I will discuss in a moment. But first, as another of the converging lines tending to corporal punishment, consider this: that for some reason or other the old full-blooded and masculine humanitarianism in this matter has weakened and fallen silent; it has weakened and fallen silent in a very curious manner, the precise reason for which I do not altogether understand. I knew the average Liberal, the average Nonconformist minister, the average Labour Member, the average middle-class Socialist, were, with all their good qualities, very deficient in what I consider a respect for the human soul. But I did imagine that they had the ordinary modern respect for the human body. The fact, however, is clear and incontrovertible. In spite of the horror of all humane people, in spite of the hesitation even of our corrupt and panic-stricken Parliament, measures can now be triumphantly passed for spreading or increasing the use of physical torture, and for applying it to the newest and vaguest categories of crime. Thirty or forty years ago, nay, twenty years ago, when Mr. F. Hugh O’Donnell and others forced a Liberal Government to drop the cat-’o-nine-tails like a scorpion, we could have counted on a mass of honest hatred of such things. We cannot count on it now.

(4) But lastly, it is not necessary that in the factories of the future the institution of physical punishment should actually \emph{remind} people of the jambok or the knout. It could easily be developed out of the many forms of physical discipline which are already used by employers on the excuses of education or hygiene. Already in some factories girls are obliged to swim whether they like it or not, or do gymnastics whether they like it or not. By a simple extension of hours or complication of exercises a pair of Swedish clubs could easily be so used as to leave their victim as exhausted as one who had come off the rack. I think it extremely likely that they will be.

\chapter{The Mask of Socialism}
\label{chapter-8}
The chief aim of all honest Socialists just now is to prevent the coming of Socialism. I do not say it as a sneer, but, on the contrary, as a compliment; a compliment to their political instinct and public spirit. I admit it may be called an exaggeration; but there really is a sort of sham Socialism that the modern politicians may quite possibly agree to set up; if they do succeed in setting it up, the battle for the poor is lost.

We must note, first of all, a general truth about the curious time we live in. It will not be so difficult as some people may suppose to make the Servile State \emph{look} rather like Socialism, especially to the more pedantic kind of Socialist. The reason is this. The old lucid and trenchant expounder of Socialism, such as Blatchford or Fred Henderson, always describes the economic power of the plutocrats as consisting in private property. Of course, in a sense, this is quite true; though they too often miss the point that private property, as such, is not the same as property confined to the few. But the truth is that the situation has grown much more subtle; perhaps too subtle, not to say too insane, for straight-thinking theorists like Blatchford. The rich man to-day does not only rule by using private property; he also rules by treating public property as if it were private property. A man like Lord Murray pulled the strings, especially the purse-strings; but the whole point of his position was that all sorts of strings had got entangled. The secret strength of the money he held did not lie merely in the fact that it was his money. It lay precisely in the fact that nobody had any clear idea of whether it was his money, or his successor’s money, or his brother’s money, or the Marconi Company’s money, or the Liberal Party’s money, or the English Nation’s money. It was buried treasure; but it was not private property. It was the acme of plutocracy because it was not private property. Now, by following this precedent, this unprincipled vagueness about official and unofficial moneys by the cheerful habit of always mixing up the money in the pocket with the money in the till, it would be quite possible to keep the rich as rich as ever in practice, though they might have suffered confiscation in theory. Mr. Lloyd George has four hundred a year as an M. P.; but he not only gets much more as a Minister, but he might at any time get immeasurably more by speculating on State secrets that are necessarily known to him. Some say that he has even attempted something of the kind. Now, it would be quite possible to cut Mr. George down, not to four hundred a year, but to fourpence a day; and still leave him all these other and enormous financial superiorities. It must be remembered that a Socialist State, in any way resembling a modern State, must, however egalitarian it may be, have the handling of huge sums, and the enjoyment of large conveniences; it is not improbable that the same men will handle and enjoy in much the same manner, though in theory they are doing it as instruments, and not as individuals. For instance, the Prime Minister has a private house, which is also (I grieve to inform that eminent Puritan) a public house. It is supposed to be a sort of Government office; though people do not generally give children’s parties, or go to bed in a Government office. I do not know where Mr. Herbert Samuel lives; but I have no doubt he does himself well in the matter of decoration and furniture. On the existing official parallel there is no need to move any of these things in order to Socialise them. There is no need to withdraw one diamond-headed nail from the carpet; or one golden teaspoon from the tray. It is only necessary to call it an official residence, like 10 Downing-street. I think it is not at all improbable that this Plutocracy, pretending to be a Bureaucracy, will be attempted or achieved. Our wealthy rulers will be in the position which grumblers in the world of sport sometimes attribute to some of the “gentlemen” players. They assert that some of these are paid like any professional; only their pay is called their expenses. This system might run side by side with a theory of equal wages, as absolute as that once laid down by Mr. Bernard Shaw. By the theory of the State, Mr. Herbert Samuel and Mr. Lloyd George might be humble citizens, drudging for their four-pence a day; and no better off than porters and coal-heavers. If there were presented to our mere senses what appeared to be the form of Mr. Herbert Samuel in an astrakhan coat and a motor-car, we should find the record of the expenditure (if we could find it at all) under the heading of “Speed Limit Extension Enquiry Commission.” If it fell to our lot to behold (with the eye of flesh) what seemed to be Mr. Lloyd George lying in a hammock and smoking a costly cigar, we should know that the expenditure would be divided between the “Condition of Rope and Netting Investigation Department,” and the “State of Cuban Tobacco Trade: Imperial Inspector’s Report.” Such is the society I think they will build unless we can knock it down as fast as they build it. Everything in it, tolerable or intolerable, will have but one use; and that use what our ancestors used to call usance or usury. Its art may be good or bad, but it will be an advertisement for usurers; its literature may be good or bad, but it will appeal to the patronage of usurers; its scientific selection will select according to the needs of usurers; its religion will be just charitable enough to pardon usurers; its penal system will be just cruel enough to crush all the critics of usurers: the truth of it will be Slavery: and the title of it may quite possibly be Socialism.



\end{document}
