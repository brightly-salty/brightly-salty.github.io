\documentclass{book}
\usepackage{fontspec}
\usepackage{xunicode}
\usepackage[english]{babel}
\usepackage{fancyhdr}
\usepackage[htt]{hyphenat}
\usepackage[a5paper, top=2cm, bottom=1.5cm, left=2.5cm,right=1.5cm]{geometry}
\makeatletter
\date{}
\pagestyle{fancy}
\fancyhead{}
\fancyhead[CO,CE]{\thepage}
\fancyfoot{}
\makeatother
\title{William Lloyd Garrison on Non-Resistance}
\author{Fanny Garrison Villard}
\begin{document}
\thispagestyle{empty}
\vspace*{\stretch{1}}
\begin{center}
	{\Huge William Lloyd Garrison on Non-Resistance   \\[5mm]}
\end{center}
\vspace*{\stretch{2}}
\newpage
\thispagestyle{empty}
\cleardoublepage
\begin{center}
	\thispagestyle{empty}
	\vspace*{\baselineskip}
	\rule{\textwidth}{1.6pt}\vspace*{-\baselineskip}\vspace*{2pt}
	\rule{\textwidth}{0.4pt}\\[\baselineskip]
	{\Huge\scshape William Lloyd Garrison on Non-Resistance   \\[5mm]}
	{\Large }
	\rule{\textwidth}{0.4pt}\vspace*{-\baselineskip}\vspace{3.2pt}
	\rule{\textwidth}{1.6pt}\\[\baselineskip]
	\vspace*{4\baselineskip}
	{\Large Fanny Garrison Villard}
	\vfill
\end{center}
\pagebreak
\newpage
\thispagestyle{empty}
\null\vfill
\noindent
\begin{center}
	{\emph{William Lloyd Garrison on Non-Resistance}, © Fanny Garrison Villard.\\[5mm]}
	{This work is free of known copyright restrictions.\\[5mm]}
\end{center}
\pagebreak
\newpage
\setcounter{tocdepth}{0}
\setcounter{secnumdepth}{0}

\chapter{Preface}
\label{chapter-0}
My inherited principles of Non-Resistance, which seem as essential to me as the breath of life and paramount to all others, and filial affection have made me yield to the urgent requests of many friends to state as best I may what part a belief in Non-Resistance played in the life of my father, William Lloyd Garrison. His undying faith in the invincible power of Non-Resistance, more than all else, in my estimation, entitles him to the gratitude of his fellow-men. “Doing evil that good may come,” he ever regarded as a false and pernicious doctrine. Therefore, his language had, he felt, to be “as harsh as truth and as uncompromising as justice.” After one of Mr. Garrison’s impassioned utterances, a warm sympathizer said to him, “Oh my friend, do try to moderate your indignation and keep more cool; why you are all on fire.” Mr. Garrison replied, “I have need to be all on fire, for I have mountains of ice about me to melt.” This is perhaps more true of Non-Resistance than of almost any other cause.

It was early given to Mr. Garrison to put his Non-Resistant principles to the test in a way that left no question as to his sincerity or as to his readiness to face death for his beliefs. On October 21, 1835, a “broadcloth” mob consisting of “5000 gentlemen of property and standing” gathered in Boston to tar and feather the English Abolitionist, George Thompson. Unable to find Mr. Thompson, who had yielded to Mr. Garrison’s urgent request to leave the city, the mob surrounded the building in which Mr. Garrison was addressing the meeting of the “Female Anti-Slavery Society” although he had been warned in advance and urged to avoid danger. “In the middle of the uproar,” my father later wrote, “an Abolition brother whose mind had not been previously settled on the peace question, in his anguish and alarm for my safety, and in view of the helplessness of the civil authority, said: ‘I must henceforth repudiate the principle of non-resistance. When the civil arm is powerless, my own rights are trodden in the dust, and the lives of my friends are put in imminent peril by ruffians, I will hereafter prepare to defend myself and them at all hazards.’ Putting my hand upon his shoulder, I said, ‘Hold, my dear brother! You know not what spirit you are of. This is the trial of our faith, and the test of our endurance. Of what value or utility are the principles of peace and forgiveness, if we may repudiate them in the hour of peril and suffering? Do you wish to become like one of those violent and bloodthirsty men who are seeking my life? Shall we give blow for blow, and array sword against sword? God forbid! I will perish sooner than raise my hand against any man, even in self-defence, and let none of my friends resort to violence for my protection. If my life be taken, the cause of emancipation will not suffer. God reigns—his throne is undisturbed by this storm—he will make the wrath of man to praise him, and the remainder he will restrain—his omnipotence will at length be victorious.’”

Had weapons been used in my father’s defence, death would certainly have been his portion. As it was the mob placed a rope around him and dragged him through the streets, intending to lynch him, until he was rescued by the Mayor of Boston and a strong force of police. It was stated by eyewitnesses that Mr. Garrison’s composure was never ruffled during this soul-searching experience.

To this summary of my father’s views I have added a sketch of his personality, as I, the last of his children, knew it. Finally, I wish to record in this volume the words of Count Tolstoy in which he affirms that it was from my father that he learned of the doctrine of Non-Resistance which he at once embraced, so that it is indissolubly connected with his name.

\textbf{Fanny Garrison Villard.}

\emph{New York, June, 1924.}

\chapter{William Lloyd Garrison in His Daughter’s Eyes, by Fanny Garrison Villard}
\label{chapter-1}
To put on paper one’s recollections of an adored parent is difficult. Yet it seems to me that a picture of my father’s rare personal and family life from my point of view may be worth recording. It was an exceptional one because of the happiness and joy which pervaded it even during the darkest days of the anti-slavery struggle. Large means my father never had and the care of his considerable family might well have hampered his spirit had it not been so joyous and serene, so fortified in the right, so certain of the eventual triumph of the good in the world.

Nowhere, in fact, did my father’s calm temperament and sweet disposition make itself more felt than in the home circle. No matter how agitated and exciting his experiences in the outside world might be, nor how uncertain his tenure of life, he forgot it all when surrounded by his wife and children. No man was ever blessed with a more faithful, devoted wife; her burdens were many, but were gladly borne for his sake. My father’s testimony was that she made his home a heaven on earth. The care of the children, which in those days included the making of all their garments at home, devolved wholly upon her, the family means affording her only one servant. Besides that, her house was a hotel, for all Abolitionists who came to Boston were made welcome under her roof, and that was one of her great contributions to the cause of the slave. It was, in fact, the only house in the city which was open to them. What would she have done without my father’s help? He it was who carried the water upstairs when water was a luxury; chopped the wood, made the fires, blacked the boots, or, in case of need, made the coffee, all the while singing. But his shining quality was that of nurse, for his love of children was unbounded and his care of them most skilful. He used to say: “I believe that I was born into the world to take care of babies” and the little ones were drawn to him as if by a magnet.

In this way he smoothed and lightened my mother’s perplexities, comforting her when she feared the children might not be worthy of their training, making light of her anxieties, and helping her at every step. Her splendid health and quick sense of the ludicrous made her duties lighter than they otherwise could have been and her beautiful presence made our home especially attractive. On account of her injured arm it was my father who prepared the food for the children, giving himself no time whatever for eating until their wants were all supplied. At table he was a most delightful host, full of conversation adapted to the guests, and his watchful eye never failed to observe and supply their needs. In short, it was a necessity of his nature to be of service to some one.

He dearly loved to take strolls in the suburbs of Boston with friends and to point out the sites that were “Just the place for a hotel” because the views were so beautiful. In whatever part of the world he was he found numberless fine spots for building and regretted that they should be overlooked, so many people might be enjoying them. He loved nature not less, but human beings more, and the country alone would have been dull and uninteresting to him, in spite of his keen appreciation of a fine landscape and fertile fields. In a mountainous region he felt great exaltation of spirit. Books of poetry he always had at hand and frequently repeated to us fine passages from them. As he possessed a peculiarly sympathetic voice the lofty sentiments he so delighted to utter never failed to stir his hearers. Newspaper reading absorbed much of his time, and he was never happier than when seated in his dressing-gown before a pile of papers, scissors in hand. My mother’s sense of order was always disturbed by the heaps of papers that were thrown upon the floor, but she never complained and even saved, where she might have destroyed. To her forbearance we owe much of the material used by my brothers in the biography of my father. His handwriting was very handsome and so plain that he who ran could read. Unfortunately he held his pen stiffly, and the mechanical drudgery of writing made him dread to take one in his hand. This increased his habit of procrastination, for which his wife gently reproved him, and the wonder is that he, after all, wrote so voluminously.

It tried him to have communications for the \emph{Liberator} written, as he expressed it, “in hieroglyphics,” and he was patience itself with the resulting bad proofs and pitied the compositors who had to set from such manuscripts. It puzzled him, however, to account for the fact that even his own clear copy was sometimes made into “pie,” as the saying is. All at the office testified to the fact that he was gentleness and consideration itself, no matter how great the trials, how tired and exhausted he might feel, or how severely his back ached from bending over the forms. How happy he would have been had he been able to dictate to a secretary, or to use a typewriter! Even his letters to members of his own family did not escape a certain formality, owing to his careful habits of thought. He once expressed himself thus: “He who attempts to appear before the public as a reformer, should so live that all his actions can be searched with candles and not found wanting.” When writing a speech or resolutions, his first desire was to make them plain and simple, and thus powerful as appeals to the heart and the understanding. As a speaker there were occasions when, inspired by a great theme, he took his rank as an orator. One felt that it would have been easy for him to die for the sake of principle, but impossible for him to swerve a hair’s breadth from the line of duty.

I was hardly more than an infant when my father came to my crib to give me a good night kiss. He said: “What a nice warm bed my darling has! The poor little slave child is not so fortunate and is torn from its mother’s arms. How good my darling ought to be!” Thus early, I was taught the lesson of pity for those less favored than myself and so tenderly that it remains after the lapse of more than three-quarters of a century, a moving appeal to me. I recall that I used to stand on my father’s shoulder, holding fast to his bald head, while I learned the names of the great and good men and women whose pictures adorned the walls of our library. Soon, I could point them out correctly for the edification of the family and friends. Sometimes I was permitted to sit on a high stool at \emph{The Liberator} office where my father’s paper was printed and play with the spaces at the compositor’s desk, a very enjoyable pastime I thought.

As my hands were exceptionally cold in winter, I often warmed them on my father’s bald head. For a long time I could not understand why he said: “You come to my incendiary head, my darling, to warm your cold hands?” One day he said to me: “I met a man this morning who thought that I had horns.” I was greatly puzzled and did my best to find them. To many friends he said: “I love all my children best, especially Fanny.” I realize more than ever before the happiness that was mine of being an only daughter. Once I was asked at school if I had been baptized. Not understanding the question, I inquired of my father when I reached home if such was the case, to which he replied: “No, my darling, you have had a good bath every morning and that is a great deal better.” This, I innocently told my school-mates the next day.

\emph{The Liberator} went to press on Wednesday, and as my father was then so busy that he did not take time to leave the office for luncheon, it was my privilege to carry a lunch to him. On such occasions he would always say: “Now you have brought it to me, my darling, I must eat it.”

My father’s call to his children on a Saturday evening to get \emph{The Liberator} “in advance of the mail”—that is, to read proof with him—was always answered cheerfully by us. When it was my turn, I was frequently rewarded by being given a story to read—often in a supplement of the Portland (Maine) \emph{Transcript}—sometimes, I must confess, much more interesting to me than the pages of \emph{The Liberator}.

The general impression of my father’s critics was that life under such conditions as ours must be gloomy, but in reality, there never was a gayer or happier family. My father’s optimistic nature and keen sense of humor could always be relied upon and in times of financial stress he would put his arm around my anxious mother, and walk up and down the room with her, saying: “My dear, the Lord will provide.”

Musical training was never vouchsafed him and he could only pick out a melody on the piano with one hand. Therefore, he was delighted when I was able to play accompaniments for him. He liked to stand behind me and beat time gently on my shoulder when singing his favorite hymns. Love of music was indeed a vital part of my father’s being. His voice waked the household in the morning with song and made the day seem bright and cheery, though it were ever so stormy and gloomy outside. As a boy he sang well and could follow, as he said, “any flute,” and he had a correct ear. Anything with a martial ring to it found an echo in his breast. When asked how a Non-Resistant could like warlike music, he said: “It is just as applicable to moral warfare and the army of the Lord.” When I was taught to play classical music on the piano, he thought that I ought to give more time to what would affect the hearts of people in general. Afterwards, however, his taste became cultivated and he finally took great delight in good concerts, enjoying Von Bülow and Rubinstein and especially the women pianists, Anna Mehlig and Mme. Essipoff; also the violinists, Ole Bull, Wieniawski, Camillo Urso, and Wilhelmy. At the time when the Monster Peace Jubilee concerts of 1869 and 1873 were held in Boston, he was so full of enthusiasm and appreciation of them that he attended almost every performance.

The company that frequented our house was so delightful that I never realized until I was fully grown that my father’s devotion to the cause of the slave made him socially ostracized. I consider it great good fortune to have listened to discussions on important reform movements of the day by such men as Wendell Phillips, Edmund Quincy, the saintly Samuel J. May and his cousin Samuel May and Theodore Parker—and such women as Maria Weston Chapman—the grande dame of the Anti-Slavery cause—Lucy Stone, Susan B. Anthony and the gifted Grimke sisters from South Carolina—all pioneers in both Anti-Slavery and Suffrage—as well as hosts of other interesting people. To us came advocates of Temperance, of the Abolition of Slavery, of the Woman’s Rights Movement, Free Religious Thought, Prison Reform and Non-Resistance—which stood on a far higher plane than the Pacifism of today. Many distinguished reformers from Great Britain paid homage to my father and were entertained at our home, always simply, as though members of the family. It would be impossible to express the joy that my four brothers and I felt in such an environment. My youngest brother on being put to bed early when guests were coming told my mother between sobs, that it was not the supper that he cared for, but the conversation; Anti-Slavery meetings were our theatre and opera, Anti-Slavery debates, meat and drink to us. What we learned was an undying devotion to the principles of justice and humanity, never-to-be abandoned, come what may. It enabled me, when pointed out at school as the daughter of an “infidel,” to glory in the accusation.

I recall the visit of a stranger to my father, who, after introducing himself, said: “Mr. Garrison, if you have immediate emancipation you will have chaos.” I looked at my father and wondered what his reply would be. He seemed absolutely serene and answered: “That is no concern of mine. I know that slavery is wrong and freedom is right, but what you will deprecate, my dear sir, will be, not the results of freedom, but of slavery.” I heard my father say during the Civil War that if slavery went down in blood this nation would realize that there is such a thing as retributive justice, for two wrongs can never make a right. He regarded human life as sacred and inviolable under all circumstances, and he carried on the struggle for Emancipation on the basis of Non-Resistance, losing many adherents in consequence of it, but gaining others who shared his firm belief that we should never do evil that good may come.

I had the privilege of being present with my father at the great celebration in Music Hall, Boston, of that happy event, the signing of the Proclamation of Emancipation by Abraham Lincoln in June, 1864. When my father saw Lincoln in Washington after attending the National Convention for the nomination of President and Vice-President, Mr. Lincoln told him that he never could have written the Proclamation of Emancipation had Mr. Garrison and his coadjutors not made the public sentiment upon which he, Lincoln, had to rely for his support.

Time had brought about a complete change in the public mind with regard to my beloved parent. He himself declared that: “The truth is, he who commences any reform which at last becomes one of transcendent importance and is crowned with victory, is always ill-judged and unfairly estimated. At the outset he is looked upon with contempt and treated in the most opprobrious manner as a wild fanatic or a dangerous disorganizer. In due time the cause grows and advances to its sure triumph, and, in proportion as it nears the goal, the popular estimate of his character changes, till, finally, excessive panegyric is substituted for outrageous abuse. The praise on the one hand and the defamation on the other, are equally unmerited. In the clear light of reason, it will be seen that he simply stood up to discharge a duty which he owed to his God, to his fellow-men, to the land of his nativity.”

The photographs and portraits of my father give one no idea of his mobile countenance. He used to say: “This is fame—a poor portrait and a worse bust——“but the bust of him made by Anne Whitney was very good. At the time when the City Fathers of Boston decided to erect a statue of him, Miss Whitney competed for it and won the prize. It is difficult to believe in these enlightened days that when it was discovered that the prize-winner was a woman she was not permitted to make the statue, but such was the case. Olin Warner was the accepted sculptor and the statue that was erected on Commonwealth Avenue is a travesty of my father.

My husband, Henry Villard, speaks of him in his memoirs as follows:

“Mr. Garrison’s exterior was a complete surprise to me. His public character as the most determined, fearless Anti-Slavery champion had so impressed me, as it did most people, that I had supposed his outward appearance must be in keeping with it. In other words, I had expected to see a fighting figure of powerful build, with thick hair, full beard and fiery defiant eyes. It seemed almost ludicrous to behold a man of middle size, completely bald and clean shaven, with kindly eyes behind spectacles, and instead of a fierce, an entirely benignant expression. He appeared, indeed, more like the typical New England minister of the Gospel than the relentless agitator that he was. The inner man corresponded fully to the outer one. He was forbearing, and mildness itself, in manner and speech.”

Miss Harriet Martineau records her first meeting with my father thus: “His aspect put to flight in an instant what prejudices his slanderers had raised in me. I was wholly taken by surprise. It was a countenance glowing with health and wholly expressive of purity, animation and gentleness.”

She further said: “One day in Michigan two friends (who happened to be Abolitionists) and I were taking a drive with the Governor of the State who was talking of some recent commotion, on the slavery question. ‘What is Garrison like?’ said he. ‘Ask Miss M.,’ said the other. I was asked accordingly and my answer was that I thought Garrison the most bewitching personage that I had met in the United States. The testimony of his personal friends, the closest watchers of his life, may safely be appealed to as to the charms of his domestic manners. Garrison gaily promised me that he would come over whenever his work is done in the United States. I believe that it would be safe to promise him a hundred thousand welcomes as warm as mine.”

On the occasion of my father’s first visit to England where he arrived in June, 1833, just at the crisis of the Anti-Slavery cause there, his mission was to expose the work of the American Colonization Society to the friends of the colored people in London and elsewhere in Great Britain. Mr. William Green described him at that time as “a young man not yet twenty-eight, without means or social standing or a numerous following; despised, hated, hunted with a price on his head; armed only with the blessings of an outcast race and the credentials of an insignificant body of ‘fanatics’ who was to present himself before the honorable, powerful and world-famous advocates of British Emancipation—before Clarkson and Wilberforce, Macaulay and Buxton.”

After arriving in London my father received an invitation to take breakfast with Sir Fowell Buxton which he gladly accepted and reached the house at the appointed time. When his name was announced Mr. Buxton inquired somewhat dubiously: “Have I the pleasure of addressing Mr. Garrison of Boston in the United States ?” The latter replied “Yes, Sir, I am he, and I am here in accordance with your invitation.” Lifting up his hands Sir Fowell exclaimed: “Why, my dear sir, I thought that you were a black man!” Referring to this episode my father said that that was the only compliment that he cared to remember or to tell, for Mr. Buxton somehow or other supposed that no white American could plead for those in bondage as he had done, therefore he must be black.

My father was of medium height and he became bald while still a young man. The remnant of his hair was like fine silk and quite black, turning gray only late in life. His eyes were large and full, of a hazel color, hidden somewhat by glasses, but giving an expression of great benevolence to his countenance, and his nose was strong and well shaped. The mouth with its firm deep lines was the feature that indicated the dauntless courage and iron will of the man who stood almost alone when he began the agitation which ended in the emancipation of the slaves. It gave to his face its intense earnestness of purpose; yet it was wonderfully mobile and the slightest movement of the lips gave him the kindest and sunniest of expressions. It was easy to note the changes of his ever-varying countenance on his smoothly shaven face. His complexion in youth was singularly white, his cheeks “like roses in the snow,” as one of his old friends told me. He had, as long as he lived, a fine color in spite of the fact that he was compelled to keep very irregular hours because of his labors in the printing office and his lecturing experiences. He held himself erect and walked with a firm, brisk step, all his movements indicating alertness and vigor of mind and body.

It seems to me that it was my father’s marvellously sympathetic nature, his complete forgetfulness of self, that drew all hearts to him. Once admitted to his presence, friend or foe felt the magic of his winning voice, expressive of a heart overflowing with kindness, and prejudices melted away before his genial smile. This endeared him to all who came in contact with him and he invariably touched responsive chords in his listeners, were they of high or low degree.

The winter of 1867 m Y husband and I spent in Paris, and in the spring my father came over as a delegate from the American Freedman’s Commission to a Paris Anti-Slavery Conference. It was convened by the British and Foreign Anti-Slavery Society. Afterwards he visited his numerous friends in England who had given him unfailing support during the days of slavery. I had the happiness of joining him there, together with my brother Frank, and attending the public breakfast given to my father in London on June 29, 1867, at St. James’ Hall by some of the most distinguished Englishmen of the time. John Bright presided, the Duke of Argyle made a memorable address, and Earl Russell improved the occasion to make open amends to the United States for his attitude during the Rebellion, when he allowed the Alabama to leave a British port to prey upon American shipping. The final address was made by John Stuart Mill—to me the culmination of that brilliant and moving occasion. He pointed out two lessons from my father’s career. The first was; u Aim at something great; aim at things which are difficult (and there is no great thing that is not difficult) . If you aim at something noble and succeed in it, you will generally find that you have succeeded not in that alone. A hundred other good and noble things which you never dreamed of will have been accomplished by the way." He also said that “though our best directed efforts may often seem wasted and lost, nothing coming of them that can be pointed to and distinctly identified as a definite gain to humanity; though this may happen ninety-nine times in every hundred, the hundredth time the result may be so great that we had never dared to hope for it and should have regarded him who had predicted it to us as sanguine beyond the bounds of mental sanity.”

My father’s reply to the eloquent addresses made on that occasion was wholly spontaneous and ended thus: “Now in parting let me say, we must not allow ourselves to be divided—England from America, America from England. By every consideration under heaven, let us resolve to keep the peace. If we have old grudges, let them be thrown to the winds. Let there be peace—a true and just peace—peace by forbearance—peace by generous concession—for the sake of the cause of mankind, and that together England and America may lead the nations of the world to freedom and glory. There is your country’s flag, there is mine. Let them be blended.”

My father’s Non-Resistant principles were fervently embraced by all his children except his eldest son, George Thompson Garrison, who, when the 54th and 55th Massachusetts regiments were formed of colored men in the Civil War, joined the former as a lieutenant. Before the final step was taken my father sent the following letter to him on June ii ? 1863. It indicates more than all else, perhaps, his spirit of tolerance and respect for the views of another—even those of a beloved son who differed from the teaching of his father on so vital a matter as war: “Though I could have wished that you had been able understandingly and truly to adopt those principles of peace which are so sacred and divine to my soul, yet you will bear me witness that I have not laid a straw in your way to prevent your acting up to your own highest convictions of duty; for nothing would be gained, but much lost, to have you violate these. Still, I tenderly hope that you will once more seriously review the whole matter before making the irrevocable decision....”

Love, they say, is blind, but love for our father was something more than that affection which commonly exists between parents and children. It was called forth by a life so pure, so tender, so earnest, so full of compassion and so joyous, that being part of it, we regarded him almost as one who had strayed from heaven to earth. Each one of us was morally stimulated by his fidelity to principle, his never-failing generosity and touching forgetfulness of self; and this not only because of his devotion to many good, though unpopular, causes, but even more for the reason that these exceptional qualities were ours to emulate in the daily round of our home life. A greater legacy, a richer moral inheritance, I cannot believe was ever bestowed upon children by both parents than that which was so abundantly ours.

We could not rise to the height of such a father, but we kept, as a solemn trust, his clear vision of unbounded love and sympathy for those suffering from cruelty and injustice. If it were only possible to give an approximate idea of so remarkable and lovely a personality and the secret of its charm my self-imposed task would be easy. But any attempt to do so must necessarily fall short of the truth, however imperishable the memory of it.

\chapter{William Lloyd Garrison on Non-Resistance}
\label{chapter-2}
As early as August 30, 1838, Mr. Garrison wrote to his intimate friend and coadjutor, Samuel J. May, of Leicester, Mass. as follows in regard to the peace convention to be held in Boston on Sept. 18-20 of that year:

“We shall probably find no difficulty in bringing a large majority of the Convention to set their seal of condemnation upon the present militia system and its ridiculous and pernicious accompaniments. They will also, I presume, reprobate all wars, defensive as well as offensive. They will not agree so cordially as to the inviolability of human life. But few, I think, will be ready to concede that Christianity forbids the use of physical force in the punishment of evil-doers; yet nothing is plainer to my understanding or more congenial to the feelings of my heart.

“... I feel the excellence and sublimity of that precept which bids me pray for those who despitefully use me; and that other precept which enjoins upon me when smitten on the one cheek to turn the other also.

“... We degrade our spirits in a brutal conflict. To talk of courts of justice and of punishing evil and disobedient men—of protecting the weak and avenging the wronged by a posse comitatus or a company of soldiers—has a taking sound but it is hollow in my ears.”\footnotemark[1]

A part of the “Declaration of Sentiments” adopted by this Peace Convention in Boston, on September 18th, 19th and 20th, 1838, which was written by Mr. Garrison at a single sitting on the forenoon of the 20th of September is as follows:

“Assembled in Convention, from various sections of the American Union, for the promotion of peace on earth and good-will among men, we, the undersigned, regard it as due to ourselves, to the cause which we love, to the country in which we live, and to the world, to publish a Declaration, expressive of the principles we cherish, the purposes we aim to accomplish, and the measures we shall adopt to carry forward the work of peaceful, universal reformation.

“Our country is the world, our countrymen are all mankind. We love the land of our nativity only as we love all other lands. The interests, rights, liberties of American citizens are no more dear to us than are those of the whole human race. Hence, we can allow no appeal to patriotism, to revenge any national insult or injury. The Prince of Peace, under whose stainless banner we rally, came not to destroy, but to save, even the worst of enemies.

“We conceive that if a nation has no right to defend itself against foreign enemies, or to punish its invaders, no individual possesses that right in his own case. The unit cannot be of greater importance than the aggregate. If one man may take life, to obtain or defend his rights, the same license must necessarily be granted to the communities, states and nations. If he may use a dagger or a pistol, they may employ cannon, bombshells, land and naval forces. The means of self-preservation must be in proportion to the magnitude of interests at stake and the number of lives exposed to destruction. But if a rapacious and bloodthirsty soldiery, thronging these shores from abroad, with intent to commit rapine and destroy life, may not be resisted by the people or magistracy, then ought no resistance to be offered to domestic troublers of the public peace or of private security. No obligation can rest upon Americans to regard foreigners as more sacred in their persons than themselves, or to give them a monopoly of wrong-doing with impunity.

“The dogma that all the governments of the world are approvingly ordained of God, and that the powers that be in the United States, in Russia, in Turkey, are in accordance with His will, is not less absurd than impious. It makes the impartial Author of human freedom and equality unequal and tyrannical.

“We register our testimony, not only against all wars, whether offensive or defensive, but all preparations for war; against every naval ship, every arsenal, every fortification; against the militia system and a standing army; against all military chieftains and soldiers; against all monuments commemorative of victory over a fallen foe, all trophies won in battle, all celebrations in honor of military or naval exploits; against all appropriations for the defence of a nation by force and arms, on the part of any legislative body; against every edict of government requiring of its subjects military service. Hence we deem it unlawful to bear arms or to hold a military office.

“The history of mankind is crowded with evidences proving that physical coercion is not adapted to moral regeneration; that the sinful dispositions of men can be subdued only by love; that evil can be exterminated from the earth only by goodness; that it is not safe to rely upon an arm of flesh, upon man whose breath is in his nostrils, to preserve us from harm; that there is great security in being gentle, harmless, long-suffering, and abundant in mercy; that it is only the meek who shall inherit the earth, for the violent who resort to the sword are destined to perish with the sword. Hence as a measure of sound policy—of safety to property, life, and liberty—of public quietude and private enjoyment—as well as on the ground of allegiance to Him who is King of Kings and Lord of Lords, we cordially adopt the Non-Resistance principle, being confident that it provides for all possible consequences, will ensure all things needful to us, is armed with omnipotent power, and must ultimately triumph over every assailing force.

“If we abide by our principles, it is impossible for us to be disorderly, or plot treason, or participate in any evil work; we shall submit to every ordinance of man; obey all the requirements of Government, except such as we deem contrary to the commands of the gospel, and in no case resist the operation of the law, except by meekly submitting to the penalty of disobedience.

“But, while we shall adhere to the doctrine of Non-Resistance and passive submission to enemies, we purpose, in a moral and spiritual sense, to speak and act boldly; to assail iniquity, in high places and in low places; to apply our principles to all existing civil, political, legal and ecclesiastical institutions.

“It appears to us a self-evident truth, that, whatever the gospel is designed to destroy at any period of the world, being contrary to it, ought now to be abandoned.

“We expect to prevail through the foolishness of preaching—striving to commend ourselves unto every man’s conscience, in the sight of God. From the press we shall promulgate our sentiments as widely as practicable. We shall endeavor to secure the cooperation of all persons, of whatever name or sect. The triumphant progress of the cause of Temperance and of Abolition in our land, through the instrumentality of benevolent and voluntary associations, encourages us to combine our own means and efforts for the promotion of a still greater cause. Hence, we shall employ lecturers, circulate tracts and publications, form societies, and petition our State and national governments, in relation to the subject of \emph{universal peace}. It will be our leading object to devise ways and means for effecting a radical change in the views, feelings and practices of society, respecting the sinfulness of war and the treatment of enemies.

“In entering upon the great work before us, we are not unmindful that, in its prosecution, we may be called to test our sincerity, even as in a fiery ordeal. It may subject us to insult, outrage, suffering, yea, even death itself. We anticipate no small amount of misconception, misrepresentation, calumny.

“Firmly relying upon the certain and universal triumph of the sentiments contained in this Declaration, however formidable may be the opposition arrayed against them—in solemn testimony of our faith in their divine origin—we hereby affix our signatures to it; commending it to the reason and conscience of mankind, giving ourselves no anxiety as to what may befall us.”\footnotemark[2]

On another occasion Mr. Garrison said of this Declaration: “This instrument contemplates nothing, repudiates nothing, but the spirit of violence in thought, word and deed. Whatever, therefore, may be done without provoking that spirit, and in accordance with the spirit of disinterested benevolence, is not touched or alluded to in the instrument. The sum total of affirmation is this—that, we are resolved, come what may, as Christians, to have long-suffering toward those who may despitefully use and persecute us—to pray for them—to forgive them in all cases. This is the head and front of our offending—nothing more, nothing less.”\footnotemark[3]

Referring to the work of this convention, Mr. Garrison wrote as follows: “Our Non-Resistance Convention is over, and the peace and blessing of heaven have attended our deliberations. Such a mass of free mind as was brought together I have never seen before in any one assembly. ‘Not many mighty,’ ‘not many rich,’ ‘not many honorable’ were found among us—of course; but there was much talent, and a great deal of soul. Not a single set speech was made by anyone, but everyone spoke in a familiar manner, just as though we constituted but a mere social party.”\footnotemark[4]

As a matter of course, the new peace organization received the condemnation of the religious press, with hardly an exception, and especially that of the American Peace Society and the New York Peace Society.

Replying to a correspondent Mr. Garrison declared that “Non-Resistance is not a state of passivity, on the contrary, it is a state of activity, ever fighting the good fight of faith, ever foremost to assail unjust power, ever struggling for ‘liberty, equality, fraternity,’ in no national sense, but in a world-wide spirit. It is passive only in this sense—that it will not return evil for evil, nor give blow for blow, nor resort to murderous weapons for protection or defense.”\footnotemark[5]

The author of the Declaration of Sentiments then asks “On what are right and wrong dependent? On recorded declarations? On ancient parchments or modern manuscripts? On sacred books? No. Though every parchment, manuscript and book in the world were given to the consuming fire, the loss would not in the least affect the right or wrong of moral actions. Truth and duty, the principle of justice and equity, the obligations of mercy and brotherly kindness, are older than all books, and more enduring than tablets of stone...

“The question at issue is—war, its nature, tendencies, results: war, whether in ancient or modern times, whether under the Jewish or Christian dispensation; is it right? was it ever justifiable?...

“Why should we go to a book to settle the character of war when we could judge of it by its fruits?...

“War is as capable of moral analysis as slavery, intemperance, licentiousness, or idolatry. It is not an abstraction, which admits of doubt or uncertainty, but as tangible as bombs, cannon, mangled corpses, smouldering ruins, desolated towns and villages, rivers of blood. It is substantially the same in all ages, and cannot change its moral features. To trace it in all its ramifications is not a difficult matter. In fact, nothing is more terribly distinct than its career; it leaves its impress on everything it touches, whether physical, mental, or moral. Why, then, not look it in the face? Why look anywhere else?”\footnotemark[6]

Mr. Garrison presided at a Non-Resistance Convention held in Worcester on March 24 and 25, 1854, and drew up several resolutions, one of them being the following:

“Resolved, that the plan of supporting governments by tariffs and other indirect taxes, is a cunning contrivance of tyrants to enable them to attain their ambitions and bloody aims without exciting the alarm of the people by a direct appeal to their pockets; therefore, one most potent way to put an end to war and tyranny is to abolish all tariffs and indirect taxes and substitute free trade and direct taxation as the means of sustaining political institutions.”\footnotemark[7]

In May, 1838, just after the dedication of Marlboro’ Chapel, which was founded in Boston “mainly by Abolitionists, upon the rock of universal emancipation, and to advance the cause of humanity and free discussion,” Mr. Garrison wrote to George W. Benson: “The spirit of mobocracy, like the pestilence, is contagious; and Boston is once more ready to re-enact the riotous scenes of 1835. The Marlboro’ Chapel, having just been completed, and standing in relation to our cause just as did Pennsylvania Hall, is an object of pro-slavery malevolence. Ever since my return, threats have been given out that the Chapel should share the fate of the Hall. Last evening was the time for its dedication; and, so threatening was the aspect of things, four companies of light infantry were ordered to be in readiness, each being provided with 100 \emph{ball} cartridges, to rush to the scene of riot on the tolling of the bells. The lancers, a powerful body of horsemen, were also in readiness. During the day placards were posted at the corners of the streets, denouncing the Abolitionists, and calling upon the citizens to rally at the Chapel in the evening, in order to put them down... Non-Resistance versus brickbats and bowie-knives. Omnipotence against a worm of the dust. Divine law against lynch law. How unequal!"\footnotemark[8]

Commenting upon Henry Ward Beecher’s statement that u there are those who will scoff at the idea of holding a sword or a rifle, in a Christian state of mind," Mr. Garrison wrote in the \emph{Liberator:} “We know not where to look for Christianity if not to its founder; and, taking the record of his life and death, of his teaching and example, we can discover nothing which even remotely, under any conceivable circumstances, justifies the use of the sword or rifle on the part of his followers; on the contrary, we find nothing but self-sacrifice, willing martyrdom (if need be), peace and good-will and the prohibition of all retaliatory feelings, enjoined upon all who would be his disciples. When he said: ‘Fear not those who kill the body,’ he broke every deadly weapon. When he said: ‘My kingdom is not of this world, else would my servants fight that I should not be delivered to the Jews,’ he plainly prohibited war in self-defence and substituted martyrdom therefore. When he said ‘Love your enemies,’ he did not mean ‘Kill them if they go too far.’ When he said, while expiring on the cross: ‘Father, forgive them; for they know not what they do,’ he did not treat them as “a herd of buffaloes,” but as poor, misguided and lost men. We believe in his philosophy; we accept his instruction; we are thrilled by his example; we rejoice in his fidelity.”\footnotemark[9]

At the New England Convention in Boston on May 26, 1858, where Thomas Wentworth Higginson and Theodore Parker expressed their opinion that the slavery question could be settled only by bloodshed, Mr. Garrison, though he had long considered such a solution inevitable, deplored the rejection of Non-Resistance principles on the part of the Abolitionists: “When the antislavery cause was launched,” he said, “it was baptized in the spirit of peace. We proclaimed to the country and the world that the weapons of our warfare were not carnal, but spiritual, and we believed them to be mighty through God to the pulling down even of the stronghold of slavery; and for several years great moral power accompanied our cause wherever presented. Alas in the course of the fearful developments of the Slave Power, and its continued aggressions on the rights of the people of the North, in my judgment a sad change has come over the spirit of anti-slavery men, generally speaking. We are growing more and more warlike, more and more disposed to repudiate the principles of peace, more and more disposed to talk about ‘finding a joint in the neck of the tyrant,’ and breaking that neck, ‘cleaving tyrants down from the crown to the groin,’ with the sword which is carnal, and so inflaming one another with the spirit of violence and for a bloody work. Just in proportion as this spirit prevails, I feel that our moral power is departing and will depart. I say this not so much as an Abolitionist as a man. I believe in the spirit of peace, and in sole and absolute reliance on truth and the application of it to the hearts and consciences of the people. I do not believe that the weapons of liberty ever have been, or ever can be, the weapons of despotism. I know that those of despotism are the sword, the revolver, the cannon, the bomb-shell; and, therefore, the weapons to which tyrants cling, and upon which they depend, are not the weapons for me, as a friend of liberty. I will not trust the war spirit anywhere in the universe of God, because the experience of six thousand years proves it not to be at all reliable in such a struggle as ours...

“I pray you, Abolitionists, still to adhere to that truth. Do not get impatient; do not become exasperated; do not attempt any new political organization; do not make yourselves familiar with the idea that blood must flow. Perhaps blood will flow—God knows, I do not; but it shall not flow through any counsel of mine. Much as I detest the oppression exercised by the Southern slaveholder, he is a man, sacred before me. He is a man, not to be harmed by my hand nor with my consent. He is a man, who is grievously and wickedly trampling upon the rights of his fellowmen; but all I have to do with him is to rebuke his sin, to call him to repentance, to leave him without excuse for his tyranny. He is a sinner before God—a great sinner; yet, while I will not cease reprobating his horrible injustice, I will let him see that in my heart there is no desire to do him harm—that I wish to bless him here, and bless him everlastingly—and that I have no other weapon to wield against him but the simple truth of God, which is the great instrument for the overthrow of all iniquity, and the salvation of the world.”\footnotemark[10]

At a meeting in Tremont Temple, held on the day of John Brown’s execution, December 2, 1859, Mr. Garrison said: “A word upon the subject of Peace. I am a Non-Resistant—a believer in the inviolability of human life, under all circumstances; I, therefore, in the name of God, disarm John Brown and every slave at the South. But I do not stop there; if I did I should be a monster. I also disarm, in the name of God, every slaveholder and tyrant in the world. For, wherever that principle is adopted all fetters must instantly melt, and there can be no oppressed and no oppressor, in the nature of things.

“... But whenever there is a contest between the oppressed and the oppressor— the weapons being equal between the parties—God knows that my heart must be with the oppressed and always against the oppressor.”\footnotemark[11]

Many questions that are asked Non-Resistants today were put to Mr. Garrison at the time of the Civil War. When thus interrogated in regard to his peace views he replied: “This question is exultingly put to the friends of peace and Non-Resistance by those whose military ardor is now at a white heat, as though it could not be satisfactorily answered, and deserved nothing but ridicule. Our reply to it is, that the peace principles are as beneficent and glorious as ever and are neither disproved nor modified by anything now transpiring in the country of a warlike character. If they had been long since embraced and carried out by the people, neither slavery nor war would now be filling the land with violence and blood. Where they prevail no man is in peril of life or liberty; where they are rejected, and precisely to the extent they are rejected, neither life nor liberty is secure. How their violation, under any circumstances, is better than a faithful adherence to them, we have not the moral vision to perceive. They are to be held responsible for nothing which they do not legitimately produce or sanction. As they neither produce nor sanction any oppression or wrongdoing, but elevate the character, control the passions and lead to the performance of all good offices, they are not to be discarded for those of a hostile character.”\footnotemark[12]

“What is war? Is it not the opposite of peace, as slavery is of liberty, as sin is of holiness, as Belial is of Christ? And is slavery sometimes to be enforced—is sin in cases of emergency to be committed—is Belial occasionally to be preferred to Christ, as circumstances may require? These are grave questions, and the redemption of the world is dependent on the answers that may be given to them.

“The better the object, the less need, the less justification there is to behave as they do, who have one that is altogether execrable. Eye for eye, tooth for tooth, life for life, is not the way to redeem or bless our race. Sword against sword, cannon against cannon, army against army, is it thus that love and good-will are diffused through the world, or that right conquers wrong? Why not, then, seek by falsehood to counteract falsehood—by cruelty to terminate cruelty—by sin to abolish sin? Can men gather grapes from thorns, or figs from thistles ?”\footnotemark[13]

“As for the governments of the world, all history shows that they cannot be maintained, except by naval and military power; that all their mandates being a dead letter without such power to enforce them in the last extremity, are virtually written in blood.”\footnotemark[14]

“To palliate crime is to be guilty of its perpetration. To ask for a postponement of the case, till a more convenient season, is to call for a suspension of the moral law, and to assume that it is right to do wrong, under present circumstances.”\footnotemark[15]

“The truth that we utter is impalpable, yet real: It cannot be thrust down by brute force, nor pierced with a dagger, nor bribed with gold, nor overcome by the application of a coat of tar and feathers. The cause that we espouse is the cause of human liberty, formidable to tyrants, and dear to the oppressed, throughout the world—containing the elements of immortality, sublime as heaven, and far-reaching as eternity—embracing every interest that appertains to the welfare of the bodies and souls of men, and sustained by the omnipotence of the Lord Almighty. The principles that we inculcate are those of equity, mercy, and love...\footnotemark[16]

“Moral influence, when in vigorous exercise, is irresistible. It has an immortal essence. It can no more be trod out of existence by the iron foot of time, or by the ponderous march of iniquity, than matter can be annihilated. It may disappear for a time; but it lives in some shape or other, in some place or other, and will rise with renovated strength.”\footnotemark[17]

\footnotetext[1]{William Lloyd Garrison, 1805-1879; the Story of His Life Told by His Children. 1885-1889. Volume 2, p. 225.

}\footnotetext[2]{Life, volume 2, p. 230.

}\footnotetext[3]{Life, volume 2, p. 237.

}\footnotetext[4]{Life, volume 2, p. 328.

}\footnotetext[5]{Selections from the Writings and Speeches of William Lloyd Garrison, 1852, p. 88.

}\footnotetext[6]{Writings, 1852, p. 89-90

}\footnotetext[7]{Life, volume 3, p. 419.

}\footnotetext[8]{Life, volume 2, p. 219.

}\footnotetext[9]{Life, volume 3, p. 437.

}\footnotetext[10]{Life, volume 3, p. 473.

}\footnotetext[11]{Life, volume 3, p. 491.

}\footnotetext[12]{Life, volume 4, p. 25.

}\footnotetext[13]{Writings, p. 80-81.

}\footnotetext[14]{Writings, p. 94.

}\footnotetext[15]{Writings, p. 140.

}\footnotetext[16]{Writings, p. 389.

}\footnotetext[17]{Writings, p. 58-59.

}\chapter{\emph{The Non-Resistant}}
\label{chapter-3}
After the formation of The Non-Resistant Society on September 18, 1838, and the appearance in the columns of the Liberator of articles espousing the cause of Non-Resistance, objections began to arise as to the use of the space of the Liberator, for any subject not directly relating to slavery. On November n, 1838, Anne Warren Weston, an ardent apostle of both causes, wrote a letter to Mr. Garrison urging him to edit a monthly paper devoted to Non-Resistance on the ground of her belief that continued advocacy of Non-Resistance by the \emph{Liberator} would damage the paper and cause vexation and discord among Abolitionists. After careful consideration of the matter the executive committee of the Non-Resistance Society accepted this point of view and determined upon the publication of a semi-monthly organ called \emph{The Non-Resistant}. When it came to be published, however, it was found that it could not be gotten out oftener than monthly and its publication was continued from January, 1839, until the 29th of June, 1842, when it came to an end, like many another hopeful publication, because of lack of means—which is clear enough proof that the Non-Resistant movement was not identical with the Abolition movement itself, for the \emph{Liberator} went its triumphal way until Abolition was achieved.

\emph{The Non-Resistant} itself was a small folio nineteen inches by twelve, its four columns to the page being of the same width as those of the \emph{Liberator} to permit the use of the same type in both publications. Its motto was, “Resist not evil—Jesus Christ.” The burden of the editorial work was carried by Mrs. Maria Weston Chapman and Edmund Quincy, together with Mr. Garrison, who stated, however, at the end of \emph{The Non-Resistant}’s existence, that he had not spent half a day during the whole period of the paper’s life in preparing editorial matter for it. The bulk of the work was, therefore, done by Mrs. Chapman and Mr. Quincy, assisted by Charles K. Whipple, the Treasurer of the Non-Resistant Society, and by Henry C. Wright, one of Mr. Garrison’s closest friends and associates. He was the sole missionary in the field and was authorized in 1 41 to go abroad as a “sort of missionary for the cause of peace, abolition, temperance, chastity, and a pure and equal Christianity”—surely a large order.

The success of the paper at the outset was very encouraging and was greater than its organizers had dared to hope. It paid for itself in 1840. At the outset Gerrit Smith, the rich Abolitionist of Peterboro, N. Y., sent \$100 towards its support, and this was considered very daring because of the intense feeling among many of the Abolitionists that it was traitorous to the great cause to espouse any other. Arthur Tappan, the New York Abolitionist, who freed Mr. Garrison from jail at the time of his incarceration in Baltimore by paying his fine, on June 5, 1830, was so incensed by \emph{The Non-Resistant} that he returned the copy sent to him and declared that he would have nothing to do with an organ that proclaimed itself against the use of force by the government and disseminated other non-governmental sentiments.

Curiously enough, it was the sending of Mr. Wright on his all-embracing mission to Europe which contributed more to the death of \emph{The Non-Resistant} than anything else. He was that rare person among the Abolitionists, a good business man and organizer, and his absence in England seriously interfered with the development of the paper. But the chief cause was, as Mr. Garrison wrote to Mr. Wright, “our time, our means, our labors are so absorbed in seeking the emancipation of our enslaved countrymen that we cannot do as much specifically and directly for non-resistance as would otherwise be in our power to perform.” The Non-Resistant Society kept alive until 1849, then it, too, gave up the ghost, and presumably for the same reason. With every year the anti-slavery struggle was growing bitterer and more thrilling, and was absorbing more and more the energy and the strength of all connected with it.

\chapter{What I Owe to Garrison, by Leo Tolstoy}
\label{chapter-4}
(A Letter to V. Tchertkoff)\footnotemark[1]

I thank you very much for sending me your biography of Garrison.

Reading it, I lived again through the spring of my awakening to true life. While reading Garrison’s speeches and articles I vividly recalled to mind the spiritual joy which I experienced twenty years ago, when I found out that the law of non-resistance—to which I had been inevitably brought by the recognition of the Christian teaching in its full meaning, and which revealed to me the great joyous ideal to be realized in Christian life—was even as far back as the forties not only recognized and proclaimed by Garrison (about Ballou I learned later), but also placed by him at the foundation of his practical activity in the emancipation of the slaves.

My joy was at that time mingled with bewilderment as to how it was that this great Gospel truth, fifty years ago explained by Garrison, could have been so hushed up that I had now to express it as something new.

My bewilderment was especially increased by the circumstance that not only people antagonistic to the progress of mankind, but also the most advanced and progressive men, were either completely indifferent to this law or actually opposed to the promulgation of that which lies at the foundation of all true progress.

But as time went on it became clearer and clearer to me that the general indifference and opposition which were then expressed, and still continue to be expressed—pre-eminently amongst political workers—towards this law of non-resistance are merely symptoms of the great significance of this law.

“The motto upon our banner,” wrote Garrison in the midst of his activity, “has been from the commencement of our moral warfare ‘OUR COUNTRY IS THE WORLD; OUR COUNTRYMEN ARE ALL MANKIND.’ We trust that it will be our only epitaph. Another motto we have chosen is, ‘UNIVERSAL EMANCIPATION.’ Up to this time we have limited its application to those who in this country are held by Southern taskmasters as marketable commodities, goods and chattels, and implements of husbandry. Henceforth we shall use it in its widest latitude—the emancipation of our whole race from the dominion of man, from the thralldom of self, from the government of brute force, from the bondage of sin, and the bringing it under the dominion of God, the control of an inward spirit, the government of the law of love...”

Garrison, as a man enlightened by the Christian teaching, having begun with the practical aim of strife against slavery, very soon understood that the cause of slavery was not the casual temporary seizure by the Southerners of a few millions of negroes, but the ancient and universal recognition, contrary to the Christian teaching, of the right of coercion on the part of certain people in regard to certain others. A pretext for recognizing this right has always been that men regarded it as possible to eradicate or diminish evil by brute force, i.e., also by evil. Having once realized this fallacy, Garrison put forward against slavery neither the suffering of slaves, nor the cruelty of slaveholders, nor the social equality of men, but the eternal Christian law of refraining from opposing evil by violence, i.e., of “non-resistance.” Garrison understood that which the most advanced among the fighters against slavery did not understand: that the only irrefutable argument against slavery is the denial of the right of any man over the liberty of another under any conditions whatsoever.

The Abolitionists endeavored to prove that slavery was unlawful, disadvantageous, cruel; that it depraved men, and so on; but the defenders of slavery in their turn proved the untimeliness and danger of emancipation, and the evil results liable to follow it. Neither the one nor the other could convince his opponent. Whereas Garrison, understanding that the slavery of the negroes was only a particular instance of universal coercion, put forward a general principle with which it was impossible not to agree— the principle that under no pretext has any man the right to dominate, i.e., to use coercion over his fellows. Garrison did not so much insist on the right of negroes to be free as he denied the right of any man whatsoever, or of any body of men, forcibly to coerce another man in any way. For the purpose of combating slavery he advanced the principle of struggle against all the evil of the world.

This principle advanced by Garrison was irrefutable, but it affected and even overthrew all the foundations of established social order, and therefore those who valued their position in that existing order were frightened at its announcement, and still more at its application to life; they endeavored to ignore it, to elude it; they hoped to attain their object without the declaration of the principle of non-resistance to evil by violence, and that application of it to life which would destroy, as they thought, all orderly organization of human life. The result of this evasion of the recognition of the unlawfulness of coercion was that fratricidal war which, having externally solved the slavery question, introduced into the life of the American people the new—perhaps still greater—evil of that corruption which accompanies every war.

Meanwhile the substance of the question remained unsolved, and the same problem, only in a new form, now stands before the people of the United States. Formerly the question was how to free the negroes from the violence of the slaveholders; now the question is how to free the negroes from the violence of all the whites, and the whites from the violence of all the blacks.

The solution of this problem in a new form is to be accomplished certainly not by the lynching of the negroes, nor by any skilful and liberal measures of American politicians, but only by the application to life of that same principle which was proclaimed by Garrison half a century ago.

The other day in one of the most progressive periodicals I read the opinion of an educated and intelligent writer, expressed with complete assurance in its correctness, that the recognition by me of the principle of non-resistance to evil by violence is a lamentable and somewhat comic delusion which, taking into consideration my old age and certain merits, can only be passed over in indulgent silence.

Exactly the same attitude towards this question did I encounter in my conversation with the remarkably intelligent and progressive American Bryan. He also, with the evident intention of gently and courteously showing me my delusion, asked me how I explained my strange principle of non-resistance to evil by violence, and as usual he brought forward the argument, which seems to everyone irrefutable, of the brigand who kills or violates a child. I told him that I recognize non-resistance to evil by violence because, having lived seventy-five years, I have never, except in discussions, encountered that fantastic brigand, who, before my eyes, desired to kill or violate a child, but that perpetually I did and do see not one but millions of brigands using violence towards children and women and men and old people and all the laborers in the name of the recognized right of violence over one’s fellows. When I said this my kind interlocutor, with his naturally quick perception, not giving me time to finish, laughed, and recognized that my argument was satisfactory.

No one has seen the fantastic brigand, but the world, groaning under violence, lies before everyone’s eyes. Yet no one sees, nor desires to see, that the strife which can liberate man from violence is not a strife with the fantastic brigand, but with those actual brigands who practise violence over men.

Non-resistance to evil by violence really means only that the mutual interaction of rational beings upon each other should consist not in violence (which can be only admitted in relation to lower organisms deprived of reason) but in rational persuasion; and that, consequently, towards this substitution of rational persuasion for coercion all those should strive who desire to further the welfare of mankind.

It would seem quite clear that in the course of the last century, fourteen million people were killed, and that now the labor and lives of millions of men are spent on wars necessary to no one, and that all the land is in the hands of those who do not work on it, and that all the produce of human labor is swallowed up by those who do not work, and that all the deceits which reign in the world exist only because violence is allowed for the purpose of suppressing that which appears evil to some people, and that therefore one should endeavor to replace violence by persuasion. That this may become possible it is necessary first of all to renounce the right of coercion.

Strange to say, the most progressive people of our circle regard it as dangerous to repudiate the right of violence and to endeavor to replace it by persuasion. These people, having decided that it is impossible to persuade a brigand not to kill a child, think it also impossible to persuade the working men not to take the land and the produce of their labor from those who do not work, and therefore these people find it necessary to coerce the laborers.

So that, however sad it is to say so, the only explanation of the non-understanding of the significance of the principle of non-resistance to evil by violence consists in this, that the conditions of human life are so distorted that those who examine the principle of nonresistance imagine that its adaptation to life and the substitution of persuasion for coercion would destroy all possibility of that social organization and of those conveniences of life which they enjoy.

But the change need not be feared; the principle of non-resistance is not a principle of coercion but of concord and love, and therefore it cannot be made coercively binding upon men. The principle of non-resistance to evil by violence, which consists in the substitution of persuasion for brute force, can be only accepted voluntarily, and in whatever measure it is freely accepted by men and applied to life—i.e., according to the measure in which people renounce violence and establish their relations upon rational persuasion—only in that measure is true progress in the life of men accomplished.

Therefore, whether men desire it or not, it is only in the name of this principle that they can free themselves from the enslavement and oppression of each other. Whether men desire it or not, this principle lies at the basis of all true improvement in the life of men which has taken place and is still to take place.

Garrison was the first to proclaim this principle as a rule for the organization of the life of men. In this is his great merit. If at the time he did not attain the pacific liberation of the slaves in America, he indicated the way of liberating men in general from the power of brute force.

Therefore Garrison will forever remain one of the greatest reformers and promoters of true human progress.

I think that the publication of this short biography will be useful to many.

\textbf{Leo Tolstoy.}

Yasnaya Polyana, January, 1904.

\footnotetext[1]{A Short Biography of William Lloyd Garrison, by V. Tchertkoff and F. Holah. London: The Free Age Press, 1904.

}\chapter{William Lloyd Garrison as Seen by a Grandson, by Oswald Garrison Villard}
\label{chapter-5}
\section{I}
When I was asked to set down on paper an estimate of William Lloyd Garrison, I at first declined. A grandson to write modestly and usefully about a beloved grandsire? It appeared an impossible undertaking, but others far more competent being not available and I being again urged, it seemed but right that I should at least make the effort and leave it to the readers to decide how largely I had failed to preserve the unbiased historical attitude.

It is perhaps in my favor that I am too far removed from the great Abolition conflict to have even a memory of that heroic era. Yet something of the spirit of it must have surrounded my cradle, since nothing in all my life has seemed as vital as the fundamental doctrines of the Abolitionists or as thrilling as the greatest of moral wars in America. As a small child I was never wholly unconscious of the gracious personality, of the kindly, benevolent, and serene spirit which dominated that home in Roxbury, to which every member of the family came in completed love and devotion in order to round out a circle of extraordinary gaiety and happiness unmarred by discord of any kind.

But even to the child there was something in those mild eyes behind the gold spectacles and those wonderfully firm lines around the generous mouth of the patriarch of the home that gave the little boy concern if he strayed from the straight and narrow path. One of his childish recollections is of the nonresistant grandfather leading a terrified, and therefore decidedly non-resistant, imp of a grandson into durance vile in a cupboard-prison, not for defending anybody’s human rights, but for having wickedly and malevolently assailed other urchins with what the Abolitionists would, in their Biblical language, have called “carnal weapons.” In the transformation which then came over the kindly eyes and in the set of the firm mouth of the Liberator, now turned jailer, the little boy found much that in later years, as he has looked back upon that incident, has made it easy for him to picture his grandfather in the heat of most uncompromising attacks upon the powers of evil, to see him, in his mind’s eye, facing the Rynders’ mob, in New York City, or writing that ringing declaration that he would not retreat a single inch and that he would be heard.

If the little boy quailed and retreated when militant destiny in grand-parental form overtook him in the midst of wrongdoing, he never had any desire to question the justice of the penalty awarded. And here again a chance bit of discipline has made the then little offender feel instinctively throughout his existence that the great moving, abiding desire of that unusual life which touched his own so briefly, yet influenced it so profoundly, was the desire for justice, for “exact justice,” to use that homely, vigorous and redundant phrase that so often appears in the grandfather s writings. No love for the offender, however dear, no hatred of wrongdoing, but was governed in him by that one great principle of doing justice that justice might come out of it all.

It was not only a happy, but a rarely serene home over which the boy’s grandfather presided from the time that he had a home of his own. The world at large might think of it as one forever stirred by intense emotions, now swayed by vibrant gusts of passion, now thrilled by the prospect of personal violence, now roused to extremes by hot fits of outraged indignation. But within its walls were singular peace and content. Where the spirit of the leader was so cheerful and so unfailingly serene there could be no question as to the correctness of the chart of life or of the methods of pursuing that great and all controlling objective of human freedom. Could such dauntless poise, such durable satisfaction in living—to borrow a phrase of ex-President Eliot’s—such equanimity of spirit, such an atmosphere of joy and of absolute faith be possible if the fundamental philosophy were wrong? No one could enter his house and doubt the uplifting nature of the struggle in which Garrison was engaged, or its righteousness. There is no more concrete example among all the many Abolition homes of similar worth and similarly unselfish devotion, of the spiritual rewards which come to those who labor in the vineyards of humanity.

\section{II}
Do we not often overlook what the cause did for the reformer while asking what the reformer did for the cause? Well, here is a case where a cause took hold of a youth, untutored and unknown, but gifted with absolute truth and the power of terrible expression. It gave him the opportunity to expand and enrich his own spirit, and every day of his living. To each wrong Garrison’s soul vibrated as the strings of aeolian harps to the winds of the forest; but the tones that his human instrument gave forth varied according to where you stood when you heard. To some who listened there came only the most rasping dissonances and discords, portending nothing but treason to the existing order of society; to other auditors the sounds were of divine sweetness. Whatever the message of the tones or their timbre, there can be no question as to their power to stir. You might hate or you might love, but listen unaffected you could not, and always the evolution of the man himself made him the truer philosopher at home and the more stinging a censor outside.

And that evolution became so noteworthy because his own self was so completely subordinated. He had no ambitions save to serve; he found in serving his causes the most complete happiness. The opportunity for leadership was his unrestricted because there was nothing to hold him back. When Wendell Phillips was paying for his audacity by social ostracism, by the loss of his Harvard friends, by forfeiting chances of political preferment, Garrison had practically no sacrifices to make. He had no social position to lose. He was in debt to nobody. No one had any hold upon him with which to padlock his utterance. And so it was only a question of how far his abilities would carry him in the leadership of the movement. The strain of caring for his family he bore more easily than most men. He had sublime faith that the Lord would provide even if he gave away his one reserve suit to someone who needed it more than he, or if he should perish at the hands of a broadcloth mob. The wolf might be at the door, but the guest was there, too, and the latch-string ever hung out, no matter how bare the cupboard or how anxious the housewife. Then he insisted on leading an absolutely blameless private life, to the great vexation of his enemies, whom his use of Biblical language and apt quotation of the Scriptures embarrassed and amazed beyond words. In short, he was the despair of all opponents.

What could you do with a man like this? You threw him in jail and he liked it immensely and utilized the opportunity to strike off his best bit of verse. You put a price upon his head and he gloried in it. You threatened him with death and dragged him through the streets with a rope around his waist and he showed his courage by failing even to be excited, and then went home to utilize your outbreak against him in a most effective sermon against the thing you were trying to uphold. You ridiculed him as a nobody and he calmly admitted it and went on preaching his gospel of liberty. You tried to close the mails to him, to undermine his influence, to destroy his reputation, his judgment and his sanity, and he went on pounding you, convicting you out of your own mouth and printing in his column in \emph{The Liberator} headed “The Refuge of Oppression” tell-tale happenings which portrayed at its worst the institution you were seeking to defend. And then when in despair you sought the aid of the law and of the officials of his State to prosecute him you found that he so walked by day and by night that you could not even indict him for running an underground railroad station, which he was careful not to because of his conspicuousness.

You called him an effeminate fanatic because he would stand up for the cause of woman in a day when there were fewer suffragists than Abolitionists, and he went calmly on insisting that his platform was not occupied unless women stood upon it. You called him crack-brained because he crossed ’the ocean to plead for the slave and then declined to speak for what was dearest to his soul because the women delegates with him were not admitted to the Convention. You denounced him as the friend and ally of Susan B. Anthony and Lucy Stone, yes, of Mary Walker, and Garrison’s hopelessly addled brain took it all as a compliment and gloried in ’these spiritual alliances which some would have dubbed his shame. How could you keep your patience with a man like this,—one who would not only let loose the savage Negro on the community, but believed that every walk of life, every sphere of intellectual activity should be open to the despised sex and actually preached that women, not men, should decide what place women should occupy in modern society.

And always he insisted upon being happy, no matter what you said or did to him. His heart was as the heart of a child even when he denounced a whole class in language of complete immoderation. It did not affect that heart when you assured him for the five thousandth time that he had thrown away his influence by his unbridled pen, ruined his cause by his vituperation and his denunciation of good men and women who did their duty by their slaves like Christians, or by taking up the despicable case of those masculine females who wished the ballot. You certainly cannot wage polemical warfare with an antagonist like this. He will not play fair; he does not follow the rules of the game. He enters the combat in such a shining armor of happiness and personal righteousness and complete unselfishness as to make it impossible for the point of your sword to enter at any point. And all the time he is belaboring you with his heavy broadsword with the utmost of calmness and most annoying vigor.

\section{III}
From Garrison came ever a torrent of denunciation of wrong-doing too Niagara-like, if you please, and marked, perhaps, as his latest biographer, Mr. John Jay Chapman avers, with a restricted and uncertain intellectual grasp. But it was what the hour called for; with all its ponderousness, with all its Biblical allusions and quotations, it was clear, wonderfully telling and still more wonderfully suited to the sturdy middle-class audience and the time to which it appealed. Mr. Chapman realizes this, for side by side with his criticism he writes:

\begin{quotation}\
	“Garrison was not a man of this kind. His mission was more lowly, more popular, more visible; and his intellectual grasp was restricted and uncertain. Garrison was a man of the market place. Language to him was not the mere means of stating truth, but a mace to break a jail. He was to be the instrument of great and rapid changes in public opinion during an epoch of terrible and fluctuating excitement. The thing which he is to see, to say, and to proclaim, from moment to moment, is as freshly given to him by prodigal nature, is as truly spontaneous, as the song of the thrush. He never calculates, he acts upon inspiration; he is always ingenuous, innocent, self-poised, and, as it were, inside of some self-acting machinery which controls his course, and rolls out the carpet of his life for him to walk on.
\end{quotation}

And he adds:

\begin{quotation}\
	“All this part of Garrison’s mental activity is his true vocation. Here he rages like a lion of Judah. By these onslaughts he is freeing people from their mental bonds; he is shaking down the palaces of Babylon.”
\end{quotation}

Out of the Bible he drew not only his spiritual weapons, but from it came much of his calm philosophy, his complete adoption of the example of Jesus as the model for himself, his belief in the necessity of non-resistance. To quote Mr. Chapman again:

\begin{quotation}\
	“To Garrison, the Bible was the many-piped organ to which he sang the song of his life, and the arsenal from which he drew the weapons of his warfare. I doubt if any man ever knew the Bible so well, or could produce a text to fit a political emergency with such startling felicity as Garrison... I doubt whether Cromwell or Milton could have rivaled Garrison in this field of quotation; and the power of quotation is as dreadful a weapon as any which the human intellect can forge. From his boyhood upward Garrison’s mind was soaked in the Bible and in no other book.”
\end{quotation}

Yet while poring thus over the greatest of books has made a gloomy or somber nature out of many a man, the spontaneity, the gaiety, the serenity of Garrison’s nature was never affected by it. John Brown by contrast was a typical Roundhead, gravely earnest and intense and usually dull, as well as ponderous, both in his writings and in the dourness of his life, while Garrison’s was full of cheer and sunshine and optimism, distinguished by its benevolence as well as by its unwavering faith in the eventual victory. This is the point I would most dwell upon—the essential richness of Garrison’s life and its complete happiness, good-humor, sweetness and kindliness. There is a story of a Southerner who, meeting Garrison on a boat, fell into conversation with him and wound up by saying:

\begin{quotation}\
	“I have been much interested, sir, in what you have said, and the exceedingly frank and temperate manner in which you have treated the subject. If all Abolitionists were like you, there would be much less opposition to your enterprise. But, sir, depend upon it, that hare-brained, reckless, violent fanatic Garrison will damage, if he does not shipwreck, any cause.”
\end{quotation}

Men and women were invariably amazed as they met him at the gentleness of his spirit and marvelled that such harsh and bitter language, such prophet-like denunciation could issue from one whose countenance was so benign. They soon saw that there was no personal animus, no sense of personal wrong save that which Jesus Himself might have felt at the wronging of one of His fellow human beings. It was the sense of impersonal justice which was the mainspring of it all. Hard as it is for even a sympathizer to read it today into Garrison’s utterances, a real desire to help the slave-holder was ever active within him. If it be said that he often concealed his affection for that particular fellow-man beneath torrential excoriation, the answer is that the grandson who writes these lines is not wrong in feeling that the abiding desire of his unusual life was absolute justice. The very fact that the grandfather loved or cared for any one or anybody made all the more vigorous the pointing out of wrong-doing; his discipline became the more stern, his blows more like those of a sledgehammer, if he but felt that he could thus serve best the erring he really cared for.

\section{IV}
The enigma of personality—what can surpass it in its profound mystery? The fundamental whys we ask about it remain unanswered though the analysis of the spirit’s feelings and of the brain’s methods go on apace. Why should the love of mankind and the readiness to give one’s life to the lowly and despised have so controlled the existence of three such different spirits as Wendell Phillips, John Brown and William Lloyd Garrison? Why did this inspiration come to their souls and not to any other three of the multitude of their generation? John Brown, at the age of fifty-five years, when most men’s thoughts begin to turn from the activities of life to the past and the future, suddenly changed from a simple guardian of flocks and tiller of the soil, a farmer, surveyor, cattle expert and wool merchant, to an arch plotter of many disguises, a belligerent pioneer, an assailant of sovereign government, bent on overthrowing human bondage by raising the banner of physical revolt.

The contagion of reform affected Garrison and Phillips, on the other hand, in early youth—but the very dissimilarity of their early lives, their social surroundings, and their prospects makes all the more amazing the fact that both should on this fundamental issue of human brotherhood have seen eye to eye. Sometimes I think this all a question of birth; that there is no such thing as changing other people’s opinions, once formed, or eradicating the seeds of hatred and oppression if once they begin to open and sprout within one; that all moral reform means marshalling those who are unfettered and unprejudiced, stirring them morally and then proving that they are in the majority, either because of numbers or because of the ethical soundness of their cause. Certainly Garrison, the son of an errant sea captain, and Wendell Phillips, the scion of aristocracy, would seem to bear out part of this theory at least. But why must they needs become latter-day disciples to revivify faith in the brotherhood of man, the fundamental doctrines of Christ?

For one thing, the three similar yet strongly dissimilar men illustrate the essential democracy of any such great reform as the liberation of the slaves. Irresistibly such a movement draws from all classes of society those made free by communion with the eternal truths; those who are willing to apply fundamental principles to any given situation and measure each by the yard-stick of an ideal, no matter how impractical the test may seem. The difference between the reformer and the non-reforming citizen is defined by the readiness of the one to take his stand on a principle and to face obloquy, ridicule and hate while striving to apply that principle to all conditions, under all circumstances. Most important of all, there enters into the soul of the reformer the divine desire to serve, no matter what the price, if only thereby the world can be made to advance by as much as the fraction of an inch.

{[}\emph{This essay reprinted by kind permission from} The World Tomorrow.{]}

\chapter{Bibliography}
\label{chapter-6}
\section{Works About William Lloyd Garrison}
An American Hero; the Story of William Lloyd Garrison. Written for Young People by Frances E. Cooke. London: Swan Sonnenschein \& Co. 1888.

A Brief Sketch of the Trial of William Lloyd Garrison for an Alleged Libel on Francis Todd, of Massachusetts. {[}Baltimore, 1830?{]}

A Discourse on William Lloyd Garrison and the Anti-Slavery Movement. Delivered at the Church of the Saviour, Brooklyn, N. Y., by A. P. Putnam, June 1, 1879. Brooklyn, N. Y., Tremlett \& Co., 1879.

The Garrison Mob in Boston, October 21, 1835. By Ellis Ames. (Proceedings of the Massachusetts Historical Society. February, 1881, p. 340-344.)

Garrison, the Non-Resistant. By Ernest Howard Crosby. Chicago, The Public Publishing Co., 1905.

The Martyr Age of the United States. By Harriet Martineau. (London and Westminster Review, December, 1838. Reprinted in pamphlet form. New York: S. W. Benedict, 1839.)

A Memorial of William Lloyd Garrison from the City of Boston. Boston: Printed by order of the City Council, 1886.

The Moral Crusader, William Lloyd Garrison; a Biographical Essay founded on The Story of Garrison’s Life told by his children. By Goldwin Smith. Toronto: Williamson \& Co. 1892. Also New York: Funk \& Wagnalls. \emph{The difference between these editions is in the typography and the frontispiece portrait.}

A National Testimonial to William Lloyd Garrison. Boston: 1866. \emph{An invitation to contribute towards a national testimonial of not less than fifty thousand dollars to William Lloyd Garrison. Signed by many subscribers and the Executive Committee, J. A. Andrew, chairman.}

Papers Relating to the Garrison Mob. Edited by Th. Lyman, 3rd. Cambridge: Welch, Bigelow \& Co. 1870.

Rebels and Reformers, Biographies for Young People. By Arthur and Dorothea P. Ponsonby. London: G. Allen \& Unwin Ltd. {[}1917.{]}

A Short Biography of William Lloyd Garrison. By V. Tchertkoff and F. Holah; with an Introductory Appreciation of his Life and Work by Leo Tolstoy. London: The Free Age Press. 1904.

A Sketch of the Character and Defence of the Principles of William Lloyd Garrison; being an Address delivered before the Managers of the “Maine Anti-Slavery Society,” in Portland, Me., on the evening of the 1st of November, 1833, by One of the Board. Published by request. New York: Printed by Henry R. Piercy, 7 Theatre Alley. 1833.

Pamphlet, with imprint: Memoir of William Lloyd Garrison. \emph{The author was James Frederic Otis, one of the founders of the American Anti-Slavery Society and signers of the Declaration of Sentiments at Philadelphia in the month following the Address. (He was afterwards a backslider from the abolition faith, but survived emancipation, dying February 2, 1867.)}

The Story of a Noble Life: William Lloyd Garrison. By William E. A. Axon. London: S. W. Partridge \& Co. 1890. (Onward Series.)

Tributes to William Lloyd Garrison at the Funeral Services, May 28, 1879. Boston: Houghton, Osgood \& Co. 1879.

The frontispiece a profile view (heliotype) of the portrait bust by Anne Whitney.

Visit of William Lloyd Garrison to Manchester. Manchester, England, United Kingdom Alliance. {[}1867?{]}

William Lloyd Garrison. By John Jay Chapman. New York: Moffat, Yard \& Co. 1913. Second edition, revised and enlarged. Boston: The Atlantic Monthly Press. 1921.

William Lloyd Garrison. By Lindsay Swift. Philadelphia: G. W. Jacobs \& Company. 1911. (American Crisis Biographies.)

William Lloyd Garrison. By N. Moore. Boston: Ginn \& Co. 1888.

William Lloyd Garrison. Von Georg Gizycki. Berlin: A. Asher \& Co. 1890. \emph{(Authorized German translation and condensation of the Story of His Life told by His Children.)}

William Lloyd Garrison. By Mary Howitt. \emph{In the “People’s Journal” for September 12, 19, 26, and October 3, 1846; with a portrait on wood drawn by H. Anelay, and probably engraved by W. J. Linton. The Memoir was reprinted in the “Pennsylvania Freeman” for March 25, 1847.}

William Lloyd Garrison; a Commemoration Discourse preached at the First Unitarian Church, Philadelphia. By Joseph May. Boston: G. H. Ellis. 1879.

William Lloyd Garrison and his Times. By Oliver Johnson; with an Introduction by John G. Whittier. Boston: B. B. Russell \& Co. 1880. (Boston: Houghton, Mifflin \& Co. 1881.)

William Lloyd Garrison, the Abolitionist; by Archibald H. Grimke. New York: Funk \& Wagnalls. 1891.

William Lloyd Garrison; the Story of his Life told by his Children. New York: The Century Co. Volumes I, II, 1885, volumes III, IV, 1889. (Boston: Houghton, Mifflin \& Co. 1894.) \emph{The documentary Life by way of eminency. It contains six portraits of Garrison, and many more of his principal coadjutors on both sides of the Atlantic; maps, facsimiles of handwriting, etc. It is the quarry from which all subsequent lives have necessarily been constructed.}

\section{Writings and Speeches of William Lloyd Garrison}
The Abolitionists and their Relations to the War; a Lecture at the Cooper Institute, New York, January 14, 1862. New York: E. D. Barker. 1862.

An Address delivered at the Broadway Tabernacle, New York, August 1, 1838, by request of the People of Color of that city, in commemoration of the complete emancipation of 600,000 slaves on that day in the British West Indies. Boston: Isaac Knapp. 1838.

An Address delivered before the Old Colony Anti-Slavery Society at South Scituate, Mass., July 4, 1839. Boston: Dow \& Jackson. 1839.

An Address delivered before the Free People of Color, in Philadelphia, New York, and other places during the month of June, 1831. Boston: Stephen Foster. 1831.

Address delivered in Boston, New York and Philadelphia before the Free People of Color in April, 1833. New York: Printed for the Free People of Color. 1833.

An Address delivered in Marlboro Chapel, Boston, July 4, 1838. Boston: Isaac Knapp. 1838.

An Address on the Progress of the Abolition Cause, delivered before the African Abolition Freehold Society of Boston, July 16, 1832. Boston: Garrison \& Knapp. 1832.

Address on the Subject of American Slavery delivered in the National Hall, Holborn, September 2, 1846. London: Richard Kinder. 1846.

Anti-Slavery Melodies for the Friends of Freedom., Prepared for the Hingham Anti-Slavery Society. Hingham, Mass.: Elijah B. Gill. 1843. \emph{Set to music. The hymns 3, 18, 27, and lyrics on pages 64, 70, are Garrison’s—the first hymn curiously razeed to fit the metre.}

Helen Eliza Garrison; a Memorial. Cambridge {[}Mass.{]}: Riverside Press. 1876.

The “Infidelity” of Abolitionism. New York. 1860.

Joseph Mazzini, His Life, Writings, and Political Principles; with an Introduction by William Lloyd Garrison. New York: Hurd \& Houghton. 1872.

Lectures of George Thompson. Compiled from various English editions; also a Brief History of His Connection with the Anti-Slavery Cause in England, by William Lloyd Garrison. Boston: Isaac Knapp. 1836.

A Letter on the Political Obligations of Abolitionists. By James G. Birney; with a Reply by William Lloyd Garrison. Boston: Dow \& Jackson. 1839.

Letter to Louis Kossuth concerning Freedom and Slavery in the United States. Boston: R. F. Wallcut. 1852.

Letters relative to the So-Called Southern Policy of President Hayes. By William E. Chandler, together with a Letter to Mr. Chandler of Mr. William Lloyd Garrison. Concord, N. H., Monitor \& Statesman Office. 1878.

The Loyalty and Devotion of the Colored Americans in the Revolution and War of 1812. Boston: R. F. Wallcut. 1861. Reprinted in 1918 by Young’s Book Exchange, New York.

“The New Reign of Terror” in the Slaveholding States, for 1859-60. Compiled by W. L. Garrison. New York: American Anti-Slavery Society. 1860. (Anti-Slavery Tracts, No. 4. New series.)

No Compromise with Slavery, an Address delivered in New York, February 14, 1854. New York: American Anti-Slavery Society, 1854.

No Fetters in the Bay State; Speech before the Committee on Federal Relations {[}of the Massachusetts Legislature{]}, Thursday, February 24, 1859. Boston. 1859.

Principles and Mode of Action of the American Anti-Slavery Society; a Speech. London: William Tweedie. {[}1853.{]} (Leeds Anti-Slavery Series, No. 86.)

Proceedings at the Public Breakfast held in honor of William Lloyd Garrison, Esq., of Boston, Massachusetts, in St. James’s Hall, London, on Saturday, June 29th, 1867. London: William Tweedie. 1868.

Proceedings of a crowded Meeting of the Colored Population of Boston, assembled the 15th July, 1846, for the purpose of bidding farewell to William Lloyd Garrison, on His Departure for England; with His Speech on the Occasion. Dublin: Webb \& Chapman. 1846.

A Selection of Anti-Slavery Hymns, for the use of the Friends of Emancipation. Boston: Garrison \& Knapp. 1834. Preface signed by William Lloyd Garrison.

Selections from the Writings and Speeches of William Lloyd Garrison. Boston: R. F. Wallcut. 1852.

Slavery in the United States of America, an

7 8 WILLIAM LLOYD GARRISON

Appeal to the Friends of Negro Emancipation throughout Great Britain. London. 1833. Signed W. L. Garrison.

Sonnets and Other Poems. Boston: Oliver Johnson. 1843.

Southern Hatred of the American Government, the People of the North, and Free Institutions. Compiled by W. L. Garrison. Boston: R. F. Wallcut. 1862.

Speeches of William Lloyd Garrison and Frederick Douglass at the great Anti-Slavery Meeting held at Paisley, Scotland, 23d September, 1846. {[}Glasgow?{]} Alex. Gardner. 1846.

Spirit of the South towards Northern Freemen and Soldiers defending the American Flag against Traitors of the Deepest Dye. Boston: R. F. Wallcut. 1861.

Thoughts on African Colonization; or, An Impartial Exhibition of the Doctrines, Principles, and Purposes of the American Colonization Society; together with the Resolutions, Addresses, and Remonstrances of the Free People of Color. “Out of thine own mouth will I condemn thee.” “Prove all things, hold fast that which is good.” Boston: Garrison \& Knapp. 1832.

Three Unlike Speeches. By W. L. Garrison, G. Davis, A. H. Stephens. The Abolitionists and their relation to the War. New York: E. D. Barker. 1862. (Pulpit and Rostrum, numbers 26-27.)

West India Emancipation; a Speech delivered at Abington, Mass., on the first day of August, 1854. Boston. 1854.

William Lloyd Garrison on State Regulation of Vice. {[}New York? 1879?{]}

Words of Garrison; a centennial selection. Boston: Houghton, Mifflin and Company. 1905. \emph{Compiled by his sons Wendell Phillips and Francis Jackson Garrison.}



\end{document}
